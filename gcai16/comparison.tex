% !TEX root = ../main.tex

Our extended \newcnf{} algorithm translates \folb{} formulas to sets of first-order clauses. It can be used in first-order theorem provers to support reasoning in \folb{}. This algorithm combines translation of \folb{} to first-order logic and clausification. This allowed us to enhance the translation by integrating clausification techniques into it. In particular, our extension integrates skolemisation, formula naming and tautology elimination. 

In what follows, we look at the translation of different features of \folb{} done by the extended \newcnf{} and point out how the integrated clausification techniques help to obtain smaller clausal normal forms. We compare the extended \newcnf{} with our translation of \folb{} formulas to full first-order logic presented in~\cite{FOOL}.

\subsection{Boolean Variables}
Our translation of \folb{} formulas to full first-order logic replaces each boolean variable $x$ occurring as formula with $x \eql \true$ and skolemises boolean variables using boolean Skolem functions. The extended \newcnf{} skolemises boolean variables using Skolem predicates and substitutes boolean variables that do not need skolemisation with constants $\true$ and $\false$. The approach taken in \newcnf{} is superior in two regards.
\begin{enumerate}
  \item The translation of \folb{} to full first-order logic converts each skolemised boolean variable $x$ occurring as formula to an equality between Skolem terms and $\true$. This translation requires a modification of superposition calculus presented in~\cite{FOOL} in order to avoid possible performance problems during superposition. \newcnf{} converts $x$ to a Skolem literal, that can be efficiently handled by standard superposition.  
  \item Substitution of a universally quantified boolean variable with $\true$ and $\false$ can decrease the size of the translation. If the variable occurs as formula, after applying the substitution, the occurrence is either removed or the whole sequent is discarded by tautology elimination.
\end{enumerate}

We note that the extended \newcnf{} naturally translates a quantified boolean formula (QBF),
as realised in the FOOL language, into a clausal normal form in effectively propositional logic (EPR). 
% Every literal in this translation is a Skolem predicate applied to boolean variables and constants $\true$ and $\false$.
This is a desirable property to have since there are first-order proving methods 
known to be efficient for dealing with this fragment (see e.g.~\cite{DBLP:conf/birthday/Korovin13}).
An alternative approach to dealing with quantifiers over interpreted sorts (such as the Booleans), 
which is complementary to skolemisation,
is employed by modern SMT solvers and relies on a guidance by counter-examples 
\cite{DBLP:journals/corr/Reynolds0K15} or on model-based projections \cite{LPAR-20:Playing_with_Quantified_Satisfaction}.

% Consider a \folb\ formula $(\forall x:\bool)(x \lor P(x)),$ where $P$ is a predicate symbol of the sort $\bool \to \bool$. The old translation converts it to $(\forall x:\bool)(x \eql true \lor P(x)),$ that yields a clause $\{ x \eql \true, P(x) \}.$ The new translation eliminate the occurrence of $x$ as formula and extends the subsitution with a binding between $x$ and $\false$. The application of the substitution results with the CNF with one clause and one literal $\{ P(\false) \}$.

\subsection{\ITE}
The extended \newcnf{} translates \ITE\ expressions occurring as formulas and as terms differently.

An \ITE\ expression that occurs as formula is translated by introducing two sequents with a copy of the condition in each one. Translation of a nesting of such \ITE\ expressions easily leads to an exponential increase in the number of sequents. This is however averted by the formula naming mechanism of \newcnf{}.

An \ITE\ expression that occurs as term is translated either by expansion or naming. Expansion doubles the number of sequents with an occurrence of the condition of \ITE, and does not introduce fresh symbols. Naming adds exactly two new sequents but introduces a fresh symbol. The naming threshold provides a trade-off between the increase of the number of sequents and the number of introduced symbols. For a large number of \ITE\ expressions it avoids the exponential increase in the number of sequents. For a small number of \ITE\ expressions inside an atomic formula it avoids growing the signature.

Formula naming averts the exponential increase in the number of sequents caused by expansion of nested \ITE\ expressions that occurs as terms. Consider for example the TPTP problem \verb'SYO500^1.003' that contains a conjecture of the form $$f_0(f_1(f_1(f_1(f_2(x))))) \eql f_0(f_0(f_0(f_1(f_2(f_2(f_2(x))))))),$$ where $f_0$, $f_1$ and $f_2$ are unary predicates that take a boolean argument and $x$ is a boolean constant. The extended \newcnf{} translates as an \ITE\ expression each application of $f_i$ that occurs as argument. Expansion of every \ITE\ expression doubles the number of sequents. However, the growth stops once the naming threshold is reached.

Our translation of FOOL formulas to full first-order logic replaces each non-boolean \ITE\ expression with an application of a fresh function symbol and adds the definition of the symbol to the set of assumptions. The definition is expressed as an equality. The extended \newcnf{} avoids introducing new equalities and uses predicate guards for naming. This avoids possible performance problems caused by self-paramodulation similar to the ones described in~\cite{FOOL}.

\subsection{\LETIN}
Our translation of \folb{} formulas to full first-order logic always names \LETIN\ expressions. The extended \newcnf{} provides the option to either name or inline \LETIN\ expressions. Naming introduces a fresh function or predicate symbol and does not multiply the number of resulting clauses. Inlining, on the other hand, does not introduce any symbols, but can drastically increase the number of clauses. Either of the translations might make a theorem prover inefficient. We point out that the number of clauses and the size of the resulting signature are not the only factors in that. For example, consider inlining of a \LETIN\ expression that defines a non-boolean term. It does not introduce a fresh function symbol and does not increase the number of clauses. However, the inlined definition might increase the size of the term with respect to the simplification ordering. This affects the order in which literals will be selected during superposition, and ultimately the performance of the prover.

Designing a syntactical criteria for choosing between naming and inlining is an interesting task for future work.