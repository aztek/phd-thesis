% !TEX root = ../main.tex

% nonnengart2001computing also have (in the simple approach): 
% elimination of equivalences as part of NNF transform 
% (but one wants to decide about equivalences during naming)
% miniscoping and variable renaming just before Skolemisation
% (but let's ignore miniscoping and assume nice variables rightaway or a detail below the level of this presentation) 
%
%???? polarity dependent elimination of equivalences section has an argument about ugly invisible tautologies 
% (like the ones we mention below)
%

Traditional approaches to
clausification in FOL \cite{nonnengart2001computing,Vampire13} produce a clausal normal
form in several stages, where each stage
represents a single pass through the formula tree. These stages may include formula simplification, translation into (equivalence) negation normal form,
formula naming, elimination of equivalences, Skolemisation, and distribution of
disjunctions over conjunctions. The \newcnf{} clausification algorithm
of~\cite{newcnf_fol} takes a different approach and employs a single top-down
traversal of the formula in which these stages are combined.  
This enables optimisations that are not available if the stages of clausification are independent. For example, compared to the traditional staged approach, \newcnf{} can introduce fewer \Skolem{} functions on formulas with complex nesting of equivalences and quantifiers. Moreover, it can detect and discard intermediate tautologies, 
which are much more difficult to recognise by the staged approach.
%
 %Another example is an easy detection of intermediate tautologies, which are discarded on the fly. \newcnf{} thus maintains a more accurate count of sub-formula occurrences, on which the decision whether to name a sub-formula is based.
%

In this paper we use the \newcnf{} algorithm and extended it to a new clausification algorithm for FOOL~\cite{FOOL}. 
The main advantage of \newcnf{} for our work, however, is that its top-down
traversal provides a suitable context not only for clausification of first-order
formulas, but also of the extension of first-order logic with \folb{}
features. In this section we overview the main features of \newcnf{}. We will follow the notation used in~\cite{newcnf_fol} and in what follows will repeat some of the definitions. 

\subsection{Preliminaries}

Our setting is that of many-sorted first-order predicate logic with equality.

% The following definition is taken from \cite{FOOL}.
%\begin{definition}\label{def:folb-signature}\em
%  A \emph{signature} of first-order logic with the Boolean sort is a triple $\Sigma = (S, F, \context)$, where:
%  \begin{enumerate}
%  \item $S$ is a set of \emph{sorts}, which contains a special sort $\bool$. A \emph{type} is either a sort or a non-empty sequence $\sigma_1,\ldots,\sigma_n,\sigma$ of sorts, written as $\sigma_1 \times \ldots \times \sigma_n \to \sigma$. When $n = 0$, we will simply write $\sigma$ instead of $\to\sigma$. We call a \emph{type assignment} a mapping from a set of variables and function symbols to types, which maps variables to sorts.
%
%    \item $F$ is a set of \emph{function symbols}. We require $F$ to contain binary function symbols $\vee$, $\wedge$, $\implies$, $\liff$ and $\lniff$, used in infix form, a unary function symbol $\neg$, used in prefix form, and nullary function symbols $\true$, $\false$.
%
%    \item $\context$ is a \emph{type assignment} which maps each function symbol $f$ into a type $\tau$. When the signature is clear from the context, we will write $\ofsort{f}{\tau}$ instead of $\context(f)=\tau$ and say that $f$ is of the type $\tau$.
%
%    We require the symbols $\vee, \wedge, \implies, \liff, \lniff$ to be of the type $\bool \times \bool \to \bool$, $\neg$ to be of the type $\bool \to \bool$ and $\true,\false$ to be of the type $\bool$. \QED
%  \end{enumerate}
%\end{definition}

A signature $\Sigma$ is a set of \emph{predicate} and \emph{function} symbols together with associated sorts. 
A \emph{term} of the sort $\tau$ is of the form $f(t_1,\ldots,t_n)$, $c$ or $x$ where
$f$ is a \emph{function symbol} of the sort $\tau_1\times\ldots\times\tau_n \to \tau$, 
$t_1,\ldots, t_n$ are terms of sorts $\tau_1,\ldots,\tau_n$, respectively, 
$c$ is a constant of sort $\tau$ and $x$ is a variable of sort $\tau$. 
%
An \emph{atom} is of the form $p(t_1,\ldots,t_n)$, $q$ or $t_1 \eql t_2$ where $p$ is a \emph{predicate symbol} of the sort $\tau_1\times\ldots\times\tau_n$, % \to \bool$
$t_1, \ldots, t_n$ are terms of sorts $\tau_1,\ldots,\tau_n$, respectively,
$q$ is a predicate symbol of sort $\bool$ and $\eql$ is the \emph{equality symbol}.  
A literal is an atom or its negation.

A \emph{formula} is of the form $\varphi_1 \wedge \ldots \wedge \varphi_n$, $\varphi_1 \vee \ldots \vee \varphi_n$, $\varphi_1 \implies \varphi_2$, $\varphi_1 \liff \varphi_2$, $\varphi_1 \lniff \varphi_2$, $\neg \varphi_1$, $\exists x : \tau. \varphi_1$, $\forall x : \tau. \varphi_1$, $\bot$, $\top$, or $l$ where $\varphi_i$  are formulas, $x$ is a variable, $\tau$ a sort and $l$ is a literal. Note that we treat conjunction and disjunction as $n$-ary operators; we assume that formulas are kept in \emph{flattened form}, e.g.\ $(\varphi_1 \wedge \varphi_2) \wedge \varphi_3$ is always represented as $\varphi_1 \wedge \varphi_2 \wedge \varphi_3$. Furthermore, we assume that usage of $\top$ and $\bot$ is simplified immediately. 
%Let ${\sf fvars}(\varphi)$ be the \emph{free variables} of formula $\varphi$, 
%i.e.~those variables in $\varphi$ with an occurrence not bound by a quantifier.

A \emph{sign} is a either $\possign$ or $\negsign$. A \emph{signed formula} is a
pair consisting of a formula $\phi$ and a sign $\sign \in \{\possign,\negsign\}$, denoted by
$\genlit{\phi}{\sign}$.
 The signed formula $\varphi^\possign$ (resp.\ $\varphi^\negsign$) means that $\varphi$ is true (resp.\ false). We will use the mapping $\formName$
from signed formulas to formulas defined as follows:
$\form{\genlit{\phi}{\possign}} = \phi$ and
$\form{\genlit{\phi}{\negsign}} = \neg \phi$. 
We call a \emph{sequent} a finite set of signed formulas. We say that a sequent
$S_1,\ldots,S_n$ is true in a FOOL interpretation if so is the universal closure
of the formula $\form{S_1} \lor \ldots \lor \form{S_n}$. Note
that if $S_1,\ldots,S_n$ are signed \emph{atomic} FOL formulas, then
$\form{S_1} \lor \ldots \lor \form{S_n}$ is a clause.

\subsection{\newcnf{}}

The \newcnf{} algorithm \cite{newcnf_fol} works with finite sets of sequents. During computation
the algorithm may construct substitutions to be applied to existing (signed) formulas. 
It is convenient for us to collect these substitutions without immediately applying
them. For this reason, instead of a sequent $D\subst$, where $\subst$ is a
substitution, we will use pairs $\genclause{D}{\subst}$ consisting of a sequent
$D$ and a substitution $\subst$. We will (slightly informally) also refer to
such pairs as sequents. 


The \newcnf{} algorithm starts with the input first-order formula $\phi$ and a
set $\GC$ of sequents that contains a single sequent
$\genclause{\{\genlit{\phi}{\possign}\}}{\emptySubst}$, where $\emptySubst$ is
the empty substitution. Then it makes a series of steps replacing sequents in
$\GC$ by other sequents until all sequents in $\GC$ contain only signed atomic
FOL formulas. Some of the steps introduce fresh (previously unused) symbols. 
% A replacement of a sequent may introduce Skolem functions and names of subformulas. 
Each update of $\GC$ preserves the following invariants: (1) if an interpretation ${\cal I}$
satisfies all sequents after the update, then ${\cal I}$ also satisfies all sequents
before the update; (2) if an interpretation ${\cal I}$
satisfies all sequents before the update, then
there exists an interpretation ${\cal I}'$ 
that extends ${\cal I}$ on fresh symbols such that ${\cal I}'$
satisfies all sequents after the update. 

The replacements of sequents are guided by the structure of $\phi$. \newcnf{}
traverses $\phi$ top-down, processing every non-atomic subformula of $\phi$
exactly once in an order that respects the subformula relation. That is, for
each two distinct subformulas $\psi_1$ and $\psi_2$ of $\phi$ such that $\psi_1$
is a subformula of $\psi_2$, $\psi_2$ is processed before $\psi_1$.
%
For every subformula of $\phi$, \newcnf{} maintains a list of its occurrences as signed formulas in the sequents of $\GC$. 
The occurrences are updated whenever sequents are removed from and added to $\GC$. 
The main role of the list is to allow for a fast enumeration and lookup of all the occurrences when a particular subformula 
is to be processed. As explained below, the number of occurrences is also used to decided whether a subformula should be named.
The replacements are governed by a set of rules that are, essentially, the standard tableau rules for first-order logic. We briefly summarise these rules below, and refer to \cite{newcnf_fol} for details.

We note that except for the rule for negation, which essentially 
flips the sign of each occurrence of $\psi = \neg \gamma$ and replaces $\psi$ with its immediate sub-formula $\gamma$
in all the sequents, the remaining rules come in pairs in which they are dual to each other. 
For instance, dealing with a disjunction $\gamma_1 \lor \gamma_2$ with a positive $\sign = \possign$
is analogous to dealing with a conjunction with a negative sign. 
For simplicity, we only show the versions for $\sign = \possign$ below.

Let $\psi$ be a subformula of $\phi$ and $\genclause{D}{\subst}$ be a sequent such that $D$ has an occurrence of $\genlit{\psi}{\possign}$.
Before proceeding to the next subformula, \newcnf{} visits and replaces all such sequents $D$.
%
Depending on the top-level connective of $\psi$ the algorithm applies the following rules.
\begin{itemize}
\item
Suppose that $\psi$ is of the form $\neg \gamma$. Add a sequent to $\GC$ obtained from $D$ by replacing the occurrence of $\genlit{\psi}{\possign}$ with $\genlit{\gamma}{\negsign}$.

\item
Suppose that $\psi$ is of the form $\gamma_1 \lor \gamma_2$. Add a sequent to $\GC$ obtained from $D$ by replacing the occurrence of $\genlit{\psi}{\possign}$ with $\genlit{\gamma_1}{\possign}, \genlit{\gamma_2}{\possign}$.

\item
Suppose that $\psi$ is of the form $\gamma_1 \land \gamma_2$. Add two sequents to $\GC$ obtained from $D$ by replacing the occurrence of $\genlit{\psi}{\possign}$ with $\genlit{\gamma_1}{\possign}$ and $\genlit{\gamma_2}{\possign}$, respectively.

\item
Suppose that $\psi$ in of the form $\gamma_1 \liff \gamma_2$. Add two sequents to $\GC$ obtained from $D$ by replacing the occurrence of $\genlit{\psi}{\possign}$ with $\genlit{\gamma_1}{\possign}, \genlit{\gamma_2}{\negsign}$ and $\genlit{\gamma_1}{\negsign}, \genlit{\gamma_2}{\possign}$, respectively.

\item
Suppose that $\psi$ in of the form $\gamma_1 \lniff \gamma_2$. Add two sequents to $\GC$ obtained from $D$ by replacing the occurrence of $\genlit{\psi}{\possign}$ with $\genlit{\gamma_1}{\possign}, \genlit{\gamma_2}{\possign}$ and $\genlit{\gamma_1}{\negsign}, \genlit{\gamma_2}{\negsign}$, respectively.

\item
Suppose that $\psi$ is of the form $(\forall x:\tau)\gamma$. Add a sequent obtained from $D$ by replacing the occurrence of $\genlit{\psi}{\possign}$ with $\genlit{\gamma}{\possign}$.

\item
Suppose that $\psi$ is of the form $(\exists x:\tau)\gamma$. Let $y_1,\ldots,y_n$ be all free variables of $\psi \subst$ and $\tau_1,\ldots,\tau_n$ be their sorts. Introduce a fresh Skolem function symbol $\sk$ of the sort $\tau_1\times\ldots\times\tau_n\to\tau$. Add a sequent $\genclause{D'}{\subst'}$, where $D'$ is obtained from $D$ by replacing the occurrence of $\genlit{\psi}{\possign}$ with $\genlit{\gamma}{\possign}$, 
and $\subst'$ extends $\subst$ with $x \mapsto \sk(y_1,\ldots,y_n)$.
\end{itemize}
%
When all subformulas of $\phi$ are traversed and the respective rules of replacing sequents are applied, the set $\GC$ only contains sequents with signed atomic formulas. $\GC$ is then converted to a set of first-order clauses by applying the substitution of each sequent to its respective formulas.

Whenever the number of occurrences of a subformula $\psi$ in sequents in $\GC$ exceeds a pre-specified {\it naming threshold}, 
$\psi$ is named as follows. Let $y_1,\ldots,y_n$ be free variables of $\psi$ and $\tau_1,\ldots,\tau_n$ be their sorts.
\newcnf{} introduces a new predicate symbol $P$ of the sort
$\tau_1\times\ldots\times\tau_n$. %  \to\bool$. 
Then, each occurrence $\genlit{\psi}{\sign}$ in sequents in $\GC$ is replaced by $\genlit{P(y_1,\ldots,y_n)}{\sign}$.
Finally, two sequents
$\genclause{\{\genlit{P(y_1,\ldots,y_n)}{\negsign},\genlit{\psi}{\possign}\}}{\emptySubst}$
and
$\genclause{\{\genlit{P(y_1,\ldots,y_n)}{\possign},\genlit{\psi}{\negsign}\}}{\emptySubst}$
are added to $\GC$ to serve as a definition of $\psi$. 
% \todo{Shouldn't we mention polarity naming? EK: is the next sentence not enough?}
As usual, in case $\psi$ always occurs in $C$ only under a single sign, 
adding only the one respective defining sequent is sufficient. 
%Also, in the case of FOOL, naming can be more complicated since variable renaming might be required.
%\todo{The previous sentence is vague and a bit weird.}

%If the number of occurrences of $\psi$ does not exceed the naming threshold, 
%each of the sequents that have an occurrence of $\genlit{\psi}{\possign}$ or $\genlit{\psi}{\negsign}$
%is replaced with one or more new sequents according to a specific rule,
%depending on the top-level connective of $\psi$.

% \newcommand{\spcr}{0.5em}
% \begin{figure}[t]
% \begin{center}
% \fbox{\begin{minipage}{\textwidth}
% \[
% \begin{array}{l}

% \text{Given a subformula $\psi$ and a sequent $\genclause{D}{\subst}$ such that $D$ has an occurrence of $\genlit{\psi}{\possign}$}

% ~\\~\\

% \begin{array}{lll}

% \vspace{\spcr}

% \text{if}~\psi = \neg \gamma & \Rightarrow & \genlit{\gamma}{\possign}

% \\ \vspace{\spcr}

% \text{if}~\psi = \gamma_1 \lor \gamma_2 & \Rightarrow & \genlit{\gamma_1}{\possign}, \genlit{\gamma_1}{\possign}

% \\ \vspace{\spcr}

% \text{if}~\psi= \gamma_1 \land \gamma_2 & \Rightarrow & \genlit{\gamma_1}{\possign} \text{~and~} \genlit{\gamma_1}{\possign}

% \\ \vspace{\spcr}

% \text{if}~\psi = \gamma_1 \liff \gamma_2 & \Rightarrow & \genlit{\gamma_1}{\possign}, \genlit{\gamma_1}{\negsign} \text{~and~} \genlit{\gamma_1}{\negsign}, \genlit{\gamma_1}{\possign}

% \\ \vspace{\spcr}

% \text{if}~\psi = \gamma_1 \lniff \gamma_2 & \Rightarrow & \genlit{\gamma_1}{\possign}, \genlit{\gamma_1}{\possign} \text{~and~} \genlit{\gamma_1}{\negsign}, \genlit{\gamma_1}{\negsign}

% \\ \vspace{\spcr}

% \text{if}~\psi = (\forall x:\tau) \gamma & \Rightarrow & \genlit{\gamma}{\possign}

% \\ \vspace{\spcr}

% \text{if}~\psi = (\exists x:\tau) \gamma & \Rightarrow & TODO

% \end{array}

% \end{array}
% \]
% \end{minipage}}
% \end{center}
% \caption{A summary of \newcnf\ rules.\label{fig:rules}}
% \end{figure}

Whenever a new sequent $\genclause{D}{\subst}$ is constructed, \newcnf{} eliminates immediate tautologies and redundant formulas. It means that
\begin{enumerate}
  \item if $D$ contains both $\genlit{\psi}{\possign}$ and $\genlit{\psi}{\negsign}$, $\genclause{D}{\subst}$ is not added to $\GC$;
  \item if $D$ contains multiple occurrences of a signed formula, only one occurrence is kept in $D$;
  \item if $D$ contains $\genlit{\top}{\possign}$ or $\genlit{\bot}{\negsign}$, $\genclause{D}{\subst}$ is not added to $\GC$;
  \item if $D$ contains a signed formula $\genlit{\bot}{\possign}$ or
    $\genlit{\top}{\negsign}$, this signed formula is removed from $D$.
\end{enumerate}
% \todo{Probably no need to mention tautology elimination anymore, since it's described in the VCNF paper}
These rules are not required for replacing sequents, however they simplify formulas and make the resulting set of clauses smaller.

\newcnf{} takes as an input a first-order formula in \emph{equivalence negation normal form} (ENNF). A formula is in ENNF if it does not contain $\implies$ and negations are only applied to atoms. ENNF is very convenient for
standard FOL, as it reduces the number of cases to consider and makes checking polarities trivial. At the same time, it is not easy to define a useful extension of ENNF for FOOL because of \LETIN\ expressions and formulas inside terms. It is straightforward, however, to extend \newcnf{} in order to support formulas in full first-order logic. For that, we need to add an extra rewriting rule for implications. In what follows we will consider a modification of \newcnf{} with this extension.