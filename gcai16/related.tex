% !TEX root = main.tex
\folb{} is a relatively new extension of FOL. We are not aware of any work that explicitly deals with clausifying formulas in this logic. However, connections can be found in work focusing on related fragments and extensions.

Most notably, Wisniewski et al.~propose in % their recent work 
\cite{DBLP:conf/cade/WisniewskiSKB16}
methods for normalising formulas in higher-order logic (HOL). Similarly to \folb{},
HOL natively contains the boolean sort. Wisniewski et al.~deal with 
formulas occurring at argument positions by a technique called \emph{argument extraction} 
which, similarly to our naming schemes, extends the signature and defines a new symbol
outside the original formula. Moreover, also Wisniewski et al. introduce skolem predicates 
instead of skolem functions when dealing with existential boolean quantifiers. 
This happens implicitly for them, since in HOL there is no distinction between formulas and terms.

\folb{} can be regarded as a superset of SMT-LIB \cite{BarFT-SMTLIB} core logic and formulas of SMT-LIB core logic can be directly expressed in \folb{}. The language of \folb{} extends the SMT-LIB core language with local function definitions, using \LETIN\ expressions defining functions of arbitrary, and not just zero, arity. 

Despite the similarity of the languages, the technology used by modern SMT solvers~\cite{DBLP:journals/jacm/NieuwenhuisOT06}
differs greatly from that of % saturation-based 
first-order theorem provers and so do the approaches to normalising the input formula.
In particular, as SMT solvers pass the propositional abstraction of the input formula to an efficient SAT solver
there is no great need to optimise extensions of the signature 
and clausification usually follows the simple Tseitin encoding~\cite{tseitin_enc} of the formula tree.
%
% E-matching \cite{DBLP:journals/jacm/DetlefsNS05,DBLP:conf/cade/MouraB07}
Moreover, modern SMT solvers employ an alternative approach to dealing with quantifiers over interpreted sorts such as the booleans, 
which is complementary to skolemisation
and relies on a guidance by counter-examples~\cite{DBLP:journals/corr/Reynolds0K15} or on model-based projections~\cite{LPAR-20:Playing_with_Quantified_Satisfaction}.

Finally, it is interesting to note that our \nfcnf{} algorithm naturally translates a quantified boolean formula (QBF),
as realised in the FOOL language, into a CNF in effectively propositional logic (EPR).
Specifically, every literal in this translation is 
a skolem predicate applied to boolean variables and constants $\true$ and $\false$.
This result is similar to the one proposed in \cite{DBLP:conf/cade/SeidlLB12},
where the authors explicitly focus on QBF as the source and EPR as the target language, respectively.
Obtaining a formula in EPR is a desirable property % to have 
since there are first-order proving methods 
known to be efficient for dealing with the fragment (see e.g.~\cite{DBLP:conf/birthday/Korovin13}).