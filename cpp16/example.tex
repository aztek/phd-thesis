\begin{figure*}[b]
  \vspace{-1em}
  \begin{center}
    \parbox{0cm}{
      \begin{tabbing}
        res \ass\ x;\\
        \IF\ (x $>$ y) \\\inc
        \THEN\ max \ass\ x;\\
        \ELSE\ max \ass\ y;\\\dec
        \IF\ (max $>$ 0) \\\inc
        \THEN\ res \ass\ res $+$ max;\\
        \ELSE\ res \ass\ res $-$ max;\\[.5em]\dec
        \reserved{assert} res $\geq$ x
      \end{tabbing}
    }
  \end{center}
  \vspace{-2em}
  \caption{Sequence of conditionals.\label{fig:seqITE}}
\end{figure*}

\begin{figure*}[tb]
{\small
\begin{lstlisting}[language=tptp]
tff(x, type, x: $int).
tff(y, type, y: $int).
tff(max, type, max: $int).
tff(res, type, res: $int).
tff(res1, type, res1: $int).

tff(transition_relation, hypothesis,
    res1 = $let(res := x,
           $let(max := $ite($greater(x, y),
                            $let(max := x, max),
                            $let(max := y, max)),
           $let(res := $ite($greater(max, 0),
                            $let(res := $sum(res, max), res),
                            $let(res := $difference(res, max),
                                 res)),
                res)))).

tff(safety_property, conjecture, $greatereq(res1, x)).
\end{lstlisting}
\caption{Representation of the partial correctness statement of the code on Figure~\ref{fig:seqITE} in Vampire\label{fig:VampireITE}.}
}
\end{figure*}

In this section we illustrate how FOOL makes first-order theorem
provers better suited to applications in program analysis and
verification.
Firstly,  we give concrete examples of the use of FOOL for
expressing program properties. We avoid various
program analysis steps, such as SSA form computations and renaming
program variables; instead we show how program properties can directly
be expressed in FOOL.
 We also present a technique for
automatically generating the next state relation of any program with
assignments, \ITE, and sequential composition.
For doing so,  we introduce a simple extension of FOOL,
allowing for a general translation that is linear in the size of the
program.
This is a new result intended to understand which extensions of
first-order logic are adequate for naturally representing fragments of
imperative programs.

%%%%%%%%%%%%%%%%%%%%%%%%%%%%%%%%%%%%%%%%%%%%%%%%
\subsection{Encoding the Next State Relation}\label{sec:foolp}

Consider the program given in
Figure~\ref{fig:seqITE}, written in a C-like syntax, using a sequence
of two conditional statements.
The program first computes the maximal value $\mathit{max}$ of two integers $x$ and
$y$ and then adds the absolute value of $\mathit{max}$ to $x$. A safety assertion,
in FOL, is specified at the end of the loop, using the
{\bf assert} construct. This program is clearly safe, the
assertion is satisfied. To prove program safety, one needs to reason
about the program's transition relation, in particular reason about
conditional statements, and express the final value of
the program variable $\mathit{res}$. The partial correctness of the program
of Figure~\ref{fig:seqITE} can be \emph{automatically} expressed in FOOL,
and then Vampire can be used to prove program safety.
This requires us to encode (i)
the next state value of $\mathit{res}$ (and $\mathit{max}$) as a hypothesis
in the extended TFF0 syntax of FOOL,
by using the \ITE\ ({\tt \$ite}) and \LETIN\ ({\tt \$let})
constructs, and (ii)
the safety property as the conjecture to be proven by Vampire.

Figure~\ref{fig:VampireITE} shows this extended TFF0 encoding.
The use of \ITE\ and \LETIN\ constructs allows us to have a
direct  encoding of the  transition relation of
Figure~\ref{fig:seqITE} in FOOL. Note that each expression from the program appears only once in the encoding.

We now explain how the encoding of the next state values of program
variables can be generated automatically.
We consider programs using assignments \texttt{:=},
\ITE\ and sequential composition $;$.
We begin by making an assumption about the structure of programs (which we relax later). A program $P$ is in \emph{restricted form} if for any subprogram of the form \IF\ e \THEN\ $P_1$ \ELSE\ $P_2$ the subprograms $P_1$ and $P_2$ only make assignments to the same single variable. Given a program $P$ in restricted form let us define its translation $[P]$ inductively as follows:
%
\begin{itemize}
	\item $[$x\ASS e$]$ is $\letin{x}{e}{x}$;
	\item $[$\IF\ e \THEN\ $P_1$ \ELSE\ $P_2]$, where $P_1$ and
          $P_2$ update $x$,  is $\letin{x}{\ite{e}{[P_1]}{[P_2]}}{x}$;
	\item $[P_1$;$P_2]$ is $\mathtt{let}~D~\mathtt{in}~[P_2]$ where $[P_1]$ is $\mathtt{let}~D~\mathtt{in}~x$.
\end{itemize}
%
Given a program $P$, the next state value for variable $x$ can be
given by $[P$; x\ASS x$]$,
i.e. by ensuring the final statement of the program updates the
variable of interest.
The restricted form is required as conditionals must be viewed
as assignments in the translation and assignments can only be made to single variables.

\begin{figure*}[tb]
  \vspace{-1em}
  \begin{center}
    \parbox{0cm}{
      \begin{tabbing}
        \IF\ (x $>$ y) \\\inc
        \THEN\ t \ass\ x; x \ass\ y; y \ass\ t;\\\dec
        \reserved{assert} y $\geq$ x
      \end{tabbing}
    }
  \end{center}
  \vspace*{-2em}
  \caption{Updating multiple variables.\label{fig:tmpSwap}}
\end{figure*}

To demonstrate the limitations of this restriction let us consider the simple program in Figure~\ref{fig:tmpSwap} that ensures that x is not larger than y. We cannot apply the translation as the conditional updates three variables. To generalise the approach we can extend FOOL with \emph{tuple expressions}, let us call this extension \foolp. In this extended logic the next state values for Figure~\ref{fig:tmpSwap} can be encoded as follows:
%
\[
\begin{array}{ll}
\mathtt{let}\; (x,y,t) =& \mathtt{if}\; x > y \;\mathtt{then} \\
&\quad\mathtt{let}\; (x,y,t) = (x,y,x) \;\mathtt{in}\\
&\quad\quad\mathtt{let}\; (x,y,t) = (y,y,t) \;\mathtt{in}\\
&\quad\quad\quad\mathtt{let}\;(x,y,t) = (x,t,t)  \;\mathtt{in}\; (x,y,t) \\
&\mathtt{else}\; (x,y,t)\\
\mathtt{in}\; (x,y,t) \\
\end{array}
\]
%
We now give a brief sketch of the extended logic \foolp\ and the associated translation. We omit details since its full definition and semantics would require essentially repeating definitions from \cite{FOOL}.  \foolp\ extends \fool\ by tuples; for all expressions $t_i$ of type $\sigma_i$ we can use a \emph{tuple expression} $(t_1,\ldots,t_n)$ of type $(\sigma_1,\ldots,\sigma_n)$. The logic should also include a suitable tuple projection function, which we do not discuss here.

This extension allows for a more general translation in two senses:
first, the previous restricted form is lifted; and second, it now
gives the next state values of {\it all} variables updated by the program. Given a program $P$ its translation $[P]$ will have the form $\letin{(x_1,\ldots,x_n)}{E}{(x_1,\ldots,x_n)}$, where $x_1,\ldots,x_n$ are all variables updated by $P$, that is, all variables used in the left-hand-side of an assignment. We inductively define $[P]$ as follows:
\begin{itemize}
	\item $[\text{x}_i \text{\ASS} e]$ is $\letin{(\ldots,x_i,\ldots)}{(\ldots,e,\ldots)}{(x_1,\ldots,x_n)}$,
	\item $[$\IF\ e \THEN\ $P_1$ \ELSE\ $P_2]$ is $\letin{(x_1,\ldots,x_n)}{\ite{e}{[P_1]}{[P_2]}}{(x_1,\ldots,x_n)},$
	\item $[P_1$;$P_2]$ is $\mathtt{let}~D~\mathtt{in}~[P_2]$ where $[P_1]$ is $\mathtt{let}~D~\mathtt{in}~(x_1,\ldots,x_n)$.
\end{itemize}
This translation is bounded by $O(v\cdot n)$, where $v$ is the number
of variables in the program and $n$ is the program size (number of
statements) as each program statement is used once with one or two
instances of $(x_1,\ldots,x_n)$.
This becomes $O(n)$ if we assume that the number of
variables is fixed. The translation could be refined so that some introduced  \LETIN\
expressions only use a subset of program variables.
Finally, this translation preserves the semantics
of the program.

\begin{theorem}\rm
  Let $P$ be a program with variables $(x_1,\ldots,x_n)$ and let $u_1,\ldots,u_n, v_1, \ldots, v_n$ be values (where $u_i$ and $v_i$ are of the same type as $x_i$). If $P$ changes the state $\{x_1\to u_1,\ldots,x_n\to u_n\}$ to $\{x_1\to v_1,\ldots,x_n\to v_n\}$ then the value of $[P]$ in $\{x_1\to u_1,\ldots,x_n\to u_n\}$ is $(v_1,\ldots,v_n)$.
\end{theorem}

This translation encodes the next state values of program variables by
directly following the structure of the program. This leads to a
succinct representation that, importantly, does not lose any
information or attempt to translate the program too early. This allows
the theorem prover to apply its own translation to FOL that it can
handle efficiently.   While \foolp{} is not yet fully supported in
Vampire, we believe experimenting with \foolp{} on
examples coming from program analysis and verification is an
interesting task for future work.


%%%%%%%%%%%%%%%%%%%%%%%%%%%%%%%%%%%%%%%%%%%%%%%%
\subsection{A Program with a Loop and Arrays}

\begin{figure*}[bt]
{
  \begin{center}
    \parbox{0cm}{
  \begin{tabbing}
    $a$ \ass\ $0$; $b$ \ass\ $0$; $c$ \ass\ $0$; \\[.5em]
    \reserved{invariant} a = b + c $\wedge$ \\
    {\color{white}\reserved{invariant}} a $\geq$ 0 $\wedge$ b $\geq$ 0 $\wedge$ c $\geq$
    0 $\wedge$ a $\leq$ k $\wedge$ \\
    {\color{white}\reserved{invariant}} $(\forall p) (0\leq p<b \implies
    (\exists i) (0 \leq i < a \wedge A[i] > 0 \wedge B[p] = A[i]))$\\[1em]
    \WHILE\ ($a \leq k$) \DO \\ \inc
      \IF\ ($A[a] > 0$) \\ \inc
        \THEN\ \=\+ $B[b]$ \ass\ $A[a]$\semicol $b$ \ass\ $b+1$\semicol \\ \dec
        \ELSE\ \=\+ $C[c]$ \ass\ $A[a]$\semicol $c$ \ass\ $c+1$\semicol \\ \dec \dec
      $a$ \ass\ $a+1$\semicol \\ \dec
    \OD\\[.5em]
    \reserved{assert} $(\forall p)(0 \leq p<b \implies B[p]> 0)$
  \end{tabbing}
    }
  \end{center}
  \caption{Array partition.\label{fig:partition}}
}
\end{figure*}

\begin{figure*}[tb]
\begin{lstlisting}[language=tptp]
tff(a, type, a: $int).
tff(b, type, b: $int).
tff(c, type, c: $int).
tff(k, type, k: $int).
tff(arrayA, type, arrayA: $array($int, $int)).
tff(arrayB, type, arrayB: $array($int, $int)).
tff(arrayC, type, arrayC: $array($int, $int)).

tff(invariant_property, hypothesis,
    inv <=> ((a = $sum(b, c)) &
             $greatereq(a, 0) & $greatereq(b, 0) &
             $greatereq(c, 0) & $lesseq(a, k) &
             ![P:$int]: ($lesseq(0, P) & $less(P, b) =>
              (?[I:$int]: ($lesseq(0, I) & $less(I, a) &
                $greater($select(arrayA, I), 0) &
                $select(arrayB,P) = $select(arrayA,I)))))).

tff(safety_property, conjecture,
    (inv & ~$lesseq(a, k)) =>
      (![P:$int]: ($lesseq(0, P) & $less(P, b) =>
                   $greater($select(arrayB, P), 0)))).
\end{lstlisting}
\caption{Representation of the partial correctness statement of the code on Figure~\ref{fig:partition} in Vampire\label{fig:loop_safety_Vampire}.}
\end{figure*}

Let us now show the use of FOOL in Vampire for reasoning about
programs with loops. Consider the program given in
Figure~\ref{fig:partition}, written in a C-like syntax.  The program
fills an integer-valued array $B$ by the strictly positive values
of a source array $A$, and an integer-valued array $C$ with
the non-positive values of $A$. A safety assertion, in FOL, is
specified at the end of the loop, using the {\bf assert}
construct. The program of Figure~\ref{fig:partition} is clearly safe
as the assertion is satisfied when the loop is exited.
However, to prove program safety we need additional
loop properties, that is loop invariants, that hold at any loop
iteration. These can be automatically generated using existing approaches, for
example the symbol elimination method for invariant generation in
Vampire~\cite{fase2009}. In this case we use the FOL property
specified in the {\bf invariant} construct of  Figure~\ref{fig:partition}. This invariant
property states that at any loop iteration, (i) the  sum of visited
array elements in $A$ is the sum of visited elements in $B$ and $C$
(that is, $a = b + c$), (ii) the number of visited array
elements in $A$, $B$, $C$ is positive (that is, $a\geq 0$, $b\geq 0$,
and $c\geq 0$), with $a\leq k$, and (iii) each array element
$B[0],\ldots,B[b-1]$ is a strictly positive element in
$A$. Formulating the latter property requires quantifier alternation
in FOL, resulting in the quantified property with $\forall\exists$
listed in the invariant of  Figure~\ref{fig:partition}.
We can verify the safety of the program using Hoare-style reasoning in Vampire.
The partial correctness property is that the invariant and the negation of the loop condition implies the safety assertion.
This is the conjecture to be proven by Vampire.
Figure~\ref{fig:loop_safety_Vampire} shows the encoding in the
extended TFF0 syntax of this partial
correctness statement; note that this uses the built-in theory of
polymorphic arrays in Vampire, where $arrayA$, $arrayB$ and $arrayC$
correspond respectively to the arrays $A$, $B$ and $C$.

So far, we assumed that the given invariant in
Figure~\ref{fig:partition} is
indeed an invariant. Using \foolp{} described in
Section~\ref{sec:foolp}, we can verify the inductiveness
property of the invariant, as follows: (i) express the the transition
relation of the loop in \foolp, and (ii) prove that, if the invariant
holds at an arbitrary loop iteration $i$, then it also holds at loop
iteration $i+1$. For proving this, we can again use \foolp\ to
formulate
the next state values of loop variables in the invariant at loop
iteration $i+1$.
Moreover, \foolp{} can also be used to express formulas as
inputs to the symbol elimination method for invariant generation in
Vampire. We leave the task of using \foolp{} for invariant generation
as further work.
