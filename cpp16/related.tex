% !TEX root = ../main.tex
FOOL was introduced in our previous work~\cite{FOOL}. This also presented a translation from FOOL to the ordinary first-order logic, and FOOL paramodulation. In this paper we describe the first practical implementation of FOOL and FOOL paramodulation.

Superposition theorem proving in finite domains, such as the Boolean domain, is also discussed in~\cite{HillenbrandWeidenbach13}. The approach of~\cite{HillenbrandWeidenbach13} sometimes falls back to enumerating instances of a clause by instantiating finite domain variables with all elements of the corresponding domains. Nevertheless, it allows one to also handle finite domains with more than two elements. One can also generalise our approach to arbitrary finite domains by using binary encodings of finite domains. However, this will necessarily result in loss of efficiency, since a single variable over a domain with $2^k$ elements will become $k$ variables in our approach, and similarly for function arguments.
Although \cite{HillenbrandWeidenbach13} reports preliminary results with the theorem prover SPASS, we could not make an experimental comparison since the SPASS implementation has not yet been made public.

Handling Boolean terms as formulas is common in the SMT community. The SMT-LIB project~\cite{SMT-LIB} defines its core logic as first-order logic extended with the distinguished first-class Boolean sort and the \LETIN\ expression used for local bindings of variables. The language of FOOL extends the SMT-LIB core language with local function definitions, using \LETIN\ expressions defining functions of arbitrary, and not just zero, arity.

A recent work \cite{SMTLIB2TPTP} presents SMTtoTPTP, a translator from SMT-LIB to TPTP. SMTtoTPTP does not fully support Boolean sort, however one can use SMTtoTPTP with the \verb'--keepBool' option to translate SMT-LIB problems to the extended TFF0 syntax, supported by Vampire.

Our implementation of the polymorphic theory of arrays uses a syntax that coincides with the TPTP's own syntax for polymorphically typed first-order logic TFF1~\cite{tff1}.
