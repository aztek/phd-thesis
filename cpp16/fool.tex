Our recent work~\cite{FOOL} presented a modification of many-sorted first-order logic that contains a boolean sort with a fixed interpretation and treats terms of the boolean sort as formulas. We called this logic FOOL, standing for first-order logic (FOL) + boolean sort. FOOL extends FOL by (i) treating boolean terms as formulas; (ii) \ITE\ expressions; and (iii) \LETIN\ expressions. There is a model-preserving transformation of FOOL formulas to FOL formulas, hence an implementation of this transformation makes it possible to prove FOOL formulas using a first-order theorem prover.

The language of FOOL is, essentially, a superset of the core language of SMT-LIB~2~\cite{SMT-LIB}, the library of problems for SMT solvers. The difference between FOOL and the core language is that the former has richer \LETIN\ expressions, which support local definitions of functions symbols of arbitrary arity, while the latter only supports local binding of variables.

FOOL can be regarded as the smallest superset of the SMT-LIB~2 Core language and TFF0. An implementation of a translation of FOOL to FOL thus also makes it possible to translate SMT-LIB problems to TPTP. Consider, for example, the following tautology, written in the SMT-LIB syntax: \lstinline'(exists ((x Bool)) x)'. It quantifies over boolean variables and uses a boolean variable as a formula. Neither is allowed in the standard TPTP language, but can be directly expressed in an extended TPTP that represents FOOL.

The rest of this section presents features of FOOL not included in FOL, explains how they are implemented in Vampire and how they can be represented in an extended TPTP syntax understood by Vampire.

\subsection{Proving with the Boolean Sort}

Vampire supports many-sorted predicate logic and the TFF0 syntax for this logic. In many-sorted predicate logic all sorts are uninterpreted, while the boolean sort should be interpreted as a two-element set. There are several ways to support the boolean sort in a first-order theorem prover, for example, one can axiomatise it by adding two constants $\true$ and $\false$ of this sort and two axioms: $(\forall x:\bool)(x \eql \true \lor x \eql \false)$ and $\true \neql \false$. However, as we discuss in \cite{FOOL}, using this axiomatisation in a superposition theorem prover may result in performance problems caused by self-paramodulation of $x \eql \true \lor x \eql \false$.

To overcome this problem, in \cite{FOOL} we proposed the following modification of the superposition calculus.
\begin{enumerate}
  \item Use a special simplification ordering that makes the constants $\true$ and $\false$ smallest terms of the sort $\bool$ and also makes $\true$ greater than $\false$.

\item Add the axiom $\true \neql \false$.

\item Add a special inference rule, called \emph{FOOL paramodulation}, of the form
  \[
    \infer[,]{C[\mathtt{\true}] \lor s \eql \mathtt{\false}}{C[s]}
  \]
where
\begin{enumerate}
\item $s$ is a term of the sort $\bool$ other than $\true$ and $\false$;
\item $s$ is not a variable;
\end{enumerate}
\end{enumerate}

Both ways of dealing with the boolean sort are supported in Vampire. The option \verb|--fool_paramodulation|, which can be set to \verb|on| or \verb|off|, chooses one of them. The default value is \verb|on|, which enables the modification.

Vampire uses the TFF0 subset of the TPTP syntax, which does not fully support FOOL. To write FOOL formulas in the input, one uses the standard TPTP notation: \tptpo\ for the boolean sort, \dtrue\ for $\true$ and \dfalse\ for $\false$. There are, however, two ways to output the boolean sort and the constants. One way will use the same notation as in the input and is the default, which is sufficient for most applications. The other way can be activated by the option \verb'--show_fool on', it will
\begin{enumerate}
  \item denote as \dbool\ every occurrence of $\bool$ as a sort of a variable or an argument (to a function or a predicate symbol);
  \item denote as \ddtrue\ every occurrence of $\true$ as an argument; and
  \item denote as \ddfalse\ every occurrence of $\false$ as an argument.
\end{enumerate}
Note that an occurrence of any of the symbols \dbool, \ddtrue\ or \ddfalse\ anywhere in an input problem is not recognised as syntactically correct by Vampire.

Setting \verb'--show_fool' to \verb'on' might be necessary if Vampire is used as a front-end to other reasoning tools. For example, one can use Vampire not only for proving, but also for preprocessing the input problem or converting it to clausal normal form. To do so, one uses the options \verb|--mode preprocess| and \verb|--mode clausify|, respectively. The output of Vampire can then be passed to other theorem provers, that either only deal with clauses or do not have sophisticated preprocessing. Setting \verb'--show_fool' to \verb'on' appends a definition of a sort denoted by \dbool\ and constants denoted by \ddtrue\ and \ddfalse\ of this sort to the output. That way the output will always contain syntactically correct TFF0 formulas, which might not be true if the option is set to \verb'off' (the default value).

Every formula of the standard FOL is syntactically a FOOL formula and has the same models. Vampire does not reason in FOOL natively, but rather translates the input FOOL formulas into FOL formulas in a way that preserves models. This is done at the first stage of preprocessing of the input problem.

Vampire implements the translation of FOOL formulas to FOL given in~\cite{FOOL}. It involves replacing parts of the problem that are not syntactically correct in the standard FOL by applications of fresh function and predicate symbols. The set of assumptions is then extended by formulas that define these symbols. Individual steps of the translation are displayed when the \verb'--show_preprocessing' option is set to \verb'on'.

In the next subsections we present the features of FOOL that are not present in FOL together with their syntax in the extended TFF0 and their implementation in Vampire.

\subsection{Quantifiers over the Boolean Sort}

FOOL allows quantification over $\bool$ and usage of boolean variables as formulas. For example, the formula $(\forall x:\bool)(x \lor \neg x)$ is a syntactically correct tautology in FOOL. It is not however syntactically correct in the standard FOL where variables can only occur as arguments.

Vampire translates boolean variables to FOL in the following way. First, every formula of the form $x \liff y$, where $x$ and $y$ are boolean variables, is replaced by $x \eql y$. Then, every occurrence of a boolean variable $x$ anywhere other than in an argument is replaced by $x \eql \true$. For example, the tautology $(\forall x:\bool)(x \lor \neg x)$ will be converted to the FOL formula $(\forall x:\bool)(x \eql \true \lor x \neql \true)$ during preprocessing.

Note that it is possible to directly express quantified boolean formulas (QBF) in FOOL, and use Vampire to reason about them.

TFF0 does not support quantification over booleans. Vampire supports an extended version of TFF0 where the sort symbol \tptpo\ is allowed to occur as the sort of a quantifier and boolean variables are allowed to occur as formulas. The formula $(\forall x:\bool)(x \lor \neg x)$ can be expressed in this syntax as \lstinline[language=tptp]{![X:$o]: (X | ~X)}. %$

\subsection[Functions and Predicates with Boolean Arguments]{Functions and Predicates with\\Boolean Arguments}

Functions and predicates in FOOL are allowed to take booleans as arguments. For example, one can define the logical implication as a binary function $\mathit{impl}$ of the type $\bool \times \bool \to \bool$ using the following axiom:
\[
  (\forall x: \bool)(\forall y: \bool)(\mathit{impl}(x, y) \liff \neg x \lor y).
\]

Since Vampire supports many-sorted logic, this feature requires no additional implementation, apart from changes in the parser.

In TFF0, functions and predicates cannot have arguments of the sort \tptpo. In the version of TFF0, supported by Vampire, this restriction is removed. Thus, the definition of $\mathit{impl}$ can be expressed as follows:
\begin{lstlisting}[language=tptp]
tff(impl, type, impl: ($o * $o) > $o).
tff(impl_definition, axiom,
    ![X: $o, Y: $o]: (impl(X, Y) <=> (~X | Y))).
\end{lstlisting}%$

\subsection{Formulas as Arguments}

Unlike the standard FOL, FOOL does not make a distinction between formulas and boolean terms. It means that a function or a predicate can take a formula as a boolean argument, and formulas can be used as arguments to equality between booleans. For example, with the definition of $\mathit{impl}$, given earlier, we can express in FOOL that
$P$ is a graph of a (partial) function of the type $\sigma \to \tau$ as follows:
\begin{equation}\label{eq:bool-arg-example}
  (\forall x:\sigma)(\forall y:\tau)(\forall z:\tau)\mathit{impl}(P(x,y) \land P(x,z), y \eql z).
\end{equation}

Note that the definition of $\mathit{impl}$ could as well use equality instead of equivalence.

In order to support formulas occurring as arguments, Vampire does the following. First, every expression of the form $\varphi \eql \psi$ is replaced by $\varphi \liff \psi$. Then, for each formula $\psi$ occurring as an argument the following translation is applied. If $\psi$ is a nullary predicate $\top$ or $\bot$, it is replaced by $\true$ or $\false$, respectively. If $\psi$ is a boolean variable, it is left as is. Otherwise, the translation is done in several steps. Let $x_1,\ldots,x_n$ be all free variables of $\psi$ and $\sigma_1,\ldots,\sigma_n$ be their sorts. Then Vampire
\begin{enumerate}
  \item introduces a fresh function symbol $g$ of the type $$\sigma_1 \times \ldots \times \sigma_n \to \bool;$$
  \item adds the definition $$(\forall x_1:\sigma_1)\ldots(\forall x_n:\sigma_n)(\psi \liff g(x_1,\ldots,x_n) \eql \true)$$ to its set of assumptions;
  \item replaces $\psi$ by $g(x_1,\ldots,x_n)$.
\end{enumerate}

For example, after this translation has been applied for both arguments of $\mathit{impl}$, \eqref{eq:bool-arg-example} becomes $$(\forall x:\sigma)(\forall y:\sigma)(\forall z:\sigma)\mathit{impl}(g_1(x, y, z), g_2(y, z)),$$ where $g_1$ and $g_2$ are fresh function symbol of the types $\sigma \times \tau \times \tau \to \bool$ and $\tau \times \tau \to \bool$, respectively, defined by the following formulas:
\begin{enumerate}
  \item $(\forall x:\sigma)(\forall y:\tau)(\forall z:\tau)(P(x,y) \land P(x,z) \liff g_1(x,y,z) \eql \true)$;
  \item $(\forall y:\tau)(\forall z:\tau)(y \eql z \liff g_2(y,z) \eql \true)$.
\end{enumerate}

TFF0 does not allow formulas to occur as arguments. The extended version of TFF0, supported by Vampire, removes this restriction for arguments of the boolean sort. Formula~\eqref{eq:bool-arg-example} can be expressed in this syntax as follows:
\begin{lstlisting}[language=tptp]
![X: s, Y: t, Z: t]: impl(p(X, Y) & p(X, Z), Y = Z)
\end{lstlisting}

For a more interesting example, consider the following logical puzzle taken from the TPTP problem \mbox{PUZ081}:
\begin{quote}
  A very special island is inhabited only by knights and knaves. Knights always tell the truth, and knaves always lie. You meet two inhabitants: Zoey and Mel. Zoey tells you that Mel is a knave. Mel says, `Neither Zoey nor I are knaves'. Who is a knight and who is a knave?
\end{quote}

\newcommand{\knight}{\mathit{Knight}}
\newcommand{\knave}{\mathit{Knave}}
\newcommand{\says}{\mathit{Says}}
\newcommand{\statement}{\mathit{statement}}
\newcommand{\person}{\mathit{person}}
\newcommand{\zoye}{\mathit{zoye}}
\newcommand{\mel}{\mathit{mel}}
To solve the puzzle, one can formalise it as a problem in FOOL and give a corresponding extended TFF0 representation to Vampire. Let $\zoye$ and $\mel$ be terms of a fixed sort $\person$ that represent Zoye and Mel, respectively. Let $\says$ be a predicate that takes a term of the sort $\person$ and a boolean term. We will write $\says(p, s)$ to denote that a person $p$ made a logical statement $s$. Let $\knight$ and $\knave$ be predicates that take a term of the sort $\person$. We will write $\knight(p)$ or $\knave(p)$ to denote that a person $p$ is a knight or a knave, respectively. We will express the fact that knights only tell the truth and knaves only lie by axioms $(\forall p:\person)(\forall s:\bool)(\knight(p) \land \says(p, s) \implies s)$ and $(\forall p:\person)(\forall s:\bool)(\knave(p) \land \says(p, s) \implies \neg s)$, respectively. We will express the fact that every person is either a knight or a knave by the axiom $(\forall p:\person)(\knight(p) \oplus \knave(p))$, where $\oplus$ is the ``exclusive or'' connective. Finally, we will express the statements that Zoye and Mel make in the puzzle by axioms $\says(\zoye, \knave(\mel))$ and $\says(\mel, \neg\knave(\zoye) \land \neg\knave(\mel))$, respectively.

The axioms and definitions, given above, can be written in the extended TFF0 syntax in the following way.
\begin{lstlisting}[language=tptp]
tff(person, type, person: $tType).
tff(says, type, says: (person * $o) > $o).

tff(knight, type, knight: person > $o).
tff(knights_always_tell_truth, axiom,
    ![P: person, S: $o]:
      (knight(P) & says(P, S) => S)).

tff(knave, type, knave: person > $o).
tff(knaves_always_lie, axiom,
    ![P: person, S: $o]:
      (knave(P) & says(P, S) => ~S)).

tff(very_special_island, axiom,
    ![P: person]: (knight(P) <~> knave(P))).

tff(zoey, type, zoey: person).
tff(mel,  type, mel:  person).

tff(zoye_says, hypothesis,
    says(zoey, knave(mel))).

tff(mel_says, hypothesis,
    says(mel, ~knave(zoey) & ~knave(mel))).
\end{lstlisting}%$

Vampire accepts this code, finds that the problem is satisfiable and outputs the saturated set of clauses. There one can see that Zoey is a knight and Mel is a knave. Note that the existing formalisations of this puzzle in TPTP (files \verb'PUZ081^1.p', \verb'PUZ081^2.p' and \verb'PUZ081^3.p') employ the language of higher-order logic (THF)~\cite{THF}. However, as we have just shown, one does not need to resort to reasoning in higher-order logic for this problem, and can enjoy the efficiency of reasoning in first-order logic.

%Running Vampire with the options \verb'--mode preprocess' and \verb'--show_fool on' produces the following output.
%\begin{lstlisting}
%tff(type_def_0, type, $bool: $tType).
%tff(func_def_0, type, $$false: $bool).
%tff(func_def_1, type, $$true: $bool).
%tff(func_def_4, type, bG0: $bool).
%tff(func_def_5, type, bG1: $bool).
%tff(pred_def_1, type, says: ($i * $bool) > $o).
%
%tff(u4, axiom,
%    $$false != $$true).
%
%tff(u5, axiom,
%    ![X0: $bool]: (($$true = X0) | ($$false = X0))).
%
%tff(u6,axiom,
%    ![X0]: ((says(X0, $$true) <~> says(X0, $$false)))).
%
%tff(u7, axiom,
%    says(mel, $$false) <=> ($$true = bG0)).
%
%tff(u8, hypothesis,
%    says(zoey, bG0)).
%
%tff(u9, axiom,
%    (~(says(zoey, $$false) | says(mel, $$false))) <=> ($$true = bG1)).
%
%tff(u10, hypothesis,
%    says(mel, bG1)).
%\end{lstlisting}
%
%Note that definitions \lstinline'type_def_0', \lstinline'func_def_0' and \lstinline'func_def_1', and units \lstinline'u4' and \lstinline'u5' axiomatise the theory of booleans. Units \verb'func_def_4' and \verb'func_def_5' introduce fresh function symbols, and units \verb'u7' and \verb'u9' are their definitions.

This example makes one think about representing sentences in various epistemic or first-order modal logics in FOOL.

\subsection{\ITE}

FOOL contains expressions of the form $\ite{\psi}{s}{t}$, where $\psi$ is a boolean term, and $s$ and $t$ are terms of the same sort. The semantics of such expressions mirrors the semantics of conditional expressions in programming languages.

\ITE\ expressions are convenient for expressing formulas coming from program analysis and interactive theorem provers. For example, consider the $\mathit{max}$ function of the type $\Z \times \Z \to \Z$ that returns the maximum of its arguments. Its definition can be expressed in FOOL as
\begin{equation}\label{eq:ite-t-example}
  (\forall x:\Z)(\forall y:\Z)(\mathit{max}(x, y) \eql \ite{x \geq y}{x}{y}).
\end{equation}

To handle such expressions, Vampire translates them to FOL. This translation is done in several steps. Let $x_1,\ldots,x_n$ be all free variables of $\psi$, $s$ and $t$, and $\sigma_1,\ldots,\sigma_n$ be their sorts. Let $\tau$ be the sort of both $s$ and $t$. The steps of translation depend on whether $\tau$ is $\bool$ or a different sort. If $\tau$ is not $\bool$, Vampire
\begin{enumerate}
  \item introduces a fresh function symbol $g$ of the type $$\sigma_1 \times \ldots \times \sigma_n \to \tau;$$
  \item adds the definitions
\begin{equation*}
\begin{aligned}
(\forall x_1:\sigma_1)\ldots(\forall x_n:\sigma_n) (\psi &\implies g(x_1,\ldots,x_n) \eql s),\\
(\forall x_1:\sigma_1)\ldots(\forall x_n:\sigma_n) (\neg\psi &\implies g(x_1,\ldots,x_n) \eql t)
\end{aligned}
\end{equation*} to its set of assumptions;
  \item replaces $\ite{\psi}{s}{t}$ by $g(x_1,\ldots,x_n)$.
\end{enumerate}

If $\tau$ is $\bool$, the following is different in the steps of translation:
\begin{enumerate}
  \item a fresh predicate symbol $g$ of the type $\sigma_1 \times \ldots \times \sigma_n$ is introduced instead; and
  \item the added definitions use equivalence instead of equality.
\end{enumerate}
\noindent
For example, after this translation \eqref{eq:ite-t-example} becomes $$(\forall x:\Z)(\forall y:\Z)(\mathit{max}(x, y) \eql g(x, y)),$$ where $g$ is a fresh function symbol of the type $\Z \times \Z \to \Z$ defined by the following formulas:
\begin{enumerate}
  \item $(\forall x:\Z)(\forall y:\Z)(x \geq y \implies g(x, y) \eql x)$;
  \item $(\forall x:\Z)(\forall y:\Z)(x \not\geq y \implies g(x, y) \eql y).$
\end{enumerate}

TPTP has two different expressions for \ITE: \ditet\ for constructing terms and \ditef\ for constructing formulas. \ditet\ takes a formula and two terms of the same sort as arguments. \ditef\ takes three formulas as arguments.

Since FOOL does not distinguish formulas and boolean terms, it does not require separate expressions for the formula-level and term-level \ITE. The extended version of TFF0, supported by Vampire, uses a new expression \dite, that unifies \ditet\ and \ditef. \dite\ takes a formula and two terms of the same sort as arguments. If the second and the third arguments are boolean, such \dite\  expression is equivalent to \ditef, otherwise it is equivalent to \ditet.

Consider, for example, the above definition of $\mathit{max}$. It can be encoded in the extended TFF0 as follows.
\begin{lstlisting}[language=tptp]
tff(max, type, max: ($int * $int) > $int).
tff(max_definition, axiom,
    ![X: $int, Y: $int]:
      (max(X, Y) = $ite($greatereq(X, Y), X, Y))).
\end{lstlisting}%$
It uses the TPTP notation \dint\ for the sort of integers and \dgreatereq\ for the greater-than-or-equal-to comparison of two numbers.

% This is working example for TPTP, should we use it instead?
%\begin{lstlisting}
%![X:$int, Y:$int]: (max(X, Y) = $ite($greatereq(X, Y), X, Y))
%\end{lstlisting}

Consider now the following valid property of $\mathit{max}$:
\begin{equation}\label{eq:ite-f-example}
  (\forall x:\Z)(\forall y:\Z)(\ite{\mathit{max}(x, y) \eql x}{x \geq y}{y \geq x}).
\end{equation}

Its encoding in the extended TFF0 can use the same \dite\ expression:
\begin{lstlisting}[language=tptp]
![X: $int, Y: $int]:
  $ite(max(X, Y) = X, $greatereq(X, Y), $greatereq(Y, X)).
\end{lstlisting}%$

Note that TFF0 without \dite\ has to differentiate between terms and formulas, and so requires to use \ditet\ in~\eqref{eq:ite-t-example} and \ditef\ in~\eqref{eq:ite-f-example}.

\subsection{\LETIN}

FOOL contains \LETIN\ expressions that can be used to introduce local function definitions. They have the form
\begin{equation}\label{eq:let}
\begin{aligned}
\mathtt{let}\;&\binding{f_1(x^1_1:\sigma^1_1,\ldots,x^1_{n_1}:\sigma^1_{n_1})}{s_1};\\
              &\ldots\\
              &\binding{f_m(x^m_1:\sigma^m_1,\ldots,x^m_{n_m}:\sigma^m_{n_m})}{s_m}\\
 \mathtt{in}\;&t,
\end{aligned}
\end{equation}
where
\begin{enumerate}
  \item $m \geq 1$;
  \item $f_1,\ldots,f_m$ are pairwise distinct function symbols;
  \item $n_i \geq 0$ for each $1 \leq i \leq m$;
  \item $x^i_1\ldots,x^i_{n_i}$ are pairwise distinct variables for each $1 \leq i \leq m$; and
  \item $s_1,\ldots,s_m$ and $t$ are terms.
\end{enumerate}

%We will write a \LETIN\ expression simply as $\letin{f(x_1:\sigma_1,\ldots,x_n:\sigma_n)}{s}{t}$ if $m = 1$.

The semantics of \LETIN\ expressions in FOOL mirrors the semantics of simultaneous non-recursive local definitions in programming languages. That is, $s_1,\ldots,s_m$ do not use the bindings of $f_1,\ldots,f_m$ created by this definition.

Note that an expression of the form \eqref{eq:let} is not in general equivalent to $m$ nested \LETIN s
\begin{equation}\label{eq:let-singles}
\begin{aligned}
&\mathtt{let}\;\binding{f_1(x^1_1:\sigma^1_1,\ldots,x^1_{n_1}:\sigma^1_{n_1})}{s_1}\;\mathtt{in}\\
&\quad\;\ddots\\
&\quad\quad\;\mathtt{let}\;\binding{f_m(x^m_1:\sigma^m_1,\ldots,x^m_{n_m}:\sigma^m_{n_m})}{s_m}\;\mathtt{in}\\
&\quad\quad\quad t.
\end{aligned}
\end{equation}
The main application of \LETIN\ expressions is in problems coming from program analysis, namely modelling of assignments. Consider for example the following code snippet featuring operations over an integer \lstinline'array'.
\begin{lstlisting}[language=cpp]
array[3] := 5;
array[2] + array[3];
\end{lstlisting}
It can be translated to FOOL in the following way. We represent the integer array as an uninterpreted function $\arrayt$ of the type $\Z \to \Z$ that maps an index to the array element at that index. The assignment of an array element can be translated to a combination of \LETIN\ and \ITE.
\begin{equation}\label{eq:let-function-example}
\begin{aligned}
  &\mathtt{let}\;\binding{\arrayt(i:\Z)}{\ite{i \eql 3}{5}{\arrayt(i)}}\;\mathtt{in}\\
  &\quad\arrayt(2) + \arrayt(3)
\end{aligned}
\end{equation}

%Let $\context$ be a type assignment. Let $f_1,\ldots,f_m$ be pairwise distinct function symbols, and %$x_{i,1}\ldots,x_{i,n}$ be pairwise distinct variables for each $1 \leq i \leq n$. If %$$\context,x_{i,1}:\sigma_{i,1}\ldots,x_{i,n}:\sigma_{i,n} \vdash s_i:\sigma_i$$ for each $1 \leq i \leq n$, and
%\begin{align*}
%\context,\;&f_1:(\sigma_{1,1} \times \ldots \times \sigma_{n,1} \to \sigma_1),\\
%           &\ldots\\
%           &f_m:(\sigma_{m,1} \times \ldots \times \sigma_{m,n} \to \sigma_m)\vdash t:\tau,
%\end{align*}
%then
%\begin{align*}
%\context \vdash (\mathtt{let}\;&\binding{f_1(x_{1,1}:\sigma_{1,1},\ldots,x_{1,n}:\sigma_{1,n})}{s_1};\\
%                               &\ldots\\
%                               &\binding{f_m(x_{m,1}:\sigma_{m,1},\ldots,x_{m,n}:\sigma_{m,n})}{s_m}\\
%                  \mathtt{in}\;&t):\tau.
%\end{align*}
%In other words, it is a term of the sort $\tau$. Note that it is equivalent to a \LETIN\ expression with a single function binding when $m = 1$.

Multiple bindings in a \LETIN\ expression can be used to concisely express simultaneous assignments that otherwise would require renaming. In the following example, constants $a$ and $b$ are swapped by a \LETIN\ expression. The resulting formula is equivalent to $f(b, a)$.
\begin{equation}\label{eq:parallel-let-example}
\letinpar{a}{b}{b}{a}{f(a, b)}
%\begin{aligned}
%  &\mathtt{let}\;\binding{a}{1};\;\binding{b}{2}\;\mathtt{in}\\
%  &\quad\mathtt{let}\;\binding{a}{b};\;\binding{b}{a}\;\mathtt{in}\\
%  &\quad\quad f(a, b)
%\end{aligned}
\end{equation}

In order to handle \LETIN\ expressions Vampire translates them to FOL. This is done in three stages for each expression in \eqref{eq:let}.
\begin{enumerate}
  \item For each function symbol $f_i$ where $0 \leq i < m$ that occurs freely in any of $s_{i+1},\ldots,s_m$, introduce a fresh function symbol $g_i$. Replace all free occurrences of $f_i$ in $t$ by $g_i$.
  \item Replace the \LETIN\ expression by an equivalent one of the form \eqref{eq:let-singles}. This is possible because the necessary condition was satisfied by the previous step.
  \item Apply a translation to each of the \LETIN\ expression with a single binding, starting with the innermost one.
\end{enumerate}

The translation of an expression of the form $$\letin{f(x_1:\sigma_1,\ldots,x_n:\sigma_n)}{s}{t}$$ is done by the following sequence of steps. Let $y_1,\ldots,y_m$ be all free variables of $s$ and $t$, and $\tau_1,\ldots,\tau_m$ be their sorts. Note that the variables in $x_1,\ldots,x_n$ are not necessarily disjoint from the variables in $y_1,\ldots,y_m$. Let $\sigma_0$ be the sort of $s$. The steps of translation depend on whether $\sigma_0$ is $\bool$ and not. If $\sigma_0$ is not $\bool$, Vampire
\begin{enumerate}
  \item introduces a fresh function symbol $g$ of the type $$\sigma_1 \times \ldots \times \sigma_n \times \tau_1 \times \ldots \times \tau_m \to \sigma_0;$$
  \item adds to the set of assumptions the definition
  \begin{align*}
    &(\forall z_1:\sigma_1)\ldots(\forall z_n:\sigma_n) (\forall y_1:\tau_1)\ldots(\forall y_m:\tau_m)\\
    &\quad(g(z_1,\ldots,z_n,y_1,\ldots,y_m) \eql s'),
  \end{align*} where $z_1,\ldots,z_n$ is a fresh sequence of variables and $s'$ is  obtained from $s$ by replacing all free occurrences of $x_1,\ldots,x_n$ by $z_1,\ldots,z_n$, respectively; and
  \item replaces $\letin{f(x_1:\sigma_1,\ldots,x_n:\sigma_n)}{s}{t}$ by $t'$, where $t'$ is obtained from $t$ by replacing all bound occurrences of $y_1,\ldots,y_m$ by fresh variables and each application $f(t_1, \ldots, t_n)$ of a free occurrence of $f$ by $g(t_1, \ldots, t_n,\allowbreak y_1, \ldots, y_m)$.
\end{enumerate}

If $\sigma_0$ is $\bool$, the steps of translation are different:
\begin{enumerate}
  \item a fresh predicate symbol of the type \[\sigma_1 \times \ldots \times \sigma_n \times \tau_1 \times \ldots \times \tau_m\] is introduced instead;
  \item the added definition uses equivalence instead of equality.
\end{enumerate}

For example, after this translation \eqref{eq:let-function-example} becomes $g(2) + g(3)$, where $g$ is a fresh function symbol of the type $\Z \to \Z$ defined by the following formula: $$(\forall i:\Z)(g(i) \eql \ite{i \eql 3}{5}{\arrayt(i)}).$$

The example~\eqref{eq:parallel-let-example} is translated in the following way. First, the \LETIN\ expression is translated to the form~\eqref{eq:let-singles}. The constant $a$ has a free occurrence in the body of $b$, therefore it is replaced by a fresh constant $a'$. The formula \eqref{eq:parallel-let-example} becomes
\begin{equation*}
\begin{aligned}
  &\mathtt{let}\;\binding{a'}{b}\;\mathtt{in}\\
  &\quad\mathtt{let}\;\binding{b}{a}\;\mathtt{in}\\
  &\quad\quad\ f(a', b).
\end{aligned}
\end{equation*}
Then, the translation is applied to both \LETIN\ expressions with a single binding and the resulting formula becomes $f(a'', b')$, where $a''$ and $b'$ are fresh constants, defined by formulas $a'' \eql b$ and $b' \eql a$.

TPTP has four different expressions for \LETIN: \dlettt\ and \dletft\ for constructing terms, and \dlettf\  and \dletff\  for constructing formulas. All of them denote a single binding. \dlettt\ and \dlettf\ denote a binding of a function symbol, whereas \dletft\ and \dletff\ denote a binding of a predicate symbol. All four expressions take a (possibly universally quantified) equation as the first argument and a term (in case of \dlettt\ and \dletft) or a formula (in case of \dlettf\ and \dletff) as the second argument. TPTP does not provide any notation for \LETIN\  expressions with multiple bindings.

Similarly to \ITE, \LETIN\ expressions in FOOL do not need different notation for terms and formulas. The modification of TFF0 supported by Vampire introduces a new \dlet\ expression, that unifies \dlettt, \dletft, \dlettf\ and \dletff, and extends them to support multiple bindings. Depending on whether the binding is of a function or predicate symbol and whether the second argument of the expression is term or formula, a \dlet\ expression is equivalent to one of \dlettt, \dletft, \dlettf\ and \dletff.

The new \dlet\ expressions use different syntax for bindings. Instead of a quantified equation, they use the following syntax: a function symbol possibly followed by a list of variable arguments in parenthesis, followed by the \lstinline':=' operator and the body of the binding. Similarly to quantified variables, variable arguments are separated with commas and each variable might include a sort declaration. A sort declaration can be omitted, in which case the variable is assumed to the be of the sort of individuals.

Formula \eqref{eq:let-function-example} can be written in the extended TFF0 with the TPTP interpreted function \dsum, representing integer addition, as follows:
\begin{lstlisting}[language=tptp]
$let(array(I:$int) := $ite(I = 3, 5, array(I)),
     $sum(array(2), array(3))).
\end{lstlisting}

%A \LETIN\ defintion of a constant \eqref{eq:let-constant-example} can be expressed as follows.
%\begin{lstlisting}
%$let(d := f(f(c)), g(d, d))
%\end{lstlisting}

The same \dlet\ expression can be used for multiple bindings. For that, the bindings should be separated by a semicolon and passed as the first argument. The formula~\eqref{eq:parallel-let-example} can be written using \dlet\ as follows.
\begin{lstlisting}[language=tptp]
$let(a := b; b := a, f(a, b)))
\end{lstlisting}

Overall, \dite\ and \dlet\ expressions provide a more concise syntax for TPTP formulas than the TFF0 variations of \ITE\ and \LETIN\  expressions. To illustrate this point, consider the following snippet of TPTP code, taken from the TPTP problem \mbox{SYN000\_2}.
\begin{lstlisting}[language=tptp]
tff(let_binders, axiom, ![X: $i]:
    $let_ff(![Y1: $i, Y2: $i]: (q(Y1, Y2) <=> p(Y1)),
      q($let_tt(![Z1: $i]:
         (f(Z1) = g(Z1, b)), f(a)), X) &
      p($let_ft(![Y3: $i, Y4: $i]: (q(Y3, Y4) <=>
         $ite_f(Y3 = Y4, q(a, a), q(Y3, Y4))),
         $ite_t(q(b, b), f(a), f(X)))))).
\end{lstlisting}

It uses both of the TFF0 variations of \ITE\ and three different variations of \LETIN. The same snippet can be expressed more concisely using \dite\ and \dlet\ expressions.
\begin{lstlisting}[language=tptp]
tff(let_binders, axiom, ![X: $i]:
    $let(q(Y1, Y2) := p(Y1),
         q($let(f(Z1) := g(Z1, b), f(a)), X) &
         p($let(q(Y3, Y4) := $ite(Y3 = Y4,
                                  q(a, a), q(Y3, Y4))),
           $ite(q(b, b), f(a), f(X)))))).
\end{lstlisting}
