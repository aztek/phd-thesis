% !TEX root = ../main.tex
The language of monomorphically typed first-order formulas TFF0 is not sufficient to express formulas of FOOL and polymorphic theory of arrays. In order to support them Vampire uses an extended version of TFF0. In the sequel we will refer to it as \extTFF. In this section we summarise the differences between TFF0 and \extTFF.

\subsection{Boolean Sort}
In TFF0 the symbol \tptpo\ can only occur as the result sort of a predicate symbol. In \extTFF\ it denotes a proper boolean sort and is allowed to occur as the sort of an argument, a result sort, and the sort of a quantifier.

In TFF0 variables are not allowed to occur as formulas. In \extTFF\ boolean variables are allowed to occur as formulas.

In TFF0 formulas are not allowed to occur as arguments. In \extTFF\ formulas are allowed to occur as boolean arguments.

In TFF0 equality cannot be used for formulas. In \extTFF\ equality can be used for formulas to indicate logical equivalence.

\subsection{\ITE}
TFF0 has two different expressions for \ITE: \ditet\ for constructing terms and \ditef\ for constructing formulas. \ditet\ takes a formula and two terms of the same sort as arguments. \ditef\ takes three formulas as arguments.

\extTFF\ instead uses the \dite\  expression, that takes a formula and two terms of the same sort as arguments. If the second and the third arguments are boolean, such \dite\ expression is equivalent to TFF0's \ditef, otherwise it is equivalent to \ditet.

\subsection{\LETIN}
TFF0 has four different expressions for \LETIN: \dlettt\ and \dletft\ for constructing terms, and \dlettf\ and \dletff\ for constructing formulas. All of them denote a single binding. \dlettt\ and \dlettf\ denote a binding of a function symbol, whereas \dletft\ and \dletff\ denote a binding of a predicate symbol. All four expressions take a (possibly universally quantified) equation as the first argument and a term (in case of \dlettt\ and \dletft) or a formula (in case of \dlettf\ and \dletff) as the second argument.

\extTFF\ instead uses the \dlet\ expression, that can have multiple bindings. \dlet\ takes a list of bindings separated with a semicolon as the first argument and the body of the expression as the second argument. Each binding has the following syntax: a function symbol possibly followed by a list of variable arguments in parenthesis, followed by the \lstinline':=' operator and the body of the binding.

A \dlet\ expression with a single binding is equivalent to one of TFF0's \dlettt, \dletft, \dlettf\ or \dletff\ expressions depending on whether the binding is of a function or a predicate symbol and whether the second argument of the expression is a term or a formula.

\subsection{Arrays}
TFF0 does not provide any syntax for arrays. Previous versions of Vampire used symbols \darrayone, \dselectone\ and \dstoreone\ for integer arrays and \darraytwo, \dselecttwo\ and \dstoretwo\ for arrays of integer arrays. \darrayone\ denoted the sort of an integer array. \dselectone\ denoted the operation of extracting an element of an integer array by its index. \dstoreone\ denoted the operation of updating an integer array at a given index with a given value. \darraytwo, \dselecttwo\ and \dstoretwo\ denoted the same for arrays of integer arrays.

\extTFF\ supports a more general syntax for polymorphic theory of arrays. For each sort $\sigma$, symbols \darraySymb, \dselect\ and \dstore can be used for arrays of the sort $\sigma$. ??? denotes the sort of an array of the sort $\sigma$, where \lstinline's' denotes $\sigma$. \dstore\ denotes the operation of extracting an element of an array of the sort $\sigma$ by its index. \dselect\ denoted the operation of updating an array of the sort $\sigma$ at a given index with a given value.

The sort of integer arrays and arrays of integer arrays can be expressed in \extTFF\ as \darray{\dint}{\dint} and \darray{\dint}{\darray{\dint}{\dint}}, respectively. The equivalent of both \dstoreone\ and \dstoretwo\ in \extTFF\ is \dstore\ and the equivalent of both \dstoreone\ and \dstoretwo\ is \dstore.
