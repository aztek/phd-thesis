Automated theorem proving is one of the central areas of computer mathematics. It studies methods and techniques for establishing validity of mathematical problems using a computer. The problems are expressed in a variety of formal logics, including first-order logic. Algorithms of automated theorem proving are implemented in computer programs called theorem provers. They find significant application in formal methods of system development and as a mean of automation in proof assistants.

This thesis contributes to automated theorem proving with an extension of many-sorted first-order logic called FOOL. In \folb{} boolean sort has a fixed interpretation and boolean terms are treated as formulas. In addition, \folb{} contains \ITE{} and \LETIN{} constructs. We argue that these extensions are useful for expressing problems coming from program analysis and interactive theorem proving. 

We give a formalisation of \folb{} and a translation of \folb{} formulas to ordinary first-order logic. This translation can be used for proving theorems of FOOL using a first-order theorem prover. We describe our implementation of this translation in the Vampire theorem prover. We extend TPTP, the standard input language of first-order provers, to support formulas of \folb{}. We simplify TPTP by providing more powerful and uniform representations of \ITE{} and \LETIN{} expressions.

We discuss a modification of superposition calculus that can reason efficiently about formulas with interpreted boolean sort. We present a superposition-friendly translation of \folb{} formulas to clausal normal form. We demonstrate usability and high performance of these modifications in Vampire on a series of benchmarks coming from various libraries of problems for automated provers.

Finally, we present an extension of \folb, aimed to be used for automated program analysis. With this extension, the next state relation of a program can be expressed as a boolean formula which is linear in the size of the program.