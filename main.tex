\documentclass[10pt,twoside]{book}
\usepackage{amssymb,amsmath,amsthm,mathrsfs,stmaryrd,dsfont,bbold,mathtools}
\usepackage[english]{babel}
\usepackage[lighttt]{lmodern}
\usepackage[inline]{enumitem}
\usepackage{yfonts}
\usepackage{pifont}
\usepackage{tikz}
\usepackage{color}
\usepackage{proof}
\usepackage{url}
\usepackage{multirow}
\usepackage{subfig}
\usepackage{wrapfig}

% !TEX root = main.tex

\usepackage[nottoc]{tocbibind}

\theoremstyle{definition}
\newtheorem{definition}{Definition}[chapter]
\newtheorem{theorem}{Theorem}[chapter]
\newtheorem{lemma}{Lemma}[chapter]
\newtheorem{example}{Example}[chapter]
\newtheorem*{example*}{Example}

\usepackage{setspace}
\renewcommand{\baselinestretch}{1.1} 

\usepackage{titlesec}

\usepackage[T1]{fontenc}
\usepackage[stretch=10]{microtype}
\usepackage{hyphenat}

\usepackage[labelsep=period]{caption}

% \widowpenalty10000
% \clubpenalty10000

% title page
\usepackage{titlepage}

\usepackage{titlesec}

\titleformat{\chapter}[display]
  {\linespread{1.0}\huge} % format
  {\vspace{-80pt}\sc\large Chapter \thechapter} % label
  {0pt} % sep
  {}[\vspace{1em}]

\titleformat{name=\chapter,numberless}[display]
  {\linespread{1.0}\huge}
  {}
  {-86pt}
  {}[\vspace{1em}]

\titlespacing*{\chapter}{0pt}{120pt}{6pt}

\titleformat{\section}
  {\normalfont\large\bfseries}
  {\thesection}
  {1em}
  {}%[\vspace{-0.1em}]
\titlespacing*{\section}{0pt}{20pt}{6pt}

% \usepackage{secdot}

% \titlelabel{\thetitle~}
% \renewcommand{\thechapter}{\arabic{chapter}.}
% \renewcommand{\thesection}{\thechapter\arabic{section}.}
% \renewcommand{\thesubsection}{\thesection\arabic{subsection}.}

\usepackage[pageref]{backref}
\usepackage{textcomp}
\renewcommand*{\backref}[1]{}
\renewcommand*{\backrefalt}[4]{%
    \ifcase #1 \\\----\ Not cited%
    \or        \\\----\ One citation on page~#2%
    \else      \\\----\ #1 citations on pages~#2%
    \fi}
\bibliographystyle{plain}

\usepackage{color}
\definecolor{lstgrey}{rgb}{0.96,0.96,0.96}
\usepackage{listings}

\lstset{
  basicstyle=\ttfamily,
  columns=flexible,
  keepspaces=true,
  backgroundcolor=\color{lstgrey},
  breaklines=false,
  sensitive=true,
  captionpos=b,
  frame=single,
  framerule=0pt,
  framesep=2pt,
  xleftmargin=3pt,
  xrightmargin=3pt,
  resetmargins=true,
  rulecolor=\color{lstgrey}
  % frameround=tttt,
%  literate={~}{{\textapprox}}1
}

\usepackage{xpatch}
\usepackage{realboxes}
\makeatletter
\xpretocmd\lstinline
  {%
   \bgroup\fboxsep=1.5pt
   \Colorbox{lstgrey}\bgroup\vphantom{\ttfamily\char`\\y}%
   \appto\lst@DeInit{\egroup\egroup}%
  }{}{}
\makeatother

% \makeatletter
% \xpretocmd\lstinline{\Colorbox{lstgrey}\bgroup\appto\lst@DeInit{\egroup}}{}{}
% \makeatother


\lstdefinelanguage{tptp}{
  keywords={tff, thf, fof, cnf, type, axiom, hypothesis, conjecture}
}

\lstdefinelanguage{cpp}{
  keywords={public, static, void, do, for, while, int, bool, if, then, else, break, assert, let, in}
}

\lstdefinelanguage{appendixcpp}{
  keywords={public, static, void, do, for, while, int, bool, if, else, break, assert}
}

\lstdefinelanguage{bnf}{}

\newcommand{\reserved}[1]{\textbf{\underline{#1}}} % reserved words in algorithms
\newcommand{\ass}{\texttt{:=}}     % assignment operator
\newcommand{\inc}{~~~~\= \+ \kill}    % used in algorithms
\newcommand{\dec}{\- \kill}         % used in algorithms
\newcommand{\semicol}{;}                  % semicolon in algorithms

\newcommand{\INPUT}{\reserved{input}}
\newcommand{\OUTPUT}{\reserved{output}}
\newcommand{\IF}{\reserved{if}}
\newcommand{\VAR}{\reserved{var}}
\newcommand{\CASE}{\reserved{case}}
\newcommand{\OF}{\reserved{of}}
\newcommand{\DO}{\reserved{do}}
\newcommand{\OD}{\reserved{end~do}}
\newcommand{\THEN}{\reserved{then}}
\newcommand{\ELSE}{\reserved{else}}
\newcommand{\WHILE}{\reserved{while}}
\newcommand{\BEGIN}{\reserved{begin}}
\newcommand{\END}{\reserved{end}}
\newcommand{\LET}{\reserved{let}}
\newcommand{\FORALL}{\reserved{forall}}
\newcommand{\ASS}{\texttt{ := }}
\newcommand{\RETURN}{\reserved{return}}
\newcommand{\REPEAT}{\reserved{repeat}}
\newcommand{\LOOP}{\reserved{loop}}

\newcommand{\extTFF}{TFF0\textsuperscript{ext}}
\newcommand{\nofoolVampire}{Vampire$\,\star$}
\newcommand{\oldcnfVampire}{Vampire$\,\star$}

\newcommand{\ite}[3]{\mathtt{if}\;{#1}\;\allowbreak\mathtt{then}\;{#2}\;\allowbreak\mathtt{else}\;{#3}}
\newcommand{\itet}[3]{\mathrm{ite\_t}({#1},\;{#2},\;{#3})}
\newcommand{\ITE}{\texttt{if}-\texttt{then}-\texttt{else}}
\newcommand{\binding}[2]{{#1}={#2}}
\newcommand{\letin}[3]{\mathtt{let}\;\binding{#1}{#2}\;\allowbreak\mathtt{in}\;{#3}}
\newcommand{\letindef}[2]{\mathtt{let}\;{#1}\;\allowbreak\mathtt{in}\;{#2}}
\newcommand{\letinpar}[5]{\mathtt{let}\;\binding{#1}{#2};\;\binding{#3}{#4}\;\mathtt{in}\;{#5}}
\newcommand{\LETIN}{\texttt{let}-\texttt{in}}
\newcommand{\ofsort}[2]{{{#1}:{#2}}}
\newcommand{\set}[2]{{\left\{\,{#1}\;|\;{#2}\,\right\}}}

\newcommand{\builtin}[1]{\mathtt{\${#1}}}
\newcommand{\true}{\mathit{true}}
\newcommand{\false}{\mathit{false}}
\newcommand{\bool}{\mathit{bool}}
\newcommand{\fool}{{FOOL}}% how should we call FOL with boolean?
\newcommand{\foolp}{{FOOL+}}

% logic
\renewcommand{\implies}{\Rightarrow}
\newcommand{\liff}{\Leftrightarrow}
\newcommand{\lniff}{\not\Leftrightarrow}
%\newcommand{\lor}{\vee}

\newcommand{\eql}{\doteq}
\newcommand{\neql}{\not\doteq}

% overstrike in math
\newcommand\hcancel[1]{\setbox0=\hbox{$#1$}%
\rlap{\raisebox{.45\ht0}{\rule{\wd0}{0.4pt}}}#1}

%
\newcommand{\context}{\eta}

\newcommand{\extension}[1]{{#1}_+}

\newcommand{\intI}{I} % interpretation I

\newcommand{\interpret}[2]{\left\llbracket\,{#1}\,\right\rrbracket_{#2}}
\newcommand{\eval}[2]{\mathrm{eval}_{#2}({#1})}
\newcommand{\replacement}[3]{{#1}_{#2}^{#3}}

\newcommand{\variant}[3]{{#1}_{#2}^{#3}}

\newcommand{\folb}{{FOOL}}% how should we call FOL with boolean?
\newcommand{\toFOL}[1]{\mathit{fol}(#1)}  % translation of syntactially FO to FO

% end of definition, theorem, proof etc.
%\newcommand{\QEDsymbol}{\text{\ding{111}}}
\newcommand{\QEDsymbol}{\qed}
\newcommand{\QED}{\hfill\QEDsymbol}

% logic
\renewcommand{\implies}{\Rightarrow}
\renewcommand{\phi}{\varphi}

% newcnf
\newcommand{\possign}{\mathtt{t}}
\newcommand{\negsign}{\mathtt{f}}
\newcommand{\genlit}[2]{{#1}^{#2}}
\newcommand{\genclause}[2]{{#1}_{#2}}

\newcommand{\sign}{\star}
\newcommand{\subst}{\theta}
\newcommand{\emptySubst}{\epsilon}

\newcommand{\formName}{\mathtt{form}}
\newcommand{\form}[1]{\formName({#1})}

\newcommand{\config}[2]{#1}

\newcommand{\GC}{\mathit{C}}

\newcommand{\sk}{\mathit{sk}}

% TPTP true, false and bool
\newcommand{\tptpo}{\lstinline'$o'} %$
\newcommand{\dbool}{\lstinline'$bool'} %$
\newcommand{\dtrue}{\lstinline'$true'} %$
\newcommand{\dfalse}{\lstinline'$false'} %$
\newcommand{\ddtrue}{\lstinline'$$true'}
\newcommand{\ddfalse}{\lstinline'$$false'}

% TPTP ite and let
\newcommand{\dite}{\lstinline'$ite'} %$
\newcommand{\ditet}{\lstinline'$ite_t'} %$
\newcommand{\ditef}{\lstinline'$ite_f'} %$

\newcommand{\dlet}{\lstinline'$let'} %$
\newcommand{\dlettt}{\lstinline'$let_tt'} %$
\newcommand{\dlettf}{\lstinline'$let_tf'} %$
\newcommand{\dletft}{\lstinline'$let_ft'} %$
\newcommand{\dletff}{\lstinline'$let_ff'} %$

\newcommand{\dint}{\lstinline'$int'} %$
\newcommand{\dgreatereq}{\lstinline'$greatereq'} %$
\newcommand{\dsum}{\lstinline'$sum'} %$

\newcommand{\di}{\lstinline'$i'} %$

\newcommand{\arrayt}{\mathit{array}}
%\newcommand{\select}[2]{{#1}[{#2}]}
%\newcommand{\selectf}{\select{\cdot\,}{\,\cdot\,}}
\newcommand{\select}[2]{\mathit{select}({#1},\allowbreak{#2})}
\newcommand{\selectf}{\mathit{select}}
%\newcommand{\store}[3]{{#1}\langle{#2}\lhd{#3}\rangle}
%\newcommand{\storef}{\store{\cdot\,}{\,\cdot}{\cdot\,}}
\newcommand{\store}[3]{\mathit{store}({#1},\allowbreak{#2},\allowbreak{#3})}
\newcommand{\storef}{\mathit{store}}

\newcommand{\darray}[2]{\darraySymb\lstinline'('{#1}\lstinline','{#2}\lstinline')'}
\newcommand{\darraySymb}{\lstinline'$array'} %$
\newcommand{\dselect}{\lstinline'$select'} %$
\newcommand{\dstore}{\lstinline'$store'} %$
\newcommand{\darrayone}{\lstinline'$array1'} %$
\newcommand{\dselectone}{\lstinline'$select1'} %$
\newcommand{\dstoreone}{\lstinline'$store1'} %$
\newcommand{\darraytwo}{\lstinline'$array2'} %$
\newcommand{\dselecttwo}{\lstinline'$select2'} %$
\newcommand{\dstoretwo}{\lstinline'$store2'} %$

\newcommand{\skolem}{skolem}

% the rest
\newcommand{\Z}{\mathds{Z}}
\newcommand{\R}{\mathds{R}}

% tuples
\newcommand{\tuple}[1]{({#1})}

\newcommand{\newcnf}{\textsc{VCNF}}
\newcommand{\nfcnf}{$\text{\textsc{VCNF}}_{\text{\textsc{FOOL}}}$}
\newcommand{\oldcnf}{\textsc{FOOL2FOL}}

\newcommand{\integer}{\mathit{int}}
\newcommand{\emptyStatement}{\mathtt{skip}}
\newcommand{\assigns}[2]{{#1}\coloneqq{#2}}
\newcommand{\seq}[2]{{#1}\,;{#2}}
\newcommand{\translateT}{\mathcal{T}}
\newcommand{\translate}[1]{\translateT({#1})}
\newcommand{\tuplifyT}{\mathcal{N}}
\newcommand{\tuplifyRel}[1]{\tuplifyT({#1})}
\newcommand{\tuplify}[2]{\tuplifyRel{{#1}}({#2})}
\newcommand{\updates}[1]{\mathrm{updates}({#1})}

\newcommand{\ifthen}[2]{\mathtt{if}\;{#1}\;\allowbreak\mathtt{then}\;{#2}}

\newcommand{\while}[2]{\mathtt{while}\;{#1}\;\mathtt{do}\;{#2}}

\newcommand{\ttrue}{\mathtt{true}}
\newcommand{\tfalse}{\mathtt{false}}

\newcommand{\letinnl}[3]{\begin{aligned}[t]&\mathtt{let}\;\binding{#1}{#2}\;\mathtt{in}\\[-0.2em]&\quad{#3}\end{aligned}}
\newcommand{\letnl}[3]{\begin{aligned}[t]\mathtt{let}\;&\binding{#1}{#2}\\[-0.2em]\mathtt{in}\;&{#3}\end{aligned}}

\newcommand{\letinparnl}[5]{\begin{aligned}[t]\mathtt{let}\;&\binding{#1}{#2};\;\binding{#3}{#4}\\\mathtt{in}\;&{#5}\end{aligned}}

\newcommand{\itenll}[3]{\begin{aligned}[t]&\mathtt{if}\;{#1}\\[-0.2em]&\mathtt{then}\;{#2}\\[-0.2em]&\mathtt{else}\;{#3}\end{aligned}}

\newcommand{\itenl}[3]{\begin{aligned}[t]\mathtt{if}\;{#1}\;&\mathtt{then}\;{#2}\\&\mathtt{else}\;{#3}\end{aligned}}


\newcommand{\interp}[1]{\left\llbracket\,{#1}\,\right\rrbracket}

\newcommand{\expr}{\mathit{e}}
\newcommand{\stmt}{\mathit{s}}
\newcommand{\State}{\mathit{st}}

\newcommand{\hoare}[3]{\{{#1}\}\,{#2}\,\{{#3}\}}

\newcommand{\pint}{\mathtt{int}}
\newcommand{\pbool}{\mathtt{bool}}
\newcommand{\parray}{\mathtt{array}}

\thesis{Thesis for the Degree of Licentiate of Engineering}
\title{Automated Theorem Proving\\in a First-Order Logic with\\First Class Boolean Sort}
\infopagetitle{Automated Theorem Proving in a First-Order Logic\\with First Class Boolean Sort}
\author{Evgenii Kotelnikov}
\department{Department of Computer Science and Engineering}
\institution{Chalmers University of Technology and University of Gothenburg}
\infopageinstitution{Chalmers University of Technology and\\[-1mm]University of Gothenburg}
\reportnumber{152L}
\issn{1652-876X}


\newcommand{\EK}[1]{{\color{red}  EK: {#1}}}

\begin{document}

\frontmatter

\maketitle

\chapter*{Abstract}
Automated theorem provers are computer programs that check if a logical conjecture follows from a set of logical statements. The conjecture and the statements are expressed in the language of a formal logic, such as first-order logic. Expressivity of first-order logic makes it convenient for writing problems coming from diverse application domains. As a result, theorem provers for first-order logic have been used for automation in proof assistants, verification of programs, static analysis of networks, and other purposes. Despite the success of theorem provers for first-order logic, their efficient use remains challenging. One of the challenges is the difficulty of translating the domain problem to first-order logic. Not only can a translation be cumbersome due to semantic differences between the domain and the logic, but it might inadvertently result in a problem that is hard for a theorem prover.

The work presented in the thesis addresses this challenge by developing an extension of first-order logic named FOOL. FOOL is friendly for translation of problems from various domains and can be efficiently supported by existing theorem provers. We describe the syntax and symantics of FOOL and present a naive translation from FOOL to plain first-order logic. We describe a more efficient clausal normal form transformation algorithm for FOOL and based on it implement a support for FOOL in the Vampire theorem prover. We illustrate the efficient use of FOOL for program verification by describing a concise encoding of next state relations of imperative programs in FOOL. We demonstrate the efficiency of automated theorem proving in FOOL with an extensive set of experiments. In these experiments we compare the performance of Vampire on a large collection of problems from various sources translated to FOOL and ordinary first-order logic. Finally, we fix the syntax for FOOL in TPTP, the standard language of first-order theorem provers.

\tableofcontents

% \chapter*{Acknowledgements}
% The first two authors were partially supported by the Wallenberg Academy Fellowship 2014, the Swedish VR grant D0497701, and the Austrian research project FWF S11409-N23. The third author was Partially supported by the EPSRC grant ``Reasoning in Verification and Security''.

\mainmatter

\chapter{Introduction}
\label{chap:intro}
% !TEX root = main.tex

\EK{

Computer mathematics studies processing of mathematical knowledge with a computer. It explores questions of how to represent mathematical problems and their proofs in a computer, how to check correctness of proofs by a computer and even how to construct proofs automatically using a computer. The latter is the domain of automated theorem proving. It is one of the central and hardest areas of computer mathematics and artificial intelligence. Automated methods of proving theorems precede the existence of computers (see e.g. \cite{bundy1999survey,davis2001early,harrison2007short} for a historical survey).

In order to be represented in a computer, a mathematical problem must be expressed in a language of some formal logic. Among the logics used for this purpose are propositional, first-order and higher-order logic, intuitionistic logic, modal, temporal, many-values logic and others.

Algorithms of automated theorem proving are implemented in computer programs called theorem provers. A theorem prover takes a logical conjecture as input and tries to either construct its proof or demonstrate that the conjecture is invalid. Theorem provers can be classified by the logic they support. Propositional, first-order and higher-order logic are among the logics that received the most attention in automated theorem proving. Reasoning in propositional logic, i.e. solving the problem of propositional satisfiability (SAT), is implemented in \emph{SAT solvers}, such as Lingeling~\cite{Lingeling} and Minisat~\cite{Minisat}. Solving the problem of satisfiability modulo theory (SMT) is implemented in \emph{SMT solvers}, such as Z3~\cite{Z3} and CVC4~\cite{CVC4}. Reasoning is first-order logic is implemented in \emph{first-order theorem provers}, such as Vampire~\cite{Vampire13}, E~\cite{E13} and iProver~\cite{iProver}. Reasoning in higher-order logic is implemented in \emph{higher-order theorem provers}, such as Satallax~\cite{Satallax} and Leo-II~\cite{LeoII}.

Generally speaking, the more expressive a logic is, the harder it is to reason in it. For example, satisfiability of a propositional problem can always be established, albeit possibly at a high computational cost. Modern SAT solvers implement elaborate algorithms that guarantee good performance characteristics in the average case. In contrast, satisfiability or unsatisfiability of a first-order problem cannot be in general established by any algorithm. Implementors of first-order provers face the challenge of making the provers succeed on as many real world problems as possible.

Theorem provers are used for software and hardware verification, information management, combinatorial reasoning, and more. They are also the most powerful mean of proof automation in interactive proof assistants. In most applications, the theorem checked by a theorem prover is generated by an external software tool and not given by a human. %Many verification tools integrate theorem provers and supply them with machine-generated theorems.

This thesis contributes to the area of automated theorem proving by presenting an extension of first-order logic that is useful for applications and can be supported by first-order theorem provers. This chapter describes the background of the thesis and is structured as follows. Section~\ref{sect:intro:fol} gives an introduction to automated theorem proving in first-order logic. Sections \ref{sect:intro:analysis} and \ref{sect:intro:itp} describe two important applications of first-order provers, automation for program verification and interactive proof assistants. Section~\ref{sect:intro:problem} states the problem addressed in the thesis, and finally Section~\ref{sect:intro:contributions} summarises the contributions of the thesis.
}

\EK{The thesis describes implementation details and challenges in the Vampire theorem prover, however the described features and their implementation can be carried out in any other first-order prover.}

\section*{Automated Theorem Proving}
\addcontentsline{toc}{section}{Automated Theorem Proving}

\EK{
\paragraph{Undecidability of first-order logic}

The problem of establishing validity of a first-order problem automatically can be traced back to Hilbert. In 1928 he posed a question, traditionally referred to as \emph{Entscheidungsproblem}\iffalse(German for ``decision problem'')\fi, stated as follows. Is there an algorithm that takes as input a statement in first-order logic and terminates with ``Yes'' or ``No'' according to whether or not the statement is valid?

This question has been answered negatively in 1936 independently by Church~\cite{church1936unsolvable} and Turing~\cite{turing1936computable}. Their proofs rely on the famous G\"{o}del's incompleteness theorem\iffalse~\cite{godel1931formal} (for an English translation see e.g. \cite[pp. 4--38]{davis1965undecidable})\fi. G\"{o}del's result entails that the problem of provability in first-order logic is not decidable, but \emph{semi-decidable}. Informally, it means that if a logical sentence of the form ``$\varphi$ implies $\psi$'' is valid, that can be established by enumerating all finite derivations in the logical system. A derivation between $\varphi$ and $\psi$ will necessarily be found in that enumeration. On the other hand, if the sentence is invalid, there is no algorithm that could in general demonstrate that.
}

\EK{
The main aim of automated first-order theorem proving is to determine satisfiability of a set of formulas, and possibly provide a certificate of the result, either in the form of a derivation of a false formula, or some representation of a model satisfying the input formulas.
}

\EK{A well known and generally best performing family of first-order proving algorithms are those based on various extensions of the calculus of ordered resolution and superposition described in \cite{NieuwenhuisRubio:HandbookAR:paramodulation:2001}.
}

\EK{
\paragraph{Refutation}

The problem of validity of a problem in first-order logic is often formulated in terms of unsatisfiability. Validity of a formula is equivalent to unsatisfiability of its negation. To prove validity of a formula one can derive contradiction from its negation, thus constructing a proof by \emph{refutation}. Conversely, invalidity of a formula is equivalent to satisfiability of its negation. To demonstrate invalidity of a formula one can find a model of its negation. Algorithms that search for satisfiability and unsatisfiability of first-order formulas are usually implemented as separate procedures.

}

\EK{
\paragraph{Clausification}
This involves the replacement of existentially quantified variables by Skolem functions, the expansion of equivalences, rewriting to negation normal form and then application of associativity rules to reach conjunctive normal form. It is well know that this process can lead to an explosion in the number of clauses.

Methods of first-order reasoning usually work not with arbitrary first-order formulas, but with first-order clauses. A first-order formula is in clausal normal form (CNF) if it has the shape $\forall x_1\ldots\forall x_n(C_1\wedge\ldots\wedge C_n)$, where each of $C_1,\ldots,C_n$ is a disjunction of literals. An alternative representation of a CNF is a set of first-order clauses. A first-order clause is an implicitly universally quantified disjunction of positive and negative first-order literals. A CNF translation converts an arbitrary first-order formula to CNF, preserving satisfiability. First-order provers that support formulas in full first-order logic implement such translations as part of their preprocessing of the input.

}

\EK{
\paragraph{Saturation-based proof search, AVATAR}
After a clause set has been produced, Vampire attempts to saturate this set with respect to some inference system I. The clause set is saturated if for every inference from I with premises in S the conclusion of the inference is also added to S. If the saturated set S contains a contradiction then the initial formulas are unsatisfiable. Otherwise, if I is a complete inference system and, importantly, the requirements for this completeness have been preserved, then the initial formulas are satisfiable. Finite saturation may not be possible and many heuristics are employed to make finding a contradiction more likely.
To compute this saturation we use a set of active clauses, with the invariant that all infer- ences between active clauses have been performed, and a set of passive clauses waiting to be activated. The algorithm then iteratively selects a given clause from passive and performs all necessary inferences to add it to active. The results of these inferences are added to passive (after undergoing some processing). This is illustrated in Figure 1. An important aspect of this process is clause selection. Clauses are selected either based on their age (youngest first) or their weight (lightest first) with these two properties being alternated in some specified ratio.
A recent addition to this story is AVATAR [50, 40], which (optionally) performs clause splitting using a SAT solver. The main point here is that the success of AVATAR is driven by the observation that saturation-based proof search does not perform well with long or heavy clauses. Therefore, encodings should avoid the introduction of such clauses. As an additional point, AVATAR can only be utilised if the boolean structure of a problem is exposed at the literal-level. For example, including a predicate implies with associated axioms would not play to AVATAR's strengths.

Vampire's main algorithm is saturation of a set of first-order clauses using the resolution and superposition calculus.
}

\EK{
\paragraph{Resolution and Superposition}

Despite the complexity of automated reasoning in first-order logic, several methods were found to be efficient for finding unsatisfiability for non-trivial problems. Modern state-or-the-art automated theorem provers are based on superposition calculus~\cite{NieuwenhuisRubio:HandbookAR:paramodulation:2001} and its refinements. Finding satisfiability is a much harder problem because a formula might only have infinite models.\iffalse In practice, the search for satisfiability is limited to finite models.\fi

Superposition-based theorem proving stems from the work of Robinson~\cite{Robinson65} on \emph{resolution calculus}. Resolution calculus establishes unsatisfiability of a set of first-order clauses by systematically and exhaustively applying a set of inference rules which include the resolution inference rule. Resolution calculus is refutationally complete, meaning that a contradiction can be deduced from any unsatisfiable set of clauses. The novelty of Robinson's work was in the usage of \emph{unification} for instantiation of variables. Unification avoids combinatorial explosion of ground instances of quantified formulas that was present e.g. in an earlier algorithm of Davis and Putnam \cite{davis1960computing}\footnote{This algorithm however retained as the prevalent method for establishing propositional satisfiability after a refinement by Logemann and Loveland~\cite{davis1962machine}.}.

Resolution calculus is refined by \emph{superposition calculus}~\cite{BG90, BG94} that employs term orderings for restricting the number of inferences. The basic idea of superposition is to only allow inferences that replace ``big'' terms by ``smaller'' ones, with respect to the given ordering.

Vampire uses resolution and superposition as its inference system I [1, 34]. A key feature of this calculus is the use of literal selection and orderings to restrict the application of inference rules, thus restricting the growth of the clause sets. Vampire uses a Knuth-Bendix term ordering (KBO) [23, 25, 32] which orders terms first by weight and then by symbol precedence whilst agreeing with a multisubset ordering on free variables. The symbol ordering is taken as a parameter but is relatively coarse in Vampire e.g. by order of occurrence in the input, arity, frequency or the reverse of these. There has been some work beginning to explore more clever things to do here [22, 38] but we have not considered treating symbols introduced by translations differently (although they will appear last in occurrence).

Saturation-based theorem provers such as E~\cite{E13}, Spass~\cite{Spass} or Vampire~\cite{Vampire13} construct proofs of unsatisfiability of first-order problems. To that end, they first convert the input problem into a set of first-order clauses and then try to derive contradiction from it. Theorem provers saturate the search space by inferring new clauses with inference rules such as binary resolution~\cite{Ganzinger01} and superposition~\cite{NieuwenhuisRubio:HandbookAR:paramodulation:2001}. They employ multiple techniques to prune the search space such as simplification orderings, selection functions and redundancy elimination.
}

\EK{
\paragraph{Term Ordering}
An important ingredient in paramodulation is the use of \emph{term orderings} for restricting the number of inferences. The basic idea of ordered paramodulation is to only perform replacements of ``big'' terms by ``smaller'' ones, with respect to the given ordering. The first instances of ordered paramodulation appeared in Knuth-Bendix completion~\cite{KB83}. Roughly, a completion procedure attempts to transform a given set of equations into an equivalent confluent one. A crucial step of the transformation process is the computation of critical pairs between equations. A critical pair is an equation obtained by \emph{superposition}, the restricted version of paramodulation in which inferences only involve left hand sides of possible rewrite steps, i.e., only the ``big'' terms (w.r.t. the given ordering) are considered. During the completion process equations are simplified by rewriting, and tautologies are removed. \EK{I don't quite undestand this, so it reads a bit vague.} Bachmair and Ganzinger refined these ideas into what got to be known as \emph{superposition calculus}~\cite{BG90, BG94}, that now forms the basis for most first-order theorem provers.
}

\EK{
\paragraph{Redundancy} Another very important concept related to saturation is the notion of redundancy. The idea is that some clauses in S are redundant in the sense that they can be safely removed from S without compromising completeness. The notion of saturation then becomes saturation-up-to- redundancy [1, 34]. An important redundancy check is subsumption. A clause A subsumes B if some subclause of B is an instance of A, in which case B can be safely removed from the search space as doing so does not change the possible models of the search space S. The fact that Vampire removes redundant formulas is good but if there is a lot of redundancy in the encoding we can still have issues as this removal can be lazy (e.g. when using the discount saturation loop that does not remove redundancies from the passive set).
}

\EK{
\paragraph{AVATAR}
Vampire also implements the AVATAR architecture~\cite{DBLP:conf/cav/Voronkov14} for splitting clauses. The idea behind AVATAR is to use a SAT or an SMT solver to guide proof search. AVATAR selects sub-problems for the saturation-based prover to tackle by making decisions over a propositional abstraction of the clause search space. The \tt{-sas} option of Vampire selects the SAT solver.
}

%\paragraph{Portfolio of proof search strategies}
Proof search strategies for first-order logic can be configured in many ways, and different proof strategies might solve different problems. Theorem provers such as E, iProver and Vampire implement not just one proof search strategy, but a portfolio of them. Based on certain characteristics of the input, theorem provers select the appropriate proof search strategies and schedules for them, and then run these strategies one by one in a time-slicing fashion. This approach allows theorem provers to succeed on a larger number of problems. Some provers extend their portfolios with proof search techniques other than saturation. For example, Vampire includes in its portfolios an implementation of the Inst-Gen calculus~\cite{DBLP:conf/birthday/Korovin13} and a finite model builder~\cite{VampireFMB}.

First-order theorem provers are currently evaluated empirically. Comparison of provers is mostly based on success rates and run times on standard corpora of problems. The main corpus is the Thousands of Problems for Theorem Provers (TPTP) library~\cite{TPTP}. The problems in this corpus are written in a variety of languages, such as FOF for untyped first-order formulas, TFF0~\cite{tff0} for typed monomorphic first-order formulas and TFF1~\cite{tff1} for typed rank-1 polymorphic first-order formulas. The TPTP library is used as a basis for the annual CASC system competition~\cite{CASC}.

\section*{Applications}
\addcontentsline{toc}{section}{Applications}
\EK{TODO}

\subsection*{Deductive Program Verification}
\addcontentsline{toc}{subsection}{Deductive Program Verification}
\EK{
Methods of program verification check that a program satisfies its specification. A program specification can be expressed with logical formulas that annotate program statements, capturing their properties. Typical examples of such properties are pre- and post-conditions, loop invariants and Craig interpolants. These program properties are checked using various tools, including theorem provers (see e.g. \cite{Bonacina10} for a detailed overview).

Automated program verification sees compliance with specification as a theorem that can be automatically checked by theorem provers. For that, program statements are first translated to logical formulas that capture the semantics of the statements. Then, a theorem is built with the translated formulas as axioms and program properties as the conjecture. Validity of the theorem is interpreted as that the program statements have their annotated properties.

Theorem provers can be used not just for checking program properties, but also for generating them. Recent approaches in interpolation and loop invariant generation~\cite{McMillan08,fase2009,hoder2012popl} present initial results of using first-order theorem provers for generating quantified program properties. First-order theorem provers can also be used to generate program properties with quantifier alternations~\cite{fase2009}; such properties could not be generated fully automatically by any previously known method.
}

\subsection*{Automation for Proof Assistants}
\addcontentsline{toc}{subsection}{Automation for Proof Assistants}
\EK{
Proof assistants are software tools that assist users in constructing proofs of mathematical problems. Proof assistants use formalisations of mathematics based on higher-order logic (Isabelle/HOL~\cite{Isabelle}), type theory (Coq~\cite{Coq}), set theory (Mizar~\cite{Mizar}) and others.

Many proof assistants enhance the workflow of their users by automatically filling in parts of the user's proof with the help of tactics. Tactics are specialised scripts that run a predefined collection of proof searching strategies. These strategies can be implemented inside the proof assistant itself or rely on third-party automated theorem provers~\cite{Sledgehammer,DBLP:conf/icms/UrbanHV10}.

To automate proof search for a problem using a theorem prover, a proof assistant first translates the problem into the logic supported by the theorem prover. Since the logics of proof assistants are usually more expressive than the logics of automated provers, this translation can be incomplete. If the theorem prover reports back a proof, the proof assistant uses it to reconstruct a proof in its own logic.
}

\section*{Motivation}
\addcontentsline{toc}{section}{Motivation}
First-order theorem provers have proved themselfes effective in several practical applications, however, their efficient usage remains challenging. One of the challenges is representation of application problems in first-order logic in a way that is efficient for automated reasoning. Systems of deductive program verification and proof assistants that rely on first-order provers usually do not deal with first-order logic natively. Instead, they translate problems in their respective domains (program properties or formulas in the logic of the proof assistant) to problems in first-order logic. There could be multiple ways of translating a problem because of the mismatch between the semantics of the domain and that of first-order logic. A theorem prover might succeed on the results of some of these translations and fail on the others. Users of a theorem prover might find designing a translation that is friendly to the prover to be a difficult task.

Theorem provers, especially first-order ones, are known to be fragile with respect to the input. Multiple, often subtle, characteristics of a first-order problem might affect the performance of the saturation-based proof search. These characteristics include, for example, the number of clauses in the problem, the size of clauses and the size of the signature. The combination of the problem and the particular setup of the proof search, such as the used selection function and term ordering, might be crucial for the success of the prover. Encoding a problem in the ``right'' format might therefore require solid knowledge of how theorem provers work and are implemented~--- something that a user of a prover might not have.

Assessing whether a translation of a certain problem to first-order logic is good might be difficult as well. Such assessment can often only be made through tedious experiments with running theorem provers, configured with different settings, on the results of the translation. A perfect translation might not necessarily exist, because different translation might work better in different scenarios. Furthermore, for some types of problems, their translations to first-order logic cannot be efficiently handled by a theorem prover at all unless it is extended with specialized rules and heuristics.

The complexity of preparing problems for first-order theorem provers can be battled by extending the logic, supported by the provers. Such extension should include theories and new syntactical features that are common in applications but sensitive to translations. The appropriate translation of these features to plain first-order logic therefore becomes the responsibility of the provers themselves. The right choice of new features and their efficient implementation in theorem provers facilitates applications of automated theorem proving. Firstly, users of theorem provers are relieved from the tedious translations and can express their problems closer to their original domain. Secondly, theorem provers are able to implement a translation that suits them best. Thirdly, theorem provers that implement portfolios of proof search strategies are able to try multiple different translations in the same run of the prover. Finally, theorem provers are able to enhance proof search for problems with specific features by implementing dedicated inference rules and other techniques.

This thesis addresses the following research question: \emph{which extensions of first-order theorem provers are useful for applications and how can these extensions be efficiently implemented?} The thesis identifies the following syntactical constructs that are generally not supported by first-order provers: first class boolean sort, \ITE\ expressions and \LETIN\ expressions. These constructs are ubiquitous in problems coming from program verification and interactive theorem provers, but all of them currently require specialised translations. Boolean values in programming languages are used both as expressions in conditional or loop statements and as boolean flags passed as arguments to functions. A natural way of translating program statements with booleans into formulas is by translating conditions as formulas and function arguments as terms. Yet we cannot mix boolean terms and formulas in the same way in first-order logic, unless the boolean sort is first class. Properties expressed in higher-order logic routinely use quantification over the interpreted boolean sort; this is not allowed in first-order logic either. Both programming languages and logic of proof assistants actively use \ITE\ and \LETIN\ statements that require non-trivial dedicated translation in first-order logic.

%\EK{The problem addressed in this thesis is the extension of the input language and underlying logic of first-order theorem provers with these constructs.}

\section*{Contributions of the Thesis}
\label{sect:intro:contributions}
\addcontentsline{toc}{section}{Contributions of the Thesis}

%The thesis focuses on practical features extending first-order theorem provers for making them better suited for applications of program verification and proof automation for interactive theorem provers.

This thesis contributes to the area of automated reasoning by exploring which extensions of first-order theorem provers facilitate their practical applications. Firstly, the thesis presents an extension FOOL of first-order logic that contains the missing syntactical constructs mentioned before. Secondly, it explores how reasoning in FOOL can be implemented in existing automated theorem provers for first-order logic. Finally, it gives evidence of usefulness of FOOL and efficiency of reasoning with it for practical applications. The rest of this section summarizes the main contributions of the thesis.

\paragraph{FOOL}
The thesis presents FOOL, standing for first-order logic (FOL) with boolean sort. \folb{} extends ordinary many-sorted FOL with \begin{enumerate*}[label=(\roman*)]\item first class boolean sort, \item boolean variables used as formulas, \item formulas used as arguments to function and predicate symbols, \item \ITE\ expressions and \item \LETIN\ expressions.\end{enumerate*} \ITE\ and \LETIN\ expressions can occur as both terms and formulas. \LETIN\ expressions can use (multiple simultaneous) definitions of function symbols, predicate symbols, and tuples. The thesis presents the definition of FOOL, its semantics, and a simple model-preserving translation from \folb{} formulas to formulas of first-order logic. This translation can be used to support \folb{} in existing first-order provers.

\paragraph{Reasoning with FOOL}
The thesis presents two approaches to an implementation of FOOL in first-order provers that improve over the simple translation of FOOL to FOL. The first approach is a new technique of dealing with the boolean sort in superposition theorem provers. This teachnique includes replacement of one of the boolean sort axioms with a specialised inference rule, called FOOL paramodulation. The second approach is a new algorithm \nfcnf{} that transforms FOOL formulas directly to first-order clauses. The thesis presents an implementation of the simple translation from FOOL to FOL and both improved approaches in Vampire.

\paragraph{Applications of FOOL}
The thesis presents an encoding of the next state relations of imperative programs in FOOL. Compared to similar methods, this encoding avoids introducing intermediate variables and results in FOOL formulas that consicely represent program fragments in logic.
%The thesis presents a translation of imperative programs annotated with their pre- and post-conditions to partial correctness properties of these programs.
The thesis present a work on verification of virtual private cloud network configurations with Vampire. The encoding of verification problems in this work relies on first class booleans, the theory of arrays and the theory of tuples.

\paragraph{Practical Evaluation}
The thesis presents extensive experimens on running Vampire, other first-order theorem provers, higher-order theorem provers and SMT solvers on FOL and FOOL problems. These problems come from various sources: benchmarks from the TPTP and SMT-LIB library, proof obligations generated by the Isabelle proof assistant, and verification conditions generated by multiple different program verification tools. The experimental results obtained with these problems show in particular that \begin{enumerate}
  \item Vampire with FOOL paramodulation performs better than Vampire with the simple translation from FOOL to FOL;
  \item Vampire with \nfcnf{} performs better that Vampire with FOOL paramodulation;
  \item Vampire performs better on verification conditions translated to FOOL than translated to FOL using current state-of-the-art methods.
\end{enumerate}

\paragraph{Impact on TPTP}
The language of FOOL is a superset of TFF0~--- the monomorphic first-order part of the TPTP language. The thesis describes a modification of the TPTP language needed to represent \folb{} formulas. This modification has been included in the TPTP standard as the TPTP Extended Typed First-Order Form (TFX).

\paragraph{Impact on Vampire}
The language of FOOL is a superset of the core theory of the SMT-LIB language~\cite{SMT-LIB}, the standard language of SMT solvers. First-order provers that support \folb{} can therefore reason about some problems from the SMT-LIB library. This opens up an opportunity to evaluate first-order provers on problems that were previously only checked by SMT solvers. Vampire gained support for SMT-LIB based on its implementation of FOOL, and since 2016 has been participating in the SMT-COMP competition~\cite{DBLP:conf/cav/BarrettMS05} where it contends against SMT solvers.

\section*{Structure of the Thesis}
\label{sect:intro:overview}
\addcontentsline{toc}{section}{Structure of the Thesis}

The work described in this thesis has been carried out in six papers, each contained in a separate chapter. Four papers (Chapters~\ref{chap:fool}, \ref{chap:implementation}, \ref{chap:cnf} and \ref{chap:boogie}) were published in peer-reviewed conferences, one (Chapter~\ref{chap:tfx}) was published in a peer-reviewed workshop, and one (Chapter~\ref{chap:aws}) is a technical report not yet submitted for publication. The references of the papers have been combined into a single bibliography at the end of the thesis. Other than that, the papers have only been edited for formatting purposes, and in general appear in their original form.

The chapters of this thesis are placed in the order in which their correspondent papers were written. Chapter~\ref{chap:fool} presents the syntax and semantics of FOOL. Chapter~\ref{chap:implementation} presents the implementation of FOOL in Vampire. Chapter~\ref{chap:cnf} presents an efficient clausification algorithm for FOOL. Chapter~\ref{chap:boogie} describes an encoding of the next state relations of imperative programs in FOOL. Chapter~\ref{chap:aws} describes an approach to network verification based on automated reasoning in first-order logic, which uses features of FOOL. Finally, Chapter~\ref{chap:tfx} describes TFX, the extension of the TPTP language that contains the syntax for FOOL.

Each of the papers contained in this thesis has been written and presented separately. As a result, the introductory remarks and preliminaries of some of the chapters necessarily overlap. Another consequence is that some ideas presented in earlier chapters are revisited and developed in later chapters. One example of such idea is the encoding of the next state relations of imperative programs in FOOL. A sketch of this encoding first appears in Chapter~\ref{chap:implementation} and preliminary experimental results are discussed in Chapter~\ref{chap:cnf}. The precise formal description of the encoding and extensive evaluation is however given later in Chapter~\ref{chap:boogie}. Another example is the set of syntactical constructs available in FOOL. The original description of FOOL in Chapter~\ref{chap:fool} does not include \LETIN\ expressions with simultaneous definitions, definitions of tuples and tuple expressions. These constructs are incuded in later chapters.

The contributions of the thesis are the cumulative contributions of all six papers. The rest of this chapter details the main contributions of each individual paper.

\subsection*{\hyperref[chap:fool]{Chapter 1.} A First Class Boolean Sort in\\First-Order Theorem Proving and TPTP}
The paper presents the syntax and semantics of \folb. We show that \folb\ is a modification of FOL and reasoning in it reduces to reasoning in FOL. We give a model-preserving \iffalse(modulo introduced definitions)\fi translation of \folb\ to FOL that can be used for proving theorems in \folb\ in a first-order prover. We discuss a modification of superposition calculus that can reason efficiently in the presence of boolean sort. This modification includes replacement of one of the boolean sort axioms with a specialised inference rule that we called \folb\ paramodulation. We note that the TPTP language can be changed to support \folb, which will also simplify some parts of the TPTP syntax. 

\paragraph{Statement of contribution.} The paper is co-authored with Laura Kov\'{a}cs and Andrei Voronkov. Evgenii Kotelnikov contributed to the formalisation of \folb{} and its translation to FOL.

\paragraph{Bibliographic information.} The paper has been published in the proceedings of the 8th Conference on Intelligent Computer Mathematics (CICM) in 2015~\cite{FOOL}.

\subsection*{\hyperref[chap:implementation]{Chapter 2.} The Vampire and the \folb{}}
The paper describes the implementation of \folb\ in Vampire. We extend and simplify the TPTP language by providing more powerful and uniform representations of \ITE\ and \LETIN\ expressions. We demonstrate usability and high performance of our implementation on two collections of benchmarks, coming from the higher-order part of the TPTP library and from the Isabelle interactive theorem prover. We compare the results of running Vampire on the benchmarks with those of SMT solvers and higher-order provers. Moreover, we compare the performance of Vampire with and without \folb{} paramodulation. We give a simple extension of \folb, allowing to express the next state relation of a program as a boolean formula which is linear in the size of the program.

\paragraph{Statement of contribution.} The paper is co-authored with Laura Kov\'{a}cs, Giles Reger and Andrei Voronkov. Evgenii Kotelnikov contributed with the implementation of \folb{} in Vampire and the experiments.

\paragraph{Bibliographic information.} The paper has been published in the proceedings of the 5th ACM SIGPLAN Conference on Certified Programs and Proofs (CPP) in 2016~\cite{VampireAndFOOL}.

\subsection*{\hyperref[chap:cnf]{Chapter 3.} A Clausal Normal Form Translation\\for \folb{}}
The paper presents a clausification algorithm that translates a FOOL formula to an equisatisfiable set of first-order clauses. This algorithm aims to minimise the number of clauses and the size of the resulting signature, especially on formulas with \ITE, \LETIN\ expressions and complex boolean structure. We demonstrate by experiments that the implementation of this algorithm in Vampire increases performance of the prover on \folb{} problems compared to the earlier translation of \folb{} formulas to full first-order logic.

\paragraph{Statement of contribution.} The paper is co-authored with Laura Kov\'{a}cs, Martin Suda and Andrei Voronkov. Evgenii Kotelnikov contributed with the extension of \newcnf{} that supports \folb{}, the implementation of this extension in Vampire and the experiments.

\paragraph{Bibliographic information.} The paper has been published in the proceedings of the 2nd Global Conference on Artificial Intelligence (GCAI) in 2016~\cite{FOOLCNF}.

\subsection*{\hyperref[chap:boogie]{Chapter 4.} A FOOLish Encoding of the Next State Relations of Imperative Programs}
The paper describes an encoding of the next state relations of imperative programs with variable updates and \ITE\ statements in FOOL. Based on this encoding the paper presents a translation of imperative programs annotated with their pre- and post-conditions to partial correctness properties of these programs. We demonstrate by experiments that this translation results in formulas that are easier for Vampire than the formulas produced by program verification tool such Boogie and BLT.

\paragraph{Statement of contribution.} The paper is co-authored with Laura Kov\'{a}cs and Andrei Voronkov. Evgenii Kotelnikov contributed with the formalised translation of imperative programs to FOOL and the experiments.

\paragraph{Bibliographic information.} The paper has been published in the proceedings of the 9th International Joint Conference on Automated Reasoning (IJCAR) in 2018~\cite{KKV18}.

\subsection*{\hyperref[chap:aws]{Chapter 5.} Checking Network Reachability Properties by Automated Reasoning in First-Order Logic}
The paper describes an approach for static verification of virtual private cloud networks using automated theorem proving for first-order logic. We model networks with Horn clauses and check first-order properties of these models using the Vampire theorem prover. We used Vampire both as a saturation-based theorem prover and a finite model builder for different kinds of checked properties.

\paragraph{Statement of contribution.} The chapter is co-authored with Pavle Suboti\'{c} and based on a joint work with Byron~Cook, Temesghen Kahsai and Sean~McLaughlin. Evgenii Kotelnikov contributed with the encoding of network rechability properties in first-order logic and the implementation of a checker for these problems based on Vampire.

\subsection*{\hyperref[chap:tfx]{Chapter 6.} TFX: The TPTP Extended Typed First-Order Form}
The paper presents the new language TFX that extends and simplifies the language of typed first-order formulas TFF. TFX includes the first class boolean sort, \ITE\ expressions, \LETIN\ expressions and tuples. The inclusion of these syntactic constructs was motivated by the work on FOOL and FOOL formulas can be directly expressed in TFX. TFX has been included in the latest release of the TPTP library.

\paragraph{Statement of contribution.} The paper is co-authored with Geoff Sutcliffe. Evgenii Kotelnikov contributed with the discussion of the TFX syntax, the description of FOOL and examples of FOOL problems.

\paragraph{Bibliographic information.} The paper has been published in the proceedings of the 6th Workshop on Practical Aspects of Automated Reasoning (PAAR) in 2018~\cite{SutcliffeK18}.

\section*{Related Work}
\addcontentsline{toc}{section}{Related Work}

\EK{

The equality can be finitely axiomatised in first-order logic as a congruence relation. However, resolution with equality axioms is known to generate enormous search spaces and thus is very inefficient. Rather than axiomatizing equality, first-order provers consider it part of the logic and implement specialized inference rules for equality reasoning. These inference rules include refinements of the paramodulation rule~\cite{WRCS67,Robinson1969}.

Treatment of sorts in FOL. Many-sorted resolution \cite{DBLP:conf/ijcai/Walther83}.

Extensionality resolution. In a more recent result~\cite{ATVA14}, Vampire was extended to support the extensionality resolution rule to efficiently reason with the extensionality axiom.

Linear integer arithmetic --- incomplete axiomatization, evaluation inference rule. Many problems, tackled by theorem provers, are written in the combination of first-order logic with theories, such as the theory of linear integer arithmetic. Theorem provers handle theories by automatically adding (possibly incomplete) theory axioms to the search space whenever an interpreted sort, function, or predicate is found in the input.

Algebraic datatypes. \cite{BPR18}

Higher-order logic in Vampire.

\folb{} contains features of higher-order logic. Some problems that previously required higher-order logic can now be expressed directly in \folb{}. For example, the current version of the TPTP library contains over a hundred of such problems. One can check these problems with first-order provers that support \folb{} rather than higher-order provers.

Finally, it is interesting to note that our \nfcnf{} algorithm naturally translates a quantified boolean formula (QBF), as realised in the FOOL language, into a CNF in effectively propositional logic (EPR). Specifically, every literal in this translation is a skolem predicate applied to boolean variables and constants $\true$ and $\false$. Obtaining a formula in EPR is a desirable property to have since there are first-order proving methods known to be efficient for dealing with the fragment (see e.g.~\cite{DBLP:conf/birthday/Korovin13}).

This example makes one think about representing sentences in various epistemic or first-order modal logics in FOOL.

}



\def\paperOneContentsTitle{A First Class Boolean Sort in\\First-Order Theorem Proving and TPTP}
\def\paperOneChapterTitle{A First Class Boolean Sort in\\First-Order Theorem Proving\\and TPTP}
\def\paperOneAuthors{Evgenii~Kotelnikov, Laura~Kov\'acs and Andrei~Voronkov}
\def\paperOneAbstract{To support reasoning about properties of programs operating with boolean values one needs theorem provers to be able to natively deal with the boolean sort. This way, program properties can be translated to first-order logic and theorem provers can be used to prove program properties efficiently. However, in the TPTP language, the input language of automated first-order theorem provers, the use of the boolean sort is limited compared to other sorts, thus hindering the use of first-order theorem provers in program analysis and verification. In this paper, we present an extension \folb\ of many-sorted first-order logic, in which the boolean sort is treated as a first-class sort. Boolean terms are indistinguishable from formulas and can appear as arguments to functions. In addition, \folb\ contains \ITE\ and \LETIN\ constructs. We define the syntax and semantics of \folb\ and its model-preserving translation to first-order logic. We also introduce a new technique of dealing with boolean sorts in superposition-based theorem provers. Finally, we discuss how the TPTP language can be changed to support \folb.}
\def\paperOnePublication{Published in the \emph{Proceedings of the 8th Conference on Intelligent\\Computer Mathematics}, pages 71--86. Springer, 2015.}
\paperchapter{\paperOneContentsTitle}
             {\paperOneChapterTitle}
             {\paperOneAuthors}
             {\paperOneAbstract}
             {\paperOnePublication}
\label{chap:fool}
% !TEX root = ../main.tex
\section{Introduction}
\label{sec:cicm15/introduction}

% !TEX root = ../main.tex

Automated program analysis and verification requires
discovering and proving program properties. Typical examples of such properties are loop invariants or Craig interpolants. These properties usually are expressed in combined theories of various data structures, such as integers and arrays, and hence require reasoning with both theories and quantifiers. Recent approaches in interpolation and loop invariant generation~\cite{McMillan08,fase2009,hoder2012popl} present initial results of using first-order theorem provers for generating quantified program properties. First-order theorem provers can also be used to generate program properties with quantifier alternations~\cite{fase2009}; such properties could not be generated fully automatically by any previously known method.
Using first-order theorem prover to generate, and not only prove program properties, opens new directions in analysis and verification of real-life programs.

First-order theorem provers, such as iProver~\cite{iProver}, E~\cite{E13}, and Vampire~\cite{Vampire13}, lack however various features that are crucial for program analysis. For example, first-order theorem provers do not yet efficiently handle (combinations of) theories;
nevertheless, sound but incomplete theory axiomatisations can be used in a first-order prover even for theories having no finite axiomatisation. Another difficulty in modelling properties arising in program analysis using theorem provers is the gap between the semantics of expressions used in programming languages and expressiveness of the logic used by the theorem prover. A similar gap exists between the language used in presenting mathematics. For example, a standard way to capture assignment in program analysis is to use a \LETIN\ expression, which introduces a local binding of a variable, or a function for array assignments, to a value. There is no local binding expression in first-order logic, which means that any modelling of imperative programs using first-order theorem provers at the backend, should implement a translation of \LETIN\ expressions. Similarly, mathematicians commonly use local definitions within definitions and proofs. Some functional programming languages also contain expressions introducing local bindings. In all three cases, to facilitate the use of first-order provers, one needs a theorem prover implementing \LETIN\ constructs natively.

Efficiency of reasoning-based program analysis largely depends on how programs are translated into a collection of logical formulas capturing the program semantics. The boolean structure of a program property that can be efficiently treated by a theorem prover is however very sensitive to the architecture of the reasoning engine of the prover. Deriving and expressing program properties in the ``right'' format therefore requires solid knowledge about how theorem provers work and are implemented~--- something that a user of a verification tool might not have. Moreover, it can be hard to efficiently reason about certain classes of program properties, unless special inference rules and heuristics are added to the theorem prover, see e.g.~\cite{ATVA14} when it comes to prove properties of data collections with extensionality axioms.

In order to increase the expressiveness of program properties generated by reasoning-based program analysis, the language of logical formulas accepted by a theorem prover needs to be extended with constructs of programming languages. This way, a straightforward translation of programs into first-order logic can be achieved, thus relieving users from designing translations which can be efficiently treated by the theorem prover.
One example of such an extension is recently added to the TPTP language~\cite{TPTP} of first-order theorem provers, resembling \ITE\ and \LETIN\ expressions that are common in programming languages. Namely, special functions \lstinline'$ite_t' and \lstinline'$ite_f' can respectively be used to express a conditional statement on the level of logical terms and formulas, and \lstinline'$let_tt', \lstinline'$let_tf', \lstinline'$let_ff' and \lstinline'$let_ft' can be used to express local variable bindings for all four possible combinations of logical terms (\lstinline't') and formulas (\lstinline'f'). While satisfiability modulo theory (SMT) solvers, such as Z3~\cite{Z3} and CVC4~\cite{CVC4}, integrate \ITE\ and \LETIN\ expressions, in the first-order theorem proving community so far only Vampire supports such expressions.

To illustrate the advantage of using \ITE\ and \LETIN\ expressions in automated provers, let us consider the following example. We are interested in verifying the partial correctness of the code fragment below:
% \pagebreak
\begin{lstlisting}[language=cpp]
if (r(a)) {
  a := a + 1
} else {
  a := a + q(a)
}
\end{lstlisting}
using the pre-condition $((\forall x) P(x) \Rightarrow x \ge 0) \wedge ((\forall x) \mathtt{q}(x) > 0) \wedge P(\mathtt{a})$ and the post-condition $\mathtt{a} > 0$.
Let $\mathtt{a1}$ denote the value of the program variable $\mathtt{a}$ after the execution of the \verb'if' statement. Using \ITE\ and \LETIN\ expressions, the next state function for $\mathtt{a}$ can naturally be expressed by the following formula:
\begin{lstlisting}[language=cpp]
a1 = if r(a) then let a = a + 1 in a
             else let a = a + q(a) in a
\end{lstlisting}

This formula can further be encoded in TPTP, and hence used by a theorem prover as a hypothesis in proving partial correctness of the above code snippet. We illustrate below the TPTP encoding of the first-order problem corresponding to the partial program correctness problem we consider.  Note that the pre-condition becomes a hypothesis in TPTP, whereas the proof obligation given by the post-condition is a TPTP conjecture. All formulas below are typed first-order formulas (\lstinline'tff') in TPTP that use the built-in integer sort (\lstinline'$int').
\begin{lstlisting}[language=tptp]
tff(1, type, p: $int > $o).
tff(2, type, q: $int > $int).
tff(3, type, r: $int > $o).
tff(4, type, a: $int).
tff(5, hypothesis, ![X: $int]: (p(X) => $greatereq(X, 0))).
tff(6, hypothesis, ![X: $int]: ($greatereq(q(X), 0))).
tff(7, hypothesis, p(a)).
tff(8, hypothesis,
    a1 = $ite_t(r(a), $let_tt(a, $sum(a, 1), a),
                      $let_tt(a, $sum(a, q(a)), a))).
tff(9, conjecture, $greater(a1, 0)).
\end{lstlisting}

Running a theorem prover that supports \lstinline'$ite_t' and \lstinline'$let_tt' on this TPTP problem would prove the partial correctness of the program we considered. Note that without the use of \ITE\ and \LETIN\ expressions, a more tedious translation is needed for expressing the next state function of the program variable $\mathtt{a}$ as a first-order formula. When considering more complex programs containing multiple conditional expressions assignments and composition,
computing the next state function of a program variable results in a formula of size exponential in the number of conditional expressions. This problem of computing the next state function of variables is well-known in the program analysis community, by computing so-called static single assignment (SSA) forms. Using the \ITE\ and \LETIN\ expressions recently introduced in TPTP and already implemented in Vampire \cite{PSI14}, one can have a linear-size translation instead.

Let us however note that the usage of conditional expressions in TPTP is somewhat limited. The first argument of \lstinline'$ite_t' and \lstinline'$ite_f' is a logical formula, which means that a boolean condition from the program definition should be translated as such. At the same time, the same condition can be treated as a value in the program, for example, in a form of a boolean flag, passed as an argument to a function. Yet we cannot mix terms and formulas in the same way in a logical statement.
A possible solution would be to map the boolean type of programs to a user-defined boolean sort, postulate axioms about its semantics, and manually convert boolean terms into formulas where needed. This approach, however, suffers the disadvantages mentioned earlier, namely the need to design a special translation and its possible inefficiency.

Handling boolean terms as formulas is needed not only in applications of reasoning-based program analysis, but also in various problems of formalisation of mathematics.
For example, if one looks at two largest kinds of attempts to formalise mathematics and proofs: those performed by interactive proof assistants, such as Isabelle~\cite{Isabelle},  and the Mizar project~\cite{Mizar}, one can see that first-order theorem provers are the main workhorses behind computer proofs in both cases~--- see e.g.~\cite{Sledgehammer,DBLP:conf/icms/UrbanHV10}.
Interactive theorem provers, such as Isabelle routinely use quantifiers over booleans.  Let us illustrate this by the
following examples, chosen among 490 properties about (co)algebraic datatypes, featuring quantifiers over booleans, generated by Isabelle and kindly found for us by Jasmin Blanchette. Consider the distributivity of a conditional expression (denoted by the $\mathrm{ite}$ function) over logical connectives, a pattern that is widely used in reasoning about properties of data structures. For lists and the $\mathtt{contains}$ function that checks that its second argument contains the first one, we have the following example:
\begin{gather}\label{formula:contains}
  \begin{aligned}
&(\forall\ofsort{p}{\bool})(\forall\ofsort{l}{list_A})(\forall\ofsort{x}{A})(\forall\ofsort{y}{A}) \\
&\quad\mathtt{contains}(l,\mathrm{ite}(p,x,y)) \doteq \\
&\quad\quad(p \Rightarrow \mathtt{contains}(l,x)) \wedge (\neg p \Rightarrow \mathtt{contains}(l,y))
 \end{aligned}
\end{gather}

A more complex example with a heavy use of booleans is the unsatisfiability of the definition of $\mathtt{subset\_sorted}$.
\begin{gather}\label{formula:subset-sorted}
\begin{aligned}
&(\forall\ofsort{l_1}{list_A})(\forall\ofsort{l_2}{list_A})(\forall\ofsort{p}{\bool}) \\
&\hspace{0.5em}\neg (\mathtt{subset\_sorted}(l_1,\,l_2) \doteq p ~\wedge \\
&\hspace{1.6em}      (\forall\ofsort{l_2'}{list_A})\neg (l_1 \doteq \mathtt{nil} \wedge l_2 \doteq l_2' \wedge p) ~\wedge \\
&\hspace{1.6em}      (\forall\ofsort{x_1}{A})(\forall\ofsort{l_1'}{list_A})\neg (l_1 \doteq \mathtt{cons}(x_1,\,l_1') \wedge l_2 \doteq \mathtt{nil} \wedge \neg p) ~\wedge \\
&\hspace{1.6em}      (\forall\ofsort{x_1}{A})(\forall\ofsort{l_1'}{list_A})(\forall\ofsort{x_2}{A})(\forall\ofsort{l_2'}{list_A}) \\
&\hspace{2.1em}       \neg (l_1 \doteq \mathtt{cons}(x_1,\,l_1') \wedge l_2 \doteq \mathtt{cons}(x_2,\,l_2') ~\wedge \\
&\hspace{3.3em}       p \doteq \mathrm{ite}(x_1 < x_2,\,\false,\\
&\hspace{6.7em}                             \mathrm{ite}(x_1 \doteq x_2,\mathtt{subset\_sorted}(l_1',\,l_2'), \\
&\hspace{12.1em}                                        \mathtt{subset\_sorted}(\mathtt{cons}(x_1,\,l_1'),\,l_2')))))
\end{aligned}
\end{gather}
The $\mathtt{subset\_sorted}$ function takes two sorted lists and checks that its second argument is a sublist of the first one.

Problems with boolean terms are also common in the SMT-LIB project~\cite{SMT-LIB}, the collection of benchmarks for SMT-solvers. Its core logic is a variant of first-order logic that treats boolean terms as formulas, in which logical connectives and conditional expressions are defined in the core theory.

%Note in particular that this formula employes quantification over boolean variables and passing boolean terms as arguments to logical connectives, that would not be admissible in ordinary first-order logic.

In this paper we propose a modification \folb\ of first-order logic, which includes a first-class boolean sort and \ITE\ and \LETIN\ expressions, aimed for being used in automated first-order theorem proving. It is the smallest logic that contains both the SMT-LIB core theory and the monomorphic first-order subset of TPTP. The syntax and semantics of the logic are given in Section~\ref{sec:folbool}.
%In this paper we propose a modification of first-order logic, similar to the SMT-LIB logic, aimed at being used for first-order theorem proving. The modification includes formalisation of \verb'if'-\verb'then'-\verb'else' and \verb'let'-\verb'in' expressions and treatment of the boolean sort as a first class sort. This way the translation of certain program fragments with boolean values into logical statements become straightforward. The syntax and semantics of the logic is given in Section~\ref{sec:folbool}.
We further describe how \folb\ can be translated to the ordinary many-sorted first-order logic in Section~\ref{sec:folb-to-fol}.
Section~\ref{sec:superposition} discusses superposition-based theorem proving and proposes a new way of dealing with the boolean sort in it.
In Section~\ref{sec:tptp} we discuss the support of the boolean sort in TPTP and propose changes to it required to support a first-class boolean sort. We point out that such changes can also partially simplify the syntax of TPTP.
Section~\ref{sec:cicm15/related} discusses related work and Section~\ref{sec:cicm15/conclusions} contains concluding remarks.

The main contributions of this paper are the following:

\begin{enumerate}
\item the definition of \folb\ and its semantics;
\item a translation from \folb\ to first-order logic, which can be used to support \folb\ in existing first-order theorem provers;
\item a new technique of dealing with the boolean sort in superposition theorem provers, allowing one to replace boolean sort axioms by special rules;
\item a proposal of a change to the TPTP language, intended to support \folb\ and also simplify \ITE\ and \LETIN\ expressions.
\end{enumerate}


%------------------------------------------------------------------------------
\section[First-Order Logic with Boolean Sort]{First-Order Logic with Boolean Sort}
\label{sec:folbool}

First-order logic with the boolean sort (\folb) extends many-sorted first-order logic (FOL) in two ways:
\begin{enumerate}
\item formulas can be treated as terms of the built-in boolean sort; and
\item one can use \ITE\ and \LETIN\ expressions defined below.
\end{enumerate}
\folb\ is the smallest logic containing both the SMT-LIB core theory and the monomorphic first-order part of the TPTP language. It extends the SMT-LIB core theory by adding \LETIN\ expressions defining functions and TPTP by the first-class boolean sort.


\subsection{Syntax}

We assume a countable infinite set of \emph{variables}.

\begin{definition}\label{def:folb-signature}\em
  A \emph{signature} of first-order logic with the boolean sort is a triple $\Sigma = (S, F, \context)$, where:

  \begin{enumerate}
  \item $S$ is a set of \emph{sorts}, which contains a special sort $\bool$. A \emph{type} is either a sort or a non-empty sequence $\sigma_1,\ldots,\sigma_n,\sigma$ of sorts, written as $\sigma_1 \times \ldots \times \sigma_n \to \sigma$. When $n = 0$, we will simply write $\sigma$ instead of $\to\sigma$. We call a \emph{type assignment} a mapping from a set of variables and function symbols to types, which maps variables to sorts.

    \item $F$ is a set of \emph{function symbols}. We require $F$ to contain binary function symbols $\vee$, $\wedge$, $\implies$ and $\liff$, used in infix form, a unary function symbol $\neg$, used in prefix form, and nullary function symbols $\true$, $\false$.

    \item $\context$ is a \emph{type assignment} which maps each function symbol $f$ into a type $\tau$. When the signature is clear from the context, we will write $\ofsort{f}{\tau}$ instead of $\context(f)=\tau$ and say that $f$ is of the type $\tau$.

    We require the symbols $\vee, \wedge, \implies, \liff$ to be of the type $\bool \times \bool \to \bool$, $\neg$ to be of the type $\bool \to \bool$ and $\true,\false$ to be of the type $\bool$. \QED
  \end{enumerate}
\end{definition}
In the sequel we assume that $\Sigma = (S,F,\context)$ is an arbitrary but fixed signature.

To define the semantics of \folb, we will have to extend the signature and also assign sorts to variables. Given a type assignment $\context$, we define $\context,x:\sigma$ to be the type assignment that maps a variable $x$ to $\sigma$ and coincides otherwise with $\context$. Likewise, we define $\context,f:\tau$ to be the type assignment that maps a function symbol $f$ to $\tau$ and coincides otherwise with $\context$.

Our next aim is to define the set of terms and their sorts with respect to a type assignment $\context$. This will be done using a relation $\context \vdash t:\sigma$, where $\sigma \in S$, terms can then be defined as all such expressions $t$.

\begin{definition}\label{def:folb-terms}\rm
  The relation $\context \vdash t:\sigma$, where $t$ is an expression and $\sigma \in S$ is defined inductively as follows. If $\context \vdash t:\sigma$, then we will say that $t$ is a \emph{term of the sort $\sigma$} w.r.t.\ $\context$.
%We will also write $\context \vdash t_1:\sigma_1,\ldots,$
  \begin{enumerate}
    \item If $\context(x) = \sigma$, then $\context \vdash x:\sigma$.

    \item If $\context(f) = \sigma_1 \times \ldots \times \sigma_n \to \sigma$, $\context \vdash t_1:\sigma_1$, \ldots, $\context \vdash t_n:\sigma_n$, then $\context \vdash  f(t_1, \ldots, t_n) : \sigma$.

    \item If $\context \vdash \phi:\bool$, $\context \vdash t_1:\sigma$ and $\context \vdash t_2:\sigma$, then $\context \vdash (\ite{\phi}{t_1}{t_2}):\sigma$.

    \item Let $f$ be a function symbol and $x_1,\ldots,x_n$ pairwise distinct variables. If $\context,x_1:\sigma_1,\ldots,x_n:\sigma_n \vdash s:\sigma$ and $\context,f:(\sigma_1\times \ldots \times\sigma_n \to\sigma) \vdash t : \tau$, then $\context \vdash (\letin{f(x_1:\sigma_1, \ldots, x_n:\sigma_n)}{s}{t}) : \tau$.

    \item If $\context \vdash  s:\sigma$ and $\context \vdash  t:\sigma$, then $\context \vdash (s \eql t) : \bool$.

    \item If $\context,x : \sigma \vdash \phi : \bool$, then $\context \vdash (\forall x : \sigma)\phi : \bool$ and $\context \vdash (\exists x:\sigma)\phi : \bool$. \QED
  \end{enumerate}
\end{definition}
We only defined a \LETIN\ expression for a single function symbol. It is not hard to extend it to a \LETIN\ expression that binds multiple pairwise distinct function symbols in parallel, the details of such an extension are straightforward.

When $\context$ is the type assignment function of $\Sigma$ and $\context \vdash t : \sigma$, we will say that $t$ is a \emph{$\Sigma$-term of the sort $\sigma$}, or simply that $t$ is \emph{a term of the sort $\sigma$}. It is not hard to argue that every $\Sigma$-term has a unique sort.

According to our definition, not every term-like expression has a sort. For example, if $x$ is a variable and $\context$ is not defined on $x$, then $x$ is a not a $term$ w.r.t.\ $\context$. To make the relation between term-like expressions and terms clear, we introduce a notion of free and bound occurrences of variables and function symbols. We call the following occurrences of variables and function symbols \emph{bound}:

\begin{enumerate}
\item any occurrence of $x$ in $(\forall x:\sigma) \phi$ or in $(\exists x:\sigma) \phi$;
\item in the term $\letin{f(x_1:\sigma_1, \ldots, x_n:\sigma_n)}{s}{t}$ any occurrence of a variable $x_i$ in $f(x_1:\sigma_1, \ldots, x_n:\sigma_n)$ or in $s$, where $i = 1,\ldots, n$.
\item in the term $\letin{f(x_1:\sigma_1, \ldots, x_n:\sigma_n)}{s}{t}$ any occurrence of the function symbol $f$ in $f(x_1:\sigma_1, \ldots, x_n:\sigma_n)$ or in $t$.
\end{enumerate}
All other occurrences are called \emph{free}. We say that a variable or a function symbol is \emph{free} in a term $t$ if it has at least one free occurrence in $t$. A term is called \emph{closed} if it has no occurrences of free variables.

\begin{theorem}\rm
  Suppose $\context \vdash t : \sigma$. Then
  \begin{enumerate}
    \item for every free variable $x$ of $t$, $\context$ is defined on $x$;
    \item for every free function symbol $f$ of $t$, $\context$ is defined on $f$;
    \item if $x$ is a variable not free in $t$, and $\sigma'$ is an arbitrary sort, then
      $\context, x : \sigma' \vdash t : \sigma$;
    \item if $f$ is a function symbol not free in $t$, and $\tau$ is an arbitrary type, then $\context, f : \tau \vdash t : \sigma$. \QED
  \end{enumerate}
\end{theorem}

\begin{definition}\rm
  A \emph{predicate symbol} is any function symbol of the type $\sigma_1 \times \ldots \times \sigma_n \to \bool$.
  A \emph{$\Sigma$-formula} is a $\Sigma$-term of the sort $\bool$. All $\Sigma$-terms that are not $\Sigma$-formulas are called \emph{non-boolean terms}. \QED
\end{definition}

Note that, in addition to the use of \LETIN\ and \ITE, \folb\ is a proper extension of first-order logic. For example, in \folb\ formulas can be used as arguments to terms and one can quantify over booleans. As a consequence, every quantified boolean formula is a formula in \folb.

\subsection{Semantics}

As usual, the semantics of \folb\ is defined by introducing a notion of \emph{interpretation} and defining how a term is evaluated in an interpretation.

\begin{definition}\label{def:folb-interpretation}\rm
  Let $\context$ be a type assignment.
  A \emph{$\context$-interpretation} $\intI$ is a map, defined as follows. Instead of $\intI(e)$ we will write $\interpret{e}{\intI}$, for every element $e$ in the domain of $\intI$.
  \begin{enumerate}
    \item Each sort $\sigma \in S$ is mapped to a nonempty domain $\interpret{\sigma}{\intI}$. We require $\interpret{\bool}{\intI} = \left\{0, 1\right\}$.

    \item If $\context \vdash x:\sigma$, then $\interpret{x}{\intI} \in \interpret{\sigma}{\intI}$.

    \item If $\context(f) = \sigma_1 \times \ldots \times \sigma_n \to \sigma$, then $\interpret{f}{\intI}$ is a function from $\interpret{\sigma_1}{\intI} \times \ldots \times \interpret{\sigma_n}{\intI}$ to $\interpret{\sigma}{\intI}$.

    \item We require $\interpret{\true}{\intI} = 1$ and $\interpret{\false}{\intI} = 0$. We require $\interpret{\wedge}{\intI}$, $\interpret{\vee}{\intI}$, $\interpret{\implies}{\intI}$, $\interpret{\liff}{\intI}$ and $\interpret{\neg}{\intI}$ respectively to be the logical conjunction, disjunction, implication, equivalence and negation, defined over $\{0,1\}$ in the standard way. \QED
  \end{enumerate}
%  We will call the symbols $\true$, $\false$, $\wedge$, $\vee$, $\implies$, $\liff$ and $\neg$ \emph{interpreted} and all other symbols \emph{uninterpreted}.\AV{don't know if this will be used}
\end{definition}

Given a $\context$-interpretation $\intI$ and a function symbol $f$, we define $\variant{\intI}{f}{g}$ to be the mapping that maps $f$ to $g$ and coincides otherwise with $\intI$.
Likewise, for a variable $x$ and value $a$ we define $\variant{\intI}{x}{a}$ to be the mapping that maps $x$ to $a$ and coincides otherwise with $\intI$.

\begin{definition}\label{def:folb-term-evaluation}\rm
  Let $\intI$ be a $\context$-interpretation, and $\context \vdash t:\sigma$. The \emph{value of $t$ in $\intI$}, denoted as $\eval{t}{\intI}$, is a value in $\interpret{\sigma}{\intI}$ inductively defined as follows:
  \[
    \begin{aligned}
      \eval{x}{\intI} &= \interpret{x}{\intI}.
      \\
      \eval{f(t_1, \ldots, t_n)}{\intI} &= \interpret{f}{\intI}(\eval{t_1}{\intI}, \ldots, \eval{t_n}{\intI}).
      \\
      \eval{s \eql t}{\intI} &=
        \left\{ \begin{aligned}
                  &\text{1, if $\eval{s}{\intI} = \eval{t}{\intI}$;} \\
                  &\text{0, otherwise.}
                \end{aligned}\right.
      \\
      \eval{(\forall x : \sigma)\phi}{\intI} &=
        \left\{ \begin{aligned}
                  &\text{1, if $\eval{\phi}{\replacement{\intI}{x}{a}} = 1$ for all $a \in \interpret{\sigma}{\intI}$;} \\
                  &\text{0, otherwise.}
                \end{aligned}\right.
      \\
      \eval{(\exists x : \sigma)\phi}{\intI} &=
        \left\{ \begin{aligned}
                  &\text{1, if $\eval{\phi}{\replacement{\intI}{x}{a}} = 1$ for some  $a \in \interpret{\sigma}{\intI}$;} \\
                  &\text{0, otherwise.}
                \end{aligned}\right.
      \\
      \eval{\ite{\phi}{s}{t}}{\intI} &=
        \left\{ \begin{aligned}
                  &\eval{s}{\intI},\text{ if $\eval{\phi}{\intI} = 1$;} \\
                  &\eval{t}{\intI},\text{ otherwise.}
                \end{aligned}\right.
    \end{aligned}
  \]
  \[
      \eval{\letin{f(x_1:\sigma_1,\ldots,x_n:\sigma_n)}{s}{t}}{\intI} = \eval{t}{\replacement{\intI}{f}{g}},
  \]
  where $g$ is such that for all $i = 1, \ldots, n$ and $a_i \in \interpret{\sigma_i}{\intI}$, we have $g(a_1, \ldots, a_n) = \eval{s}{\replacement{\intI}{x_1 \ldots x_n}{a_1 \ldots a_n}}$. \QED
\end{definition}

\begin{theorem}\label{thm:semantics}\rm
  Let $\context \vdash \phi : \bool$ and $\intI$ be a $\context$-interpretation. Then
  \begin{enumerate}
    \item for every free variable $x$ of $\phi$, $\intI$ is defined on $x$;
    \item for every free function symbol $f$ of $\phi$, $\intI$ is defined on $f$;
    \item if $x$ is a variable not free in $\phi$, $\sigma$ is an arbitrary sort, and $a \in \interpret{\sigma}{\intI}$ then $\eval{\phi}{\intI} = \eval{\phi}{\replacement{\intI}{x}{a}}$;
    \item if $f$ is a function symbol not free in $\phi$, $\sigma_1,\ldots,\sigma_n,\sigma$ are arbitrary sorts and $g \in \interpret{\sigma_1}{\intI} \times \ldots \times \interpret{\sigma_n}{\intI} \to \interpret{\sigma}{\intI}$, then $\eval{\phi}{\intI} = \eval{\phi}{\replacement{\intI}{f}{g}}$. \QED
  \end{enumerate}
\end{theorem}

Let $\context \vdash \phi : \bool$. A $\context$-interpretation $\intI$ is called a \emph{model} of $\phi$, denoted by $\intI \models \phi$, if $\eval{\phi}{\intI} = 1$. If $\intI \models \phi$, we also say that $\intI$ \emph{satisfies} $\phi$. We say that $\phi$ is \emph{valid}, if $\intI \models \phi$ for all $\context$-interpretations $\intI$, and \emph{satisfiable}, if $\intI \models \phi$ for at least one $\context$-interpretation $\intI$. Note that Theorem~\ref{thm:semantics} implies that any interpretation, which coincides with $\intI$ on free variables and free function symbols of $\phi$ is also a model of $\phi$.


%------------------------------------------------------------------------------
\section{Translation of \folb{} to FOL}
\label{sec:folb-to-fol}

% !TEX root = ../main.tex
\folb\ is a modification of FOL. Every FOL formula is syntactically a \folb\ formula and has the same models, but not the other way around. In this section we present a translation from \folb\ to FOL, which preserves models. This translation can be used for proving theorems of \folb\ using a first-order theorem prover. We do not claim that this translation is efficient -- more research is required on designing translations friendly for first-order theorem provers.

We do not formally define many-sorted FOL with equality here, since FOL is essentially a subset of \folb, which we will discuss now.  

We say that an occurrence of a subterm $s$ of the sort $\bool$ in a term $t$ is in a \emph{formula context} if it is an argument of a logical connective or the occurrence in either $(\forall x:\sigma)s$ or $(\exists x:\sigma)s$. We say that an occurrence of $s$ in $t$ is in a \emph{term context} if this occurrence is an argument of a function symbol, different from a logical connective, or an equality. We say that a formula of \folb\ is \emph{syntactically first order} if it contains no \ITE\ and \LETIN\ expressions, no variables occurring in a formula context and no formulas occurring in a term context. By restricting the definition of terms to the subset of syntactically first-order formulas, we obtain the standard definition of many-sorted first-order logic, with the only exception of having a distinguished Boolean sort and constants $\true$ and $\false$ occurring in a formula context.

Let $\phi$ be a closed $\Sigma$-formula of \folb{}. We will perform the following steps to translate $\phi$ into a first-order formula. During the translation we will maintain a set of formulas $D$, which initially is empty. The purpose of $D$ is to collect a set of formulas (definitions of new symbols), which guarantee that the transformation preserves models.

\begin{enumerate}
\item Make a sequence of translation steps obtaining a syntactically first order formula $\phi'$. During this translation we will introduce new function symbols and add their types to the type assignment $\context$. We will also add formulas describing properties of these symbols to $D$. The translation will guarantee that the formulas $\phi$ and $\bigwedge_{\psi \in D}\psi \wedge \phi'$ are equivalent, that is, have the same models restricted to $\Sigma$.

\item Replace the constants $\true$ and $\false$, standing in a formula context, by nullary predicates $\top$ and $\bot$ respectively, obtaining a first-order formula.

\item Add special Boolean sort axioms.
\end{enumerate}
During the translation, we will say that a function symbol or a variable is \emph{fresh} if it neither appears in $\phi$ nor in any of the definitions, nor in the domain of $\context$.

We also need the following definition. Let $\context \vdash t:\sigma$, and $x$ be a variable occurrence in $t$. The \emph{sort of this occurrence of $x$} is defined as follows:

\begin{enumerate}
\item any free occurrence of $x$ in a subterm $s$ in the scope of $(\forall x:\sigma')s$ or $(\exists x:\sigma')s$ has the sort $\sigma'$.
\item any free occurrence of $x_i$ in a subterm $s_1$ in the scope of \\$\letin{f(x_1:\sigma_1, \ldots, x_n:\sigma_n)}{s_1}{s_2}$ has the sort $\sigma_i$, where $i = 1,\ldots,n$. 
\item a free occurrence of $x$ in $t$ has the sort $\context(x)$.
\end{enumerate}
If $\context \vdash t:\sigma$, $s$ is a subterm of $t$ and $x$ a free variable in $s$, we say that $x$ has a sort $\sigma'$ in $s$ if its free occurrences in $s$ have this sort.

The translation steps are defined below. We start with an empty set $D$ and an initial \folb\ formula $\phi$, which we would like to change into a syntactically first-order formula. At every translation step we will select a formula $\chi$, which is either $\phi$ or a formula in $D$, which is not syntactically first-order, replace a subterm in $\chi$ it by another subterm, and maybe add a formula to $D$. The translation steps can be applied in any order.

\begin{enumerate}
  \item Replace a Boolean variable $x$ occurring in a formula context, by $x \eql \true$.

  \item Suppose that $\psi$ is a formula occurring in a term context such that (i) $\psi$ is different from $\true$ and $\false$, (ii) $\psi$ is not a variable, and (iii) $\psi$ contains no free occurrences of function symbols bound in $\chi$. Let $x_1,\ldots,x_n$ be all free variables of $\psi$ and $\sigma_1,\ldots,\sigma_n$ be their sorts. Take a fresh function symbol $g$, add the formula $(\forall x_1:\sigma_1)\ldots(\forall x_n:\sigma_n) (\psi \liff g(x_1,\ldots,x_n) \eql \true)$ to $D$ and replace $\psi$ by $g(x_1,\ldots,x_n)$. Finally, change $\context$ to $\context,g : \sigma_1 \times \ldots \times \sigma_n \to \bool$.

  \item Suppose that $\ite{\psi}{s}{t}$ is a term containing no free occurrences of function symbols bound in $\chi$. Let $x_1,\ldots,x_n$ be all free variables of this term and $\sigma_1,\ldots,\sigma_n$ be their sorts. Take a fresh function symbol $g$, add the formulas $(\forall x_1:\sigma_1)\ldots(\forall x_n:\sigma_n) (\psi \implies g(x_1,\ldots,x_n) \eql s)$ and $(\forall x_1:\sigma_1)\ldots(\forall x_n:\sigma_n) (\neg\psi \implies g(x_1,\ldots,x_n) \eql t)$ to $D$ and replace this term by $g(x_1,\ldots,x_n)$. Finally, change $\context$ to $\context,g : \sigma_1 \times \ldots \times \sigma_n \to \sigma_0$, where $\sigma_0$ is such that $\context,x_1:\sigma_1,\ldots,x_n:\sigma_n \vdash s : \sigma_0$.

  \item Suppose that $\letin{f(x_1:\sigma_1, \ldots, x_n:\sigma_n)}{s}{t}$ is a term containing no free occurrences of function symbols bound in $\chi$. Let $y_1,\ldots,y_m$ be all free variables of this term and $\tau_1,\ldots,\tau_m$ be their sorts. Note that the variables in $x_1,\ldots,x_n$ are not necessarily disjoint from the variables in $y_1,\ldots,y_m$. 

Take a fresh function symbol $g$ and fresh sequence of variables $z_1,\ldots,z_n$. Let the term $s'$ be obtained from $s$ by replacing all free occurrences of $x_1,\ldots,x_n$ by $z_1,\ldots,z_n$, respectively. Add the formula $(\forall z_1:\sigma_1)\ldots(\forall z_n:\sigma_n) (\forall y_1:\tau_1)\ldots(\forall y_m:\tau_m) (g(z_1,\ldots,z_n,y_1,\ldots,\allowbreak y_m) \eql s')$ to $D$. Let the term $t'$ be obtained from $t$ by replacing all bound occurrences of $y_1,\ldots,y_m$ by fresh variables and each application $f(t_1, \ldots, t_n)$ of a free occurrence of $f$ in $t$ by $g(t_1, \ldots, t_n,\allowbreak y_1, \ldots, y_m)$. Then replace $\letin{f(x_1:\sigma_1, \ldots, x_n:\sigma_n)}{s}{t}$ by $t'$. Finally, change $\context$ to $\context,g : \sigma_1 \times \ldots \times \sigma_n \times \tau_1 \times \ldots \times \tau_m \to \sigma_0$, where $\sigma_0$ is such that $\context,x_1:\sigma_1,\ldots,x_n:\sigma_n,y_1:\tau_1,\ldots,y_m:\tau_m \vdash s : \sigma_0$. 
\end{enumerate}
The translation terminates when none of the above rules apply.

We will now formulate several of properties of this translation, which will imply that, in a way, it preserves models. These properties are not hard to prove, we do not include proofs in this paper.

\begin{lemma}\label{lemma:step-preserves-equivalence}\rm
  Suppose that a single step of the translation changes a formula $\phi_1$ into $\phi_2$, $\delta$ is the formula added at this step (for step 1 we can assume $\true=\true$ is added), $\context$ is the type assignment before this step and $\context'$ is the type assignment after. Then for every $\context'$-interpretation $\intI$ we have $\intI \models \delta \implies (\phi_1 \liff \phi_2)$. \QED
\end{lemma}

By repeated applications of this lemma we obtain the following result.

\begin{lemma}\label{lemma:definitions-preserve-models}\rm
  Suppose that the translation above changes a formula $\phi$ into $\phi'$, $D$ is the set of definitions obtained during the translation, $\context$ is the initial type assignment and $\context'$ is the final type assignment of the translation. Let $I'$ be any interpretation of $\context'$. Then $I' \models \bigwedge_{\psi \in D} \psi \implies (\phi \Leftrightarrow \phi')$. \QED
\end{lemma}

We also need the following result.

\begin{lemma}\label{lem:termination}\rm
  Any sequence of applications of the translation rules terminates. \QED
\end{lemma}

The lemmas proved so far imply that the translation terminates and the final formula is equivalent to the initial formula in every interpretation satisfying all definitions in $D$. To prove model preservation, we also need to prove some properties of the introduced definitions. 

\begin{lemma}\label{lem:satisfy:definitions}\rm
  Suppose that one of the steps 2--4 of the translation translates a formula $\phi_1$ into $\phi_2$, $\delta$ is the formula added at this step, $\context$ is the type assignment before this step, $\context'$ is the type assignment after, and $g$ is the fresh function symbol introduced at this step. Let also $\intI$ be $\context$-interpretation. Then there exists a function $h$ such that $\replacement{\intI}{g}{h} \models \delta$. \QED
\end{lemma}

These properties imply the following result on model preservation.

\begin{theorem}\label{thm:model:preservation}\rm
  Suppose that the translation above translates a formula $\phi$ into $\phi'$, $D$ is the set of definitions obtained during the translation, $\context$ is the initial type assignment and $\context'$ is the final type assignment of the translation. 
  \begin{enumerate}
    \item Let $\intI$ be any $\context$-interpretation. Then there is a $\context'$-interpretation $I'$ such that $\intI'$ is an extension of $\intI$ and $\intI' \models \bigwedge_{\psi \in D} \psi \wedge \phi'$.
    \item Let $\intI'$ be a $\context'$-interpretation and $\intI' \models \bigwedge_{\psi \in D} \psi \wedge \phi'$. Then $\intI' \models \phi$. \QED
  \end{enumerate}
\end{theorem}
This theorem implies that $\phi$ and $\bigwedge_{\psi \in D} \psi \wedge \phi'$ have the same models, as far as the original type assignment (the type assignment of $\Sigma$) is concerned. The formula $\bigwedge_{\psi \in D} \psi \wedge \phi'$ in this theorem is syntactically first-order. Denote this formula by $\gamma$. Our next step is to define a model-preserving translation from syntactically first-order formulas to first-order formulas.

To make $\gamma$ into a first-order formula, we should get rid of $\true$ and $\false$ occurring in a formula context. To preserve the semantics, we should also add axioms for the Boolean sort, since in first-order logic all sorts are uninterpreted, while in \folb\ the interpretations of the Boolean sort and constants $\true$ and $\false$ are fixed. 

To fix the problem, we will add axioms expressing that the Boolean sort has two elements and that $\true$ and $\false$ represent the two distinct elements of this sort.
\begin{equation}\label{axiom:bool}
  \forall (x:\bool)(x \eql \true \vee x \eql \false) \wedge \true \not\eql \false.
\end{equation}
Note that this formula is a tautology in \folb, but not in FOL.

Given a syntactically first-order formula $\gamma$, we denote by $\toFOL{\gamma}$ the formula obtained from $\gamma$ by replacing all occurrences of $\true$ and $\false$ in a formula context by logical constants $\top$ and $\bot$ (interpreted as always true and always false), respectively and adding formula \eqref{axiom:bool}.

\begin{theorem}\label{thm:model:preservation:2}\rm
  Let $\context$ is a type assignment and $\gamma$ be a syntactically first-order formula such that $\context \vdash \gamma:\bool$.
  \begin{enumerate}
  \item Suppose that $\intI$ is a $\context$-interpretation and $\intI \models \gamma$ in \folb. Then $\intI \models \toFOL{\gamma}$ in first-order logic.
  \item Suppose that $\intI$ is a $\context$-interpretation and $\intI \models \toFOL{\gamma}$ in first-order logic. Consider the \folb-interpretation $\intI'$ that is obtained from $\intI$ by changing the interpretation of the Boolean sort $\bool$ by $\{0,1\}$ and the interpretations of $\true$ and $\false$ by the elements $1$ and $0$, respectively, of this sort. Then $\intI' \models \gamma$ in \folb. \QED
  \end{enumerate}
\end{theorem}

Theorems~\ref{thm:model:preservation} and~\ref{thm:model:preservation:2} show that our translation preserves models. Every model of the original formula can be extended to a model of the translated formulas by adding values of the function symbols introduced during the translation. Likewise, any first-order model of the translated formula becomes a model of the original formula after changing the interpretation of the Boolean sort to coincide with its interpretation in \folb.

%------------------------------------------------------------------------------
\section{Superposition for \folb{}}
\label{sec:superposition}

% !TEX root = ../main.tex
In
Section~\ref{sec:folb-to-fol} we presented a model-preserving
syntactic translation of \folb{} to FOL.
Based on this translation, automated reasoning about \folb{} formulas
can be done by translating a \folb{} formula into a FOL
formula, and using an automated first-order theorem prover on the resulting FOL formula.
State-of-the-art first-order theorem provers, such as Vampire~\cite{Vampire13}, E~\cite{E13} and
Spass~\cite{Spass}, implement superposition calculus for proving first-order formulas. Naturally, we would like to have a translation exploiting such provers in an efficient manner.

Note however that our translation adds the two-element domain axiom
$\forall (x:\bool)\allowbreak(x \eql \true \vee x \eql \false)$ for the boolean sort. This axioms will be converted to the clause
\begin{equation}\label{clause:T|F}
  x \eql \true \vee x \eql \false,
\end{equation}
where $x$ is a boolean variable. In this section we
explain why this axiom requires a special treatment and propose a solution to overcome problems caused by its presence.
%
%ththerefore be reduced to reasoning about the translated FOL formula
%using established technics such as superposition calculi. The extra
%axioms, added to the set of definitions at the last step of the
%translation, however, might not be treated efficiently by a
%superposition inference system. In this section
%we will explain the difficulties raised by the presence of these axioms and formulate a property that must be satisfied by a superposition inference system in order to be able to reason in \folb{} efficiently.

We assume some basic understanding of first-order theorem proving and superposition calculus, see, e.g.~\cite{Ganzinger01,NieuwenhuisRubio:HandbookAR:paramodulation:2001}. We fix a superposition inference system for first-order logic with equality, parametrised by a simplification ordering $\succ$ on literals and a well-behaved literal selection function \cite{Vampire13}, that is a function that guarantees completeness of the calculus. We denote selected literals by underlining them. We assume that equality literals are treated by a dedicated inference rule, namely, the ordered paramodulation rule~\cite{Robinson1969}:
\[
\infer[\quad\text{if}\ \theta = \mathrm{mgu}(l, s),]{(L[r] \vee C \vee D)\theta}%
{\underline{l \eql r} \vee C & \underline{L[s]} \vee D}
\]
where $C,D$ are clauses, $L$ is a literal, $l,r,s$ are terms, $\mathrm{mgu}(l, s)$ is a most general unifier of $l$ and $s$, and $r\theta \not\succeq l\theta$.
The notation $L[s]$ denotes that $s$ is a subterm of $L$, then $L[r]$ denotes the result of replacement of $s$ by $r$.

Suppose now that we use an off-the-shelf superposition theorem prover to reason about FOL formulas obtained by our translation. W.l.o.g, we assume that $\true \succ \false$ in the term ordering used by the prover. Then self-paramodulation (from $\true$ to $\true$) can be applied to clause~\eqref{clause:T|F} as follows:
\[
\infer{x \eql y \vee x \eql \false \vee y \eql \false}%
{\underline{x \eql \true} \vee x \eql \false & \underline{y \eql \true} \vee y \eql \false}
\]

The derived clause $x \eql y \vee x \eql false \vee y \eql \false$ is a recipe for disaster, since the literal $x \eql y$ must be selected and can be used for paramodulation into every non-variable term of a boolean sort. Very soon the search space will contain many clauses obtained as logical consequences of clause \eqref{clause:T|F} and results of paramodulation from variables applied to them. This will cause a rapid degradation of performance of superposition provers.

To get around this problem, we propose the following solution. First, we will choose term
orderings $\succ$ having the following properties: $\true\succ\false$ and $\true$ and
$\false$ are the smallest ground terms w.r.t.\ $\succ$. Consider now all ground instances of \eqref{clause:T|F}. They have the form $s \eql \true \vee s \eql \false$, where $s$ is a ground term. When $s$ is either $\true$ or $\false$, this instance is a tautology, and hence redundant. Therefore, we should only consider instances for which $s \succ \true$. This prevents self-paramodulation of \eqref{clause:T|F}.

Now the only possible inferences with \eqref{clause:T|F} are inferences of the form
\[
\infer[,]{C[\true] \vee s \eql \false}%
{\underline{x \eql \true} \vee x \eql \false & C[s]}
\]
where $s$ is a non-variable term of the sort $\bool$.
To implement this, we can remove clause \eqref{clause:T|F} and add as an extra inference rule to the superposition calculus the following rule:
\[
\infer[,]{C[\true] \vee s \eql \false}%
{C[s]}
\]
where $s$ is a non-variable term of the sort $\bool$ other than $\true$ and $\false$.


%------------------------------------------------------------------------------
\section{TPTP Support for \folb{}}
\label{sec:tptp}

The typed monomorphic first-order formulas subset, called TFF0, of the TPTP language~\cite{TPTP}, is a representation language for many-sorted first-order logic. It contains \verb'if'-\verb'then'-\verb'else' and \verb'let'-\verb'in' constructs (see below), which is useful for applications, but is inconsistent in its treatment of the boolean sort. It has a predefined atomic sort symbol \verb'$o' denoting the boolean sort. However, unlike all other sort symbols, \verb|$o| can only be used to declare the return type of predicate symbols. This means that one cannot define a function having a boolean argument, use boolean variables or equality between booleans. 

Such an inconsistent use of the boolean sort results in having two kinds of \verb'if'-\verb'then'-\verb'else' expressions and four kinds of \verb'let'-\verb'in' expressions. For example, a \folb-term $\letin{f(x_1:\sigma_1, \ldots, x_n:\sigma_n)}{s}{t}$ can be represented using one of the four TPTP alternatives \verb|$let_tt|, \verb|$let_tf|, \verb|$let_ft| or \verb|$let_ff|, depending on whether $s$ and $t$ are terms or formulas. 

Since the boolean type is second-class in TPTP, one cannot directly represent formulas coming from program analysis and interactive theorem provers, such as formulas \eqref{formula:contains} and \eqref{formula:subset-sorted} of Section~\ref{sec:cicm15/introduction}.

We propose to modify the TFF0 language of TPTP to coincide with \folb. It is not late to do so, since there is no general support for \verb'if'-\verb'then'-\verb'else' and \verb'let'-\verb'in'. To the best of our knowledge, Vampire is currently the only theorem prover supporting full TFF0. Note that such a modification of TPTP would make multiple forms of \verb'if'-\verb'then'-\verb'else' and \verb'let'-\verb'in' redundant. It will also make it possible to directly represent the SMT-LIB core theory.

We note that our changes and modifications on TFF0 can also be applied to the TFF1 language of TPTP~\cite{tff1}. TFF1 is  a polymorphic extension of TFF0 and its formalisation  does not treat the boolean sort. Extending our work to TFF1 should not be hard but has to be done in detail.

%------------------------------------------------------------------------------
\section{Related Work}
\label{sec:cicm15/related}

% !TEX root = ../main.tex
Handling Boolean terms as formulas is common in the SMT community. The SMT-LIB project~\cite{SMT-LIB} defines its core logic as first-order logic extended with the distinguished first-class Boolean sort and the \verb'let'-\verb'in' expression used for local bindings of variables. The core theory of SMT-LIB defines logical connectives as Boolean functions and the ad-hoc polymorphic \verb'if'-\verb'then'-\verb'else' ($ite$) function, used for conditional expressions. 
% SMT-solvers do not reason in the core logic, but use quantifier-free fragments of it with theories. 
The language \folb\ defined here extends the SMT-LIB core language with local function definitions,
using \verb'let'-\verb'in' expressions defining functions of arbitrary, and not just zero, arity. Thus, \folb\ contains both this language and the TFF0 subset of TPTP. Further, we present a translation of \folb\ to FOL and show how one can improve superposition theorem provers to reason with the Boolean sort. 

% Unlike SMT-LIB, \folb{} defines logical connectives as interpreted functions and not as part of a theory, and the \verb'if'-\verb'then'-\verb'else' construct as part of the logic language.

Efficient superposition theorem proving in finite domains, such as the Boolean domain, is also discussed in~\cite{HillenbrandWeidenbach13}. The approach of~\cite{HillenbrandWeidenbach13} sometimes falls back to enumerating instances of a clause by instantiating finite domain variables with all elements of the corresponding domains. We point out here that for the Boolean (i.e., two-element) domain there is a simpler solution. However, the approach of~\cite{HillenbrandWeidenbach13} also allows one to handle domains with more than two elements. One can also generalise our approach to arbitrary finite domains by using binary encodings of finite domains, however, this will necessarily result in loss of efficiency, since a single variable over a domain with $2^k$ elements will become $k$ variables in our approach, and similarly for function arguments.


%------------------------------------------------------------------------------
\section{Conclusion}
\label{sec:cicm15/conclusions}

% !TEX root = ../main.tex
We defined first-order logic with the first class Boolean sort (\folb{}). It extends ordinary many-sorted first-order logic (FOL) with (i) the Boolean sort such that terms of this sort are indistinguishable from formulas and (ii) \ITE\ and \LETIN\ expressions. The semantics of \LETIN\ expressions in \folb{} is essentially their semantics in functional programming languages, when they are not used for recursive definitions. In particular, non-recursive local functions can be defined and function symbols can be bound to a different sort in nested \verb'let'-\verb'in' expressions.

We argued that these extensions are useful in reasoning about problems coming from program analysis and interactive theorem proving. The extraction of properties from certain program definitions (especially in functional programming languages) into \folb{} formulas is more straightforward than into ordinary FOL formulas and potentially more efficient. In a similar way, a more straightforward translation of certain higher-order formulas into \folb{} can facilitate proof automation in interactive theorem provers.

\folb{} is a modification of FOL and reasoning in it reduces to reasoning in FOL. We gave a translation of \folb{} to FOL that can be used for proving theorems in \folb{} in a first-order theorem prover. We further discussed a modification of superposition calculus that can reason efficiently in presence of the Boolean sort. Finally, we pointed out that the TPTP language can be changed to support \folb{}, which will also simplify some parts of the TPTP syntax.

Implementation of theorem proving support for \folb{}, including its super\-po\-sition-friendly translation to CNF, is an important task for future work. Further, we are also interested in extending \folb{} with theories, such as the theory of integer linear arithmetic and arrays.

%------------------------------------------------------------------------------
\section*{Acknowledgements}
\label{sec:cicm15/acknowledgements}

The first two authors were partially supported by the Wallenberg Academy Fellowship 2014, the Swedish VR grant D0497701, and the Austrian research project FWF S11409-N23. The third author was Partially supported by the EPSRC grant ``Reasoning in Verification and Security''.


\def\paperTwoContentsTitle{The Vampire and the FOOL}
\def\paperTwoChapterTitle{The Vampire and the FOOL}
\def\paperTwoAuthors{Evgenii~Kotelnikov, Laura~Kov\'{a}cs,\\Giles~Reger and Andrei~Voronkov}
\def\paperTwoAbstract{This paper presents new features recently implemented in the theorem prover Vampire, namely support for first-order logic with a first class boolean sort (\folb{}) and polymorphic arrays. In addition to having a first class boolean sort, \folb{} also contains \ITE\ and \LETIN\ expressions. We argue that presented extensions facilitate reasoning-based program analysis, both by increasing the expressivity of first-order reasoners and by gains in efficiency.}
\def\paperTwoPublication{Published in the \emph{Proceedings of the 5th ACM SIGPLAN Conference on Certified Programs and Proofs}, pages 37--48. ACM New York, 2016.}
\paperchapter{\paperTwoContentsTitle}
             {\paperTwoChapterTitle}
             {\paperTwoAuthors}
             {\paperTwoAbstract}
             {\paperTwoPublication}
\label{chap:implementation}
\section{Introduction}
\label{sect:introduction}

Automated program analysis and verification requires discovering and proving program properties. These program properties are checked using various tools, including theorem provers. The translation of program properties into formulas accepted by a theorem prover is not straightforward because of a mismatch between the semantics of the programming language constructs and that of the input language of the theorem prover. If program properties are not directly expressible in the input language, one should implement a translation of such program properties to the language. Such translations can be very complex and thus error prone.

The performance of a theorem prover on the result of a translation crucially depends on whether the translation introduces formulas potentially making the prover inefficient. Theorem provers, especially first-order ones, are known to be very fragile with respect to the input. Expressing program properties in the ``right'' format therefore requires solid knowledge about how theorem provers work and are implemented~--- something that a user of a verification tool might not have. Moreover, it can be hard to efficiently reason about certain classes of program properties, unless special inference rules and heuristics are added to the theorem prover. For example, \cite{ATVA14} shows a considerable gain in performance on proving properties of data collections by using a specially designed extensionality resolution rule.

If a theorem prover natively supports expressions that mirror the semantics of programming language constructs, we solve both above mentioned problems. First, the users do not have to design translations of such constructs. Second, the users do not have to possess a deep knowledge of how the theorem prover works~--- the efficiency becomes the responsibility of the prover itself.

In this work we present new features recently implemented in the theorem prover Vampire~\cite{Vampire13} to natively support mirroring programming language constructs in its input language. They include (i) FOOL~\cite{FOOL}, that is the extension of first-order logic by a first-class boolean sort, \ITE\ and \LETIN\ expressions, and (ii) polymorphic arrays.

This paper is structured as follows. Section~\ref{sect:fool} presents how FOOL is implemented in Vampire and focuses on new extensions to the TPTP input language~\cite{TPTP} of first-order provers. Section~\ref{sect:fool}  extends the TPTP language of monomorphic many-sorted first-order formulas, called TFF0~\cite{tff0}, and allows users to treat the built-in boolean sort \tptpo\ as a first class sort. Moreover, it introduces expressions \dite\ and \dlet, which unify various TPTP \ITE\ and \LETIN\ expressions.

Section~\ref{sect:arrays} presents a formalisation of a polymorphic theory of arrays in TPTP and its implementation in Vampire. It extends TPTP with features of the TFF1 language~\cite{tff1} of rank-1 polymorphic  first-order formulas, namely, sort arguments for the built-in array sort constructor \darraySymb. Sort variables however are not supported.

We argue that these extensions make the translation of properties of some programs to TPTP easier. To support this claim, in Section~\ref{sect:example} we discuss representation of various programming and other constructs in the extended TPTP language. We also give a linear translation of  the next state relation for any program with assignments, \ITE, and sequential composition.

Experiments with theorem proving with FOOL formulas are described in Section~\ref{sect:experiments}. In particular, we show that the implementation of a new inference rule, called FOOL paramodulation, improves performance of theorem provers using superposition calculus.

Finally, Section~\ref{sect:related} discusses related work and Section~\ref{sect:future} outlines future work.

% \LK{added the text below, addressing the CFP}

\noindent\paragraph{Summary of the main results.}
\begin{itemize}
\item We describe an implementation of first-order logic with a first-class boolean sort. This bridges the gap between input languages for theorem provers and logics and tools used in program analysis. We believe it is a first ever implementation of first-class boolean sorts in superposition theorem provers.

\item We extend and simplify the TPTP language~\cite{TPTP}, by providing more powerful and more uniform representations of \ITE\ and \LETIN\ expressions. To the best of our knowledge, Vampire is the only superposition theorem prover implementing these constructs.

\item We formalise and describe an implementation in Vampire of a polymorphic theory of arrays. Again, we believe that Vampire is the only superposition theorem prover implementing this theory.

\item We give a simple extension of FOOL, allowing to express the next state relation of a program as a boolean formula which is linear in the size of the program. This  boolean formula captures the exact semantics of the program and can be used by a first-order theorem prover. We are not aware of any other work on extending theorem provers with support for representing fragments of imperative programs.

\item We demonstrate usability and high performance of our implementation on two collections of examples, coming from the higher-order part of the TPTP library and from the Isabelle interactive theorem prover~\cite{Isabelle}. Our experimental results show that Vampire outperforms systems which could previously be used to solve such problems:
higher-order theorem provers and satisfiability modulo theory (SMT) solvers.
\end{itemize}

The paper focuses on new, practical features extending first-order theorem provers for making them better suited for applications of reasoning in various theories, program analysis and verification. While the paper describes implementation details and challenges in the Vampire theorem prover, the described features and their implementation can be carried out in any other first-order prover.

Summarising, we believe that our paper advances the state-of-the-art in formal certification of programs and proofs. With the use of FOOL and polymorphic arrays, we bring first-order theorem proving closer to program logics and make first-order theorem proving better suited for program analysis and verification. We also believe that an implementation of FOOL advances automation of mathematics, making many problems using the boolean type directly understood by a first-order theorem prover, while they previously were treated as higher-order problems.


%------------------------------------------------------------------------------
\section{First Class Boolean Sort}
\label{sect:fool}

This section presents a clausification algorithm for \folb{}. This algorithm takes a \folb{} formula as input and produces a set of first-order clauses. The conjunction of these clauses is equisatisfiable to the input formula.

\folb{} extends many-sorted first-order logic with an interpreted boolean sort and the following syntactical constructs:
\begin{enumerate}
  \item boolean variables used as formulas;
  \item formulas used as arguments to function and predicate symbols;
  \item \ITE\ expressions that can occur as terms and formulas;
  \item \LETIN\ expressions that can occur as terms and formulas and can define an arbitrary number of function and predicate symbols.
\end{enumerate}

There are several ways to support the interpreted boolean sort in a first-order logic. The approach taken in~\cite{FOOL} proposes to axiomatise it by adding two constants $\true$ and $\false$ of this sorts and two axioms: $\true \neql \false$ and $(\forall x:\bool)(x \eql \true \lor x \eql \false)$. Furthermore, \cite{FOOL} proposed a modification of superposition calculus that included a replacement of the second axiom with the specialised \folb{} paramodulation rule. This modification prevents possible performance problems of a superposition theorem prover caused by self-paramodulation of $x \eql \true \lor x \eql \false$. The translation to first-order clauses presented in this section does not require boolean axioms or modifications of superposition calculus to correctly support the boolean sort. This property is explained at the end of this section.

Our algorithm is an extension of \newcnf{} that adds support for \folb{} formulas. In order to enable \newcnf{} to translate \folb{} and not just first-order formula we make the following changes to it.
\begin{itemize}
  \item We allow intermediate clauses to contain signed \folb{} formulas, and not just first-order formulas.
  \item We extend the \newcnf{} tautology elimination with the support for boolean variables. Whenever a boolean variable occurs in an intermediate clause twice with the opposite signs, that intermediate clause is not added to $\GC$. Whenever a boolean variable occurs in an intermediate clause multiple times with the same sign, only one occurrence is kept in the intermediate clause.
  \item We add extra rules that guide how intermediate clauses are replaced in the set $\GC$, detailed below. These rules correspond to syntactical constructs available in \folb{} but not in ordinary first-order logic.
  \item We change the rule that translates existentially quantified formulas to skolemise boolean variables using Skolem predicates and not Skolem functions. For that, we also allow substitutions to map boolean variables to Skolem literals. 
  \item We add an extra step of translation. After the input formula has been traversed, we apply substitutions of boolean variables to every formula in each respective intermediate clause. The resulting set of intermediate clauses might have Skolem literals occurring as terms. We run the clausification algorithm again on this set of intermediate clauses. The second run does not introduce new substitutions and results with a set of intermediate clauses that only contains atomic formulas and substitutions of non-boolean variables.
\end{itemize}

We extend the rules of replacing intermediate clauses with the cases detailed below. We will not distinguish formulas used as arguments as a separate syntactical construct, but rather treat each such formula $\phi$ as an \ITE\ expression of the form $\ite{\phi}{\true}{\false}$. We will assume that every \LETIN\ expression defines exactly one function or predicate symbol. Every \LETIN\ expression that defines more that one symbol can be transformed to multiple nested \LETIN\ expressions, each defining a single symbol, possibly by renaming some of the symbols. Moreover, we will assume that \LETIN\ expressions only occur as formulas. Every formula that contains a \LETIN\ expression that occurs as non-boolean term can be transformed to a \LETIN\ expression that defines the same symbol and occurs as formula. % EK: Should be careful with let inside let bindings!

Let $\psi$ be a subformula of the input formula $\phi$ and $\genclause{C}{\subst}$ be an intermediate clause such that $C$ has an occurrence of $\genlit{\psi}{\sign}$.
\begin{itemize}
  \item
    Suppose that $\psi$ is a boolean variable $x$. If $\subst$ does not map $x$, add the intermediate clause $\genclause{C'}{\subst'}$ to $\GC$, where $C'$ is obtained from $C$ by removing the occurrence of $\genlit{\psi}{\sign}$ and $\subst'$ extends $\subst$ with $x \mapsto \false$ if $\sign=\possign$, and $x \mapsto \true$ if $\sign=\negsign$. If $\subst$ does map $x$, add $\genclause{C}{\subst}$ to $\GC$. 

  \item
    Suppose that $\psi$ is $\gamma_1 \eql \gamma_2$, where $\gamma_1$ and $\gamma_2$ are formulas. Add two intermediate clauses to $\GC$ obtained from $C$ by replacing the occurrence of $\psi$ with $\genlit{\gamma_1}{-\sign}$, $\genlit{\gamma_2}{\possign}$ and $\genlit{\gamma_1}{\sign}$, $\genlit{\gamma_2}{\negsign}$, respectively.

  \item
    Suppose that $\psi$ is $\ite{\chi}{\gamma_1}{\gamma_2}$. Add two intermediate clauses to $\GC$ obtained from $C$ by replacing the occurrence of $\genlit{\psi}{\sign}$ with $\genlit{\chi}{\negsign}$, $\genlit{\gamma_1}{\sign}$ and $\genlit{\chi}{\possign}$, $\genlit{\gamma_2}{\sign}$, respectively.

  \item
    Suppose that $\psi$ is an atomic formula that contains one or more \ITE\ expressions occurring as terms. Each of the \ITE\ expressions is translated in one of two ways, either by expanding or by naming. We will describe both ways for a single \ITE\ expressions and then generalise for an arbitrary number of \ITE\ expressions. Suppose that $\psi$ is an atomic formula $L[\ite{\gamma}{s}{t}]$.

    \paragraph{Expanding.} Add two intermediate clauses to $\GC$ obtained from $C$ by replacing the occurrence of $\genlit{\phi}{\sign}$ with $\genlit{\gamma}{\negsign}$, $\genlit{L[s]}{\sign}$ and $\genlit{\gamma}{\possign}$, $\genlit{L[t]}{\sign}$, respectively.
    
    \paragraph{Naming.} Let $x_1,\ldots,x_n$ be all free variables of $\phi$, and $\tau_1,\ldots,\tau_n$ be their sorts. Let $\tau$ be be the sort of both $s$ and $t$. Then,
    \begin{enumerate}
      \item introduce a fresh predicate symbol $P$ of the sort $\tau\times\tau_1\times\ldots\times\tau_n$;
      \item introduce a fresh variable $y$ of the sort $\tau$;
      \item add an intermediate clause to $\GC$ that is obtained from $C$ by replacing the occurrence of $\genlit{\psi}{\sign}$ with $\genlit{L[y]}{\sign}$, $\genlit{P(y,x_1,\ldots,x_n)}{\negsign}$;
      \item add intermediate clauses $\genclause{\{\genlit{\gamma}{\negsign},\genlit{P(s,x_1,\ldots,x_n)}{\possign}\}}{\emptySubst}$ and\\$\genclause{\{\genlit{\gamma}{\possign},\allowbreak\genlit{P(t,\allowbreak x_1,\allowbreak \ldots,x_n)}{\possign}\}}{\emptySubst}$ to $\GC$.
    \end{enumerate}

    In order to eliminate all \ITE\ expressions we apply either expanding or naming to each of the \ITE\ expressions. We assume that a pre-specified expansion threshold limits the maximal number of expanded \ITE\ expressions inside one atomic formula. We start by expanding all \ITE\ expression and once the expansion threshold is reached, name the remaining \ITE\ expressions.

  \item
    Suppose that $\psi$ is $\letin{f(x_1:\sigma_1,\ldots,x_n:\sigma_n)}{t}{\gamma}$. It is translated in one of two ways, either by inlining or by naming. The choice of inlining or naming of \LETIN\ expressions in the problem is determined by a pre-specified boolean option. % provided by the user of the algorithm.
    
    \paragraph{Inlining.} Add an intermediate clause to $\GC$ that is obtained from $C$ by replacing the occurrence of $\genlit{\psi}{\sign}$ with $\genlit{\gamma'}{\sign}$. $\gamma'$ is obtained from $\gamma$ by replacing each application $f(t_1,\ldots,t_n)$ of a free occurrence of $f$ in $\gamma$ with $t'$, that is obtained from $t$ by replacing each free occurrence of $x_1,\ldots,x_n$ in $t$ with $t_1,\ldots,t_n$, respectively. We point out that inlining predicate symbols of zero arity does not hinder identification of tautologies thanks to intermediate tautology removal inside intermediate clauses.

    \paragraph{Naming.} Add an intermediate clause to $\GC$ that is obtained from $C$ by replacing the occurrence of $\genlit{\psi}{\sign}$ with $\genlit{\gamma}{\sign}$. Let $\tau$ be the sort of $t$. If $\tau$ is $\bool$, add intermediate clauses $\genclause{\{\genlit{f(x_1,\ldots,x_n)}{\negsign},\genlit{t}{\possign}\}}{\emptySubst}$ and $\genclause{\{\genlit{f(x_1,\ldots,x_n)}{\possign},\genlit{t}{\negsign}\}}{\emptySubst}$ to $\GC$. Otherwise, add an intermediate clause $\genclause{\{f(x_1,\ldots,x_n) \eql t\}}{\emptySubst}$ to $\GC$.
\end{itemize}

The extra step of translation that eliminates Skolem literals occurring as terms amounts to application of the expansion threshold-based procedure for \ITE\ expressions.

The extended \newcnf{} algorithm produces a set of first-order clauses. This set does not require boolean axioms to be equisatisfiable to the original \folb{} formula. The resulting set of clauses has the following two properties.
\begin{enumerate}
  \item It can only contain boolean variables and constants $\true$ and $\false$ as boolean terms. Every boolean term that occurs in $\phi$ is translated as formula and no boolean terms other than variables, $\true$ and $\false$ are introduced. 
  \item It does not contain equalities between boolean terms. Every boolean equality occurring in the input is translated as equivalence between its arguments, and no new boolean equalities are introduced.
\end{enumerate}
These two properties ensure that boolean variables will only be unified with $\true$ and $\false$ during superposition. Constants $\true$ and $\false$ cannot be unified with each other, therefore no logical inference can violate the properties of the boolean sort. % TODO: A more accurate argument is needed here!

%------------------------------------------------------------------------------
\section{Polymorphic Theory of Arrays}
\label{sect:arrays}

% !TEX root = ../main.tex
Using built-in arrays and reasoning in the first-order theory of arrays are common in program analysis, for example for finding loop invariants in programs using arrays~\cite{fase2009}. Previous versions of Vampire supported theories of integer arrays and arrays of integer arrays~\cite{Vampire13}. No other array sorts were supported and in order to implement one it would be necessary to hardcode a new sort and add the theory axioms corresponding to that sort. In this section we describe a polymorphic theory of arrays implemented in Vampire.

\subsection{Definition}
The polymorphic theory of arrays is the union of theories of arrays parametrised by two sorts: sort $\tau$ of indexes and sort $\sigma$ of values. It would have been proper to call these theories the theories of maps from $\tau$ to $\sigma$, however we decided to call them arrays for the sake of compatibility with arrays as defined in SMT-LIB.

A theory of arrays is a first-order theory that contains a sort
$\arrayt(\tau,\sigma)$, function symbols $\selectf :
\arrayt(\tau,\sigma) \times \tau \to \sigma$ and $\storef :
\arrayt(\tau,\sigma) \times \tau \times \sigma \to
\arrayt(\tau,\sigma)$, and three axioms.
The function symbol $\selectf$ represents a binary operation of
extracting an array element by its index.
The function symbol $\storef$ represents a ternary operation of updating an array at a given index with a given value. The array axioms are:
\begin{enumerate}
%  \item array congruence $$(\forall a:\arrayt(\tau,\sigma))(\forall i:\tau)(\forall j:\tau)(i \eql j \implies \select{a}{i} \eql \select{a}{j});$$
  \item read-over-write 1
        \begin{align*}
          &(\forall a:\arrayt(\tau,\sigma))(\forall v:\sigma)(\forall i:\tau)(\forall j:\tau)\\
          &\quad(i \eql j \implies \select{\store{a}{i}{v}}{j} \eql v);
        \end{align*}
  \item read-over-write 2
        \begin{align*}
          &(\forall a:\arrayt(\tau,\sigma))(\forall v:\sigma)(\forall i:\tau)(\forall j:\tau)\\
          &\quad(i \neql j \implies \select{\store{a}{i}{v}}{j} \eql \select{a}{j});
        \end{align*}
%  \item extensionality $$(\forall a:\arrayt(\tau,\sigma))(\forall b:\arrayt(\tau,\sigma))((\forall i:\tau)(\select{a}{i} \eql \select{b}{i}) \liff a\eql b).$$
  \item extensionality
        \begin{align*}
          &(\forall a:\arrayt(\tau,\sigma))(\forall b:\arrayt(\tau,\sigma))\\
          &\quad((\forall i:\tau)(\select{a}{i} \eql \select{b}{i}) \implies a\eql b).
        \end{align*}
\end{enumerate}
We will call every concrete instance of the theory of arrays for
concrete sorts $\tau$ and $\sigma$ the \emph{$(\tau,\sigma)$-instance}.

%One can use the polymorphic theory of arrays to express program properties. For example, the following formula expresses the fact that an integer array $a$ of size $n$ is sorted.
%\begin{equation}\label{eq:arrays-example}
%  (\forall i:\tau)(i \geq 0 \land i < n \implies \select{a}{i} \leq \select{a}{i + 1})
%\end{equation}

One can use the polymorphic theory of arrays to express program properties. Recall the code snippet involving arrays mentioned in Section~\ref{sect:fool}:
\begin{lstlisting}[language=cpp]
array[3] := 5;
array[2] + array[3];
\end{lstlisting}
Formula~\eqref{eq:let-function-example} used an interpreted function to represent the array in this code. We can alternatively use arrays to represent it as follows
\begin{equation}\label{eq:arrays-example}
\begin{aligned}
&\mathtt{let}\;\binding{\arrayt}{\store{\arrayt}{3}{5}}\;\mathtt{in}\\
&\quad\select{\arrayt}{2} + \select{\arrayt}{3}
\end{aligned}
\end{equation}

\subsection{Implementation in Vampire}

Vampire implements reasoning in the polymorphic theory of arrays by adding corresponding sorts axioms when the input uses array sorts and/or functions.

Whenever the input problem uses a sort $\arrayt(\tau,\sigma)$, Vampire adds this sort and function symbols $\selectf$ and $\storef$ of the types $\arrayt(\tau,\sigma) \times \tau \to \sigma$ and $\arrayt(\tau,\sigma) \times \tau \times \sigma \to \arrayt(\tau,\sigma)$, respectively.

If the input problem contains $\storef$, Vampire adds the following axioms for the sorts $\tau$ and $\sigma$ used in the corresponding array theory instance:

\begin{equation}\label{eq:array-axiom-1}
  \begin{aligned}
    &(\forall a:\arrayt(\tau,\sigma))(\forall i:\tau)(\forall v:\sigma)\\
    &\quad(\select{\store{a}{i}{v}}{i} \eql v)
  \end{aligned}
\end{equation}
\begin{equation}\label{eq:array-axiom-2}
  \begin{aligned}
    &(\forall a:\arrayt(\tau,\sigma))(\forall i:\tau)(\forall j:\tau)(\forall v:\sigma)\\
    &\quad(i \neql j \implies \select{\store{a}{i}{v}}{j} \eql \select{a}{j})
  \end{aligned}
\end{equation}
\begin{equation}\label{eq:array-axiom-3}
  \begin{aligned}
    &(\forall a:\arrayt(\tau,\sigma))(\forall b:\arrayt(\tau,\sigma))\\
    &\quad(a \not\eql b \implies (\exists i:\tau)(\select{a}{i} \neql \select{b}{i}))
  \end{aligned}
\end{equation}
These axioms are equivalent to the axioms read-over-write~1, read-over-write~2 and extensionality.

If the input contains only $\selectf$ but not $\storef$ for this instance, then only extensionality \eqref{eq:array-axiom-3} is added.

Theory axioms are not added when the \verb'--theory_axioms' option is set to \verb'off' (the default value is \verb'on'), which leaves an option for the user to try her or his own axiomatisation of arrays.

Vampire uses the extensionality resolution rule~\cite{ATVA14} to efficiently reason with the extensionality axiom.

To express arrays, the TPTP syntax extension supported by Vampire
allows, for every pair of sorts $\tau$ and $\sigma$, denoted by
\lstinline't' and \lstinline's' in the TFF0 syntax, to denote the sort
$\arrayt(\tau,\sigma)$ by \lstinline'$array(s, t)'. Function symbols $\selectf$
and $\storef$ can be expressed as ad-hoc polymorphic \dselect\ and
\texttt{\$store}, respectively for every pairs of sorts
$\tau,\sigma$. Previously,
the theories of integer arrays and arrays of integer arrays were
represented as sorts \darrayone\ and \darraytwo\ in Vampire,
with the corresponding sort-specific function symbols \dselectone, \dselecttwo, \dstoreone\  and
\dstoretwo. Our new implementation in Vampire, with
support for the polymorphic theory of arrays, deprecates these
two concrete array theories. Instead, one can now use the sorts
\lstinline'$array($int, $int)' and \lstinline'$array($int, $array($int, $int))'.
%The example~\eqref{eq:arrays-example} can be expressed in the extended TFF0 using the built-in theory of integers. Here \lstinline'$greatereq', \lstinline'$less' and \lstinline'$sum' represent operations of greater-or-equal comparison, less comparison and integer addition, respectively.
%\begin{lstlisting}
%![I : $int]: (($greatereq(i, 0) & $less(i, n)) =>
%                 $lesseq($select(a, i), $select(a, $sum(i, 1))))
%\end{lstlisting}
For example, formula~\eqref{eq:arrays-example} can be written in the extended TFF0 syntax as follows:
\begin{lstlisting}
$let(array := $store(array, 3, 5),
     $sum($select(array, 2), $select(array, 3))).
\end{lstlisting}%$

%------------------------------------------------------------------------------
\subsection{Theory of Boolean Arrays}

An interesting special case of the polymorphic theory of arrays is the theory of Boolean arrays. In that theory the $\selectf$ function has the type $\arrayt(\tau,\bool) \times \tau \to \bool$ and the $\storef$ function has the type $\arrayt(\tau,\bool) \times \tau \times \bool \to \arrayt(\tau,\bool)$. This means that applications of $\selectf$ can be used as formulas and $\storef$ can have a formula as the third argument.

Vampire implements the theory of Booleans arrays similarly to other sorts, by adding theory axioms when the option \verb'--theory_axioms' is enabled. However, the theory axioms are different for the following reason. The axioms of the theory of Boolean arrays are syntactically correct in FOOL but not in FOL, because they use quantification over Booleans. However, Vampire adds theory axioms only after a translation of FOOL to FOL. For this reason, Vampire uses the following set of axioms for Boolean arrays:
\begin{equation*}
  \begin{aligned}
    &(\forall a:\arrayt(\tau,\bool))(\forall i:\tau)(\forall v:\bool)\\
    &\quad(\select{\store{a}{i}{v}}{i} \liff (v \eql \true))
  \end{aligned}
\end{equation*}
\begin{equation*}
  \begin{aligned}
    &(\forall a:\arrayt(\tau,\bool))(\forall i:\tau)(\forall j:\tau)(\forall v:\bool)\\
    &\quad(i \neql j \implies \select{\store{a}{i}{v}}{j} \liff \select{a}{j})
  \end{aligned}
\end{equation*}
\begin{equation*}
  \begin{aligned}
    &(\forall a:\arrayt(\tau,\bool))(\forall b:\arrayt(\tau,\bool))\\
    &\quad(a \not\eql b \implies (\exists i:\tau)(\select{a}{i} \oplus \select{b}{i}))
  \end{aligned}
\end{equation*}
where $\oplus$ is the ``exclusive or'' connective.

\newcommand{\encrypt}{\mathit{encrypt}}
\newcommand{\key}{\mathit{key}}
\newcommand{\msg}{\mathit{message}}
\newcommand{\plaintext}{\mathit{plaintext}}
\newcommand{\cipher}{\mathit{cipher}}

One can use the theory of Boolean arrays, for example, to express properties of bit vectors. In the following example we give a formalisation of a basic property of XOR encryption, where the key, the message and the cipher are bit vectors. Let $\encrypt$ be a function of the type $\arrayt(\Z,\bool) \times \arrayt(\Z,\bool) \to \arrayt(\Z,\bool)$. We will write $\encrypt(\msg, \key)$ to denote the result of bit-wise application of the XOR operation to $\msg$ and $\key$. For simplicity we will assume that the message and the key are of equal length. The definition of $\encrypt$ can be expressed with the following axiom:
\begin{align*}
  &(\forall \msg:\arrayt(\Z,\bool))(\forall \key:\arrayt(\Z,\bool))(\forall i:\Z)\\
  &\quad(\select{\encrypt(\msg,\key)}{i} \eql \\
  &\quad\quad\select{\msg}{i} \oplus \select{\key}{i}).
\end{align*}

An important property of XOR encryption is its vulnerability to the known plaintext attack. It means that knowing a message and its cipher, one can obtain the key that was used to encrypt the message by encrypting the message with the cipher. This property can be expressed by the following formula.
\begin{align*}
  &(\forall \plaintext:\arrayt(\Z,\bool))(\forall \cipher:\arrayt(\Z,\bool))\\
  &\quad(\forall \key:\arrayt(\Z,\bool))(\cipher \eql \encrypt(\plaintext,\key) \implies\\
  &\quad\quad\key \eql \encrypt(\plaintext,\cipher))
\end{align*}

The sort $\arrayt(\Z,\bool)$ is represented in the extended TFF0 syntax as \darray{\dint}{\dbool}. The presented property of XOR encryption can be expressed in the extended TFF0 in the following way.
\begin{lstlisting}[language=tptp]
tff(encrypt, type, encrypt: ($array($int, $o) *
    $array($int, $o)) > $array($int, $o)).

tff(xor_encryption, axiom,
    ![Message: $array($int, $o),
      Key: $array($int, $o), I: $int]:
      ($select(encrypt(Message, Key), I) =
        ($select(Message, I) <~> $select(Key,I)))).

tff(known_plaintext_attack, conjecture,
    ![Plaintext: $array($int, $o),
      Cipher: $array($int, $o), Key: $array($int, $o)]:
        ((Cipher = encrypt(Plaintext, Key)) =>
          (Key = encrypt(Plaintext, Cipher)))).
\end{lstlisting}


%------------------------------------------------------------------------------
\section[Program Analysis with the~New~Extensions]{Program Analysis with\\the~New~Extensions}
\label{sect:example}

\begin{figure*}[b]
  \vspace{-1em}
  \begin{center}
    \parbox{0cm}{
      \begin{tabbing}
        res \ass\ x;\\
        \IF\ (x $>$ y) \\\inc
        \THEN\ max \ass\ x;\\
        \ELSE\ max \ass\ y;\\\dec
        \IF\ (max $>$ 0) \\\inc
        \THEN\ res \ass\ res $+$ max;\\
        \ELSE\ res \ass\ res $-$ max;\\[.5em]\dec
        \reserved{assert} res $\geq$ x
      \end{tabbing}
    }
  \end{center}
  \vspace{-2em}
  \caption{Sequence of conditionals.\label{fig:seqITE}}
\end{figure*}

\begin{figure*}[tb]
{\small
\begin{lstlisting}[language=tptp]
tff(x, type, x: $int).
tff(y, type, y: $int).
tff(max, type, max: $int).
tff(res, type, res: $int).
tff(res1, type, res1: $int).

tff(transition_relation, hypothesis,
    res1 = $let(res := x,
           $let(max := $ite($greater(x, y),
                            $let(max := x, max),
                            $let(max := y, max)),
           $let(res := $ite($greater(max, 0),
                            $let(res := $sum(res, max), res),
                            $let(res := $difference(res, max),
                                 res)),
                res)))).

tff(safety_property, conjecture, $greatereq(res1, x)).
\end{lstlisting}
\caption{Representation of the partial correctness statement of the code on Figure~\ref{fig:seqITE} in Vampire\label{fig:VampireITE}.}
}
\end{figure*}

In this section we illustrate how FOOL makes first-order theorem
provers better suited to applications in program analysis and
verification.
Firstly,  we give concrete examples of the use of FOOL for
expressing program properties. We avoid various
program analysis steps, such as SSA form computations and renaming
program variables; instead we show how program properties can directly
be expressed in FOOL.
 We also present a technique for
automatically generating the next state relation of any program with
assignments, \ITE, and sequential composition.
For doing so,  we introduce a simple extension of FOOL,
allowing for a general translation that is linear in the size of the
program.
This is a new result intended to understand which extensions of
first-order logic are adequate for naturally representing fragments of
imperative programs.

%%%%%%%%%%%%%%%%%%%%%%%%%%%%%%%%%%%%%%%%%%%%%%%%
\subsection{Encoding the Next State Relation}\label{sec:foolp}

Consider the program given in
Figure~\ref{fig:seqITE}, written in a C-like syntax, using a sequence
of two conditional statements.
The program first computes the maximal value $\mathit{max}$ of two integers $x$ and
$y$ and then adds the absolute value of $\mathit{max}$ to $x$. A safety assertion,
in FOL, is specified at the end of the loop, using the
{\bf assert} construct. This program is clearly safe, the
assertion is satisfied. To prove program safety, one needs to reason
about the program's transition relation, in particular reason about
conditional statements, and express the final value of
the program variable $\mathit{res}$. The partial correctness of the program
of Figure~\ref{fig:seqITE} can be \emph{automatically} expressed in FOOL,
and then Vampire can be used to prove program safety.
This requires us to encode (i)
the next state value of $\mathit{res}$ (and $\mathit{max}$) as a hypothesis
in the extended TFF0 syntax of FOOL,
by using the \ITE\ ({\tt \$ite}) and \LETIN\ ({\tt \$let})
constructs, and (ii)
the safety property as the conjecture to be proven by Vampire.

Figure~\ref{fig:VampireITE} shows this extended TFF0 encoding.
The use of \ITE\ and \LETIN\ constructs allows us to have a
direct  encoding of the  transition relation of
Figure~\ref{fig:seqITE} in FOOL. Note that each expression from the program appears only once in the encoding.

We now explain how the encoding of the next state values of program
variables can be generated automatically.
We consider programs using assignments \texttt{:=},
\ITE\ and sequential composition $;$.
We begin by making an assumption about the structure of programs (which we relax later). A program $P$ is in \emph{restricted form} if for any subprogram of the form \IF\ e \THEN\ $P_1$ \ELSE\ $P_2$ the subprograms $P_1$ and $P_2$ only make assignments to the same single variable. Given a program $P$ in restricted form let us define its translation $[P]$ inductively as follows:
%
\begin{itemize}
	\item $[$x\ASS e$]$ is $\letin{x}{e}{x}$;
	\item $[$\IF\ e \THEN\ $P_1$ \ELSE\ $P_2]$, where $P_1$ and
          $P_2$ update $x$,  is $\letin{x}{\ite{e}{[P_1]}{[P_2]}}{x}$;
	\item $[P_1$;$P_2]$ is $\mathtt{let}~D~\mathtt{in}~[P_2]$ where $[P_1]$ is $\mathtt{let}~D~\mathtt{in}~x$.
\end{itemize}
%
Given a program $P$, the next state value for variable $x$ can be
given by $[P$; x\ASS x$]$,
i.e. by ensuring the final statement of the program updates the
variable of interest.
The restricted form is required as conditionals must be viewed
as assignments in the translation and assignments can only be made to single variables.

\begin{figure*}[tb]
  \vspace{-1em}
  \begin{center}
    \parbox{0cm}{
      \begin{tabbing}
        \IF\ (x $>$ y) \\\inc
        \THEN\ t \ass\ x; x \ass\ y; y \ass\ t;\\\dec
        \reserved{assert} y $\geq$ x
      \end{tabbing}
    }
  \end{center}
  \vspace*{-2em}
  \caption{Updating multiple variables.\label{fig:tmpSwap}}
\end{figure*}

To demonstrate the limitations of this restriction let us consider the simple program in Figure~\ref{fig:tmpSwap} that ensures that x is not larger than y. We cannot apply the translation as the conditional updates three variables. To generalise the approach we can extend FOOL with \emph{tuple expressions}, let us call this extension \foolp. In this extended logic the next state values for Figure~\ref{fig:tmpSwap} can be encoded as follows:
%
\[
\begin{array}{ll}
\mathtt{let}\; (x,y,t) =& \mathtt{if}\; x > y \;\mathtt{then} \\
&\quad\mathtt{let}\; (x,y,t) = (x,y,x) \;\mathtt{in}\\
&\quad\quad\mathtt{let}\; (x,y,t) = (y,y,t) \;\mathtt{in}\\
&\quad\quad\quad\mathtt{let}\;(x,y,t) = (x,t,t)  \;\mathtt{in}\; (x,y,t) \\
&\mathtt{else}\; (x,y,t)\\
\mathtt{in}\; (x,y,t) \\
\end{array}
\]
%
We now give a brief sketch of the extended logic \foolp\ and the associated translation. We omit details since its full definition and semantics would require essentially repeating definitions from \cite{FOOL}.  \foolp\ extends \fool\ by tuples; for all expressions $t_i$ of type $\sigma_i$ we can use a \emph{tuple expression} $(t_1,\ldots,t_n)$ of type $(\sigma_1,\ldots,\sigma_n)$. The logic should also include a suitable tuple projection function, which we do not discuss here.

This extension allows for a more general translation in two senses:
first, the previous restricted form is lifted; and second, it now
gives the next state values of {\it all} variables updated by the program. Given a program $P$ its translation $[P]$ will have the form $\letin{(x_1,\ldots,x_n)}{E}{(x_1,\ldots,x_n)}$, where $x_1,\ldots,x_n$ are all variables updated by $P$, that is, all variables used in the left-hand-side of an assignment. We inductively define $[P]$ as follows:
\begin{itemize}
	\item $[\text{x}_i \text{\ASS} e]$ is $\letin{(\ldots,x_i,\ldots)}{(\ldots,e,\ldots)}{(x_1,\ldots,x_n)}$,
	\item $[$\IF\ e \THEN\ $P_1$ \ELSE\ $P_2]$ is $\letin{(x_1,\ldots,x_n)}{\ite{e}{[P_1]}{[P_2]}}{(x_1,\ldots,x_n)},$
	\item $[P_1$;$P_2]$ is $\mathtt{let}~D~\mathtt{in}~[P_2]$ where $[P_1]$ is $\mathtt{let}~D~\mathtt{in}~(x_1,\ldots,x_n)$.
\end{itemize}
This translation is bounded by $O(v\cdot n)$, where $v$ is the number
of variables in the program and $n$ is the program size (number of
statements) as each program statement is used once with one or two
instances of $(x_1,\ldots,x_n)$.
This becomes $O(n)$ if we assume that the number of
variables is fixed. The translation could be refined so that some introduced  \LETIN\
expressions only use a subset of program variables.
Finally, this translation preserves the semantics
of the program.

\begin{theorem}\rm
  Let $P$ be a program with variables $(x_1,\ldots,x_n)$ and let $u_1,\ldots,u_n, v_1, \ldots, v_n$ be values (where $u_i$ and $v_i$ are of the same type as $x_i$). If $P$ changes the state $\{x_1\to u_1,\ldots,x_n\to u_n\}$ to $\{x_1\to v_1,\ldots,x_n\to v_n\}$ then the value of $[P]$ in $\{x_1\to u_1,\ldots,x_n\to u_n\}$ is $(v_1,\ldots,v_n)$.
\end{theorem}

This translation encodes the next state values of program variables by
directly following the structure of the program. This leads to a
succinct representation that, importantly, does not lose any
information or attempt to translate the program too early. This allows
the theorem prover to apply its own translation to FOL that it can
handle efficiently.   While \foolp{} is not yet fully supported in
Vampire, we believe experimenting with \foolp{} on
examples coming from program analysis and verification is an
interesting task for future work.


%%%%%%%%%%%%%%%%%%%%%%%%%%%%%%%%%%%%%%%%%%%%%%%%
\subsection{A Program with a Loop and Arrays}

\begin{figure*}[bt]
{
  \begin{center}
    \parbox{0cm}{
  \begin{tabbing}
    $a$ \ass\ $0$; $b$ \ass\ $0$; $c$ \ass\ $0$; \\[.5em]
    \reserved{invariant} a = b + c $\wedge$ \\
    {\color{white}\reserved{invariant}} a $\geq$ 0 $\wedge$ b $\geq$ 0 $\wedge$ c $\geq$
    0 $\wedge$ a $\leq$ k $\wedge$ \\
    {\color{white}\reserved{invariant}} $(\forall p) (0\leq p<b \implies
    (\exists i) (0 \leq i < a \wedge A[i] > 0 \wedge B[p] = A[i]))$\\[1em]
    \WHILE\ ($a \leq k$) \DO \\ \inc
      \IF\ ($A[a] > 0$) \\ \inc
        \THEN\ \=\+ $B[b]$ \ass\ $A[a]$\semicol $b$ \ass\ $b+1$\semicol \\ \dec
        \ELSE\ \=\+ $C[c]$ \ass\ $A[a]$\semicol $c$ \ass\ $c+1$\semicol \\ \dec \dec
      $a$ \ass\ $a+1$\semicol \\ \dec
    \OD\\[.5em]
    \reserved{assert} $(\forall p)(0 \leq p<b \implies B[p]> 0)$
  \end{tabbing}
    }
  \end{center}
  \caption{Array partition.\label{fig:partition}}
}
\end{figure*}

\begin{figure*}[tb]
\begin{lstlisting}[language=tptp]
tff(a, type, a: $int).
tff(b, type, b: $int).
tff(c, type, c: $int).
tff(k, type, k: $int).
tff(arrayA, type, arrayA: $array($int, $int)).
tff(arrayB, type, arrayB: $array($int, $int)).
tff(arrayC, type, arrayC: $array($int, $int)).

tff(invariant_property, hypothesis, inv <=>
    ((a = $sum(b, c)) &
     $greatereq(a, 0) & $greatereq(b, 0) &
     $greatereq(c, 0) & $lesseq(a, k) &
     ![P: $int]: ($lesseq(0, P) & $less(P, b) =>
       (?[I: $int]: ($lesseq(0, I) & $less(I, a) &
          $greater($select(arrayA, I), 0) &
          $select(arrayB, P) = $select(arrayA, I)))))).

tff(safety_property, conjecture,
    (inv & ~$lesseq(a, k)) =>
      (![P: $int]: ($lesseq(0, P) & $less(P, b) =>
                    $greater($select(arrayB, P), 0)))).
\end{lstlisting}
\caption{Representation of the partial correctness statement of the code on Figure~\ref{fig:partition} in Vampire\label{fig:loop_safety_Vampire}.}
\end{figure*}

Let us now show the use of FOOL in Vampire for reasoning about
programs with loops. Consider the program given in
Figure~\ref{fig:partition}, written in a C-like syntax.  The program
fills an integer-valued array $B$ by the strictly positive values
of a source array $A$, and an integer-valued array $C$ with
the non-positive values of $A$. A safety assertion, in FOL, is
specified at the end of the loop, using the {\bf assert}
construct. The program of Figure~\ref{fig:partition} is clearly safe
as the assertion is satisfied when the loop is exited.
However, to prove program safety we need additional
loop properties, that is loop invariants, that hold at any loop
iteration. These can be automatically generated using existing approaches, for
example the symbol elimination method for invariant generation in
Vampire~\cite{fase2009}. In this case we use the FOL property
specified in the {\bf invariant} construct of  Figure~\ref{fig:partition}. This invariant
property states that at any loop iteration, (i) the  sum of visited
array elements in $A$ is the sum of visited elements in $B$ and $C$
(that is, $a = b + c$), (ii) the number of visited array
elements in $A$, $B$, $C$ is positive (that is, $a\geq 0$, $b\geq 0$,
and $c\geq 0$), with $a\leq k$, and (iii) each array element
$B[0],\ldots,B[b-1]$ is a strictly positive element in
$A$. Formulating the latter property requires quantifier alternation
in FOL, resulting in the quantified property with $\forall\exists$
listed in the invariant of  Figure~\ref{fig:partition}.
We can verify the safety of the program using Hoare-style reasoning in Vampire.
The partial correctness property is that the invariant and the negation of the loop condition implies the safety assertion.
This is the conjecture to be proven by Vampire.
Figure~\ref{fig:loop_safety_Vampire} shows the encoding in the
extended TFF0 syntax of this partial
correctness statement; note that this uses the built-in theory of
polymorphic arrays in Vampire, where $arrayA$, $arrayB$ and $arrayC$
correspond respectively to the arrays $A$, $B$ and $C$.

So far, we assumed that the given invariant in
Figure~\ref{fig:partition} is
indeed an invariant. Using \foolp{} described in
Section~\ref{sec:foolp}, we can verify the inductiveness
property of the invariant, as follows: (i) express the the transition
relation of the loop in \foolp, and (ii) prove that, if the invariant
holds at an arbitrary loop iteration $i$, then it also holds at loop
iteration $i+1$. For proving this, we can again use \foolp\ to
formulate
the next state values of loop variables in the invariant at loop
iteration $i+1$.
Moreover, \foolp{} can also be used to express formulas as
inputs to the symbol elimination method for invariant generation in
Vampire. We leave the task of using \foolp{} for invariant generation
as further work.


%------------------------------------------------------------------------------
\section{Experimental Results}
\label{sect:experiments}

% !TEX root = ../main.tex
The extension of Vampire to support FOOL and the polymorphic theory of
arrays comprises about 3,100 lines of C++ code, of which the
translation of FOOL to FOL and FOOL paramodulation takes about 2,000
lines, changes in the parser about 500 lines and
the implementation of the polymorphic theory of arrays about 600 lines.
Our implementation is available at \url{www.cse.chalmers.se/~evgenyk/fool-experiments/} and will be included
in the forthcoming official release of Vampire.

In the sequel, by Vampire we mean its version including support for
FOOL and the polymorphic theory of arrays. We write \nofoolVampire\ for
its version with FOOL paramodulation turned off.

In this section we present experimental results obtained by running Vampire on FOOL problems. Unfortunately, no large collections of such problems are available, because FOOL was not so far supported by any first-order theorem prover. What we did was to extract such benchmarks from other collections.

\begin{enumerate}
\item We noted that many problems in the higher-order part of the TPTP library~\cite{TPTP} are FOOL problems, containing no real higher-order features. We converted them to FOOL problems.

\item We used a collection of first-order problems about (co)al\-ge\-braic datatypes, generated by the Isabelle theorem prover~\cite{Isabelle}, see Subsection~\ref{subsec:Isabelle} for more details.
\end{enumerate}
Our results are summarised in Tables~\ref{table:thf-results}--\ref{table:smt-lib-nontrivial} and discussed below. These results were obtained on a MacBook Pro with a 2,9 GHz Intel Core i5 and 8 Gb RAM, and using the time limit of 60 seconds per problem. Both the benchmarks and the results are available at \url{www.cse.chalmers.se/~evgenyk/fool-experiments/}.

\subsection{Experiments with TPTP Problems}
The higher-order part of the TPTP library contains 3036 problems. Among these problems, 134 contain either boolean arguments in function applications or quantification over booleans, but contain no lambda abstraction, higher-order sorts or higher-order equality. We used these 134 problems, since they belong to FOOL but not to FOL. We translated these problems from THF0 to the modification of TFF0, supported by Vampire using the following syntactic transformation: \begin{enumerate*}[label=(\alph*)]
\item every occurrence of the keyword \lstinline'thf' was replaced by \lstinline'tff';
\item every occurrence of a sort definition of the form \lstinline's_1 >  ... > s_n > s' was replaced by \lstinline's_1 * ... * s_n > s';
\item every occurrence of a function application of the form \lstinline'f @  t_1 @ ... @ t_n' was replaced by \lstinline'f(t_1, ..., t_n)'.
\end{enumerate*}

Out of 134 problems, 123 were marked as Theorem and 5 as
Unsatisfiable, 5 as CounterSatisfiable, and 1 as Satisfiable, using
the SZS status of TPTP. Essentially, this means that among their
satisfiability-checking analogues, 128 are unsatisfiable and 6 are
satisfiable. Vampire was run with the \verb'--mode casc' option for
unsatisfiable (Theorem and Unsatisfiable) problems and with \verb'--mode casc_sat' for satisfiable (CounterSatisfiable and Satisfiable) problems. These options correspond to the CASC competition modes of
Vampire for respectively proving validity (i.e. unsatisfiability) and
satisfiability of an input problem.

For this experiment, we compared the performance of Vampire with those of the higher-order theorem provers used in the the latest edition of CASC \cite{CASC25}:
Satallax~\cite{Satallax}, Leo-II~\cite{LeoII}, and Isabelle~\cite{Isabelle}. We note that all of them used the first-order theorem prover E~\cite{E13} for first-order reasoning (Isabelle also used several other provers).

\begin{table}[t]
  \caption{Runtimes in seconds of provers on the set of 134 higher-order TPTP problems.}
  \begin{center}
  \begin{tabular}{lrr}
    \hline Prover & Solved & Total time on solved problems \\ \hline
    Vampire & 134 & 3.59 \\
    \nofoolVampire & 134 & 7.28 \\
    Satallax & 134 & 23.93 \\
    Leo-II & 127 & 27.42 \\
    Isabelle & 128 & 893.80
  \end{tabular}
  \end{center}
  \label{table:thf-results}
\end{table}

Table~\ref{table:thf-results} summarises our results on these problems. Only Vampire, \nofoolVampire\ and Satallax were able to solve all of them, while
Vampire was the fastest among all provers. We believe these results
are significant for two reasons. First, for solving these problems
previously one 
needed higher-order theorem provers, but now can they be proven using first-order reasoners. Moreover, even on such simple problems there is a clear gain from using FOOL paramodulation.

% \LK{should we add table with some non-trivial HOL problems?}
% AV: no, they are all trivial, alas

\subsection[Experiments with Algebraic Datatypes Problems]{Experiments with\\Algebraic Datatypes Problems}
\label{subsec:Isabelle}

For this experiment, we used 152 problems generated by the Isabelle theorem prover. These
problems express various properties of (co)algebraic datatypes and are written in the SMT-LIB~2 syntax~\cite{SMT-LIB}. All 152 problems contain quantification over booleans, boolean arguments in function/predicate applications and \ITE\ expressions. These examples were generated and given to us by Jasmin Blanchette, following the recent work on reasoning about (co)datatypes~\cite{Blanchette15}. To run the benchmark we first translated the SMT-LIB files to the TPTP syntax using the SMTtoTPTP translator~\cite{SMTLIB2TPTP} version 0.9.2.
Let us note that this version of SMTtoTPTP does not fully support the
boolean type in SMT-LIB. However, by setting the option
\verb'--keepBool' in SMTtoTPTP, we managed to translate these 152
problems into an extension of TFF0, which Vampire can read.
We also modified the source code of  SMTtoTPTP so that  \ITE\
expressions in the SMT-LIB files are not expanded but translated to \dite\
in FOOL. A similar modification would have been needed for translating
\LETIN\ expressions; however, none of our 152 examples used \LETIN.

After translating these 152 problems into an extended TFF0 syntax
supporting FOOL, we ran Vampire twice on each benchmark: once using the
option \verb'--mode casc', and once using
\verb'--mode' \verb'casc_sat'.  For each problem, we recorded the
fastest successful run of Vampire. We used a similar setting for
evaluating \nofoolVampire.
In this experiment, we then compared Vampire with
the best available SMT solvers, namely with CVC4~\cite{CVC4} and
Z3~\cite{Z3}.

\begin{table}[tb]
  \caption{Runtimes in seconds of provers on the set of 152 algebraic datatypes problems.}%  \nofoolVampire\ denotes Vampire with disabled FOOL paramodulation.}
  \begin{center}
  \begin{tabular}{lrr}
    \hline Prover & Solved & Total time on solved problems \\ \hline
    Vampire & 59 & 26.580 \\
    Z3 & 57 & 4.291 \\
    \nofoolVampire & 56 & 26.095 \\
    CVC4 & 53 & 25.480
  \end{tabular}
  \end{center}
  \label{table:smt-lib-results}
\end{table}

Table~\ref{table:smt-lib-results} summarises the results of our experiments on these 152 problems. Vampire solved the largest number of problems, and all problems solved by \nofoolVampire\ were also solved by Vampire.
Figure~\ref{fig:smt-lib-diagram} shows the Venn diagram of the sets of
problems solved by Vampire, CVC4 and Z3, where the numbers denote the numbers of solved problems.
All problems apart from 11 were either solved by all systems or not solved by all systems. Table~\ref{table:smt-lib-nontrivial} details performance results on these 11 problems.

\begin{figure}[tb]
  %\vspace{-0.3em}
  \centering
  \begin{tikzpicture}
    \draw (0,0) circle (1.5cm);
    \draw (50:1cm) circle (1.55cm);
    \draw (0cm:0.8cm) circle (1.4cm);
    \node at (0.8cm:0.5cm) {$51$}; %
    \node at (-2.2cm:1.2cm) {$1$};
    \node at (1cm:-1.1cm) {$2$};
    \node at (-2cm:-1.1cm) {$3$};
    \node at (-3.8cm:-1.9cm) {$4$}; %
    \node at (-0.95cm:1.775cm) {$0$}; %
    \node at (0.45cm:1.775cm) {$1$}; %
    \node at (2.7cm:2.65cm) {Vampire};
    \node at (-3.7cm:1.8cm) {Z3};
    \node at (-1.6cm:2.3cm) {CVC4};
  \end{tikzpicture}
  \vspace{-0.3em}
  \caption{Venn diagram of the subsets of the algebraic datatypes problems, solved by Vampire, CVC4 and Z3.}
  \label{fig:smt-lib-diagram}
\end{figure}

\newcommand{\timeout}{---}
\newcommand{\gaveup}{---}
\begin{table}[tb]
  \caption{Runtimes in seconds of provers on selected algebraic datatypes problems. Dashes mean the solver failed to find a solution.}% \nofoolVampire\ denotes Vampire with disabled FOOL paramodulation.}
  \begin{center}
  \begin{tabular}{lrrr}
    \hline Problem & Vampire & CVC4 & Z3 \\ \hline
    \verb'afp/abstract_completeness/1830522' & \timeout & \timeout & 0.172 \\
    \verb'afp/bindag/2193162' & \timeout & \gaveup & 0.388 \\
    \verb'afp/coinductive_stream/2123602' & \timeout & 0.373 & 0.101 \\
    \verb'afp/coinductive_stream/2418361' & 3.392 & \timeout & \timeout \\
    \verb'afp/huffman/1811490' & 0.023 & \gaveup & \timeout \\
    \verb'afp/huffman/1894268' & 0.025 & \gaveup & 0.052 \\
    \verb'distro/gram_lang/3158791' & 0.047 & 0.179 & \timeout \\
    \verb'distro/koenig/1759255' & 0.070 & \timeout & \timeout \\
    \verb'distro/rbt_impl/1721121' & 4.523 & \timeout & \timeout \\
    \verb'distro/rbt_impl/2522528' & 0.853 & \gaveup & 0.064 \\
    \verb'gandl/bird_bnf/1920088' & 0.037 & \timeout & 0.077
  \end{tabular}
  \end{center}
  \label{table:smt-lib-nontrivial}
\end{table}

Based on our experimental results shown in Tables~\ref{table:smt-lib-results} and \ref{table:smt-lib-nontrivial}, we make the following observations. On the given set of problems the implementation of FOOL reasoning in Vampire was efficient enough to compete with state-of-the-art SMT solvers. This is significant because the problems were tailored for SMT reasoning. Vampire not only solved the largest number of problems, but also yielded runtime results that are comparable with those of CVC4. Whenever successful, Z3 turned out to be faster than Vampire; we believe this is because of the sophisticated preprocessing steps in Z3. Improving FOOL preprocessing in Vampire, for example for more efficient CNF translation of FOOL formulas, is an interesting task for further research. We note that the usage of FOOL paramodulation showed improvement.

%------------------------------------------------------------------------------
\section{Related Work}
\label{sect:related}

FOOL was introduced in our previous work~\cite{FOOL}. This also presented a translation from FOOL to the ordinary first-order logic, and FOOL paramodulation. In this paper we describe the first practical implementation of FOOL and FOOL paramodulation.

Superposition theorem proving in finite domains, such as the boolean domain, is also discussed in~\cite{HillenbrandWeidenbach13}. The approach of~\cite{HillenbrandWeidenbach13} sometimes falls back to enumerating instances of a clause by instantiating finite domain variables with all elements of the corresponding domains. Nevertheless, it allows one to also handle finite domains with more than two elements. One can also generalise our approach to arbitrary finite domains by using binary encodings of finite domains. However, this will necessarily result in loss of efficiency, since a single variable over a domain with $2^k$ elements will become $k$ variables in our approach, and similarly for function arguments.
Although \cite{HillenbrandWeidenbach13} reports preliminary results with the theorem prover SPASS, we could not make an experimental comparison since the SPASS implementation has not yet been made public.

Handling boolean terms as formulas is common in the SMT community. The SMT-LIB project~\cite{SMT-LIB} defines its core logic as first-order logic extended with the distinguished first-class boolean sort and the \LETIN\ expression used for local bindings of variables. The language of FOOL extends the SMT-LIB core language with local function definitions, using \LETIN\ expressions defining functions of arbitrary, and not just zero, arity.

A recent work \cite{SMTLIB2TPTP} presents SMTtoTPTP, a translator from SMT-LIB to TPTP. SMTtoTPTP does not fully support boolean sort, however one can use SMTtoTPTP with the \verb'--keepBool' option to translate SMT-LIB problems to the extended TFF0 syntax, supported by Vampire.

Our implementation of the polymorphic theory of arrays uses a syntax that coincides with the TPTP's own syntax for polymorphically typed first-order logic TFF1~\cite{tff1}.


%------------------------------------------------------------------------------
\section{Conclusion and Future Work}
\label{sect:future}

% !TEX root = ../main.tex
We presented new features recently implemented in Vampire. They include FOOL: the extension of first-order logic by a first-class boolean sort, \ITE\ and \LETIN\ expressions, and polymorphic arrays. Vampire implements FOOL by translating FOOL formulas into FOL formulas. We described how this translation is done for each of the new features. Furthermore, we described a modification of the superposition calculus by FOOL paramodulation that makes Vampire reasoning in FOOL more efficient. 
We also gave a simple extension to FOOL that allows one to express the next state relation of a program as a boolean formula which is linear in the size of the program.

Neither FOOL nor polymorphic arrays can be expressed in TFF0. In order to support them Vampire uses a modification of the TFF0 syntax with the following features:

\begin{enumerate}
  \item the boolean sort \tptpo\ can be used as the sort of arguments and quantifiers;
  \item boolean variables can be used as formulas, and formulas can be used as boolean arguments;
  \item \ITE\ expressions are represented using a single keyword \dite\ rather than two different keywords \ditet\ and \ditef;
  \item \LETIN\ expressions are represented using a single keyword \dlet\ rather than four different keywords \dlettt, \mbox{\dlettf,} \dletft\ and \dletff;
  \item \darraySymb, \dselect\ and \dstore\ are used to represent arrays of arbitrary types.
\end{enumerate}

Our experimental results have shown that our implementation, and especially FOOL paramodulation, are efficient and can be used to solve hard problems.

Many program analysis problems, problems used in the SMT community, and problems generated by interactive provers, which previously required (sometimes complex) ad hoc translations to first-order logic, can now be understood by Vampire without any translation. Furthermore, Vampire can be used to translate them to the standard TPTP without \ITE\ and \LETIN\ expressions, that is, the format understood by essentially all modern first-order theorem provers and used at recent CASC competitions. One should simply use \texttt{--mode preprocess} and Vampire will output the translated problem to \texttt{stdout} in the TPTP syntax. 

The translation to FOL described here is only the first step to the efficient handling of FOOL. It can be considerably improved. For example, the translation of \LETIN\ expressions always introduces a fresh function symbol together with a definition for it, whereas in some cases inlining the function would produce smaller clauses. Development of a better translation of FOOL is an important future work.

FOOL can be regarded as the smallest superset of the SMT-LIB~2 Core language and TFF0. A native implementation of an SMT-LIB parser in Vampire is an interesting future work. Note that such an implementation can also be used to translate SMT-LIB to FOOL or to FOL.

Another interesting future work is extending FOOL to handle polymorphism and implementing it in Vampire. This would allow us to parse and prove problems expressed in the TFF1~\cite{tff1} syntax. Note that the current usage of \darraySymb\ conforms with the TFF1 syntax for type constructors.


\section*{Acknowledgements}
%\acks
We acknowledge funding from the Austrian FWF National Research Network RiSE
S11409-N23, the Swedish VR grant D049770~--- GenPro, the
Wallenberg Academy Fellowship 2014, and the EPSRC grant ``Reasoning in Verification and Security''.

\def\paperThreeContentsTitle{A Clausal Normal Form Translation for \folb{}}
\def\paperThreeChapterTitle{A Clausal Normal Form\\Translation for \folb{}}
\def\paperThreeAuthors{Evgenii~Kotelnikov, Laura~Kov\'{a}cs,\\Martin~Suda and Andrei~Voronkov}
\def\paperThreeAbstract{Automated theorem provers for first-order logic usually operate on sets of first-order clauses. It is well-known that the translation of a formula in full first-order logic to a clausal normal form (CNF) can crucially affect performance of a theorem prover. In our recent work we introduced a modification of first-order logic extended by the first class boolean sort and syntactical constructs that mirror features of programming languages. We called this logic FOOL. Formulas in FOOL can be translated to ordinary first-order formulas and checked by first-order theorem provers. While this translation is straightforward, it does not result in a CNF that can be efficiently handled by state-of-the-art theorem provers which use superposition calculus. In this paper we present a new CNF translation algorithm for FOOL that is friendly and efficient for superposition-based first-order provers. We implemented the algorithm in the Vampire theorem prover and evaluated it on a large number of problems coming from formalisation of mathematics and program analysis. Our experimental results show an increase of performance of the prover with our CNF translation compared to the naive translation.}
\def\paperThreePublication{Published in the \emph{Proceedings of the 2nd Global Conference on Artificial Intelligence}, pages 53--71. EPiC Series in Computing, 2016.}
\paperchapter{\paperThreeContentsTitle}
             {\paperThreeChapterTitle}
             {\paperThreeAuthors}
             {\paperThreeAbstract}
             {\paperThreePublication}
\label{chap:cnf}
% !TEX root = ../main.tex

% \author{
% Evgenii Kotelnikov\inst{1}
% \and
% Laura Kov\'acs\inst{1,2}
% \and
% Martin Suda\inst{2}
% \and
% Andrei Voronkov\inst{1,3,4}%
% }

% \institute{
% Chalmers University of Technology, Gothenburg, Sweden
% %\\\email{evgenyk@chalmers.se}
% \and
% TU Wien, Vienna
% %\\ \email{msuda@forsyte.at, lkovacs@forsyte.at}
% \and
% The University of Manchester
% \and EasyChair
% %\\ \email{andrei@voronkov.com}
% }

% \authorrunning{Kotelnikov, Kov\'acs, Suda, Voronkov}

\section{Introduction}
\label{sec:newcnf/introduction}
% !TEX root = main.tex


% Formal verification and analysis of software heavily use theorem provers for
% various logics to check properties of programs.  
% Among the main methods for proving software correctness or deriving logical explanations for faulty software behaviour is SAT and SMT solving~\cite{Z3,CVC4}, automated and interactive theorem proving~\cite{Vampire13,Isabelle}. These methods are inter-related
% and modern program analysis and verification tools often use a combination of them.

% Nearly all modern automated theorem provers first translate formulas into a clausal normal form (CNF), and then perform reasoning on clauses.
%
%
%Automated theorem provers for first-order logic usually operate on sets of first-order clauses. In order to check a formula in full first-order logic, theorem provers first translate it to clausal normal form (CNF). 
%
% CNF translation affects the performance of a theorem prover. While there is no
% absolute criterion of what the best CNF for a formula is, theorem provers
% usually try to make the CNF smaller according to some measure. This measure can
% include the number of clauses, the number of literals, the lengths of the
% clauses and the size of the resulting signature, i.e.~the number of function and
% predicate symbols. Implementors of CNF translations commonly employ formula
% simplification~\cite{nonnengart2001computing}, (generalised) formula
% naming~\cite{nonnengart2001computing,azmy2013computing}, and other
% clausification techniques, aimed to make the CNF smaller.

Automated theorem provers for first-order logic usually operate on sets of first-order clauses. In order to check a formula in full first-order logic, theorem provers first translate it to clausal normal form (CNF). It is well-known that the quality of this translation affects the performance of the theorem prover. While there is no absolute criterion of what the best CNF for a formula is, theorem provers usually try to make the CNF smaller according to some measure. This measure can include the number of clauses, the number of literals, the lengths of the clauses and the size of the resulting signature, i.e.~the number of function and predicate symbols. Implementors of CNF translations commonly employ formula simplification~\cite{nonnengart2001computing}, (generalised) formula naming~\cite{nonnengart2001computing,azmy2013computing}, and other clausification techniques, aimed to make the CNF smaller.

Our recent work~\cite{FOOL} presented a modification of many-sorted first-order logic with first-class boolean sort. We called this logic \folb{}, standing for first-order logic (FOL) with boolean sort. \folb{} extends standard FOL by (i) treating boolean terms as formulas, (ii) \ITE\ expressions, (iii) \LETIN\ expressions, and (iv) tuple expressions. While \ITE\ and \LETIN\ expressions are also available in the SMT-LIB core language~\cite{BarFT-SMTLIB}, the standard input language for SMT solvers, FOOL is a strict superset of SMT-LIB as tuple expressions are not part of SMT-LIB and \LETIN\ expressions in FOOL can define non-constant functions and predicate symbols. 

There is a model-preserving translation of \folb{} formulas to FOL (see \cite{FOOL})
that works by replacing parts of a \folb{} formula with applications of fresh function and predicate symbols and extending the set of assumptions with definitions of these symbols.
% We implemented this translation in the Vampire theorem prover~\cite{Vampire13}. 
To reason about a FOOL formula, one can thus 
%To check a \folb{} problem Vampire 
first translate it to a FOL formula and then convert the FOL formula into a set of clauses
using the usual first-order clausification techniques. 
While this translation provides an easy way to support \folb{} in existing first-order provers,
it is not necessarily efficient.
A more efficient translation can convert a \folb{} formula directly to a set of first-order clauses, skipping the intermediate step of converting FOOL to FOL. This way, the translation can integrate known clausification techniques and improve the quality of the resulting clausal normal form. 

In this paper  we present a new clausification algorithm, called \nfcnf{},  that translates a \folb{} formula to an equisatisfiable set of first-order clauses. 
Our algorithm 
avoids producing large numbers  of duplicate clauses and new symbols during clausification and 
also avoids clauses that can make theorem provers inefficient.
We show that in practice this leads to a significant increase in the performance of a theorem prover).
%that implements it compared e.g. to the use of the translation to full first-order logic from~\cite{FOOL}.

Our \nfcnf{} algorithm  is a non-trivial  extension of the recent \newcnf{} clausification algorithm for FOL~\cite{newcnf_fol}. The extension employs several clausification techniques for handling the non-FOL features of FOOL, namely boolean terms and \ITE, \LETIN\ and tuple expressions. These techniques comprise the contributions of this work and are listed below.

% To this end,
% we (i) skolemise boolean variables using predicates and not functions and thus avoid introducing boolean equalities, 
% (ii) name common subexpressions of FOOL formulas by using guards, 
% and (iii) control the translation of \ITE\ and \LETIN\ expressions by inlining or using new definitions, depending on a threshold level counting formula occurrences. 
%The extension employes several clausification techniques for handling features of FOOL. These techniques comprise the contributions of this work. 
%Section~\ref{sec:newcnf/cnf} revisits the essentials of \newcnf{} which are required for our extension presented in Section~\ref{sec:newcnf/fool}. Our algorithm combines translation of \folb{} formulas to first-order logic and clausification. 
%Sect.~\ref{sec:newcnf/comparison} discusses the advantage of our clausification algorithm for producing small clausal normal forms of FOOL formulas.  %Section~\ref{sec:newcnf/experiments} describes the experiments on theorem proving with FOOL formulas using different translations. Finally,
 %Section~\ref{sec:newcnf/related} discusses related work and Section~\ref{sec:newcnf/conclusions} outlines future work.

\paragraph{Contributions.} The main contributions of this paper are the following:
\begin{enumerate}
  \item We present a new clausification algorithm for translating \folb{} formulas to an equisatisfiable set of first-order clauses. 
  \item We handle boolean variables in FOOL formulas by skolemising them using \skolem{} predicates instead of \skolem{} functions, thus avoiding the introduction of new boolean equalities. 
  \item We control the clausification of FOOL formulas with \ITE\ and \LETIN\ expressions by a threshold level on the number of formula occurrences. Depending on the threshold, our algorithms decides on the fly whether to inline \ITE\ and \LETIN\ expressions or introduce a new name and definition for them. 
\item We handle tuple expressions in FOOL by introducing so-called projection functions  and use these projection functions in the translation of \LETIN\ expressions with tuple definition. 
  \item We implemented our work in the Vampire theorem prover~\cite{Vampire13}, 
  offering this way an automated support to reason about FOOL formulas. 
  \item We evaluate our work on three benchmark suites coming from verification
    and analysis of software and described in
    Section~\ref{sec:newcnf/experiments}, and show experimentally that our method
    significantly improves over~\cite{VampireAndFOOL} by the number of solved problems and the runtime.
\end{enumerate}

\section{Clausal Normal Form for First-Order Logic}
\label{sec:newcnf/cnf}
% !TEX root = ../main.tex

% nonnengart2001computing also have (in the simple approach): 
% elimination of equivalences as part of NNF transform 
% (but one wants to decide about equivalences during naming)
% miniscoping and variable renaming just before skolemisation
% (but let's ignore miniscoping and assume nice variables rightaway or a detail below the level of this presentation) 
%
%???? polarity dependent elimination of equivalences section has an argument about ugly invisible tautologies 
% (like the ones we mention below)
%

Traditional approaches to
clausification in FOL \cite{nonnengart2001computing,Vampire13} produce a clausal normal
form in several stages, where each stage
represents a single pass through the formula tree. These stages may include formula simplification, translation into (equivalence) negation normal form,
formula naming, elimination of equivalences, skolemisation, and distribution of
disjunctions over conjunctions. The \newcnf{} clausification algorithm
of~\cite{newcnf_fol} takes a different approach and employs a single top-down
traversal of the formula in which these stages are combined.  
This enables optimisations that are not available if the stages of clausification are independent. For example, compared to the traditional staged approach, \newcnf{} can introduce fewer \skolem{} functions on formulas with complex nesting of equivalences and quantifiers. Moreover, it can detect and discard intermediate tautologies, 
which are much more difficult to recognise by the staged approach.
%
 %Another example is an easy detection of intermediate tautologies, which are discarded on the fly. \newcnf{} thus maintains a more accurate count of sub-formula occurrences, on which the decision whether to name a sub-formula is based.
%

In this paper we use the \newcnf{} algorithm and extended it to a new clausification algorithm for FOOL~\cite{FOOL}. 
The main advantage of \newcnf{} for our work, however, is that its top-down
traversal provides a suitable context not only for clausification of first-order
formulas, but also of the extension of first-order logic with \folb{}
features. In this section we overview the main features of \newcnf{}. We will follow the notation used in~\cite{newcnf_fol} and in what follows will repeat some of the definitions. 

\subsection{Preliminaries}

Our setting is that of many-sorted first-order predicate logic with equality.

% The following definition is taken from \cite{FOOL}.
%\begin{definition}\label{def:folb-signature}\em
%  A \emph{signature} of first-order logic with the boolean sort is a triple $\Sigma = (S, F, \context)$, where:
%  \begin{enumerate}
%  \item $S$ is a set of \emph{sorts}, which contains a special sort $\bool$. A \emph{type} is either a sort or a non-empty sequence $\sigma_1,\ldots,\sigma_n,\sigma$ of sorts, written as $\sigma_1 \times \ldots \times \sigma_n \to \sigma$. When $n = 0$, we will simply write $\sigma$ instead of $\to\sigma$. We call a \emph{type assignment} a mapping from a set of variables and function symbols to types, which maps variables to sorts.
%
%    \item $F$ is a set of \emph{function symbols}. We require $F$ to contain binary function symbols $\vee$, $\wedge$, $\implies$, $\liff$ and $\lniff$, used in infix form, a unary function symbol $\neg$, used in prefix form, and nullary function symbols $\true$, $\false$.
%
%    \item $\context$ is a \emph{type assignment} which maps each function symbol $f$ into a type $\tau$. When the signature is clear from the context, we will write $\ofsort{f}{\tau}$ instead of $\context(f)=\tau$ and say that $f$ is of the type $\tau$.
%
%    We require the symbols $\vee, \wedge, \implies, \liff, \lniff$ to be of the type $\bool \times \bool \to \bool$, $\neg$ to be of the type $\bool \to \bool$ and $\true,\false$ to be of the type $\bool$. \QED
%  \end{enumerate}
%\end{definition}

A signature $\Sigma$ is a set of \emph{predicate} and \emph{function} symbols together with associated sorts. 
A \emph{term} of the sort $\tau$ is of the form $f(t_1,\ldots,t_n)$, $c$ or $x$ where
$f$ is a \emph{function symbol} of the sort $\tau_1\times\ldots\times\tau_n \to \tau$, 
$t_1,\ldots, t_n$ are terms of sorts $\tau_1,\ldots,\tau_n$, respectively, 
$c$ is a constant of sort $\tau$ and $x$ is a variable of sort $\tau$. 
%
An \emph{atom} is of the form $p(t_1,\ldots,t_n)$, $q$ or $t_1 \eql t_2$ where $p$ is a \emph{predicate symbol} of the sort $\tau_1\times\ldots\times\tau_n$, % \to \bool$
$t_1, \ldots, t_n$ are terms of sorts $\tau_1,\ldots,\tau_n$, respectively,
$q$ is a predicate symbol of sort $\bool$ and $\eql$ is the \emph{equality symbol}.  
A literal is an atom or its negation.

A \emph{formula} is of the form $\varphi_1 \wedge \ldots \wedge \varphi_n$, $\varphi_1 \vee \ldots \vee \varphi_n$, $\varphi_1 \implies \varphi_2$, $\varphi_1 \liff \varphi_2$, $\varphi_1 \lniff \varphi_2$, $\neg \varphi_1$, $\exists x : \tau. \varphi_1$, $\forall x : \tau. \varphi_1$, $\bot$, $\top$, or $l$ where $\varphi_i$  are formulas, $x$ is a variable, $\tau$ a sort and $l$ is a literal. Note that we treat conjunction and disjunction as $n$-ary operators; we assume that formulas are kept in \emph{flattened form}, e.g.\ $(\varphi_1 \wedge \varphi_2) \wedge \varphi_3$ is always represented as $\varphi_1 \wedge \varphi_2 \wedge \varphi_3$. Furthermore, we assume that usage of $\top$ and $\bot$ is simplified immediately. 
%Let ${\sf fvars}(\varphi)$ be the \emph{free variables} of formula $\varphi$, 
%i.e.~those variables in $\varphi$ with an occurrence not bound by a quantifier.

A \emph{sign} is a either $\possign$ or $\negsign$. A \emph{signed formula} is a
pair consisting of a formula $\phi$ and a sign $\sign \in \{\possign,\negsign\}$, denoted by
$\genlit{\phi}{\sign}$.
 The signed formula $\varphi^\possign$ (resp.\ $\varphi^\negsign$) means that $\varphi$ is true (resp.\ false). We will use the mapping $\formName$
from signed formulas to formulas defined as follows:
$\form{\genlit{\phi}{\possign}} = \phi$ and
$\form{\genlit{\phi}{\negsign}} = \neg \phi$. 
We call a \emph{sequent} a finite set of signed formulas. We say that a sequent
$S_1,\ldots,S_n$ is true in a FOOL interpretation if so is the universal closure
of the formula $\form{S_1} \lor \ldots \lor \form{S_n}$. Note
that if $S_1,\ldots,S_n$ are signed \emph{atomic} FOL formulas, then
$\form{S_1} \lor \ldots \lor \form{S_n}$ is a clause.

\subsection{\newcnf{}}

The \newcnf{} algorithm \cite{newcnf_fol} works with finite sets of sequents. During computation
the algorithm may construct substitutions to be applied to existing (signed) formulas. 
It is convenient for us to collect these substitutions without immediately applying
them. For this reason, instead of a sequent $D\subst$, where $\subst$ is a
substitution, we will use pairs $\genclause{D}{\subst}$ consisting of a sequent
$D$ and a substitution $\subst$. We will (slightly informally) also refer to
such pairs as sequents. 


The \newcnf{} algorithm starts with the input first-order formula $\phi$ and a
set $\GC$ of sequents that contains a single sequent
$\genclause{\{\genlit{\phi}{\possign}\}}{\emptySubst}$, where $\emptySubst$ is
the empty substitution. Then it makes a series of steps replacing sequents in
$\GC$ by other sequents until all sequents in $\GC$ contain only signed atomic
FOL formulas. Some of the steps introduce fresh (previously unused) symbols. 
% A replacement of a sequent may introduce Skolem functions and names of subformulas. 
Each update of $\GC$ preserves the following invariants: (1) if an interpretation ${\cal I}$
satisfies all sequents after the update, then ${\cal I}$ also satisfies all sequents
before the update; (2) if an interpretation ${\cal I}$
satisfies all sequents before the update, then
there exists an interpretation ${\cal I}'$ 
that extends ${\cal I}$ on fresh symbols such that ${\cal I}'$
satisfies all sequents after the update. 

The replacements of sequents are guided by the structure of $\phi$. \newcnf{}
traverses $\phi$ top-down, processing every non-atomic subformula of $\phi$
exactly once in an order that respects the subformula relation. That is, for
each two distinct subformulas $\psi_1$ and $\psi_2$ of $\phi$ such that $\psi_1$
is a subformula of $\psi_2$, $\psi_2$ is processed before $\psi_1$.
%
For every subformula of $\phi$, \newcnf{} maintains a list of its occurrences as signed formulas in the sequents of $\GC$. 
The occurrences are updated whenever sequents are removed from and added to $\GC$. 
The main role of the list is to allow for a fast enumeration and lookup of all the occurrences when a particular subformula 
is to be processed. As explained below, the number of occurrences is also used to decided whether a subformula should be named.
The replacements are governed by a set of rules that are, essentially, the standard tableau rules for first-order logic. We briefly summarise these rules below, and refer to \cite{newcnf_fol} for details.

We note that except for the rule for negation, which essentially 
flips the sign of each occurrence of $\psi = \neg \gamma$ and replaces $\psi$ with its immediate sub-formula $\gamma$
in all the sequents, the remaining rules come in pairs in which they are dual to each other. 
For instance, dealing with a disjunction $\gamma_1 \lor \gamma_2$ with a positive $\sign = \possign$
is analogous to dealing with a conjunction with a negative sign. 
For simplicity, we only show the versions for $\sign = \possign$ below.

Let $\psi$ be a subformula of $\phi$ and $\genclause{D}{\subst}$ be a sequent such that $D$ has an occurrence of $\genlit{\psi}{\possign}$.
Before proceeding to the next subformula, \newcnf{} visits and replaces all such sequents $D$.
%
Depending on the top-level connective of $\psi$ the algorithm applies the following rules.
\begin{itemize}
\item
Suppose that $\psi$ is of the form $\neg \gamma$. Add a sequent to $\GC$ obtained from $D$ by replacing the occurrence of $\genlit{\psi}{\possign}$ with $\genlit{\gamma}{\negsign}$.

\item
Suppose that $\psi$ is of the form $\gamma_1 \lor \gamma_2$. Add a sequent to $\GC$ obtained from $D$ by replacing the occurrence of $\genlit{\psi}{\possign}$ with $\genlit{\gamma_1}{\possign}, \genlit{\gamma_2}{\possign}$.

\item
Suppose that $\psi$ is of the form $\gamma_1 \land \gamma_2$. Add two sequents to $\GC$ obtained from $D$ by replacing the occurrence of $\genlit{\psi}{\possign}$ with $\genlit{\gamma_1}{\possign}$ and $\genlit{\gamma_2}{\possign}$, respectively.

\item
Suppose that $\psi$ in of the form $\gamma_1 \liff \gamma_2$. Add two sequents to $\GC$ obtained from $D$ by replacing the occurrence of $\genlit{\psi}{\possign}$ with $\genlit{\gamma_1}{\possign}, \genlit{\gamma_2}{\negsign}$ and $\genlit{\gamma_1}{\negsign}, \genlit{\gamma_2}{\possign}$, respectively.

\item
Suppose that $\psi$ in of the form $\gamma_1 \lniff \gamma_2$. Add two sequents to $\GC$ obtained from $D$ by replacing the occurrence of $\genlit{\psi}{\possign}$ with $\genlit{\gamma_1}{\possign}, \genlit{\gamma_2}{\possign}$ and $\genlit{\gamma_1}{\negsign}, \genlit{\gamma_2}{\negsign}$, respectively.

\item
Suppose that $\psi$ is of the form $(\forall x:\tau)\gamma$. Add a sequent obtained from $D$ by replacing the occurrence of $\genlit{\psi}{\possign}$ with $\genlit{\gamma}{\possign}$.

\item
Suppose that $\psi$ is of the form $(\exists x:\tau)\gamma$. Let $y_1,\ldots,y_n$ be all free variables of $\psi \subst$ and $\tau_1,\ldots,\tau_n$ be their sorts. Introduce a fresh Skolem function symbol $\sk$ of the sort $\tau_1\times\ldots\times\tau_n\to\tau$. Add a sequent $\genclause{D'}{\subst'}$, where $D'$ is obtained from $D$ by replacing the occurrence of $\genlit{\psi}{\possign}$ with $\genlit{\gamma}{\possign}$, 
and $\subst'$ extends $\subst$ with $x \mapsto \sk(y_1,\ldots,y_n)$.
\end{itemize}
%
When all subformulas of $\phi$ are traversed and the respective rules of replacing sequents are applied, the set $\GC$ only contains sequents with signed atomic formulas. $\GC$ is then converted to a set of first-order clauses by applying the substitution of each sequent to its respective formulas.

Whenever the number of occurrences of a subformula $\psi$ in sequents in $\GC$ exceeds a pre-specified {\it naming threshold}, 
$\psi$ is named as follows. Let $y_1,\ldots,y_n$ be free variables of $\psi$ and $\tau_1,\ldots,\tau_n$ be their sorts.
\newcnf{} introduces a new predicate symbol $P$ of the sort
$\tau_1\times\ldots\times\tau_n$. %  \to\bool$. 
Then, each occurrence $\genlit{\psi}{\sign}$ in sequents in $\GC$ is replaced by $\genlit{P(y_1,\ldots,y_n)}{\sign}$.
Finally, two sequents
$\genclause{\{\genlit{P(y_1,\ldots,y_n)}{\negsign},\genlit{\psi}{\possign}\}}{\emptySubst}$
and
$\genclause{\{\genlit{P(y_1,\ldots,y_n)}{\possign},\genlit{\psi}{\negsign}\}}{\emptySubst}$
are added to $\GC$ to serve as a definition of $\psi$. 
% \todo{Shouldn't we mention polarity naming? EK: is the next sentence not enough?}
As usual, in case $\psi$ always occurs in $C$ only under a single sign, 
adding only the one respective defining sequent is sufficient. 
%Also, in the case of FOOL, naming can be more complicated since variable renaming might be required.
%\todo{The previous sentence is vague and a bit weird.}

%If the number of occurrences of $\psi$ does not exceed the naming threshold, 
%each of the sequents that have an occurrence of $\genlit{\psi}{\possign}$ or $\genlit{\psi}{\negsign}$
%is replaced with one or more new sequents according to a specific rule,
%depending on the top-level connective of $\psi$.

% \newcommand{\spcr}{0.5em}
% \begin{figure}[t]
% \begin{center}
% \fbox{\begin{minipage}{\textwidth}
% \[
% \begin{array}{l}

% \text{Given a subformula $\psi$ and a sequent $\genclause{D}{\subst}$ such that $D$ has an occurrence of $\genlit{\psi}{\possign}$}

% ~\\~\\

% \begin{array}{lll}

% \vspace{\spcr}

% \text{if}~\psi = \neg \gamma & \Rightarrow & \genlit{\gamma}{\possign}

% \\ \vspace{\spcr}

% \text{if}~\psi = \gamma_1 \lor \gamma_2 & \Rightarrow & \genlit{\gamma_1}{\possign}, \genlit{\gamma_1}{\possign}

% \\ \vspace{\spcr}

% \text{if}~\psi= \gamma_1 \land \gamma_2 & \Rightarrow & \genlit{\gamma_1}{\possign} \text{~and~} \genlit{\gamma_1}{\possign}

% \\ \vspace{\spcr}

% \text{if}~\psi = \gamma_1 \liff \gamma_2 & \Rightarrow & \genlit{\gamma_1}{\possign}, \genlit{\gamma_1}{\negsign} \text{~and~} \genlit{\gamma_1}{\negsign}, \genlit{\gamma_1}{\possign}

% \\ \vspace{\spcr}

% \text{if}~\psi = \gamma_1 \lniff \gamma_2 & \Rightarrow & \genlit{\gamma_1}{\possign}, \genlit{\gamma_1}{\possign} \text{~and~} \genlit{\gamma_1}{\negsign}, \genlit{\gamma_1}{\negsign}

% \\ \vspace{\spcr}

% \text{if}~\psi = (\forall x:\tau) \gamma & \Rightarrow & \genlit{\gamma}{\possign}

% \\ \vspace{\spcr}

% \text{if}~\psi = (\exists x:\tau) \gamma & \Rightarrow & TODO

% \end{array}

% \end{array}
% \]
% \end{minipage}}
% \end{center}
% \caption{A summary of \newcnf\ rules.\label{fig:rules}}
% \end{figure}

Whenever a new sequent $\genclause{D}{\subst}$ is constructed, \newcnf{} eliminates immediate tautologies and redundant formulas. It means that
\begin{enumerate}
  \item if $D$ contains both $\genlit{\psi}{\possign}$ and $\genlit{\psi}{\negsign}$, $\genclause{D}{\subst}$ is not added to $\GC$;
  \item if $D$ contains multiple occurrences of a signed formula, only one occurrence is kept in $D$;
  \item if $D$ contains $\genlit{\top}{\possign}$ or $\genlit{\bot}{\negsign}$, $\genclause{D}{\subst}$ is not added to $\GC$;
  \item if $D$ contains a signed formula $\genlit{\bot}{\possign}$ or
    $\genlit{\top}{\negsign}$, this signed formula is removed from $D$.
\end{enumerate}
% \todo{Probably no need to mention tautology elimination anymore, since it's described in the VCNF paper}
These rules are not required for replacing sequents, however they simplify formulas and make the resulting set of clauses smaller.

\newcnf{} takes as an input a first-order formula in \emph{equivalence negation normal form} (ENNF). A formula is in ENNF if it does not contain $\implies$ and negations are only applied to atoms. ENNF is very convenient for
standard FOL, as it reduces the number of cases to consider and makes checking polarities trivial. At the same time, it is not easy to define a useful extension of ENNF for FOOL because of \LETIN\ expressions and formulas inside terms. It is straightforward, however, to extend \newcnf{} in order to support formulas in full first-order logic. For that, we need to add an extra rewriting rule for implications. In what follows we will consider a modification of \newcnf{} with this extension.

\section{Clausal Normal Form for \folb}
\label{sec:newcnf/fool}
% !TEX root = ../main.tex

In this section we describe our new clausification algorithm for \folb{}. 
The algorithm takes a \folb{} formula as input and produces an equisatisfiable set of first-order clauses. 
We write \nfcnf{} to refer to this algorithm, and \oldcnf{} to refer to the algorithm of~\cite{FOOL} for translating \folb{} formulas to arbitrary FOL formulas. In what follows, we first briefly overview the \folb{} logic and then describe \nfcnf{} and compare the CNFs produces by it and \oldcnf{}.

\subsection{\folb{}}

\folb{} \cite{FOOL} extends the standard many-sorted FOL with an interpreted boolean sort. 
Boolean variables can be used as formulas in \folb{} and formulas may be used as
arguments to function and predicate symbols. In addition to its first-class boolean sort,
\folb{} extends standard FOL with following constructs:
\begin{enumerate}
  %\item boolean variables used as formulas; %LK: these are not constructs
 % \item formulas used as arguments to function and predicate symbols;%LK: these are not constructs
  \item \ITE\ expressions that can occur as terms and formulas;
  \item \LETIN\ expressions that can occur as terms and formulas and can define an arbitrary number of function and predicate symbols.
\end{enumerate}
Finally, \folb{} also includes tuple expressions and \LETIN\ expressions with tuple definitions. 
A \LETIN\ expression with a tuple definition has the form $\letin{\tuple{c_1,\ldots,c_n}}{s}{t}$, where $n > 1$, $t$ is a term, $c_1,\ldots,c_n$ are constants, and $s$ is a tuple expression. A tuple expression is inductively defined as follows:
\begin{enumerate}
  \item $\tuple{s_1,\ldots,s_n}$, where $s_1,\ldots,s_n$ are terms;
  \item $\ite{\phi}{s_1}{s_2}$, where $s_1$ and $s_2$ are tuple expressions;
  \item a \LETIN\ expression of the form $\letindef{D}{t}$, where $D$ is tuple, function, or predicate definition, and $t$ is a tuple expression.
\end{enumerate}

Note that tuple expressions are not first class terms. They can only occur on the right-hand side of tuple definitions, but not as arguments to function or predicate symbols. Moreover, we do not assign sorts to tuple expressions and do not allow nested tuple expressions. It is however straightforward to extend \folb{} with a theory of first class tuples. For that, one needs to assign tuple sorts of the form $\tuple{\tau_1,\ldots,\tau_n}$ to tuple expressions of the form $\tuple{s_1,\ldots,s_n}$ if $s_1:\tau_1,\ldots,s_n:\tau_n$, and allow tuple expression to appear as terms. Such extension is not considered in this paper.

There are several ways to support the interpreted boolean sort in first-order theorem proving. 
The approach taken in~\cite{FOOL} proposes to axiomatise it by adding two constants $\true$ and $\false$ of this sorts and two axioms: $\true \neql \false$ and $(\forall x:\bool)(x \eql \true \lor x \eql \false)$. Furthermore, \cite{FOOL} proposes a modification in superposition calculus of first-order provers: it (i)
changes the  simplification ordering of first-order prover by making $\true$ and $\false$ the smallest terms of boolean sort 
and (ii) replaces the second axiom with a so-called \folb{} paramodulation rule. These modifications block self-paramodulation of $x \eql \true \lor x \eql \false$ and hence prevent performance problems arising from self-paramodulation in superposition theorem proving. 
In this paper, we however argue that neither boolean axiom nor modifications of superposition calculus are needed to support the interpreted boolean sort. 
Rather, we propose special processing of boolean variables and boolean equalities during clausification and avoid the introduction of new boolean equalities. 

\subsection{Introducing \nfcnf{}}

The \nfcnf{} clausification algorithm introduced in this paper is a non-trivial extension of the \newcnf{} algorithm. 
%In the sequel, for the simplicity of the presentation, we write \nfcnf{} to refer to our new clausification algorithm for \folb{}. 
%Further, we write \oldcnf{} to mean the algorithm of~\cite{FOOL} for translating \folb{} formulas to arbitrary FOL formulas. 
Compared to \oldcnf{}, % our 
\nfcnf{} 
% algorithms 
clausifies FOOL formulas directly, without first translating them to general FOL formulas and only then to CNF. 
The \nfcnf{} algorithm   extends \newcnf{} by adding support for \folb{} formulas, as follows. 
%  that adds support for \folb{} formulas. In order to enable 
%\newcnf{} to translate \folb{} and not just first-order formula we make the following changes to it.
\begin{itemize}
  \item We allow sequents to contain signed \folb{} formulas, and not just first-order formulas.
  \item We extend the \newcnf{} tautology elimination with the support for boolean variables. Whenever a boolean variable occurs in a sequent twice with the opposite signs, that sequent is not added to $\GC$. Whenever a boolean variable occurs in a sequent multiple times with the same sign, only one occurrence is kept in the sequent.
  \item We add extra rules that guide how sequents are replaced in the set $\GC$ detailed below. These rules correspond to syntactical constructs available in \folb{} but not in ordinary first-order logic.
  \item We change the rule that translates existentially quantified formulas to sko\-le\-mi\-se boolean variables using \skolem{} predicates and not \skolem{} functions. For that, we also allow substitutions to map boolean variables to \skolem{} literals. 
  \item We add an extra step of translation. After the input formula has been traversed, we apply substitutions of boolean variables to every formula in each respective sequent. The resulting set of sequents might have \skolem{} literals occurring as terms. We run the clausification algorithm again on this set of sequents. The second run does not introduce new substitutions and results with a set of sequents that only contain atomic formulas and substitutions of non-boolean variables.
\end{itemize}

In the sequel, we detail the rules of \nfcnf{} for replacing sequents. To simplify the exposition and without the loss of generality, we make the following assumptions about the input FOOL formula.
\begin{itemize}
\item
We do not distinguish formulas used as arguments as a separate syntactical construct, but rather treat each such formula $\phi$ as an \ITE\ expression of the form $\ite{\phi}{\true}{\false}$. 
\item
We assume that every \LETIN\ expression defines exactly one function or predicate symbol.
%
Every \LETIN\ expression that defines more than one symbol can be transformed to multiple nested \LETIN\ expressions,
each defining a single symbol, possibly by renaming some of the symbols. 
\item
We assume that \LETIN\ expressions only occur as formulas. 
%
Every atomic formula that contains a \LETIN\ expression can be transformed to a \LETIN\ expression that defines the same symbol and occurs as a formula. 
% EK: Should be careful with let inside let bindings!
\item
Finally, we assume that each function or predicate symbol is defined by a \LETIN{} expression at most once. 
%
This can be achieved by a standard renaming policy.
\end{itemize}

\subsection{\nfcnf{} Rules}
This section presents the rewriting rules of \nfcnf{} for syntactic construct available in FOOL, but not in standard first-order logic. For each such construct we present a rewriting rule for it in \nfcnf{}, give an example of a FOOL formula with that construct, and compare its CNFs obtained using \nfcnf{} and \oldcnf{}.

Let us now fix an input formula $\phi$ and let $\psi$ be one of its subformulas. 
In the sequel we assume that $\phi$ and $\psi$ are fixed and give all definitions relative to them. 
Let $\genclause{D}{\subst}$ be a sequent such that $D$ has an occurrence of $\genlit{\psi}{\sign}$. 

\subsection*{Boolean Variables}
Suppose that $\psi$ is a boolean variable $x$. 
If $\subst$ does map $x$, \nfcnf{} adds $\genclause{D}{\subst}$ to $\GC$.
% Explanation, idea of correctness:
This corresponds to the case in which $x$ was an existentially quantified variable skolemised in some previous step.
        
If $\subst$ does not map $x$, \nfcnf{} adds the sequent $\genclause{D'}{\subst'}$ to $\GC$, where $D'$ is obtained from $D$ by removing the occurrence of $\genlit{\psi}{\sign}$ and $\subst'$ extends $\subst$ with $x \mapsto \false$ if $\sign=\possign$, and $x \mapsto \true$ if $\sign=\negsign$. 
% Explanation, idea of correctness:
This corresponds to the case in which $x$ was a universally quantified variable.
Treating the boolean universal quantifier as a conjunction, we are implicitly replacing the sequent $D$ with two extensions, one for $x \mapsto \false$ and the other for $x \mapsto \true$. However, one of them is always true due to the occurrence of $\genlit{\psi}{\sign}$ in $D$ and so is not considered anymore. Thus only $\genclause{D'}{\subst'}$ is further processed by \nfcnf{}.

\begin{example} Let $\psi_1 = (\forall x:\bool)(x \lor P(x))$, $\psi_2 = (\exists y:\bool)(P(y) \land y)$,
where $P$ is a predicate symbol of the sort $\bool \to \bool$ and let us consider the formula $\varphi = \psi_1 \lor \psi_2$.

To process $\varphi$, \nfcnf{} first applies the rule for disjunction inherited from $\newcnf$,
obtaining the sequent $\{\psi_1^\possign, \psi_2^\possign\}_\epsilon$. The following are the steps corresponding to processing 
$\psi_1$ and its subformulas:
\[
\begin{array}{ll}
\{(\forall x:\bool)(x \lor P(x))^\possign, \psi_2^\possign\}_\epsilon & \Rightarrow \\
\{(x \lor P(x))^\possign, \psi_2^\possign\}_\epsilon & \Rightarrow \\
\{x^\possign, P(x)^\possign, \psi_2^\possign\}_\epsilon & \Rightarrow \\
\{x^\possign, P(x)^\possign, \psi_2^\possign\}_{\{x \mapsto \false\}}.\\
\end{array}
\]
Notice how the substitution is extended by $x \mapsto \false$ because of the positive occurrence of the boolean variable $x$.

Next, we show how $\psi_2$ and its subformulas get processed. 
We introduce $\mathit{sk}$, a nullary skolem predicate symbol for the existential quantifier:
\[
\begin{array}{ll}
\{x^\possign, P(x)^\possign, (\exists y:\bool)(P(y) \land y)^\possign\}_{\{x \mapsto \false\}} & \Rightarrow\\
\{x^\possign, P(x)^\possign, (P(y) \land y)^\possign\}_{\{x \mapsto \false, y \mapsto \mathit{sk}\}} & \Rightarrow\\
\{x^\possign, P(x)^\possign, P(y)^\possign\}_{\{x \mapsto \false, y \mapsto \mathit{sk}\}}, 
\{x^\possign, P(x)^\possign, y^\possign\}_{\{x \mapsto \false, y \mapsto \mathit{sk}\}}.
\end{array}
\]
Recall that dealing with boolean variables in \nfcnf{} requires an extra stage in which boolean substitutions are applied:
\[
\{\false^\possign, P(\false)^\possign, P(\mathit{sk})^\possign\}_\epsilon, 
\{\false^\possign, P(\false)^\possign, \mathit{sk}^\possign\}_\epsilon.
\]
Next, \nfcnf{} eliminates the tautology $\genlit{\false}{\possign}$ in both sequents. The literal $P(\mathit{sk})$ contains a formula inside, therefore \nfcnf{} translates it as the formula $P(\ite{\mathit{sk}}{\true}{\false})$ according to the rules given in Section~\ref{subsect:term-ite}:
\[
\{P(\false)^\possign, P(\ite{\mathit{sk}}{\true}{\false})^\possign\}_\epsilon, 
\{P(\false)^\possign, \mathit{sk}^\possign\}_\epsilon.
\]
Finally, \nfcnf{} converts signed atomic formulas to literals and we obtain the following three clauses:\footnote{Notice that the last two clauses are identical and one of them could be dropped. 
However, \nfcnf{} is not designed to do that.}
\[
\{P(\false), \neg \mathit{sk}, P(\true)\}, \{P(\false), \mathit{sk}\}, \{P(\false), \mathit{sk}\}.
\]

\oldcnf{} converts $\varphi$ to the following set of clauses:
$$\{ x \eql \true, P(x), P(\mathit{sk}) \}, \{ x \eql \true, P(x), \mathit{sk}\eql \true \},$$
where $\mathit{sk}$ is a skolem constant of the sort $\bool$.
\QED\end{example}

%The translation of \folb{} formulas to full first-order logic in~\cite{FOOL} 
The \oldcnf{} algorithm of~\cite{FOOL} 
replaces each boolean variable $x$ occurring as formula with $x \eql \true$ and skolemises boolean variables using boolean \skolem{} functions. 
Unlike \oldcnf{}, \nfcnf{} skolemises boolean variables using \skolem{} predicates and substitutes boolean variables that do not need skolemisation with constants $\true$ and $\false$. 
The approach taken in \nfcnf{} is superior in two regards.
\begin{enumerate}
  \item \oldcnf{} converts each skolemised boolean variable $x$ occurring as formula to an equality between \skolem{} terms and $\true$. \nfcnf{} converts $x$ to a \skolem{} literal which can be handled by standard superposition more efficiently. 
  \item Substitution of a universally quantified boolean variable with $\true$ and $\false$ can decrease the size of the translation. If the boolean variable occurs as formula, after applying the substitution,  the occurrence is either removed or the whole sequent is discarded by tautology elimination in \nfcnf{}.
\end{enumerate}

% Hence, \oldcnf{} requires a modification of the superposition calculus in order to avoid self-paramodulation during superposition theorem proving. 

% That is, \nfcnf{} brings no changes in the superposition calculus of the theorem prover. 

Our treatment of boolean variables never introduces new equalities and uses \skolem{} predicates instead of \skolem{} functions.
%\todo{Aren't we repeating this?} EK: Perhapse, but it's an important thing to mention, without it, we cannot claim that modifications of superposition calculus are not needed
We process boolean equalities as logical equivalences and use guards to name \ITE\ expressions occurring as terms. The usage of these techniques give the resulting set of clauses the following two properties.
\begin{enumerate}
  \item It can only contain boolean variables and constants $\true$ and $\false$ as boolean terms. 
  
  Every boolean term that occurs in $\phi$ is translated as formula and no boolean terms other than variables, $\true$ and $\false$ are introduced. 
  \item It does not contain equalities between boolean terms. \label{item_no_eq} 
  
  Every boolean equality occurring in the input is translated as equivalence between its arguments, and no new boolean equalities are eventually introduced.
\end{enumerate}
These two properties ensure that 
% boolean variables will only be unified with $\true$ and $\false$ during superposition. Constants $\true$ and $\false$ cannot be unified with each other, therefore all inferences that involve clauses with boolean terms are sound. As a result, 
no extra axioms or inference rules are required to handle the interpreted boolean sort in a theorem prover.
In particular, thanks to the second property we do not need any form of equational reasoning for this sort.

% Proof idea: 
% 1) as with any other proving, when there is no equality, we do not need superposition
% 2) if SAT, we should interpret the sort as a two element domain.
% We set the Herbrand universe of the sort to \{$\true$, $\false$\} (add them if not present) and run the model operator.

\subsection*{Boolean Equalities}
Suppose that $\psi$ is $\gamma_1 \eql \gamma_2$, where $\gamma_1$ and $\gamma_2$ are formulas.
\nfcnf{} adds a sequent to $\GC$ that is obtained from $D$ by replacing the occurrence of $\genlit{\psi}{\sign}$ 
with $(\gamma_1 \liff \gamma_2)^{\sign}$.

In effect, \nfcnf{} reduces the case of boolean equality to that of formula equivalence, 
delegating the processing to the respective rule inherited from \newcnf.

% The \nfcnf{} algorithm adds two sequents to $\GC$ obtained from $D$ by replacing the occurrence of $\psi$ with 
% $\genlit{\gamma_1}{-\sign}$, $\genlit{\gamma_2}{\possign}$ and $\genlit{\gamma_1}{\sign}$, $\genlit{\gamma_2}{\negsign}$, respectively.
% Explanation, idea of correctness:
% This means our algorithm treats equality between variables the same way as equivalence.

\subsection*{\ITE\ Expressions as Terms}
\label{subsect:term-ite}
Suppose that $\psi$ is an atomic formula that contains one or more \ITE\ expressions occurring as terms. \nfcnf{} translates each of the expressions either by expanding or naming it. 
We first describe this step of \nfcnf{} for a single \ITE\ expression and then generalise for an arbitrary number of \ITE\ expressions inside one atomic formula. Suppose that $\psi$ is an atomic formula $L[\ite{\gamma}{s}{t}]$.

\paragraph{Expanding.} \nfcnf{} adds two sequents to $\GC$ obtained from $D$ by replacing the occurrence of $\genlit{\psi}{\sign}$ with $\genlit{\gamma}{\negsign}$, $\genlit{L[s]}{\sign}$ and $\genlit{\gamma}{\possign}$, $\genlit{L[t]}{\sign}$, respectively.
    
\paragraph{Naming.} Let $x_1,\ldots,x_n$ be all the free variables of $\gamma$, and $\tau_1,\ldots,\tau_n$ be their sorts. Let $\tau$ be the common sort of both $s$ and $t$. Then, the \nfcnf{} algorithm 
\begin{enumerate}
  \item introduces a fresh predicate symbol $P$ of the sort $\tau\times\tau_1\times\ldots\times\tau_n$;
  \item introduces a fresh variable $y$ of the sort $\tau$;
  \item adds a sequent to $\GC$ that is obtained from $D$ by replacing the occurrence of $\genlit{\psi}{\sign}$ with $\genlit{L[y]}{\sign}$, $\genlit{P(y,x_1,\ldots,x_n)}{\negsign}$;
  \item adds sequents $\genclause{\{\genlit{\gamma}{\negsign},\genlit{P(s,x_1,\ldots,x_n)}{\possign}\}}{\emptySubst}$ and $\genclause{\{\genlit{\gamma}{\possign},\allowbreak\genlit{P(t,\allowbreak x_1,\allowbreak \ldots,x_n)}{\possign}\}}{\emptySubst}$ to $\GC$.
\end{enumerate}

\begin{example} Consider a definition of the $\mathit{max}$ function using \ITE\ taken from \cite{VampireAndFOOL}:
\begin{equation}\label{eq:formula-ite-example}
  (\forall x:\Z)(\forall y:\Z)(\mathit{max}(x, y) \eql \ite{x \geq y}{x}{y}).
\end{equation}

To translate \eqref{eq:formula-ite-example}, \nfcnf{} first applies twice the rule for existential quantifier inherited from \newcnf, obtaining the sequent $$\genclause{ \{ \genlit{(\mathit{max}(x, y) \eql \ite{x \geq y}{x}{y})}{\possign} \} }{\emptySubst}.$$ Then, either expanding or naming processes the result.
\begin{itemize}
  \item Expanding results in $$\genclause{ \{ \genlit{(x \geq y)}{\negsign}, \genlit{(\mathit{max}(x,y) \eql x)}{\possign} \} }{\emptySubst},
  \genclause{ \{ \genlit{(x \geq y)}{\possign}, \genlit{(\mathit{max}(x,y) \eql y)}{\negsign} \} }{\emptySubst}.$$
  \item Naming results in
  \begin{align*}
  &\genclause{ \{ \genlit{(\mathit{max}(x,y)\eql z)}{\possign}, \genlit{P(z, x, y)}{\negsign} \} }{\emptySubst},\\
  &\genclause{ \{ \genlit{(x \geq y)}{\negsign}, \genlit{P(x,x,y)}{\possign} \} }{\emptySubst},
  \genclause{ \{ \genlit{(x \geq y)}{\possign}, \genlit{P(y,x,y)}{\possign} \} }{\emptySubst},
  \end{align*}
  where $z$ is a fresh variable of the sort $\Z$ and $P$ is a fresh predicate symbol of the sort $\Z \times \Z \times \Z$.
\end{itemize}

Finally, \newcnf{} converts signed formulas to literals, and we obtain
\begin{itemize}
  \item $\{ x \not\geq y, \mathit{max}(x,y) \eql x \}, \{ x \geq y, \mathit{max}(x,y) \not\eql y \}$ in case of expanding;
  \item $\{ \mathit{max}(x,y)\eql z, \neg P(z, x, y) \},
  \{ x \not\geq y, P(x,x,y) \},
  \{ x \geq y, P(y,x,y) \}$ in case of naming.
\end{itemize}

\oldcnf{} translates \eqref{eq:formula-ite-example} to $(\forall x:\Z)(\forall y:\Z)(\mathit{max}(x, y) \eql g(x, y)),$ where $g$ is a fresh function symbol defined by the following formulas:
\begin{enumerate}
  \item $(\forall x:\Z)(\forall y:\Z)(x \geq y \implies g(x, y) \eql x)$;
  \item $(\forall x:\Z)(\forall y:\Z)(x \not\geq y \implies g(x, y) \eql y).$
\end{enumerate}
This translation ultimately yields the set of three clauses with two new equalities
\begin{equation*}
\{\mathit{max}(x,y) \eql g(x,y)\}, \{x \not\geq y, g(x,y) \eql x\}, \{x \geq y, g(x,y) \eql y\}.\QED
\end{equation*}
\end{example}

Both excessive expanding and excessive naming can result in a big CNF. Expanding \ITE\ expressions in \nfcnf{} doubles the number of sequents with occurrences of $L$, but does not introduce fresh symbols. Naming, on the other hand, adds exactly two new sequents, but introduces a fresh symbol. Both expanding and naming duplicate the condition of the \ITE\ expression. As discussed previously, \nfcnf{} keeps track of the number of occurrences of this condition and names it if this number exceeds the naming threshold. At the same time, expanding constructs two new literals that cannot be named because they might be syntactically distinct and \nfcnf{} does not count occurrences of literals. If the constructed literals contain more \ITE\ expressions inside, rewriting them might cause exponential increase of the number of sequents.

To balance between these two strategies, we introduce a parameter to \nfcnf{} called the \ITE\ expansion threshold.
By default, we heuristically set the \ITE\ expansion threshold of \nfcnf{} to $\log_2 n$, where $n$ is the naming threshold of \newcnf{}. The \ITE\ expansion threshold of \nfcnf{} limits the maximal number of expanded \ITE\ expressions inside one atomic formula. We start by expanding all \ITE\ expression and once the expansion threshold is reached, \nfcnf{} names the remaining \ITE\ expressions.

Similarly to the naming threshold inherited from \newcnf{},
the expansion threshold 
provides a trade-off between the increase of the number of sequents and the number of introduced symbols. For a large number of \ITE\ expressions it avoids the exponential increase in the number of sequents. For a small number of \ITE\ expressions inside an atomic formula it avoids growing the signature.

% Formula naming averts the exponential increase in the number of sequents caused by expansion of nested \ITE\ expressions that occur as terms. 
% To illustrate this point, consider the TPTP problem \verb'SYO500^1.003' that contains a conjecture of the form $$f_0(f_1(f_1(f_1(f_2(x))))) \eql f_0(f_0(f_0(f_1(f_2(f_2(f_2(x))))))),$$ where $f_0$, $f_1$ and $f_2$ are predicates of the sort $\bool \to \bool$, and $x$ is a boolean constant. The \nfcnf{} translates as an \ITE\ expression each application of $f_i$ that occurs as argument. Expansion of every \ITE\ expression doubles the number of sequents. However, the growth stops once the naming threshold is reached.

To compare to \oldcnf{}, we recall that \oldcnf{} replaces each non-boolean \ITE\ expression with an application of a fresh function symbol and adds the definition of the symbol to the set of assumptions. The definition is expressed as an equality. 
Unlike~\oldcnf{}, our new \nfcnf{} algorithm avoids introducing new equalities and uses predicate guards for naming, thus avoiding possible self-paramodulation triggered by equality literals. % This avoids possible performance problems caused by self-paramodulation similar to the ones described in~\cite{FOOL}.

\subsection*{\ITE\ Expressions as Formulas}
Suppose that $\psi$ is of the form $\ite{\chi}{\gamma_1}{\gamma_2}$. 
Then, \nfcnf{} adds two sequents to $\GC$ obtained from $D$ by replacing the occurrence of $\genlit{\psi}{\sign}$ with $\genlit{\chi}{\negsign}$, $\genlit{\gamma_1}{\sign}$ and $\genlit{\chi}{\possign}$, $\genlit{\gamma_2}{\sign}$, respectively.

If done unconditionally, the translation of nested \ITE\ expressions could lead to an exponential increase in the number of sequents,
as the condition formula $\chi$ is being copied. However, \nfcnf{} inherits from \newcnf{} the mechanism 
for naming subformulas with many occurrences (as explained in the previous section) which prevents such blowup.

\begin{example} Consider the following property of the $\mathit{max}$ function
\begin{equation}\label{eq:term-ite-example}
  (\forall x:\Z)(\forall y:\Z)(\ite{\mathit{max}(x, y) \eql x}{x \geq y}{y \geq x}).
\end{equation}

To process \eqref{eq:term-ite-example}, \nfcnf{} first applies twice the rule for existential quantifier inherited from \newcnf{}, obtaining the sequent $$\genclause{\genlit{\{(\ite{\mathit{max}(x, y) \eql x}{x \geq y}{y \geq x})}{\possign}\}}{\emptySubst}.$$ Then, \nfcnf{} applies the rule for the \ITE\ expression: $$\genclause{ \{ \genlit{(\mathit{max}(x,y) \eql x)}{\negsign} , \genlit{(x \geq y)}{\possign} \} }{\emptySubst}, \genclause{ \{ \genlit{(\mathit{max}(x,y) \eql x)}{\possign} , \genlit{(x \geq y)}{\negsign} \} }{\emptySubst}.$$ Finally, \nfcnf{} converts signed formulas to literals and obtains the resulting set of clauses $$\{ \mathit{max}(x,y) \not\eql x, x \geq y \}, \{ \mathit{max}(x,y) \eql x, y \not\geq y \}.$$

In contrast, \oldcnf{} introduces a name for the \ITE\ expression and translates \eqref{eq:term-ite-example} to $(\forall x:\Z)(\forall y:\Z)P(x,y)$, where $P$ is a fresh predicate symbol of the sort $\Z \times \Z$ with the following definitions:
\begin{enumerate}
  \item $(\forall x:\Z)(\forall y:\Z)(\mathit{max}(x, y) \eql     x \implies P(x,y) \liff x \geq y)$;
  \item $(\forall x:\Z)(\forall y:\Z)(\mathit{max}(x, y) \not\eql x \implies P(x,y) \liff y \geq x)$.
\end{enumerate}

These three formulas ultimately yield the following set of clauses:
\begin{align*}
&\{P(x,y)\},\\
&\{\mathit{max}(x, y) \not\eql x, \neg P(x,y), x \geq y\},\{\mathit{max}(x, y) \not\eql x, P(x,y), x \not\geq y\},\\
&\{\mathit{max}(x, y) \eql y, \neg P(x,y), y \geq x\},\{\mathit{max}(x, y) \eql y, P(x,y), y \not\geq x\}.\QED
\end{align*}
\end{example}

% We point out that the translation of nested \ITE\ expressions may easily lead to an exponential increase in the number of sequents. % We handle this situation in \nfcnf{} by extending the formula naming mechanism of \nfcnf{}, as detailed below. 

\subsection*{\LETIN\ Expressions}
\label{subsect:letin}
Suppose that $\psi$ is $\letin{f(x_1:\tau_1,\ldots,x_n:\tau_n)}{t}{\gamma}$. 
The \nfcnf{} algorithms translates $\psi$ either by inlining or by naming, as discussed below.  
The choice of inlining or naming of \LETIN\ expressions in the problem is determined by a pre-specified boolean parameter of the algorithm. %boolean option. % provided by the user of the algorithm.
    
\paragraph{Inlining.} \nfcnf{} adds a sequent to $\GC$ that is obtained from $D$ by replacing the occurrence of $\genlit{\psi}{\sign}$ with $\genlit{\gamma'}{\sign}$. $\gamma'$ is obtained from $\gamma$ by replacing each application $f(s_1,\ldots,s_n)$ of an % free -- not needed anymore (we assume f is not bound twice)
occurrence of $f$ in $\gamma$ with $t'$ and renaming of binding variables. $t'$ is obtained from $t$ by replacing each free occurrence of $x_1,\ldots,x_n$ in $t$ with $s_1,\ldots,s_n$, respectively. We point out that inlining predicate symbols of zero arity does not hinder identification of tautologies thanks to tautology removal inside sequents.

\paragraph{Naming.} \nfcnf{} adds a sequent to $\GC$ that is obtained from $D$ by replacing the occurrence of $\genlit{\psi}{\sign}$ with $\genlit{\gamma}{\sign}$.
Further, \nfcnf{} also adds the sequent $\genclause{\{f(x_1,\ldots,x_n) \eql t\}}{\emptySubst}$ to $\GC$.
    
% Let $\tau$ be the sort of $t$. If $\tau$ is $\bool$, add sequents $\genclause{\{\genlit{f(x_1,\ldots,x_n)}{\negsign},\genlit{t}{\possign}\}}{\emptySubst}$ and $\genclause{\{\genlit{f(x_1,\ldots,x_n)}{\possign},\genlit{t}{\negsign}\}}{\emptySubst}$ to $\GC$. Otherwise, add a sequent $\genclause{\{f(x_1,\ldots,x_n) \eql t\}}{\emptySubst}$ to $\GC$.

Naming introduces a fresh function or predicate symbol and does not multiply the number of resulting clauses. Inlining, on the other hand, does not introduce any symbols, but can drastically increase the number of clauses. Either of the translations might make a theorem prover inefficient. We point out that the number of clauses and the size of the resulting signature are not the only factors in that. For example, consider inlining of a \LETIN\ expression that defines a non-boolean term. It does not introduce a fresh function symbol and does not increase the number of clauses. However, the inlined definition might increase the size of the term with respect to the simplification ordering. This affects the order in which literals will be selected during superposition, and ultimately the performance of the prover.

% Designing a syntactical criteria for choosing between naming and inlining is an interesting task for future work.

\subsection*{\LETIN\ Expressions with Tuple Definitions}
Suppose that $\psi$ is $\letin{\tuple{c_1,\ldots,c_n}}{s}{\gamma}$ where $n > 1$. Let $\tau_1,\ldots,\tau_n$ be the sorts of $c_1,\ldots,c_n$, respectively. Then, 
the \nfcnf{} algorithm 
\begin{enumerate}
  \item introduces a fresh sort $\tau$, a fresh function symbol $t$ of the sort $\tau$, a fresh function symbol $g$ of the sort $\tau_1\times\ldots\times\tau_n\to\tau$, and $n$ fresh function symbols $\pi_1,\ldots,\pi_n$ (called projection functions), where for each $1 \le i \le n$, $\pi_i$ is of the sort $\tau \to \tau_i$;
  \item adds a sequent to $\GC$ that is obtained from $D$ by replacing every occurrence of $\genlit{\psi}{\sign}$ with $\genlit{(\letin{t}{s'}{\gamma'})}{\sign}$. $\gamma'$ is obtained from $\gamma$ by replacing each free occurrence of $c_i$ with $\pi_i(t)$ for each $1 \le i \le n$. $s'$ is obtained from $s$ by replacing every tuple expression $\tuple{s_1,\ldots,s_n}$ with $g(s_1,\ldots,s_n)$;
  \item adds sequents to $\GC$ that axiomatise functions $g,\pi_1,\ldots,\pi_n$. In particular, these state
  that $\pi_i(g(s_1,\ldots,s_n)) \eql s_i$ for every $i=1,\ldots, n$ and that $t_1 \eql t_2 \liff \bigwedge_{i=1}^n \pi_i(t_1) \eql \pi_i(t_2)$.
\end{enumerate} 

\begin{example} Consider a formula that uses a tuple \LETIN\ expression to swap two constants $x$ and $y$ of the sort $\Z$ before applying a predicate $P$ of the sort $\Z \times \Z$ to them:
\begin{equation*}\label{eq:tuple-let-example}
  \letin{(x,y)}{(y,x)}{P(x,y)}.
\end{equation*}
To clausify this formula, \nfcnf{} firstly converts it to the formula $$\letin{t}{g(y,x)}{P(\pi_1(t), \pi_2(t))},$$ where $t$ is a fresh term of the fresh sort $\tau$, and $g$ is a fresh function symbol of the sort $\Z \times \Z \to \tau$, and $\pi_1$ and $\pi_2$ are projection functions with appropriate axiomatisation. Then, depending on whether inlining or naming is enabled, \nfcnf{} result with clauses $$\{P(\pi_1(g(y,x)),\pi_2(g(y,x)))\} \text{~or~} \{P(\pi_1(t'),\pi_2(t'))\}, \{t' \eql g(y,x)\}$$ respectively, where $t'$ is a fresh constant symbol of the sort $\tau$.
\QED\end{example}

\oldcnf{}, as described in~\cite{FOOL}, cannot handle \LETIN\ expression with tuple definitions.

\section{Experimental Results}
\label{sec:newcnf/experiments}
% !TEX root = main.tex

%Vampire is the first theorem prover to implement \newcnf{}. 
We extended Vampire's \newcnf{} clausification algorithm for standard FOL with our \nfcnf{} clausification algorithm for \folb{} formulas. 
The implementation of \nfcnf{} comprised about 500 lines of C++ code. 
%It will be included in the forthcoming official release of Vampire.
Our implementation, benchmarks and results are available at \url{www.cse.chalmers.se/~evgenyk/fool-cnf-experiments/}. 

% Our implementation is available at \url{www.cse.chalmers.se/~evgenyk/fool-cnf-experiments/} and will be included in the forthcoming official release of Vampire.

In what follows, we report on our experimental results obtained by running Vampire on \folb{} problems. 
Whenever we refer to Vampire, we mean the Vampire version extended with our new \nfcnf{} clausification algorithm for FOOL. 
We will write \oldcnfVampire{} for the previous version of Vampire
with the \oldcnf{} algorithm of~\cite{VampireAndFOOL};
\oldcnfVampire{} translates \folb{} formulas to FOL (after which they are clausified in a standard way)
and uses a special inference rule to avoid \folb{} self-paramodulation. 
% with the translation of \folb{} formulas to FOL presented in~\cite{FOOL}. In the sequel, by Vampire we will mean its version with the extended \newcnf{}. We will write %\oldcnfVampire{} for its version with the translation of \folb{} formulas to FOL and enabled \folb{} paramodulation~\cite{FOOL}.

For our experiments, we used three sets of benchmarks: 
(i) problems taken from~\cite{Blanchette15} on reasoning about (co)al\-ge\-braic datatypes (see Sect.~\ref{subsec:ADT}), 
(ii) examples with both quantifiers and uninterpreted functions taken from the SMT-LIB library~\cite{SMT-LIB} (see Sect.~\ref{subsec:SMT}), and
(iii) benchmarks on proving the partial correctness of loop-free programs (see Sect.~\ref{subsec:PrgAn}). The last benchmark suite is constructed by us to 
illustrate the use of FOOL in program analysis and verification. 
As Vampire is the only automated first-order theorem prover supporting FOOL, and in particular \ITE\ and \LETIN\ expressions, 
we could not compare Vampire with any other first-order prover. 
Further,  \oldcnfVampire{} did not yet support tuple expressions in
FOOL. Tuple expressions are also not supported by state-of-the-art SMT
solvers. For these reasons, we compared Vampire against \oldcnfVampire{}
and the SMT solvers CVC4~\cite{CVC4} and Z3~\cite{Z3}  only on the experiments from
Sect.~\ref{subsec:ADT}--\ref{subsec:SMT}.
%Our results are summarised in Tables~\ref{table:isabelle-results}--\ref{table:smt-lib-results2} and discussed below. 

%The first set is taken from our previous work~\cite{VampireAndFOOL} on the implementation of \folb{} in Vampire. The seconds set consists of problems from the SMT-LIB library~\cite{SMT-LIB}, a corpus of benchmarks for satisfiability modulo theory (SMT) solvers.
%
%In our previous work~\cite{VampireAndFOOL}, we experimented with our initial implementation of \folb{} in Vampire. 
%For that experiment, we generated two sets of \folb{} problems.
%\begin{enumerate}
 % \item Problems from the higher-order part of the TPTP library~\cite{TPTP} that can be directly expressed in \folb{}. We translated these problems from the TPTP language of higher-order logic to the modification of TPTP that supports \folb{}.
%  \item Problems about properties of (co)algebraic datatypes generated by the Isabelle theorem prover~\cite{Isabelle} to be checked by SMT solvers. We translated these problems from the SMT-LIB~2 language to TPTP using the SMTtoTPTP tool~\cite{SMTLIB2TPTP}.
%\end{enumerate}
%
%For this work we used the second set of problems only 
%as problems in the first set are easy and all of them were already solved by Vampire before.


\subsection{Experiments with Algebraic Datatypes Problems}\label{subsec:ADT}
We used 152 problems about (co)algebraic datatypes taken from~\cite{Blanchette15}. 
These examples were generated by Isabelle and translated by us to the TPTP syntax~\cite{TPTP}. 
These examples are expressed in \folb{}, as they use boolean variables occurring as formulas, formulas occurring as arguments to function and predicate symbols, and \ITE\ expressions. None of the 152 problems use \LETIN\ expressions.

We evaluated the performance of Vampire, \oldcnfVampire,
CVC4 and Z3 on the unsatisfiable problems in this set. In order to filter out satisfiable problems, we run all the provers on all the problems 
and only recorded the runs where at least one of the provers reported unsatisfiability. That gave us 57 problems.

We ran both Vampire and \oldcnfVampire\ with the option \verb'--mode casc'. For the runs of Vampire, the naming threshold was set to 8. We run CVC4 and Z3 with their default options.

Table~\ref{table:isabelle-results} summarises our results. They were obtained on a MacBook Pro with a 2,9 GHz Intel Core i5 and 8 Gb RAM, with a 60 seconds time limit for each benchmark. Vampire and \oldcnfVampire\ solved the largest number of problems, both provers solved the same problems. 51 problems were solved by all provers. Both Vampire and \oldcnfVampire\ solved 3 problems, not solved by either CVC4 or Z3. CVC4 and Z3 solved one problem, not solved by either Vampire or \oldcnfVampire. Compared to \oldcnfVampire, Vampire showed significantly smaller runtime. We therefore conclude that our clausification algorithm for FOOL improved the performance of Vampire on this set of problems.

\begin{table}[tb]
  \caption{Runtimes in seconds of provers on the set of 57 unsatisfiable algebraic datatypes problems.}
  \begin{center}
  \begin{tabular}{|l|r|r|}
    \hline
    Prover         & Solved & Total time on solved problems \\
    \hline
    Vampire        & 56     & 23.470 \\
    \oldcnfVampire & 56     & 31.121 \\
    Z3             & 53     & 3.615  \\
    CVC4           & 53     & 25.480 \\
    \hline
  \end{tabular}
  \end{center}
  \label{table:isabelle-results}
\end{table}

% \begin{figure}[tb]
%   %\vspace{-0.3em}
%   \centering
%   \begin{tikzpicture}
%     \draw (0,0) circle (1.5cm);
%     \draw (50:1cm) circle (1.55cm);
%     \draw (0cm:0.8cm) circle (1.4cm);
%     \node at (0.8cm:0.5cm) {$51$}; %
%     \node at (-2.2cm:1.2cm) {$1$};
%     \node at (1cm:-1.1cm) {$0$};
%     \node at (-2cm:-1.1cm) {$1$};
%     \node at (-3.8cm:-1.9cm) {$3$}; %
%     \node at (-0.95cm:1.775cm) {$0$}; %
%     \node at (0.45cm:1.775cm) {$1$}; %
%     \node at (2.7cm:2.65cm) {Vampire};
%     \node at (-3.7cm:1.8cm) {Z3};
%     \node at (-1.6cm:2.3cm) {CVC4};
%   \end{tikzpicture}
%   \vspace{-0.3em}
%   \caption{Venn diagram of the subsets of the algebraic datatypes problems, solved by Vampire, CVC4 and Z3.}
%   \label{fig:isabelle-diagram}
% \end{figure}

\subsection{Experiments with SMT-LIB Problems}\label{subsec:SMT}

As explained in more detail later on (see Section~\ref{sec:related}), \folb{} can be regarded as a superset of the SMT-LIB core logic.
% \folb{} can be regarded as a superset of SMT-LIB core logic and problems of SMT-LIB core logic can be directly expressed in \folb{}. The language of \folb{} extends the SMT-LIB core language with local function definitions, using \LETIN\ expressions defining functions of arbitrary, and not just zero, arity. 
A theorem prover that supports \folb{} can be straightforwardly extended to read problems written in the SMT-LIB syntax.
%
For our experiments using SMT-LIB problems, we used problems in quantified predicate logic with uninterpreted functions stored in the UF subspace of SMT-LIB.
These problems use \ITE\ expressions, \LETIN\ expressions that define constants, and formulas 
occurring as arguments to equality. None of the problems use quantifiers over the boolean sort. 
 The  problems taken from SMT-LIB are written in the SMT-LIB~2
 syntax. In order to read these problems, we implemented a parser for
 a sufficient subset of the SMT-LIB~2 language in Vampire. The
 implementation of the parser comprised about 2,500 lines of C++ code. 

We evaluated the performance of Vampire, \oldcnfVampire, and CVC4 on unsatisfiable problems of the UF subspace. Each problem in the SMT-LIB library is marked with one of the statuses \verb'sat', \verb'unsat' and \verb'unknown'. A problem is marked as \verb'sat' or \verb'unsat' 
when at least two SMT solvers proved it to be satisfiable or unsatisfiable, respectively.
Otherwise, a problem is marked as \verb'unknown'. In order to filter out satisfiable problems,
we ran Vampire, \oldcnfVampire, and CVC4 on the problems marked as \verb'unsat' and \verb'unknown' and then recorded the results on the problems that were proven unsatisfiable by at least one prover. That gave us 2191 problems.


We ran Vampire twice on each problem: once with naming of \LETIN\ expressions and once with inlining (see Sect.~\ref{subsect:letin}).
For each run the naming threshold was set to 8. In both runs we also used the option \verb'--mode casc'. For each problem, we recorded the fastest successful run of Vampire. We ran \oldcnfVampire\ once on each problem with the option \verb'--mode casc'.

\begin{table}[tb]
  \caption{Runtimes in seconds of provers on the set of 2191 unsatisfiable SMT-LIB problems.}
  \begin{center}
  \begin{tabular}{|l|r|r|r|}
    \hline
    Prover         & Solved & Uniquely solved & Total time on solved problems \\
    \hline
    CVC4           & 2084   & 55              & 26,309.47 \\
    Vampire        & 2076   & 12              & 22,920.50 \\
    \oldcnfVampire & 1984   & 9               & 19,911.69 \\
    Z3             & 1729   & 4               & 18,102.96 \\
    \hline
  \end{tabular}
  \end{center}
  \label{table:smt-lib-results2}
\end{table}

% \begin{figure}[tb]
%   \centering
%   \begin{tikzpicture}
%     \draw (0,0) circle (1.5cm);
%     \draw (50:1.5cm) circle (1.6cm);
%     \draw (0cm:1.3cm) circle (1.45cm);
%     \node at (0.8cm:0.75cm) {$1885$}; %
%     \node at (-1.85cm:1.05cm) {$9$};
%     \node at (0.8cm:-0.8cm) {$91$};
%     \node at (-2.8cm:-1.0cm) {$99$};
%     \node at (-4.1cm:-2.3cm) {$12$}; %
%     \node at (-0.75cm:2cm) {$10$}; %
%     \node at (0.65cm:1.95cm) {$80$}; %
%     \node at (2.55cm:3.2cm) {Vampire};
%     \node at (-3.7cm:1.8cm) {CVC4};
%     \node at (-1.6cm:2.4cm) {\oldcnfVampire};
%   \end{tikzpicture}
%   \vspace{-0.3em}
%   \caption{Venn diagram of the subsets of the 2191 unsatisfiable SMT-LIB problems, solved by Vampire, \oldcnfVampire\ and CVC4.}
%   \label{fig:smt-lib-newcnf-diagram}
% \end{figure}

Table~\ref{table:smt-lib-results2} summarises the results of our experiments on the SMT-LIB problems. These results were obtained on the StarExec compute cluster~\cite{starexec} using the time limit of 5 minutes per problem. 
CVC4 solved the largest number of problems, Vampire solved significantly more than \oldcnfVampire, and Z3 solved the least number of problems. None of the provers solved a superset of problems solved by another prover. The ``Uniquely solved'' column of Table~\ref{table:smt-lib-results2} shows the number of problems that were solved by each of the provers, but not any of the other ones. 1675 problems were solved by all of the provers, and 2190 problems were solved by at least one of the provers. Vampire solved 111 problems not solved by \oldcnfVampire, and \oldcnfVampire\ solved 19 problems not solved by Vampire.

We also recorded how different translations of \LETIN\ affected the performance of Vampire. Vampire with inlining of \LETIN\ expressions solved 61 problems not solved by Vampire without inlining of \LETIN\ expressions. Vampire without inlining of \LETIN\ expressions solved 45 problems not solved by Vampire without inlining of \LETIN\ expressions.
%\todo{Does it mean the Venn diagram reports the parallel best of these two versions?} EK: yes, it says so two paragraphs before

Based on the results of this experiment we make the following observations. Vampire solved new problems by inlining \LETIN\ expressions and expanding \ITE\ expressions. Vampire could not solve some of the problems that were solved by \oldcnfVampire, we explain it by the fact that \oldcnfVampire\ always names \ITE\ expressions, which turns out to be important for solving some problems. Both inlining and naming of \LETIN\ expressions can make a prover inefficient.

\subsection[Experiments with FOOL Reasoning about Programs]{Experiments with\\FOOL Reasoning about Programs}
\label{subsec:PrgAn}
% !TEX root = ../main.tex


%FOOL extends ordinary first-order logic with first-class boolean sort,
%\ITE\ and \LETIN\ expressions. 
In this experiment we evaluated Vampire on FOOL problems that express partial correctness property of imperative programs. We obtained these problems manually from a collection of loop-free programs that we in turn generated from a small set of programs with loops by unrolling their loops. Both the benchmarks and the results are available at \url{www.cse.chalmers.se/~evgenyk/fool-tuple-experiments/}.


%We now report on our experiments of using FOOL for the safety
%verification of progams. 
%
%properties, in particular for proving partial correctness 
%These extensions allow one to automatically express logical properties of imperative programs as FOOL formulas. In this section we
%\begin{enumerate}
 % \item illustrate how the partial correctness property can be encoded as a FOOL problem for a collection of realistic loop-free programs;
%  \item present experimental results obtained by running Vampire on these FOOL problems.
%\end{enumerate}
%For this experiment 
We used five small programs with loops annotated with a safety property using the {\tt assert} command. They are listed in Appendix A. Each program contains one loop with one or more
\ITE\ expressions, assignments and tests over integers,
integer arrays and booleans. Table~\ref{table:examples-results}
summarises the programs used in our experiments: the programs
\verb'count_two',  \verb'count_two_flag' and \verb'count_three'
implement versions of counting elements in an input array using
different criteria and ensure that the sum of counted elements equal
to the array length; \verb'two_arrays' and \verb'three_arrays' sort
and compare 
two, and respectively three arrays element-wise. 
We unrolled these program loops 2, 3, 4 and 5
times, resulting in a set of 20 annotated loop-free programs.
%We used a collection of 20 code fragments that we extracted from five programs with loops by unrolling each program's loop 2, 3, 4, or 5 times. Appendix~\ref{sec:newcnf/examples} lists the code of the five programs with loops, and we give their description below. Each program uses a loop with one or many \ITE\ expressions inside, assignments for integers, integer arrays and booleans. Each program has an empty pre-condition and one or many post-conditions. 
%\begin{itemize}
%  \item \verb'count_two' counts the number of negative and non-negative elements of an array. Its post-condition checks that the sum of the number of negative and non-negative elements of the array is equal to the length of the array.
%  \item \verb'count_two_flag' does the same thing as \verb'count_two' but assigns a condition to a boolean flag.
%  \item \verb'count_three' counts the number of elements of an array that are negative, non-negative and greater than 5, and non-negative and less or equal to 5. Its post-condition checks that the sum of values of the three counters is equal to the length of the array.
%  \item \verb'two_arrays' sorts two arrays element-wise. Its post-conditions check that every element of one array is less or equal than the element of the other array at the same index.
%  \item \verb'three_arrays' sorts three arrays element-wise.
%\end{itemize}
%
%We unrolled each program's loop by fixing the length of the arrays
%used in the program and repeating the body of the loop. 
Figure~\ref{fig:examples/count_two} shows the
program \verb'count_two' and the program obtained by unrolling three times the loop of \verb'count_two'.

\begin{figure}[ht]
\begin{minipage}{.515\textwidth}
  {\begin{lstlisting}[language=cpp]
int a[];
int x = 0, y = 0;
for (int i = 0; i < n; i++) {
  if (a[i] > 0) x++; else y++;
}
assert(x + y == n);
\end{lstlisting}}
\end{minipage}%
\hspace{1pt}
\begin{minipage}{.48\textwidth}
  {\begin{lstlisting}[language=cpp]
int a[];
int x = 0, y = 0;
if (a[0] > 0) x++; else y++;
if (a[1] > 0) x++; else y++;
if (a[2] > 0) x++; else y++;
assert(x + y == 3);
\end{lstlisting}}
\end{minipage}
  \caption{The \texttt{count\_two} program and the program obtained by unrolling it three times.}
  \label{fig:examples/count_two}
\end{figure}

%Each one of the 20 code fragments with unrolled loops is an imperative
%program with assignements, \ITE, and sequential composition. 
For each one of the 20 loop-free benchmarks, we expressed its partial
correctness as a TPTP problem using FOOL in the combination of the  theory of
linear integer arithmetic and the polymorphic theory of
arrays~\cite{VampireAndFOOL}. To this end, (i) we formulated the
safety assertion as a TPTP conjecture and (ii) expressed the
transition relation of the program as a FOOL formula with tuple
expressions and \LETIN\ expressions with tuple definitions.
We refer to \cite{VampireAndFOOL} for the details of the translation of a program's transition relation to FOOL. In particular, the correctness of this translation is stated in Theorem 1 of that work. Each FOOL formula produces by the translation is linear in the size of the program. Figure~\ref{fig:examples/count_two_tptp} shows the TPTP translation of the safety property of the \verb'cout\_two\_tptp' program. It uses the {\tt thf} subset of the TPTP language, which is the standard subset that contains features of FOOL.

%we (i) encode a conjunction of post-conditions as the safety condition and (ii) encode the next state values of program variables. The former is straightforward, and the latter employs the translation described in our previous work~\cite{VampireAndFOOL}. We briefly recap it here.
%
%The translation works with programs using assignments $\ass$, \ITE, and sequential composition. Given a program $P$ its translation $[P]$ has the form $\letin{(x_1,\ldots,x_n)}{E}{(x_1,\ldots,x_n)}$, where $x_1,\ldots,x_n$ are all variables updated by $P$, that is, all variables used in the left-hand-side of an assignment. $[P]$ is inductively defined as follows:
%
%\begin{itemize}
%  \item $[\text{x}_i \text{\ASS} e]$ is $\letin{(\ldots,x_i,\ldots)}{(\ldots,e,\ldots)}{(x_1,\ldots,x_n)}$,
%  \item $[$\IF\ e \THEN\ $P_1$ \ELSE\ $P_2]$ is $\letin{(x_1,\ldots,x_n)}{\ite{e}{[P_1]}{[P_2]}}{(x_1,\ldots,x_n)},$
%  \item $[P_1$;$P_2]$ is $\mathtt{let}~D~\mathtt{in}~[P_2]$ where $[P_1]$ is $\mathtt{let}~D~\mathtt{in}~(x_1,\ldots,x_n)$.
%\end{itemize}

%The crucial aspect of this translation is that it relies on tuple expressions and tuple definitions in \LETIN\ expressions available in FOOL.

%The translation takes a program statement $s$ and a FOOL formula $\psi$ as the input and produces a FOOL formula $[s]\psi$ that expresses the next state values of program variables. The translation is inductively defined as follows.
%\begin{enumerate}
%  \item $[x \ass\ e]\psi$ is $\letin{x}{e}{\psi}$;
%  \item $[s_1 ; s_2]\psi$ is $\letin{a}{b}{c}$
%  \item $[\IF~\phi~\THEN~s_1~\ELSE~s_2]\psi$ is $\letin{\tuple{x_1}{\ldots}{x_n}}{\ite{\phi}{[s_1,\tuple{x_1}{\ldots}{x_n}]}{[s_2,\tuple{x_1}{\ldots}{x_n}]}}{\psi}$, where $x_1,\ldots,x_n$ are all program variables of $s_1$ and $s_2$.
%\end{enumerate}
%Figure~\ref{fig:examples/count_two_tptp} shows the translation of code fragment from Figure~\ref{fig:examples/count_two_unrolled} written in the TPTP language. In this and other examples we translated assignments to an element of an array as an assignment of an updated version of an array to itself.

\begin{figure}[ht]
\begin{lstlisting}[language=tptp]
thf(a, type, a: $array($int, $int)).
thf(x, type, x: $int).
thf(y, type, y: $int).

thf(count_two, conjecture,
    $let(x := 0,
    $let(y := 0,
    $let([x, y] := $ite($greater($select(a, 0), 0),
                        $let(x := $sum(x, 1), [x, y]),
                        $let(y := $sum(y, 1), [x, y])),
    $let([x, y] := $ite($greater($select(a, 1), 0),
                        $let(x := $sum(x, 1), [x, y]),
                        $let(y := $sum(y, 1), [x, y])),
    $let([x, y] := $ite($greater($select(a, 2), 0),
                        $let(x := $sum(x, 1), [x, y]),
                        $let(y := $sum(y, 1), [x, y])),
         $sum(x, y) = 3)))))).
\end{lstlisting}
  \caption{A FOOL translation of the unrolled program in Figure~\ref{fig:examples/count_two} written in the TPTP language.}
  \label{fig:examples/count_two_tptp}
\end{figure}

The results of the experiments are summarised in Table~\ref{table:examples-results}. These results were obtained on a MacBook Pro with a 2,9 GHz Intel Core i5 and 8 Gb RAM, and using the time limit of 60 seconds per problem.  The first column of the table lists the
names of the programs with loops, and columns~2--5 indicate how many
time the program loop was unrolled and gives the time needed by
Vampire to prove the correctness of the corresponding loop-free
program. %Dashes mean Vampire failed to construct a
%proof; when investigating the failing cases, we realised that more
%efficient implementation of tuples in FOOL is needed. We leave this
%task for future work. 

Based on the results of this experiment we conclude that Vampire can be used for verification of bounded safety properties of imperative programs.

%The experiment shows shows that Vampire can be used for proving bounded safety of imperative programs.
%This is mainly thanks to the FOOL logic and, in particular, the support for the tuple construct,
%which make first-order theorem proving better suited for applications of program analysis and verification. 
%The results of the experiments are summarised in
%Table~\ref{table:examples-results}. 
%\EK{Are there any observation we can make about the results?} These results were obtained on a MacBook Pro with a 2,9 GHz Intel Core i5 and 8 Gb RAM, and using the time limit of 60 seconds per problem. Both the benchmarks and the results are available at \url{www.cse.chalmers.se/~evgenyk/fool-tuple-experiments/}.

\begin{table}[ht]
  \caption{Runtimes in seconds of Vampire on 20 problems encoding partial program correctness.}
  \begin{center}
  \begin{tabular}{lrrrr}
    \hline Problem & 2 & 3 & 4 & 5 \\ \hline
    \verb'count_two'        &  0.011  &  0.016  &  0.030  &  0.053 \\
    \verb'count_two_flag'~  &  0.011  &  0.017  &  0.028  &  0.041 \\
    \verb'count_three'      &  0.023  &  0.042  &  0.128  &  0.522 \\
    \verb'two_arrays'       &  0.026  &  0.091  &  0.237  &  0.263 \\
    \verb'three_arrays'     &  0.446  &  5.368  &  8.719  & 14.886
  \end{tabular}
  \end{center}
  \label{table:examples-results}
  \vspace{-1em}
\end{table}

\section{Related Work}
\label{sec:related}
% !TEX root = main.tex
\folb{} is a relatively new extension of FOL. We are not aware of any work that explicitly deals with clausifying formulas in this logic. However, connections can be found in work focusing on related fragments and extensions.

Most notably, Wisniewski et al.~propose in % their recent work 
\cite{DBLP:conf/cade/WisniewskiSKB16}
methods for normalising formulas in higher-order logic (HOL). Similarly to \folb{},
HOL natively contains the boolean sort. Wisniewski et al.~deal with 
formulas occurring at argument positions by a technique called \emph{argument extraction} 
which, similarly to our naming schemes, extends the signature and defines a new symbol
outside the original formula. Moreover, also Wisniewski et al. introduce skolem predicates 
instead of skolem functions when dealing with existential boolean quantifiers. 
This happens implicitly for them, since in HOL there is no distinction between formulas and terms.

\folb{} can be regarded as a superset of SMT-LIB \cite{BarFT-SMTLIB} core logic and formulas of SMT-LIB core logic can be directly expressed in \folb{}. The language of \folb{} extends the SMT-LIB core language with local function definitions, using \LETIN\ expressions defining functions of arbitrary, and not just zero, arity. 

Despite the similarity of the languages, the technology used by modern SMT solvers~\cite{DBLP:journals/jacm/NieuwenhuisOT06}
differs greatly from that of % saturation-based 
first-order theorem provers and so do the approaches to normalising the input formula.
In particular, as SMT solvers pass the propositional abstraction of the input formula to an efficient SAT solver
there is no great need to optimise extensions of the signature 
and clausification usually follows the simple Tseitin encoding~\cite{tseitin_enc} of the formula tree.
%
% E-matching \cite{DBLP:journals/jacm/DetlefsNS05,DBLP:conf/cade/MouraB07}
Moreover, modern SMT solvers employ an alternative approach to dealing with quantifiers over interpreted sorts such as the booleans, 
which is complementary to skolemisation
and relies on a guidance by counter-examples~\cite{DBLP:journals/corr/Reynolds0K15} or on model-based projections~\cite{LPAR-20:Playing_with_Quantified_Satisfaction}.

Finally, it is interesting to note that our \nfcnf{} algorithm naturally translates a quantified boolean formula (QBF),
as realised in the FOOL language, into a CNF in effectively propositional logic (EPR).
Specifically, every literal in this translation is 
a skolem predicate applied to boolean variables and constants $\true$ and $\false$.
This result is similar to the one proposed in \cite{DBLP:conf/cade/SeidlLB12},
where the authors explicitly focus on QBF as the source and EPR as the target language, respectively.
Obtaining a formula in EPR is a desirable property % to have 
since there are first-order proving methods 
known to be efficient for dealing with the fragment (see e.g.~\cite{DBLP:conf/birthday/Korovin13}).

\section{Conclusion and Future Work}
\label{sec:newcnf/conclusions}
% !TEX root = main.tex

Applications of program analysis and verification rely on SAT/SMT solvers and/or
theorem provers to reason about program properties formulated in various logics.
%SAT/SMT solvers and theorem provers first translate arbitrary logical properties into an equisatisfiable set of first-order clauses (CNF form) and reason further only about the CNF form of the input formula. 
The efficiency of SAT/SMT solvers and theorem provers critically depends on the used clausification algorithm. 
%a set Efficient algorithms for computing small CNF forms are therefore of critical importance in automated reasoning. 
In this paper we presented a new clausification algorithm, called \nfcnf{}, for formulas expressed in \folb{}. 
Our algorithm is a non-trivial extension of the recent \newcnf{} clausification
algorithm for standard first-order logic.  \nfcnf{} for FOOL    
%combines the translation of \folb{} to first-order logic and clausification. This combination allowed us to integrate into the translation clausification techniques such as 
introduces \skolem{} predicates over boolean variables, avoids equalities over
boolean variables, and uses formula naming and tautology elimination on complex
formulas. 
%The algorithm aims to produce sets of clauses that can be efficiently checked by first-order theorem provers. 
It also avoids excessively duplicating clauses and introducing too many new
symbols. 
Thanks to the our new  \nfcnf{}  algorithm, proving FOOL formulas requires neither an
axiomatisation of 
the boolean sort nor modifications in superposition calculus. 
We implemented our work in Vampire and
experimentally showed its benefits on a large number of examples.  
For future work we are interested in developing further criteria for
controlling naming and inlining expressions during clausification. Using FOOL
for more complex applications of program analysis is another interesting venue to
exploit.


%Our algorithms employs several new techniques for handling features of FOOL. In comparison with our old approach to translation of FOOL~\cite{VampireAndFOOL}
%\begin{enumerate}
%  \item boolean variables are skolemised with predicates and not functions;
%  \item boolean variables that do not need to be skolemised are exhaustively instantiated with the two possible boolean constants $\true$ and $\false$ in a way that does not increase the size of the translation;
%  \item no new equalities are ever introduced, definitions for common subexpressions always use guards;
%  \item naming of \ITE\ and \LETIN\ expressions is controlled by parameters.
%\end{enumerate}
%Moreover, unlike with the old translation, the set of clauses produced by the new algorithm requires neither an axiomatisation of the boolean sort nor modifications of superposition calculus.

%We implemented the extended \newcnf{} algorithm in the Vampire theorem prover. Our experimental results showed an increase of performance of Vampire compared to its version with the translation of \folb{} formulas to full first-order logic. We observed that new problems can be solved by expansion of \ITE\ and instantiation of boolean variables with boolean constants. We observed that both inlining and naming of \LETIN\ expressions can make a theorem prover succeed or fail. 

%For future work we are interested in developing syntactical criteria that determine whether a given \LETIN\ or \ITE\ expression should be named or inlined, or expanded or inlined, respectively. 

%We do not have problems that use let with functions with arguments. They would've been useful though.

\section*{Acknowledgments}
\label{sec:newcnf/acknowledgments}
We thank Andrew Reynolds for an explanation on how state-of-the-art SMT solvers deal with clausification and quantifiers. 

This work has been supported by the ERC Starting Grant 2014 SYMCAR 639270, the Wallenberg Academy Fellowship 2014, the Swedish VR grant D0497701 and the Austrian research project FWF S11409-N23.

% \appendix

\newpage
\section*{Appendix \thechapter.A. Imperative Programs with Loops and \ITE}
\label{sec:newcnf/examples}
{\small
\begin{tabular}[!h]{l@{\hskip 0pt}l}
  \begin{tabular}[!h]{l}
    \begin{minipage}[t]{0.45\textwidth}
      {\begin{lstlisting}[language=appendixcpp,title=\texttt{count\_two}]
int a[];
int x = 0, y = 0;
for (int i = 0; i < n; i++) {
  if (a[i] > 0) x++;
  else y++;
}
assert(x + y == n);
\end{lstlisting}}
    \end{minipage}
  \\
    \begin{minipage}[h]{0.45\textwidth}
      {\begin{lstlisting}[language=appendixcpp,title=\texttt{count\_three}]
int a[];
int x = 0, y = 0, z = 0;
for (int i = 0; i < n; i++) {
  if (a[i] < 0) {
    x++;
  } else {
    if (a[i] > 5) y++;
    else z++;
  }
}
assert(x + y + z == n);
\end{lstlisting}}
    \end{minipage}
  \\ \vspace{-10pt}
    \begin{minipage}[t]{0.45\textwidth}
      {\begin{lstlisting}[language=appendixcpp,title=\texttt{count\_two\_flag}]
int a[];
bool b;
int x = 0, y = 0;
for (int i = 0; i < n; i++) {
  b = a[i] > 0;
  if (b) x++;
  else y++;
}
assert(x + y == n);
\end{lstlisting}}
    \end{minipage}
  \end{tabular}
&
  \begin{tabular}[!h]{l}
    \begin{minipage}[t]{0.45\textwidth}
      {\begin{lstlisting}[language=appendixcpp,title=\texttt{two\_arrays}]
int a[], b[];
for (int i = 0; i < n; i++) {
  if (a[i] > b[i]) {
    int t = a[i];
    a[i] = b[i];
    b[i] = t;
  }
}
for (int i = 0; i < n; i++) {
  assert(a[i] <= b[i]);
}
\end{lstlisting}}
    \end{minipage}
  \\
    \begin{minipage}[t]{0.45\textwidth}
      {\begin{lstlisting}[language=appendixcpp,title=\texttt{three\_arrays}]
int a[], b[], c[];
for (int i = 0; i < n; i++) {
  if (a[i] > b[i]) {
    int t = a[i];
    a[i] = b[i];
    b[i] = t;
  }
  if (b[i] > c[i]) {
    int t = b[i];
    b[i] = c[i];
    c[i] = t;

    if (a[i] > b[i]) {
      t = a[i];
      a[i] = b[i];
      b[i] = t;
    }
  }
}
for (int i = 0; i < n; i++) {
  assert(a[i] <= b[i]);
  assert(b[i] <= c[i]);
}
\end{lstlisting}}
    \end{minipage}
  \end{tabular}
\end{tabular}
}


\def\paperFourContentsTitle{A FOOLish Encoding of the Next State Relations\\of Imperative Programs}
\def\paperFourChapterTitle{A FOOLish Encoding of the Next State Relations of Imperative Programs}
\def\paperFourAuthors{Evgenii~Kotelnikov, Laura~Kov\'{a}cs and Andrei~Voronkov}
\def\paperFourAbstract{Automated theorem provers are routinely used in program analysis and verification for checking program properties. These properties are translated from program fragments to formulas expressed in the logic supported by the theorem prover. Such translations can be complex and require deep knowledge of how theorem provers work in order for the prover to succeed on the translated formulas. Our previous work introduced FOOL, a modification of first-order logic that extends it with syntactical constructs resembling features of programming languages. One can express program properties directly in FOOL and leave translations to plain first-order logic to the theorem prover. In this paper we present a FOOL encoding of the next state relations of imperative programs. Based on this encoding we implement a translation of imperative programs annotated with their pre- and post-conditions to partial correctness properties of these programs. We present experimental results that demonstrate that program properties translated using our method can be efficiently checked by the first-order theorem prover Vampire.}
\def\paperFourPublication{Published in \EK{TODO: IJCAR 2018}}
\paperchapter{\paperFourContentsTitle}
             {\paperFourChapterTitle}
             {\paperFourAuthors}
             {\paperFourAbstract}
             {\paperFourPublication}
\label{chap:boogie}
% !TEX root = ../main.tex

% \documentclass{llncs}

% \usepackage{listings}
% \usepackage{color}
% \usepackage{amssymb,amsmath,mathrsfs,stmaryrd,dsfont,bbold,mathtools}
% \usepackage{lmodern}
% \usepackage{url}
% \usepackage{yfonts}
% \usepackage[inline]{enumitem}
% \usepackage{pifont}
% \usepackage{tikz}
% \usepackage{multirow}
% \usepackage[caption=false]{subfig}
% \usepackage{wrapfig}

% \usepackage[textsize=tiny,prependcaption]{todonotes}

\section{Introduction}
\label{sec:boogie/introduction}
% !TEX root = ../main.tex

Automated program analysis and verification requires discovering and proving program properties ensuring program correctness. 
 These program properties are usually expressed in combined theories of various data structures, such as integers and arrays. 
SMT solvers and first-order theorem provers that are used to check these properties need efficient handling of both theories and quantifiers. Moreover, 
formalisation of the program properties in the logic supported by the SMT solver or theorem prover plays a crucial role in making the prover succeed proving program correctness. 

The translation of program properties into logical formulas accepted by a theorem prover is not straightforward. The reason for this is a mismatch between the semantics of the programming language constructs and that of the input language of the theorem prover. If program properties are not directly expressible in the input language, one needs to implement a translation of these properties to the language of the theorem prover. Such translations can be complex and error prone. Furthermore, one might need deep knowledge of how theorem provers work to obtain formulas in a form that theorem provers can handle efficiently.

Program verification systems reduce the mismatch between program properties and their formalisation as logical formulas from two ends. On the one hand, intermediate verification languages, such as Boogie~\cite{leino2008boogie} and WhyML~\cite{DBLP:conf/esop/FilliatreP13}, are designed to represent programs and their properties in a way that is friendly for translations to logic. On the other hand, theorem provers extend their supported logics with syntactic constructs that mirror those of programming languages.

Our previous work introduced FOOL~\cite{FOOL}, a modification of many-sorted first-order logic (FOL). FOOL extends FOL with syntactical constructs such as \ITE\ and \LETIN\ expressions. These constructs can be used to naturally express program properties about conditional statements and variable updates. Users of a theorem prover that supports FOOL do not need to invent translations for these features of programming languages and can use features of FOOL directly. It allows the theorem prover to apply its own translation to FOL that it can use efficiently. We extended the Vampire theorem prover~\cite{Vampire13} to support FOOL~\cite{VampireAndFOOL} and designed an efficient clausification algorithm VCNF~\cite{FOOLCNF} for FOOL.

In summary, FOOL extends FOL with the following constructs. 
\begin{itemize}
  \item First-class boolean sort~--- one can define function and predicate symbols with boolean arguments and use quantifiers over the boolean sort.
  \item Boolean variables used as formulas.
  \item Formulas used as arguments to function and predicate symbols.
  \item Expressions of the form $\ite{\varphi}{s}{t}$, where $\varphi$ is a formula, and $s$ and $t$ are either both terms or formulas.
  \item Expressions of the form $\letindef{D_1;\ldots;D_k}{t}$, where $k > 0$, $t$ is either a term or a formula, and $D_1,\ldots,D_k$ are simultaneous definitions, each of the form
    \begin{enumerate}
      \item $\binding{f(x_1:\sigma_1,\ldots,x_n:\sigma_n)}{s}$, where $n \geq 0$, $f$ can be a function or a predicate symbol, and $s$ is either a term or a formula;
      \item $\binding{(c_1,\ldots,c_n)}{s}$, where $n > 1$, $c_1,\ldots,c_n$ are constant symbols of the sorts $\sigma_1,\ldots,\sigma_n$, respectively, and $s$ is a tuple expression. A tuple expression is inductively defined to be either
      \begin{enumerate}
        \item $(s_1,\ldots,s_n)$, where $s_1,\ldots,s_n$ are terms of the sorts $\sigma_1,\ldots,\allowbreak\sigma_n$, respectively;
        \item $\ite{\phi}{s_1}{s_2}$, where $\phi$ is a formula, and $s_1$ and $s_2$ are tuple expressions; or
        \item $\letindef{D_1;\ldots;D_k}{s'}$, where $D_1;\ldots;D_k$ are definitions, and $s'$ is a tuple expression.
      \end{enumerate}
    \end{enumerate}
\end{itemize}

To our knowledge, no other logic, efficiently implemented in automated theorem provers, contains these constructs. Some constructs of FOOL have been previously implemented in interactive and higher-order theorem provers. However, there was no special emphasis on the efficiency or friendliness of the translation for the following processing by automatic provers.

In this paper, we extend our previous work on FOOL by demonstrating the efficient use of FOOL for program analysis. To this end, we give an efficient encoding of the next state relations of imperative programs in FOOL. Let us motivate our work with the simple program on Figure~\ref{fig:boogie/simple-if}. This program contains an \verb'if' statement and assignments to integer variables. The \lstinline'assert' statement ensures that \lstinline'x' is never greater than \lstinline'y' after execution of the \verb'if' statement.

\begin{figure}
  \parbox{4.7cm}{
    \vspace{2em}\hspace{0.6cm}
    \begin{minipage}{3.2cm}
    \begin{lstlisting}[language=cpp]^^J
if (x > y) \{^^J
\ \ t := x;^^J
\ \ x := y;^^J
\ \ y := t;^^J
\}^^J
assert(x <= y);^^J
    \end{lstlisting}
    \end{minipage}
    \vspace{1.5em}
    \caption{An imperative program with an \texttt{if} statement.}
    \label{fig:boogie/simple-if}
  }
\quad
  \begin{minipage}{6cm}
\[
  \letnl{(x,y,t)}{\itenll{x > y}
                 {\letinnl{t}{x}
                          {\letinnl{x}{y}
                                   {\letinnl{y}{t}
                                            {(x,y,t)}}}}
                 {(x,y,t)}}
        {x \le y}
\]
    \caption{A FOOL encoding of the program assertion on Figure~\ref{fig:boogie/simple-if}.}
    \label{fig:boogie/simple-if-fool}
  \end{minipage}
\end{figure}

To check that the given program assertion holds using an automated theorem prover, one has to express this assertion as a logical formula. For that, one has to express the updated values of \verb'x' and \verb'y' after the sequence of assignments. For example, one can compute the updated value of each individual variable separately for each possible execution trace. However, this approach suffers from a bloated resulting formula that will contain duplicating parts of the program. A more common technique is to first convert a program to a static single assignment (SSA) form. This conversion introduces a new intermediate variable for each assignment and creates a smaller translated formula.

Both excessive naming and excessive duplication of program expressions can make the resulting logical formula very hard for a first-order theorem prover. The encoding of the next state relations of imperative programs given in this paper avoids both by using a FOOL formula that closely matches the structure of the original program (Section~\ref{sec:boogie/technique}). This way the decision between introducing new symbols and duplicating program expressions is left to the theorem prover that is better equipped to make it. The assertion of the program in Figure~\ref{fig:boogie/simple-if} is concisely expressed with our encoding as the FOOL formula on Figure~\ref{fig:boogie/simple-if-fool}.

While FOOL offers a concise representation of some programming constructs, the efficient implementation of FOOL poses a challenge for first-order theorem provers since their performance on various translations to CNF can be hampered by the (unintended) use of constructs interfering with their internal implementation, including the use of orderings, selection and the saturation algorithm. For example, to deal with the boolean sort, it is not uncommon to add an axiom like $(\forall x)(x = 0 \vee x = 1)$ for this sort. Even this simple axiom can cause a considerable growth of the search space, especially when used with certain term orderings. To address the challenge of dealing with full FOOL, one needs experimental comparison of various translations or various implementations of FOOL. Our paper is the first one to make such an experimental comparison.

Our encoding uses tuple expressions and \LETIN\ expressions with tuple definitions, available in FOOL. We extend and generalise the use of tuples in first-order theorem provers by introducing a polymorphic theory of first class tuples (Section~\ref{sec:boogie/tuples}). In this theory one can define tuple sorts and use tuples as terms.

Our encoding can be efficiently used in automated program analysis and verification. To demonstrate this, we report on our experimental results obtained by running Vampire on program verification problems (Section~\ref{sec:boogie/experiments}). These verification problems are partial correctness properties that we generated from a collection of imperative programs using an implementation of our encoding to FOOL as well as other tools.

\paragraph*{Contributions.} We summarise the main contributions of this paper below.
\begin{enumerate}
  \item We define an encoding of the next state relation of imperative programs in FOOL and show that it is sound (Section~\ref{sec:boogie/technique}). Using this encoding, we define a translation of certain properties of imperative programs to FOOL formulas. 
  \item We present a polymorphic theory of first class tuples and its implementation in Vampire (Section~\ref{sec:boogie/tuples}). To our knowledge, Vampire is the only superposition-based theorem prover to support this theory.
  %\item We present a collection of simple imperative programs annotated with their quantified properties and our encoding of partial correctness properties of these programs (Section~\ref{sec:boogie/experiments}). This collection of programs can be used for benchmarking program verification and program analysis tools.
  \item We present experimental results obtained by running Vampire on a collection of benchmarks expressing partial correctness properties of imperative programs (Section~\ref{sec:boogie/experiments}). We generated these benchmarks using an implementation of our encoding to FOOL and other tools. Our results show Vampire is more efficient on the FOOL encoding of partial correctness properties, compared with other translations.
\end{enumerate}

% This paper extends our previous work in~\cite{VampireAndFOOL} by (i) formalising the polymorphic theory of first class tuples in FOOL, (ii) implementing FOOL tuples in Vampire and using them to express partial program correctness, and (iii) experimental evaluation of using FOOL in program analysis and verification. 

\section{Polymorphic Theory of First Class Tuples}
\label{sec:boogie/tuples}
The use of tuple expressions in FOOL is limited. They can only occur on the right hand side of a tuple definition in \LETIN. One cannot use a tuple expression elsewhere, for example, as an argument to a function or predicate symbol.

In this section we describe the theory of first class tuples that enables a more generic use of tuples. This theory contains tuple sorts and tuple terms. Both of them are first class~--- one can define function and predicate symbols with tuple arguments, quantify over the tuple sort, and use tuple terms as arguments to function and predicate symbols. Tuple expressions in FOOL, combined with the polymorphic theory of tuples, are tuple terms.

\paragraph*{\bf Definition.}
The polymorphic theory of tuples is the union of theories of tuples parametrised by tuple arity $n > 0$ and sorts $\tau_1,\ldots,\tau_n$.

A theory of first class tuples is a first-order theory that contains a sort $(\tau_1,\ldots,\tau_n)$, function symbols $t:\tau_1\times\ldots\times\tau_n\to(\tau_1,\ldots,\tau_n)$, $\pi_1:(\tau_1,\ldots,\tau_n)\to\tau_1,\ldots,\pi_n:(\tau_1,\ldots,\tau_n)\to\tau_n$, and two axioms. The function symbol $t$ constructs a tuple from given terms, and function symbols $\pi_1,\ldots,\pi_n$ project a tuple to its individual elements. For simplicity we will write $\tuple{t_1,\ldots,t_n}$ instead of $t(t_1,\ldots,t_n)$ to mean a tuple of terms $t_1,\ldots,t_n$. The tuple axioms are
\begin{enumerate}
  \item exhaustiveness
    $$(\forall x_1:\tau_1)\ldots(\forall x_n:\tau_n)(\pi_1(\tuple{x_1,\ldots,x_n})\eql x_1 \wedge \ldots \wedge \pi_n(\tuple{x_1,\ldots,x_n})\eql x_n);$$
        % \begin{align*}
        %   &(\forall x_1:\tau_1)\ldots(\forall x_n:\tau_n)\\
        %   &\quad(\pi_1(\tuple{x_1,\ldots,x_n})\eql x_1 \wedge \ldots \wedge \pi_n(\tuple{x_1,\ldots,x_n})\eql x_n);
        % \end{align*}
  \item injectivity
        \begin{align*}
          &(\forall x_1:\tau_1)\ldots(\forall x_n:\tau_n)(\forall y_1:\tau_1)\ldots(\forall y_n:\tau_n)\\
          &\quad(\tuple{x_1,\ldots,x_n} \eql \tuple{y_1,\ldots,y_n} \implies x_1 \eql y_1 \wedge \ldots \wedge x_n \eql y_n).
        \end{align*}
\end{enumerate}

Tuples are ubiquitous in mathematics and programming languages. For example, one can use the tuple sort $(\R,\R)$ as the sort of complex numbers. Thus, the term $\tuple{a,b}$, where $a:\R$ and $b:\R$ represents a complex number $a+bi$. One can define the addition function $\mathit{plus}: (\R,\R) \times (\R,\R) \to (\R,\R)$ for complex numbers with the formula
\begin{equation}\label{eq:complex-tuples}
  % (\forall x:(\R,\R))(\forall y:(\R,\R))(\mathit{plus}(x,y)\eql \tuple{\pi_1(x) + \pi_1(y), \pi_2(x) + \pi_2(y)}),
  \begin{aligned}
    &(\forall x:(\R,\R))(\forall y:(\R,\R)) \\
    &\quad(\mathit{plus}(x,y)\eql \tuple{\pi_1(x) + \pi_1(y), \pi_2(x) + \pi_2(y)}),
  \end{aligned}
\end{equation} where $+$ denotes addition for real numbers.

Tuple terms can be used as tuple expressions in FOOL. If $\binding{(c_1,\ldots,c_n)}{s}$ is a tuple definition inside a \LETIN, where $c_1,\ldots,c_n$ are constant symbols of sorts $\tau_1,\ldots,\tau_n$, respectively, then tuple expression $s$ is a term of the sort $(\tau_1,\ldots,\tau_n)$.

It is not hard to extend tuple definitions to allow arbitrary tuple terms of the correct sort on the right hand side of $=$. For example, one can use a variable of the tuple sort. With such extension, Formula~\ref{eq:complex-tuples} can be equivalently expressed using a \LETIN\ with two simultaneous tuple definitions as follows
\begin{equation}\label{eq:complex-tuples2}
  % (\forall x:(\R,\R))(\forall y:(\R,\R))(\mathit{plus}(x,y)\eql \letinpar{(a,b)}{x}{(c,d)}{y}{\tuple{a+c,b+d}).}
  \begin{aligned}
    &(\forall x:(\R,\R))(\forall y:(\R,\R))\\
    &\quad(\mathit{plus}(x,y)\eql \letinpar{(a,b)}{x}{(c,d)}{y}{\tuple{a+c,b+d}).}
  \end{aligned}
\end{equation}

\paragraph*{\bf Implementation.}
Vampire implements reasoning with the polymorphic theory of tuples by adding corresponding tuple axioms when the input uses tuple sorts and/or tuple functions. For each tuple sort $(\tau_1,\ldots,\tau_n)$ used in the input, Vampire defines a term algebra~\cite{DBLP:conf/popl/KovacsRV17} with the single constructor $t$ and $n$ destructors $\pi_1,\ldots,\pi_n$. Then Vampire adds the corresponding term algebra axioms, which coincide with the tuple theory axioms.

Vampire reads formulas written in the TPTP language~\cite{tff0}. The TFX subset\footnote{\url{http://www.cs.miami.edu/~tptp/TPTP/Proposals/TFXTHX.html}} of TPTP contains a syntax for tuples and \LETIN\ expressions with tuple definitions. The sort $(\R,\R)$ is represented in TFX as \lstinline'[$real,$real]' and the term $\tuple{a+c,b+d}$ is represented as \lstinline'[$sum(a,c),$sum(b,d)]'. Formula~\ref{eq:complex-tuples2} can be expressed in TPTP as
\begin{lstlisting}[language=tptp]
tff(plus,type,plus:([$real,$real]*[$real,$real])>[$real,$real]).
tff(plus_def,axiom,
    ![X:[$real,$real],Y:[$real,$real]]:
      (plus(X,Y)=$let([[a:$real,b:$real],[c:$real,d:$real]],
                        [a,b]:=X;[c,d]:=Y,
                        [$sum(a,c),$sum(b,d)]))).
\end{lstlisting}

Vampire translates \LETIN\ with tuple definitions to clausal normal form of first-order logic using the \newcnf\ clausification algorithm~\cite{FOOLCNF}.


\section{Imperative Programs to FOOL}
\label{sec:boogie/technique}
In this section we discuss an efficient translation of imperative programs to FOOL. To formalise the translation we define a restricted imperative programming language and its denotational semantics in Section~\ref{sec:boogie/next-state/programming-language}. This language is capable of expressing variable updates, \ITE, and sequential composition. Then, we define an encoding of the next state relation for programs of this language, and state the soundness property of this encoding in Section~\ref{sec:boogie/next-state/encoding}. Finally, in Section~\ref{sec:boogie/next-state/translation} we show a translation that converts a program, annotated with its pre-conditions and post-conditions, to a FOOL formula that expresses the partial correctness property of that program. %We formulate the soundness property of the translation as well.

We give (rather standard) definitions of our programming language and its semantics and use them to define the main contributions of this section: the encoding of the next state relation (Definition~\ref{def:boogie/encoding}) and soundness of this encoding (Theorem~\ref{thm:boogie/encoding-soundness}).

\subsection{An Imperative Programming Language}\label{sec:boogie/next-state/programming-language}

% Our goal is to define a program language that on the one hand is simple enough to be illustrative, and on the other hand expressive enough to allow non-trivial meaningful programs.

We define a programming language with assignments to typed variables, \ITE, and sequential composition. We omit variable declarations in our language and instead assume for each program a set of program variables $V$ and a type assignment $\context$. $\context$ is a function that maps each program variable into a type. Each type is either $\pint$, $\pbool$, or $\parray(\sigma,\tau)$, where $\sigma$ and $\tau$ are types of array indexes and array values, respectively. In the sequel we will assume that $V$ and $\context$ are arbitrary but fixed.

Programs in our language select and update elements of arrays, including multidimentional arrays. We do not introduce a distinguished type for multidimentional arrays but instead use nested arrays. We write $\parray(\sigma_1,\ldots,\sigma_n,\tau)$, $n > 1$, to mean the nested array type $$\parray(\sigma_1,\parray(\ldots,\parray(\sigma_n,\tau)\ldots)).$$

\begin{definition}\label{def:boogie/expression}
An \emph{expression} of the type $\tau$ is defined inductively as follows.
\begin{enumerate}
  \item An integer $n$ is an expression of the type $\pint$.
  \item Symbols $\ttrue$ and $\tfalse$ are expressions of the type $\pbool$.
  \item If $\context(x)=\tau$, then $x$ is an expression of the type $\tau$.
  \item If $\context(x)=\parray(\sigma_1,\ldots,\sigma_n,\tau)$, $n > 0$, $\expr_1,\ldots,\expr_n$ are expressions of types $\sigma_1,\allowbreak\ldots,\allowbreak\sigma_n$, respectively, then $x[\expr_1,\ldots,\expr_n]$ is an expression of the type $\tau$.
  \item If $\expr_1$ and $\expr_2$ are expressions of the type $\tau$, then $\expr_1 \eql \expr_2$ is an expression of the type $\pbool$.
  \item If $\expr_1$ and $\expr_2$ are expressions of the type $\pint$, then $-\expr_1$, $\expr_1+\expr_2$, $\expr_1-\expr_2$, $\expr_1\times\expr_2$ are expressions of the type $\pint$.
  \item If $\expr_1$ and $\expr_2$ are expressions of the type $\pint$, then $\expr_1<\expr_2$ is an expression of the type $\pbool$.
  \item If $\expr_1$ and $\expr_2$ are expression of the type $\pbool$, then $\neg\expr_1$, $\expr_1\vee\expr_2$, $\expr_1\wedge\expr_2$ are expressions of the type $\pbool$. \QED
\end{enumerate}
\end{definition}

%\begin{definition}
%An \emph{lvalue} $\ell$ of the type $\tau$ is defined inductively as follows.
%\begin{enumerate}
%  \item If $\context(x)=\tau$, then $x$ is an lvalue of the type $\tau$.
%  \item If $\ell$ is an lvalue of the type $\arrayt(\sigma,\tau)$ and $i$ is an expression of the type $\sigma$, then $\ell[i]$ is an lvalue of the type $\tau$.
%\end{enumerate}
%\end{definition}

\begin{definition}\label{def:boogie/statement}
A \emph{statement} is defined inductively as follows.
\begin{enumerate}
  \item $\emptyStatement$ is an empty statement.
  \item If $\context(x_1)=\tau_1,\ldots,\context(x_n)=\tau_n,n\ge1$ and $\expr_1,\ldots,\expr_n$ are expressions of the types $\tau_1,\ldots,\tau_n$, respectively, then $\assigns{x_1,\ldots,x_n}{\expr_1,\ldots,\expr_n}$ is a statement.
  \item If $\context(x)=\parray(\sigma_1,\ldots,\sigma_n,\tau)$, $n\ge1$, and $\expr_1,\ldots,\allowbreak\expr_n,\allowbreak\expr$ are expressions of types $\sigma_1,\ldots,\sigma_n,\tau$, respectively, then $\assigns{x[\expr_1,\ldots,\expr_n]}{\expr}$ is a statement.
  \item If $\expr$ is an expression of the type $\pbool$, $\stmt_1$ and $\stmt_2$ are statements, and at least one of $s_1$, $s_2$ is not $\emptyStatement$, then $\ite{\expr}{\stmt_1}{\stmt_2}$ is a statement.
  \item If $\stmt_1$ and $\stmt_2$ are statements and neither of them is $\emptyStatement$, then $\seq{\stmt_1}{\stmt_2}$ is a statement. \QED
\end{enumerate}
\end{definition}

We say that $x_1,\ldots,x_n$ in the statement $\assigns{x_1,\ldots,x_n}{\expr_1,\ldots,\expr_n}$ and $x$ in the statement $\assigns{x[\expr_1,\ldots,\expr_n]}{\expr}$ are \emph{assigned program variables}. For each statement $s$ we denote by $\updates{s}$ the set of all assigned program variables that occur in $s$.

We define the semantics of the programming language by an interpretation function $\interp{-}$ for types, expressions and statements. The interpretation of a type is a set: $\interp{\pint}=\Z$, $\interp{\pbool}=\{0,1\}$, and $\interp{\parray(\tau,\sigma)} = \interp{\tau}\to\interp{\sigma}$. The interpretation of expressions and statements is defined using \emph{program states}, that is, mappings of program variables $x\in V$, $\context(x)=\tau$ to elements of $\interp{\tau}$.

\begin{definition}\label{def:boogie/interpret-expression}
Let $\expr$ be an expression of the type $\tau$. The \emph{interpretation} $\interp{\expr}$ is a mapping from program states to $\interp{\tau}$ defined inductively as follows.
\begin{enumerate}
  \item $\interp{n}$ maps each state to $n$, where $n$ is an integer.
  \item $\interp{\ttrue}$ maps each state to 1.
  \item $\interp{\tfalse}$ maps each state to 0.
  \item $\interp{x}$ maps each $\State$ to $\State(x)$.
  \item $\interp{x[\expr_1,\ldots,\expr_n]}$ maps each $\State$ to $\State(x)(\interp{\expr_1}(\State))\ldots(\interp{\expr_n}(\State))$.
  \item $\interp{\expr_1 \oplus \expr_2}$ maps each $\State$ to $\interp{\expr_1}(\State) \oplus \interp{\expr_2}(\State), \text{where } \oplus \in \{ \eql, +, -, \times, <, \vee, \wedge \}.$
  \item $\interp{\neg\expr}$ maps each $\State$ to $\neg\interp{\expr}(\State)$.\QED
\end{enumerate}
% \[
% \begin{aligned}
% \interp{n}(\State) &= n,\text{ where $n$ is an integer.} \\
% \interp{\ttrue}(\State) &= 1. \\
% \interp{\tfalse}(\State) &= 0. \\
% \interp{x}(\State) &= \State(x). \\
% \interp{x[\expr_1,\ldots,\expr_n]}(\State) &= \State(x)(\interp{\expr_1}(\State))\ldots(\interp{\expr_n}(\State)). \\
% % \interp{\expr_1 \oplus \expr_2}(\State) &= \begin{aligned}[t]
% %                                              &\interp{\expr_1}(\State) \oplus \interp{\expr_2}(\State),\\
% %                                              &\text{where $\oplus \in \{ \eql, +, -, \times, <, \vee, \wedge \}$.}
% %                                            \end{aligned}\\
% \interp{\expr_1 \oplus \expr_2}(\State) &= \interp{\expr_1}(\State) \oplus \interp{\expr_2}(\State), \text{where } \oplus \in \{ \eql, +, -, \times, <, \vee, \wedge \}.\\
% \interp{-\expr}(\State) &= -\interp{\expr}(\State). \\
% \interp{\neg\expr}(\State) &= \neg\interp{\expr}(\State).\quad\quad\quad\quad\quad\quad\quad\quad\quad\quad\quad\quad\quad\quad\quad\quad\quad\quad\quad\;\QED
% \end{aligned}
% \]
\end{definition}

\begin{definition}\label{def:boogie/interpret-statement}
Let $\stmt$ be a statement. The \emph{interpretation} $\interp{\stmt}$ is a mapping between program states defined inductively as follows.
\begin{enumerate}
  \item $\interp{\emptyStatement}$ is the identity mapping.
  \item $\interp{\assigns{x_1,\ldots,x_n}{\expr_1,\ldots,\expr_n}}$ maps each $\State$ to $\State'$ such that $\State'(x_i)=\interp{\expr_i}(\State)$ for each $1 \le i \le n$ and otherwise coincides with $\State$.
  \item $\interp{\assigns{x[\expr_1,\ldots,\expr_n]}{\expr}}$ maps each $\State$ to $\State'$ such that $$\State'(x)(\interp{\expr_1}(\State))\ldots(\interp{\expr_n}(\State))=\interp{e}(\State)$$ and otherwise coincides with $\State$.
  \item $\interp{\ite{\expr}{\stmt_1}{\stmt_2}}$ maps each $\State$ to $\interp{\stmt_1}(\State)$ if $\interp{\expr}(\State)=1$ and to $\interp{\stmt_2}(\State)$ otherwise.
  \item $\interp{\seq{\stmt_1}{\stmt_2}}$ is $\interp{\stmt_2}\circ\interp{\stmt_1}$. \QED  
\end{enumerate}
% \[
% \begin{aligned}
% \interp{\emptyStatement}(\State) &= \State.\\
% \interp{\assigns{x_1,\ldots,x_n}{\expr_1,\ldots,\expr_n}}(\State) &= \State',
%   \begin{aligned}[t]
%     &\text{such that $\State'(x_i)=\interp{\expr_i}(\State)$ for each $1 \le i \le n$}\\*[-0.5ex]
%     &\text{and otherwise coincides with $\State$.}
%   \end{aligned}\\
% \interp{\assigns{x[\expr_1,\ldots,\expr_n]}{\expr}}(\State) &= \EK{TODO}. \\
% \interp{\ite{\expr}{\stmt_1}{\stmt_2}}(\State) &= \left\{ \begin{array}{l}
%                                                             \interp{\stmt_1}(\State),\text{ if }\interp{\expr}(\State)=1;\\
%                                                             \interp{\stmt_2}(\State),\text{ otherwise.}
%                                                           \end{array} \right.\\
% \interp{\seq{\stmt_1}{\stmt_2}}(\State) &= \interp{\stmt_2}(\interp{\stmt_1}(\State)). \\
% \end{aligned}
% \]\QED
\end{definition}


\subsection{Encoding the Next State Relation}\label{sec:boogie/next-state/encoding}

Our setting is FOOL extended with the theory of linear integer arithmetic, the polymorphic theory of arrays~\cite{VampireAndFOOL}, and the polymorphic theory of first class tuples (Section~\ref{sec:boogie/tuples}). The theory of linear integer arithmetic includes the sort $\Z$, the predicate symbol $<$, and the function symbols $+$, $-$, and $\times$. The theory of arrays includes the sort $\arrayt(\tau,\sigma)$ for all sorts $\tau$ and $\sigma$, and function symbols $\selectf$ and $\storef$. The function symbol $\selectf$ represents a binary operation of extracting an array element by its index. The function symbol $\storef$ represents a ternary operation of updating an array at a given index with a given value. We point out that sorts $\bool$, $\Z$, and $\arrayt(\sigma,\tau)$ mirror types $\pbool$, $\pint$ and $\parray(\sigma,\tau)$ of our programming language, and have the same interpretations.

We represent multidimentional arrays in FOOL as nested arrays\footnote{Multidimentional arrays can be represented in FOOL also as arrays with tuple indexes. We do not discuss such representation in this work.}. To this end we
\begin{enumerate*}[label=(\roman*)]
%  \item write $\arrayt(\sigma_1,\ldots,\sigma_n,\tau)$, where $n > 1$, to mean $\arrayt(\sigma_1,\allowbreak\arrayt(\ldots,\allowbreak\arrayt(\sigma_n,\allowbreak\tau)\ldots));$
  \item inductively define $\select{a}{i_1,\allowbreak\ldots,\allowbreak i_n}$, where $n > 1$, to be $\select{\select{a}{i_1}}{i_2,\ldots,i_n}$; and
  % \select{a}{i_1,\ldots,i_n} = \select{\select{a}{i_1}}{i_2,\ldots,i_n}
  \item inductively define $\store{a}{i_1,\allowbreak\ldots,\allowbreak i_n}{e}$, where $n > 1$, to be $\store{a}{i_1}{\store{\select{a}{i_1}}{i_2,\allowbreak\ldots,\allowbreak i_n}{e}}.$
  % \store{a}{i_1,\ldots,i_n}{e} = \store{a}{i_1}{\store{\select{a}{i_1}}{i_2,\ldots,i_n}{e}}
\end{enumerate*}

Our encoding of the next state relation produces FOOL terms that use program variables as constants and do not use any other uninterpreted function or predicate symbols. In the sequel we will only consider such FOOL terms. For these FOOL terms, $\context$ is a type assignment and each program state can be extended to a $\context$-interpretation, the details of this extension are straightforward (we refer to~\cite{FOOL} for the semantics of FOOL). We will use program states as $\context$-interpretations for FOOL terms. For example we will write $\eval{t}{\State}$ for the value of $t$ in $\State$, where $t$ is a FOOL term and $\State$ is a program state. We will say that a program state $\State$ satisfies a FOOL formula $\phi$ if $\eval{\phi}{\State} = 1$.

To define the encoding of the next state relation we first define a translation of expressions to FOOL terms. Our encoding applies this translation to each expression that occurs inside a statement.

\begin{definition}
Let $\expr$ be an expression of the type $\tau$. $\translate{\expr}$ is a FOOL term of the sort $\tau$, defined inductively as follows.
\[
\begin{aligned}
\translate{n} &= n, \text{where $n$ is an integer}. \\
\translate{\ttrue} &= \true. \\
\translate{\tfalse} &= \false. \\
\translate{x} &= x. \\
\translate{x[\expr_1,\ldots,\expr_n]} &= \select{x}{\translate{\expr_1},\ldots,\translate{\expr_n}}. \\
\translate{\expr_1\oplus\expr_2} &= \translate{\expr_1} \oplus \translate{\expr_2}, \text{where } \oplus \in \{ \eql, +, -, <, \times, \vee, \wedge \}. \\
% \translate{\expr_1\oplus\expr_2} &= \begin{aligned}[t]
%                                       &\translate{\expr_1} \oplus \translate{\expr_2},\\
%                                       &\text{where $\oplus \in \{ \eql, +, -, <, \times, \vee, \wedge \}$.}
%                                     \end{aligned} \\
\translate{-\expr} &= -\translate{\expr}. \\
\translate{\neg\expr} &= \neg\translate{\expr}.\quad\quad\quad\quad\quad\quad\quad\quad\quad\quad\quad\quad\quad\quad\quad\quad\quad\quad\,\,\QED
\end{aligned}
\]
\end{definition}

\begin{lemma}\label{lemma:boogie/transform-expressions}
$\eval{\translate{\expr}}{\State} = \interp{\expr}(\State)$ for each expression $\expr$ and state $\State$. \QED
\end{lemma}
\begin{proof}
By structural induction on $\expr$. %\QED
\end{proof}

%\begin{definition}\label{def:boogie/updates}
%Let $\stmt$ be a statement. The set of variables updated in $\stmt$, denoted as $\updates{s}$, is defined inductively as follows.
%\[
%\begin{aligned}
%\updates{\emptyStatement} &= \emptyset. \\
%\updates{\assigns{x}{\expr}} &= \{ x \}. \\
%\updates{\assigns{x[\expr_1,\ldots,\expr_n]}{\expr}} &= \{ x \}. \\
%\updates{\ite{\expr}{\stmt_1}{\stmt_2}} &= \updates{\stmt_1} \cup \updates{\stmt_2}. \\
%\updates{\seq{\stmt_1}{\stmt_2}} &= \updates{\stmt_1} \cup \updates{\stmt_2}.
%\end{aligned}
%\]\QED
%\end{definition}

\begin{definition}\label{def:boogie/encoding}
Let $\stmt$ be a statement. $\tuplifyRel{\stmt}$ is a mapping between FOOL terms of the same sort, defined inductively as follows.
\begin{enumerate}
  \item $\tuplifyRel{\emptyStatement}$ is the identity mapping.
  \item $\tuplifyRel{\assigns{x_1,\ldots,x_n}{\expr_1,\ldots,\expr_n}}$ maps $t$ to $$\letin{(x_1,\ldots,x_n)}{(\translate{\expr_1},\ldots,\translate{\expr_n})}{t}.$$
  \item $\tuplifyRel{\assigns{x[\expr_1,\ldots,\expr_n]}{\expr}}$ maps $t$ to $$\letin{x}{\storef(x,\translate{\expr_1},\ldots,\translate{\expr_n},\translate{\expr})}{t}.$$
%  \item $\tuplifyRel{\ite{\expr}{\stmt_1}{\stmt_2}}$ maps $t$ to $$\letin{(x_1,\ldots,x_n)}{t'}{t},$$ where $$t' = \ite{\translate{\expr}}{\tuplify{\stmt_1}{(x_1,\ldots,x_n)}}{\tuplify{\stmt_2}{(x_1,\ldots,x_n)}}$$ and $\updates{s_1}\cup\updates{s_2} = \{x_1,\ldots,x_n\}.$
  \item $\tuplifyRel{\ite{\expr}{\stmt_1}{\stmt_2}}$ maps $t$ to \[
    \letnl{(x_1,\ldots,x_n)}{\itenl{\translate{\expr}}
                                   {\tuplify{\stmt_1}{(x_1,\ldots,x_n)}}
                                   {\tuplify{\stmt_2}{(x_1,\ldots,x_n)}}}
          {t,}
  \] where $\updates{s_1}\cup\updates{s_2} = \{x_1,\ldots,x_n\}$.\\[-0.5em]
  \item $\tuplifyRel{\seq{\stmt_1}{\stmt_2}}$ is $\tuplifyRel{\stmt_1}\circ\tuplifyRel{\stmt_2}$. \QED
\end{enumerate}
% \[
% \begin{aligned}
%   \tuplify{\emptyStatement}{t} &= t. \\
%   \tuplify{\assigns{x_1,\ldots,x_n}{\expr_1,\ldots,\expr_n}}{t} &=
%     \letin{(x_1,\ldots,x_n)}{(\translate{\expr_1},\ldots,\translate{\expr_n})}
%           {t}. \\
%   \tuplify{\assigns{x[\expr_1,\ldots,\expr_n]}{\expr}}{t} &=
%     \letin{x}{\storef(x,\translate{\expr_1},\ldots,\translate{\expr_n},\translate{\expr})}
%           {t}. \\
%   \tuplify{\ite{\expr}{\stmt_1}{\stmt_2}}{t} &=
%     \letnl{(x_1,\ldots,x_n)}{\itenl{\translate{\expr}}
%                                    {\tuplify{\stmt_1}{(x_1,\ldots,x_n)}}
%                                    {\tuplify{\stmt_2}{(x_1,\ldots,x_n)}}}
%           {t\text{, where $\updates{s_1}\cup\updates{s_2} = \{x_1,\ldots,x_n\},n>0$.}} \\
%   \tuplify{\seq{\stmt_1}{\stmt_2}}{t} &= \tuplify{\stmt_1}{\tuplify{\stmt_2}{t}}. \\
% \end{aligned}
% \]\QED
\end{definition}

The following theorem is the soundness property of translation $\tuplifyT$. Essentially, it states that $\tuplifyT$ encodes the semantics of a given statement as a FOOL formula.

\begin{theorem}\label{thm:boogie/encoding-soundness}
$\eval{\tuplify{\stmt}{t}}{\State} = \eval{t}{\interp{\stmt}(\State)}$ for each statement $\stmt$, state $\State$ and FOOL term $t$. \QED
\end{theorem}
\begin{proof}
By structural induction on $\stmt$. %\QED
\end{proof}

\subsection{Encoding the Partial Correctness Property}\label{sec:boogie/next-state/translation}

We use the encoding of the next state relation to generate partial correctness properties of programs annotated with their pre-conditions and post-conditions. %These properties can then be checked automatically by a theorem prover. 

We define an \emph{annotated program} to be a Hoare triple $\hoare{\varphi}{\stmt}{\psi}$, where $\stmt$ is a statement, and $\varphi$ and $\psi$ are formulas in first-order logic. We say that $\hoare{\varphi}{\stmt}{\psi}$ is correct if for each program state $\State$ that satisfies $\varphi$, $\interp{\stmt}(\State)$ satisfies $\psi$. We translate each annotated program $\hoare{\varphi}{\stmt}{\psi}$ to the FOOL formula $\varphi \implies \tuplify{\stmt}{\psi}$.

\begin{theorem}\label{thm:boogie/translation-soundness}
Let $\hoare{\varphi}{\stmt}{\psi}$ be an annotated program. The FOOL formula $\varphi \implies \tuplify{\stmt}{\psi}$ is valid iff $\hoare{\varphi}{\stmt}{\psi}$ is correct. \QED
\end{theorem}
\begin{proof}
Directly follows from Theorem~\ref{thm:boogie/encoding-soundness}. %\QED
\end{proof}

We point out the following two properties of the encoding $\tuplifyT$. First, the size of the encoded formula is $O(v\cdot n)$, where $v$ is the number of variables in the program and $n$ is the program size as each program statement is used once with one or two instances of $(x_1,\ldots,x_n)$. Second, the encoding does not introduce any new symbols. When we translate program correctness properties to FOL, both an excessive number of new symbols and an excessive size of the translation might make the encoded formula hard for a theorem prover. Instead of balancing between the two, encoding to FOOL leaves the decision to the theorem prover.


\section{Experiments}
\label{sec:boogie/experiments}
% !TEX root = ../main.tex

%Section~\ref{sec:boogie/next-state/translation} described a translation from imperative programs to FOOL formulas that encode the partial correctness properties of these programs. These partial correctness properties can be automatically checked by a theorem prover that supports reasoning with FOOL. 
In this section we describe our experiments on comparing the performance of the Vampire theorem prover~\cite{Vampire13} on FOOL and on translations of program properties to FOL. We used a collection of 50 programs written in the Boogie verification language~\cite{leino2008boogie}. Each of these programs uses only variable assignments, \ITE\ statements, and sequential composition and is annotated with its pre-conditions and post-conditions, expressed in first-order logic. From this collection of programs we generated the following three sets of benchmarks.

% We also compare our work to SMT-based verification, in particular to the SMT solver Z3~\cite{Z3} for proving partial correctness of programs formulated as SMT-LIB formulas~\cite{BarFT-SMTLIB}.

%We run Vampire on a set of benchmarks that encode partial correctness problems. We generated these problems from our collection of imperative programs using \begin{enumerate*}[label=(\roman*)]\item the front end of the Boogie~\cite{DBLP:conf/fmco/BarnettCDJL05} verifier, \item our implementation of the translation from Section~\ref{sec:boogie/next-state/translation} named Voogie, and \item the BLT~\cite{CF-iFM17} translator\end{enumerate*}.

\begin{enumerate}
  \item 50 problems in first-order logic written in the SMT-LIB language~\cite{SMT-LIB}. We generated these problems by running the front end of the Boogie~\cite{DBLP:conf/fmco/BarnettCDJL05} verifier.
  \item 50 FOOL problems with tuples generated by running our implementation of the translation from Section~\ref{sec:boogie/next-state/translation}, named Voogie.
  \item 50 FOOL problems generated by running the BLT~\cite{CF-iFM17} translator.
\end{enumerate}

We point out that in our experiments we do not aim to compare methods of program verification or specific verification tools. Rather, we compare different ways of translating realistic verification problems for theorem provers.

In what follows, we describe the collection of imperative programs used in our experiments (Section~\ref{sec:boogie/experiments/programs}) and discuss our set of benchmarks (Section~\ref{sec:boogie/experiments/problems}). All properties that we deal with use integers and arrays, as well as universal and existential quantifiers. To verify these properties one has to reason in the combination of theories and quantifiers. We briefly describe how Vampire implements this kind of reasoning in Section~\ref{sec:boogie/experiments/vampire}. Our experimental results are summarised in Tables~\ref{table:boogie/boogie-results}--\ref{table:boogie/blt-results} and discussed in Section~\ref{sec:boogie/experiments/results}.

\newcommand{\bad}{\mathit{bad}}
\newcommand{\iter}{i}

\subsection{Examples of Imperative Programs}\label{sec:boogie/experiments/programs}
We demonstrate the work of our translation on a collection of imperative programs that only use variable assignments, \ITE\ statements, and sequential composition. Unfortunately, no large collections of such programs are available. There are many benchmarks for software verification tools, but most of them use control flow statements not covered in this work, such as gotos and exceptions. We also cannot use benchmarks from the hardware verification and model checking communities, because they are mostly about boolean values and bit-vectors. For our experiments we generated our own imperative programs in two steps described below.

%\begin{enumerate}
%  \item 
First, we crafted 10 programs that implement textbook algorithms and solutions to program verification competitions. Each program uses variables of the integer, boolean, and array type. Each program contains a single \verb'while' loop
of the form $\while{\expr}{\stmt}$, where $\expr$ is a boolean expression and $\stmt$ is a statement. In addition, each program contains 
variable assignments, \ITE\ statements, and sequential composition. We annotated each program with its pre-condition $\varphi$ and each loop with its invariant $\psi$. The formulas $\varphi$ and $\psi$ are expressed in first-order logic.
%  \item

Then, we unrolled the loop of each program $k$ times, where $k$ is an integer between 1 and 5. This resulted in 50 loop-free programs that retain the annotated properties.
%Using the notation above, the pre-condition of this loop-free annotated program is $\varphi \wedge \chi$. The post-condition states that $\psi$ holds after exactly $k$ iterations of the loop. We expressed the post-condition as $\neg\bad \implies \psi$, where $\bad$ is a fresh boolean variable. $\bad$ encodes the under-specified state of the program. $\bad$ is set to $\true$ before each unrolled iteration of the loop if $\expr$ does not hold, and at the end of the program if $\expr$ does hold.
%
%\end{enumerate}
%
%Each of the 10 programs with loops consists of declarations of variables and a single loop of the form $\while{\expr}{\stmt}$, where $\expr$ is a boolean expression and $\stmt$ is a statement. $\expr$ and $\stmt$ fit the definitions of an expression and a statement, respectively, given in Section~\ref{sec:boogie/technique}. Each program is annotated with its pre-condition $\varphi$ and its post-condition $\psi$, and each loop is annotated with its loop invariant $\chi$. $\varphi$, $\psi$, and $\chi$ are formulas in first-order logic.
%
%While transforming programs with loops into loop-free programs by taking finite unrollings of the loop, we also annotated the resultin loop-free programs We transformed each program with a loop into an annotated program by unrolling the loop $k \ge 1$ times. 
%The pre-condition of this annotated program is $\varphi \wedge \chi$. The post-condition states that $\psi$ holds after exactly $k$ iterations of the loop. We expressed the post-condition as $\neg\bad \implies \psi$, where $\bad$ is a fresh boolean variable. $\bad$ encodes the underspecified state of the program. $\bad$ is set to $\true$ before each unrolled iteration of the loop if $\expr$ does not hold, and at the end of the program if $\expr$ does hold.
%\end{enumerate}
Each program encodes the loop invariant property of the original program. 
% AV : commented out since not true in view of pre-conditions:
% While in order to verify a loop invariant one only needs to unroll the loop once,
Multiple unrollings provide us with programs with long sequences of variables updates, \ITE\ statements and compositions, which are convenient for our experiments.
%
%In summary, our loop unrolling program transformation consisted of the following steps.
Our loop unrolling program transformation consisted of the following steps.
\begin{enumerate}
  \item Introduce a fresh boolean variable $\bad$ that encodes the under-specified state of the program.
  \item Construct a guarded loop iteration $\iter$ as $$\seq{\ite{\neg e}{\assigns{\bad}{\true}}{\emptyStatement}}{\stmt}.$$
  \item Construct a sequence of iterations $\seq{\seq{\iter}{\ldots}}{\iter}$, where $\iter$ is repeated $k$ times. 
  \item Finally, construct the annotated program $$\hoare{\varphi \wedge \psi}{\seq{\seq{\iter}{\ldots}}{\iter}}{\neg\bad \implies \psi}.$$
\end{enumerate}

It is not hard to show that if a program with a loop satisfies its specification, then the Hoare triple resulting in step 4 of the above transformation also holds. 

%Let us make the following remark regarding our examples and their formalisation in Boogie.
We wrote our example programs with loops as well as their loop-free unrolled versions in the Boogie language. Boogie can unroll loops automatically, but introduces \verb'goto' statements that our translation does not support. For this reason, we used the loop unrolling described above. 

%Each of our Boogie programs implements a single procedure \verb'main'. We annotated every procedure with its pre-conditions using the keyword \verb'requires' and its post-conditions using the keyword \verb'ensures'. We annotated each loop with its invariants using the keyword \verb'invariant'. Some of our programs use multidimensional arrays. Boogie supports both multidimensional arrays and nested arrays. We used the latter to represent our programs.

An example of our loop unrolling is available at \url{http://www.cse.chalmers.se/~evgenyk/ijcar18/}. It shows the \texttt{maxarray} program with a loop from our collection and a program generated from \texttt{maxarray} by unrolling its loop twice.

%Appendix~\ref{sec:boogie/appendix} contains an example of our loop unrolling. Figure~\ref{fig:boogie/loop} shows the \texttt{maxarray} program with a loop\footnote{This program is a solution to Challenge 1 of COST IC0701 Verification Competition 2011.} from our collection and Figure~\ref{fig:boogie/loop-free} shows a program generated from \verb'maxarray' by unrolling its loop twice. 

\subsection{Benchmarks}\label{sec:boogie/experiments/problems}
We used the 50 annotated loop-free programs and generated their partial correctness statements using Boogie, Voogie and BLT. These statements are encoded as unsatisfiable problems in first-order logic and FOOL.
Our collection of imperative programs with loops, their loop-free unrollings and benchmarks expressed in the TPTP language \cite{TPTP} is available at \url{http://www.cse.chalmers.se/~evgenyk/ijcar18/}. The TPTP benchmarks are also available, along with other FOOL problems, on the TPTP website \url{http://tptp.org}.

The Boogie verifier generates verification conditions as formulas in first-order logic written in the SMT-LIB language and uses the SMT solver Z3~\cite{Z3} to check these formulas. We ran Boogie with the option \verb'/proverLog' to print the generated formulas on each of our annotated loop-free programs and in this way obtained a collection of 50 SMT-LIB benchmarks.

Voogie is our implementation of the translation described in Section~\ref{sec:boogie/technique}. It takes as input programs written in a fragment of the Boogie language and generates FOOL formulas written in the TPTP language. The source code of Voogie is available at \url{https://github.com/aztek/voogie}.

The fragment of the Boogie language supported by Voogie can be seen as the smallest fragment that is sufficient to represent the loop-free programs in our collection. This fragment consists of
\begin{enumerate*}[label=(\roman*)]
  \item top level variable declarations;
  \item a single procedure \verb'main' annotated with its pre- and post-conditions;
  \item assignments to variables, including parallel assignments, and assignments to array elements;
  \item \ITE\ statements; and
  \item arithmetic and boolean operations.
\end{enumerate*}
% \begin{enumerate}
%   \item Top level variable declarations of types \verb'int', \verb'bool', or the type of nested arrays.
%   \item A single procedure \verb'main' with property annotations marked with keywords \verb'requires' and \verb'ensures'.
%   \item Assignments to variables, including parallel assignments, and assignments to array elements.
%   \item \ITE\ statements.
%   \item Arithmetic and boolean operations.
% \end{enumerate}
Running Voogie on each loop-free program in our collection gave us 50 TPTP benchmarks.
An example of the TPTP benchmark obtained from running Voogie on the \texttt{maxarray} program with its loops unrolled twice is available at \url{http://www.cse.chalmers.se/~evgenyk/ijcar18/}.
%Figure~\ref{fig:boogie/loop-free-translation} in Appendix~\ref{sec:boogie/appendix} shows the TPTP code of the partial correctness property of the program from Figure~\ref{fig:boogie/loop-free} obtained with Voogie. 

BLT (Boogie Less Triggers)~\cite{CF-iFM17} is an automatic tool that takes Boogie programs as input and generates their verification conditions in first-order logic written in the TPTP language. BLT has an experimental feature of generating FOOL formulas with tuple \LETIN\ and tuple expressions to represent next state values of program variables in a style similar to Voogie. At the time of our experiments, this feature was not stable enough, and we did not enable it. Running BLT with its default configuration on each of the 50 loop-free programs in our collection gave us 50 TPTP benchmarks.

The representation of program expressions coincides in all three translations. All translations use the theory of linear integer arithmetic and the theory of arrays as realised in their respective languages.

% The translations represent sequences of \ITE\ statements and variable assignments differently. We summarise the differences below.

% The Boogie translation (see~\cite{DBLP:journals/ipl/Leino05} for details)
% \begin{enumerate}
%   \item introduces a fresh constant for each state of each program variable;
%   \item encodes each variable assignment as logical equality and each sequence of assignments as logical conjunction;
%   \item introduces a fresh boolean constant for each branch of each \ITE\ statement. Each such constant is defined using a \LETIN\ expression and encodes the next state relation of the branch.
% \end{enumerate}

% The Voogie translation 
% \begin{enumerate}
%   \item does not introduce any new symbols;
%   \item encodes each variable assignment using a \LETIN\ expression;
%   \item encodes each \ITE\ statement as a \ITE\ expression inside a \LETIN\ expression with a tuple definition.
% \end{enumerate}

% The BLT translation (see~\cite{CF-iFM17} for details)
% \begin{enumerate}
%   \item encodes each variable assignment using a \LETIN\ expression;
%   \item encodes each \ITE\ statement as a \ITE\ expression;
%   \item duplicates the post-condition of the program in each branch of \ITE.
% \end{enumerate}

\subsection{Theories and Quantifiers in Vampire}\label{sec:boogie/experiments/vampire}
Vampire's main algorithm is saturation of a set of first-order clauses using the resolution and superposition calculus. Vampire also implements the AVATAR architecture~\cite{DBLP:conf/cav/Voronkov14} for splitting clauses. The idea behind AVATAR is to use a SAT or an SMT solver to guide proof search. AVATAR selects sub-problems for the saturation-based prover to tackle by making decisions over a propositional abstraction of the clause search space. The \verb'-sas' option of Vampire selects the SAT solver.

Vampire handles theories by automatically adding theory axioms to the search space whenever an interpreted sort, function, or predicate is found in the input. This approach is incomplete for theories such as linear and non-linear integer and real arithmetic, but shows good results in practice. The \verb'-tha' option of Vampire with values \verb'on' and \verb'off' controls whether theory axioms are added.

A recent work~\cite{DBLP:conf/gcai/RegerB0V16} lifted AVATAR to be modulo theories by replacing the SAT solver by an SMT solver, ensuring that the sub-problem is theory-consistent in the ground part. The result is that the saturation prover and the SMT solver deal with the parts of the problem to which they are best suited. Vampire implements AVATAR modulo theories using Z3.

Our experience with running Vampire on theory- and quan\-ti\-fi\-er-in\-ten\-si\-ve problems shows that some of the theory axioms can degrade the performance of Vampire. These axioms make Vampire infer many theory tautologies making the search space larger. We found that, among others, axioms of commutativity, associativity, left and right identity, and left and right inverse of arithmetic operations are in this sense ``expensive''. Our solution to this problem is a more refined control over which theory axioms Vampire adds to the search space. We added to the \verb'-tha' option of Vampire a new value named \verb'some' that makes Vampire only add ``cheap'' axioms to the search space. \verb'some' implements our empirical criterion for choosing theory axioms. Designing other criteria for axiom selection is an interesting task for future work.

%Cheap: non-reflexivity, transitivity, total order, plus one greater.
%Expensive: commutativity, associativity, right identity, left identity, commutative group axioms, right inverse, monotonicity of $+$.

\subsection{Experimental Results}\label{sec:boogie/experiments/results}
For our experiments, we compared the performance of Vampire on the Boogie, Voogie, and BLT translations of our benchmarks. 
%We also compared the performance of Vampire with that of Z3 on the SMT-LIB benchmarks. Z3 cannot read input in TPTP and does not support tuples, therefore we could not run it on the TPTP benchmarks.

We ran Vampire on all three sets of benchmarks with options \verb'-tha some' and \verb'-sas z3'. Vampire supports both TPTP and SMT-LIB syntax, the input language is selected by setting the \verb'--input_syntax' option to \verb'tptp' and \verb'smtlib2', respectively. We performed our experiments on the StarExec compute cluster~\cite{starexec} using the time limit of 5 minutes per problem. The detailed experimental results are available at \url{http://www.cse.chalmers.se/~evgenyk/ijcar18/}.

\begin{table}\center
  \caption{Runtimes in seconds of Vampire on the Boogie translation of the benchmarks.}
  \label{table:boogie/boogie-results}
  \begin{tabular}[ht]{lrrrrr}
\hline
\multirow{2}{*}{Benchmark} & \multicolumn{5}{c}{Number of loop unrollings} \\ %\cline{2-6}
& \multicolumn{1}{c}{1} & \multicolumn{1}{c}{2} & \multicolumn{1}{c}{3} & \multicolumn{1}{c}{4} & \multicolumn{1}{c}{5} \\
\hline
binary-search    & 0.884 & 2.420 & 3.364 & 10.709 & 27.648 \\
bubble-sort      & -- & -- & -- & -- & -- \\
dutch-flag       & 24.789 & -- & -- & -- & -- \\
insertion-sort   & 122.354 & -- & -- & -- & -- \\
matrix-transpose & 1.311 & -- & 1.078 & -- & -- \\
maxarray         & 0.205 & 0.587 & 1.197 & 1.702 & 1.692 \\
maximum          & 0.066 & 0.078 & 0.082 & 0.095 & 0.129 \\
one-duplicate    & -- & -- & -- & -- & -- \\
select-k         & 96.993 & -- & -- & -- & -- \\
two-way-sort     & 0.191 & 0.205 & 0.647 & 1.384 & 1.344 \\
  \end{tabular}
\end{table}

\begin{table}\center
  \caption{Runtimes in seconds of Vampire on the Voogie translation of the benchmarks.}
  \label{table:boogie/voogie-results}
  \begin{tabular}[ht]{lrrrrr}
\hline
\multirow{2}{*}{Benchmark} & \multicolumn{5}{c}{Number of loop unrollings} \\ %\cline{2-6}
& \multicolumn{1}{c}{1} & \multicolumn{1}{c}{2} & \multicolumn{1}{c}{3} & \multicolumn{1}{c}{4} & \multicolumn{1}{c}{5} \\
\hline
binary-search    &  1.979 & 25.135 &   6.560 &     -- & 163.803 \\
bubble-sort      &  0.394 & 53.192 &   2.073 &     -- &      -- \\
dutch-flag       & 11.384 &     -- &      -- &     -- &      -- \\
insertion-sort   & 18.262 & 38.169 &   3.369 & 21.698 &  11.639 \\
matrix-transpose &  0.266 &  8.362 &      -- &     -- &      -- \\
maxarray         &  0.170 &  0.587 &   0.489 &  2.635 &   6.325 \\
maximum          &  0.062 &  0.065 &   0.070 &  0.087 &   0.102 \\
one-duplicate    &  0.125 &  2.402 &   2.231 & 93.746 & 145.243 \\
select-k         &  0.216 &  0.612 & 203.655 &     -- &      -- \\
two-way-sort     &  0.464 &  5.360 &      -- &     -- &      -- \\
  \end{tabular}
\end{table}

Tables~\ref{table:boogie/boogie-results} and \ref{table:boogie/voogie-results} summarise the results of Vampire on the Boogie and Voogie translations of the benchmarks, respectively. A dash means that Vampire does not solve the problem within the given time limit.
\begin{itemize}
  \item Vampire solves 25 of the problems, translated by Boogie, and 36 problems, translated by Voogie.
  \item For 16 benchmark programs, Vampire solves their Voogie translations, but not the Boogie translations.
  \item For 5 benchmark programs, Vampire solves their Boogie translations, but not the Voogie translations.
  \item For 20 benchmark programs, Vampire solves both of their translations, and is faster on the Voogie translations for 12 of them.
%  \item Vampire solved either one of the two translations for 43 of the 50 benchmark programs.
\end{itemize}

Table~\ref{table:boogie/blt-results} summarises the results of Vampire on the BLT translations of the benchmarks.
\begin{itemize}
  \item Vampire solves 19 of the problems, translated by BLT.
  \item For all benchmark programs whose BLT translation Vampire is able to solve, Vampire also solves their Voogie translations. There are 3 benchmark programs for which Vampire solves their BLT translations but not their Boogie translations.
\end{itemize}

\begin{table}\center
  \caption{Runtimes in seconds of Vampire on the BLT translation of the benchmarks.}
  \label{table:boogie/blt-results}
  \begin{tabular}[ht]{lrrrrr}
\hline
\multirow{2}{*}{Benchmark} & \multicolumn{5}{c}{Number of loop unrollings} \\ %\cline{2-6}
& \multicolumn{1}{c}{1} & \multicolumn{1}{c}{2} & \multicolumn{1}{c}{3} & \multicolumn{1}{c}{4} & \multicolumn{1}{c}{5} \\
\hline
binary-search    & 0.821 & 163.790 &     -- &     -- & -- \\
bubble-sort      & 3.511 &      -- &     -- &     -- & -- \\
dutch-flag       & 4.049 &      -- &     -- &     -- & -- \\
insertion-sort   & 1.780 &      -- &     -- &     -- & -- \\
matrix-transpose & 0.465 &  12.437 &     -- &     -- & -- \\
maxarray         & 0.174 &   1.567 & 47.724 &     -- & -- \\
maximum          & 0.069 &   0.140 &  0.724 & 12.234 & -- \\
one-duplicate    & 0.307 &  10.039 &     -- &     -- & -- \\
select-k         & 3.142 &      -- &     -- &     -- & -- \\
two-way-sort     & 0.319 &  24.622 &     -- &     -- & -- \\
  \end{tabular}
\end{table}

Based on the results presented in Tables~\ref{table:boogie/boogie-results}--\ref{table:boogie/blt-results} we make the following observation. The problems translated from our benchmarks by Voogie are easier for Vampire than the problems translated by Boogie and BLT. Vampire is more efficient both in terms of the number of solved problems and runtime on the problems translated by Voogie. 
%We explain this observation by the fact that the Voogie translation uses tuple expressions and \LETIN\ with tuple definitions, which results in a succinct representation of the program inside the theorem prover. We conclude that by representing partial correctness of imperative programs directly as FOOL formulas we leverage the capabilities of Vampire.
This confirms our conjecture that the use of (efficient translations of) FOOL is better for saturation theorem provers than translations to FOL designed for other purposes. It would be interesting to run these experiments for theorem provers other than Vampire, however Vampire is currently the only prover implementing FOOL.

%Table~\ref{table:boogie/z3-results} summarises the results of Z3 on the SMT-LIB benchmarks. Z3 solved all the problems and showed good performance on all of them. 

%Based on our experimental resultn shows in Tables~\ref{table:boogie/tptp-results} and \ref{table:boogie/smtlib-results} we make the following observations.

%The detailed experimental results are available at \url{http://www.cse.chalmers.se/~evgenyk/ijcar18/}.



%\begin{table}
%  \caption{Runtimes in seconds of Z3 on the SMT-LIB translations of the benchmarks.}
%\scalebox{\tableScale}{
 % \begin{tabular}{lrrrrr}
%\hline
%\multirow{2}{*}{Benchmark} & \multicolumn{5}{c}{Number of loop unrollings} \\ %\cline{2-6}
%& \multicolumn{1}{c}{1} & \multicolumn{1}{c}{2} & \multicolumn{1}{c}{3} & \multicolumn{1}{c}{4} & \multicolumn{1}{c}{5} \\
%\hline
%binary-search    & 0.018 & 0.025 & 0.040 & 0.066 & 0.114 \\
%bubble-sort      & 0.024 & 0.041 & 0.070 & 0.114 & 0.181 \\
%dutch-flag       & 0.021 & 0.048 & 0.130 & 0.395 & 1.327 \\
%insertion-sort   & 0.022 & 0.032 & 0.051 & 0.086 & 0.133 \\
%matrix-transpose & 0.017 & 0.022 & 0.028 & 0.042 & 0.052 \\
%maxarray         & 0.018 & 0.023 & 0.031 & 0.048 & 0.072 \\
%maximum          & 0.015 & 0.016 & 0.019 & 0.022 & 0.028 \\
%one-duplicate    & 0.017 & 0.020 & 0.026 & 0.039 & 0.050 \\
%select-k         & 0.020 & 0.029 & 0.044 & 0.074 & 0.129 \\
%two-way-sort     & 0.017 & 0.026 & 0.047 & 0.098 & 0.218 \\
 % \end{tabular}
%}
%  \label{table:boogie/z3-results}
%\end{table}

\section{Related Work}
\label{sec:boogie/related}
Our previous work introduced FOOL~\cite{FOOL}, its implementation in Vampire~\cite{VampireAndFOOL}, and an efficient clausification algorithm for FOOL formulas~\cite{FOOLCNF}.

In~\cite{VampireAndFOOL} we sketched a tuple extension of FOOL and an algorithm for computing the next state relations of imperative programs that uses this extension. This paper extends and improves the algorithm. In particular, \begin{enumerate*}[label=(\roman*)]\item we described an encoding that uses FOOL in its current form, available in Vampire, \item we refined the encoding to only use in \LETIN\ the variables updated in program statements, \item we gave the definition of the encoding formally and in full detail, and \item we presented experimental results that confirm the described benefits of the encoding.\end{enumerate*}

Boogie is used as the name of both the intermediate verification language~\cite{leino2008boogie} and the automated verification framework~\cite{DBLP:conf/fmco/BarnettCDJL05}. The Boogie verifier encodes the next state relations of imperative programs in first-order logic by naming intermediate states of program variables~\cite{DBLP:journals/ipl/Leino05}.

BLT~\cite{CF-iFM17} is a tool that automatically generates verification conditions of Boogie programs. The aim of the BLT project is to use first-order theorem provers rather than SMT solvers for checking quantified program properties. BLT produces formulas written in the TPTP language and uses \ITE\ and \LETIN\ constructs of FOOL. BLT has an experimental option that introduces tuples for encoding of the next state relation. This option implements the encoding described in our earlier work~\cite{VampireAndFOOL}.

\section{Conclusion and Future Work}
\label{sec:boogie/conclusions}
% !TEX root = ../main.tex
We presented an encoding of the next state relations of imperative programs in FOOL. Based on this encoding we defined a translation from imperative programs, annotated with their pre- and post-conditions, to FOOL formulas that encode partial correctness properties of these programs. We presented experimental results obtained by running the theorem prover Vampire on such properties. We generated these properties using our translation and verification tools Boogie and BLT. We described a polymorphic theory of first class tuples and its implementation in Vampire.

The formulas produced by our translation can be efficiently checked by automated theorem provers that support FOOL. The structure of our encoding closely resembles the structure of the program. The encoding contains neither new symbols nor duplicated parts of the program. This way, the efficient representation of the problem in plain first-order logic is left to the theorem prover that is better equipped to do it.

Our encoding is useful for automated program analysis and verification. Our experimental results show that Vampire was more efficient in terms of the number of solved problems and runtime on the problems obtained using our translation.

FOOL reduces the gap between programming languages and languages of automated theorem provers. Our encoding relies on tuple expressions and \LETIN\ with tuple definitions, available in FOOL. To our knowledge, these constructs are not available in any other logic efficiently implemented in automated theorem provers.

The polymorphic theory of first class tuples is a useful addition to a first-order theorem prover. On the one hand, it generalises and simplifies tuple expressions in FOOL. On the other hand, it is a convenient theory on its own, and can be used for expressing problems of program analysis and computer mathematics.

For future work we are interested in making automated first-order theorem provers friendlier to program analysis and verification. One direction of this work is design of an efficient translation of features of programming languages to languages of automated theorem provers. Another direction is extensions of first-order theorem provers with new theories, such as the theory of bit vectors. Finally, we are interested in further improving automated reasoning in combination of theories and quantifiers.

\section*{Acknowledgements}
This work has been supported by the ERC Starting Grant 2014 SYMCAR 639270, the Wallenberg Academy Fellowship 2014, the Swedish VR grant D0497701, the Austrian research project FWF S11409-N23 and the EPSRC grant EP/P03408X/1-QuTie.


\bibliography{refs}

\end{document}