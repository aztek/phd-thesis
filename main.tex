\documentclass[10pt,twoside]{book}
\usepackage{amssymb,amsmath,amsthm,mathrsfs,stmaryrd,dsfont,bbold}
\usepackage[english]{babel}
\usepackage[lighttt]{lmodern}
\usepackage[inline]{enumitem}
\usepackage{pifont}
\usepackage{tikz}
\usepackage{color}
\usepackage{proof}
\usepackage{url}

% !TEX root = main.tex

\theoremstyle{definition}
\newtheorem{definition}{Definition}[chapter]
\newtheorem{theorem}{Theorem}[chapter]
\newtheorem{lemma}{Lemma}[chapter]
\newtheorem{example}{Example}[chapter]
\newtheorem*{example*}{Example}

\renewcommand{\baselinestretch}{1.1} 

\usepackage[labelsep=period]{caption}

\widowpenalty10000
\clubpenalty10000

% title page
\usepackage{titlesec}

\titleformat{\chapter}[display]
  {\linespread{1.0}\huge} % format
  {\vspace{-80pt}\sc\large Chapter \thechapter} % label
  {0pt} % sep
  {}[\vspace{1em}]

\titleformat{name=\chapter,numberless}[display]
  {\linespread{1.0}\huge}
  {}
  {-86pt}
  {}[\vspace{1em}]

\titlespacing*{\chapter}{0pt}{120pt}{6pt}

\titleformat{\section}
  {\normalfont\large\bfseries}
  {\thesection}
  {1em}
  {}%[\vspace{-0.1em}]
\titlespacing*{\section}{0pt}{20pt}{6pt}

\renewcommand*{\backref}[1]{}
\renewcommand*{\backrefalt}[4]{%
    \ifcase #1 \\\----\ Not cited%
    \or        \\\----\ One citation on page~#2%
    \else      \\\----\ #1 citations on pages~#2%
    \fi}
\bibliographystyle{plain}

\usepackage{color}
\definecolor{lstgrey}{rgb}{0.96,0.96,0.96}
\usepackage{listings}

\lstset{
  basicstyle=\ttfamily,
  columns=flexible,
  keepspaces=true,
  backgroundcolor=\color{lstgrey},
  breaklines=false,
  sensitive=true,
  captionpos=b,
  frame=single,
  framesep=2pt,
  xleftmargin=3pt,
  xrightmargin=3pt,
  resetmargins=true,
  rulecolor=\color{lstgrey}
  % frameround=tttt,
%  literate={~}{{\textapprox}}1
}

\lstdefinelanguage{tptp}{
  keywords={tff, thf, fof, cnf, type, axiom, hypothesis, conjecture}
}

\lstdefinelanguage{cpp}{
  keywords={public, static, void, do, for, while, int, bool, if, then, else, break, assert, let, in}
}

\lstdefinelanguage{appendixcpp}{
  keywords={public, static, void, do, for, while, int, bool, if, else, break, assert}
}

\lstdefinelanguage{bnf}{}

\newcommand{\reserved}[1]{\textbf{\underline{#1}}} % reserved words in algorithms
\newcommand{\ass}{\texttt{:=}}     % assignment operator
\newcommand{\inc}{~~~~\= \+ \kill}    % used in algorithms
\newcommand{\dec}{\- \kill}         % used in algorithms
\newcommand{\semicol}{;}                  % semicolon in algorithms

\newcommand{\INPUT}{\reserved{input}}
\newcommand{\OUTPUT}{\reserved{output}}
\newcommand{\IF}{\reserved{if}}
\newcommand{\VAR}{\reserved{var}}
\newcommand{\CASE}{\reserved{case}}
\newcommand{\OF}{\reserved{of}}
\newcommand{\DO}{\reserved{do}}
\newcommand{\OD}{\reserved{end~do}}
\newcommand{\THEN}{\reserved{then}}
\newcommand{\ELSE}{\reserved{else}}
\newcommand{\WHILE}{\reserved{while}}
\newcommand{\BEGIN}{\reserved{begin}}
\newcommand{\END}{\reserved{end}}
\newcommand{\LET}{\reserved{let}}
\newcommand{\FORALL}{\reserved{forall}}
\newcommand{\ASS}{\texttt{ := }}
\newcommand{\RETURN}{\reserved{return}}
\newcommand{\REPEAT}{\reserved{repeat}}
\newcommand{\LOOP}{\reserved{loop}}

\newcommand{\extTFF}{TFF0\textsuperscript{ext}}
\newcommand{\nofoolVampire}{Vampire$\,\star$}
\newcommand{\oldcnfVampire}{Vampire$\,\star$}

\newcommand{\ite}[3]{\mathtt{if}\;{#1}\;\allowbreak\mathtt{then}\;{#2}\;\allowbreak\mathtt{else}\;{#3}}
\newcommand{\itet}[3]{\mathrm{ite\_t}({#1},\;{#2},\;{#3})}
\newcommand{\ITE}{\texttt{if}-\texttt{then}-\texttt{else}}
\newcommand{\binding}[2]{{#1}={#2}}
\newcommand{\letin}[3]{\mathtt{let}\;\binding{#1}{#2}\;\allowbreak\mathtt{in}\;{#3}}
\newcommand{\letindef}[2]{\mathtt{let}\;{#1}\;\allowbreak\mathtt{in}\;{#2}}
\newcommand{\letinpar}[5]{\mathtt{let}\;\binding{#1}{#2};\;\binding{#3}{#4}\;\mathtt{in}\;{#5}}
\newcommand{\LETIN}{\texttt{let}-\texttt{in}}
\newcommand{\ofsort}[2]{{{#1}:{#2}}}
\newcommand{\set}[2]{{\left\{\,{#1}\;|\;{#2}\,\right\}}}

\newcommand{\builtin}[1]{\mathtt{\${#1}}}
\newcommand{\true}{\mathit{true}}
\newcommand{\false}{\mathit{false}}
\newcommand{\bool}{\mathit{bool}}
\newcommand{\fool}{{FOOL}}% how should we call FOL with boolean?
\newcommand{\foolp}{{FOOL+}}

% logic
\renewcommand{\implies}{\Rightarrow}
\newcommand{\liff}{\Leftrightarrow}
\newcommand{\lniff}{\not\Leftrightarrow}
%\newcommand{\lor}{\vee}

\newcommand{\eql}{\doteq}
\newcommand{\neql}{\not\doteq}

% overstrike in math
\newcommand\hcancel[1]{\setbox0=\hbox{$#1$}%
\rlap{\raisebox{.45\ht0}{\rule{\wd0}{0.4pt}}}#1}

%
\newcommand{\context}{\eta}

\newcommand{\extension}[1]{{#1}_+}

\newcommand{\intI}{I} % interpretation I

\newcommand{\interpret}[2]{\left\llbracket\,{#1}\,\right\rrbracket_{#2}}
\newcommand{\eval}[2]{\mathrm{eval}_{#2}({#1})}
\newcommand{\replacement}[3]{{#1}_{#2}^{#3}}

\newcommand{\variant}[3]{{#1}_{#2}^{#3}}

\newcommand{\folb}{{FOOL}}% how should we call FOL with boolean?
\newcommand{\toFOL}[1]{\mathit{fol}(#1)}  % translation of syntactially FO to FO

% end of definition, theorem, proof etc.
%\newcommand{\QEDsymbol}{\text{\ding{111}}}
\newcommand{\QEDsymbol}{\qed}
\newcommand{\QED}{\hfill\QEDsymbol}

% logic
\renewcommand{\implies}{\Rightarrow}
\renewcommand{\phi}{\varphi}

% newcnf
\newcommand{\possign}{\mathtt{t}}
\newcommand{\negsign}{\mathtt{f}}
\newcommand{\genlit}[2]{{#1}^{#2}}
\newcommand{\genclause}[2]{{#1}_{#2}}

\newcommand{\sign}{\star}
\newcommand{\subst}{\theta}
\newcommand{\emptySubst}{\epsilon}

\newcommand{\formName}{\mathtt{form}}
\newcommand{\form}[1]{\formName({#1})}

\newcommand{\config}[2]{#1}

\newcommand{\GC}{\mathit{C}}

\newcommand{\sk}{\mathit{sk}}

% TPTP true, false and bool
\newcommand{\tptpo}{\lstinline'$o'} %$
\newcommand{\dbool}{\lstinline'$bool'} %$
\newcommand{\dtrue}{\lstinline'$true'} %$
\newcommand{\dfalse}{\lstinline'$false'} %$
\newcommand{\ddtrue}{\lstinline'$$true'}
\newcommand{\ddfalse}{\lstinline'$$false'}

% TPTP ite and let
\newcommand{\dite}{\lstinline'$ite'} %$
\newcommand{\ditet}{\lstinline'$ite_t'} %$
\newcommand{\ditef}{\lstinline'$ite_f'} %$

\newcommand{\dlet}{\lstinline'$let'} %$
\newcommand{\dlettt}{\lstinline'$let_tt'} %$
\newcommand{\dlettf}{\lstinline'$let_tf'} %$
\newcommand{\dletft}{\lstinline'$let_ft'} %$
\newcommand{\dletff}{\lstinline'$let_ff'} %$

\newcommand{\dint}{\lstinline'$int'} %$
\newcommand{\dgreatereq}{\lstinline'$greatereq'} %$
\newcommand{\dsum}{\lstinline'$sum'} %$

\newcommand{\di}{\lstinline'$i'} %$

\newcommand{\arrayt}{\mathit{array}}
%\newcommand{\select}[2]{{#1}[{#2}]}
%\newcommand{\selectf}{\select{\cdot\,}{\,\cdot\,}}
\newcommand{\select}[2]{\mathit{select}({#1},\allowbreak{#2})}
\newcommand{\selectf}{\mathit{select}}
%\newcommand{\store}[3]{{#1}\langle{#2}\lhd{#3}\rangle}
%\newcommand{\storef}{\store{\cdot\,}{\,\cdot}{\cdot\,}}
\newcommand{\store}[3]{\mathit{store}({#1},\allowbreak{#2},\allowbreak{#3})}
\newcommand{\storef}{\mathit{store}}

\newcommand{\darray}[2]{\darraySymb\lstinline'('{#1}\lstinline','{#2}\lstinline')'}
\newcommand{\darraySymb}{\lstinline'$array'} %$
\newcommand{\dselect}{\lstinline'$select'} %$
\newcommand{\dstore}{\lstinline'$store'} %$
\newcommand{\darrayone}{\lstinline'$array1'} %$
\newcommand{\dselectone}{\lstinline'$select1'} %$
\newcommand{\dstoreone}{\lstinline'$store1'} %$
\newcommand{\darraytwo}{\lstinline'$array2'} %$
\newcommand{\dselecttwo}{\lstinline'$select2'} %$
\newcommand{\dstoretwo}{\lstinline'$store2'} %$

\newcommand{\skolem}{skolem}

% the rest
\newcommand{\Z}{\mathds{Z}}
\newcommand{\R}{\mathds{R}}

% tuples
\newcommand{\tuple}[1]{({#1})}

\newcommand{\newcnf}{\textsc{VCNF}}
\newcommand{\nfcnf}{$\text{\textsc{VCNF}}_{\text{\textsc{FOOL}}}$}
\newcommand{\oldcnf}{\textsc{FOOL2FOL}}

\newcommand{\integer}{\mathit{int}}
\newcommand{\emptyStatement}{\mathtt{skip}}
\newcommand{\assigns}[2]{{#1}\coloneqq{#2}}
\newcommand{\seq}[2]{{#1}\,;{#2}}
\newcommand{\translateT}{\mathcal{T}}
\newcommand{\translate}[1]{\translateT({#1})}
\newcommand{\tuplifyT}{\mathcal{N}}
\newcommand{\tuplifyRel}[1]{\tuplifyT({#1})}
\newcommand{\tuplify}[2]{\tuplifyRel{{#1}}({#2})}
\newcommand{\updates}[1]{\mathrm{updates}({#1})}

\newcommand{\ifthen}[2]{\mathtt{if}\;{#1}\;\allowbreak\mathtt{then}\;{#2}}

\newcommand{\while}[2]{\mathtt{while}\;{#1}\;\mathtt{do}\;{#2}}

\newcommand{\ttrue}{\mathtt{true}}
\newcommand{\tfalse}{\mathtt{false}}

\newcommand{\letinnl}[3]{\begin{aligned}[t]&\mathtt{let}\;\binding{#1}{#2}\;\mathtt{in}\\[-0.2em]&\quad{#3}\end{aligned}}
\newcommand{\letnl}[3]{\begin{aligned}[t]\mathtt{let}\;&\binding{#1}{#2}\\[-0.2em]\mathtt{in}\;&{#3}\end{aligned}}

\newcommand{\letinparnl}[5]{\begin{aligned}[t]\mathtt{let}\;&\binding{#1}{#2};\;\binding{#3}{#4}\\\mathtt{in}\;&{#5}\end{aligned}}

\newcommand{\itenll}[3]{\begin{aligned}[t]&\mathtt{if}\;{#1}\\[-0.2em]&\mathtt{then}\;{#2}\\[-0.2em]&\mathtt{else}\;{#3}\end{aligned}}

\newcommand{\itenl}[3]{\begin{aligned}[t]\mathtt{if}\;{#1}\;&\mathtt{then}\;{#2}\\&\mathtt{else}\;{#3}\end{aligned}}


\newcommand{\interp}[1]{\left\llbracket\,{#1}\,\right\rrbracket}

\newcommand{\expr}{\mathit{e}}
\newcommand{\stmt}{\mathit{s}}
\newcommand{\State}{\mathit{st}}

\newcommand{\hoare}[3]{\{{#1}\}\,{#2}\,\{{#3}\}}

\newcommand{\pint}{\mathtt{int}}
\newcommand{\pbool}{\mathtt{bool}}
\newcommand{\parray}{\mathtt{array}}

\thesis{Thesis for the Degree of Doctor of Philosophy}
\title{Automated Theorem Proving with\\Extensions of First-Order Logic}
\infopagetitle{Automated Theorem Proving with Extensions of First-Order Logic}
\author{Evgenii Kotelnikov}
\department{Department of Computer Science and Engineering}
\institution{Chalmers University of Technology and University of Gothenburg}
\infopageinstitution{Chalmers University of Technology and\\[-1mm]University of Gothenburg}
\reportnumber{152L}
\issn{1652-876X}


\newcommand{\EK}[1]{{\color{red}  EK: {#1}}}

\begin{document}

\frontmatter

\maketitle

\chapter*{Abstract}
Automated theorem provers are computer programs that check if a logical conjecture follows from a set of logical statements. The conjecture and the statements are expressed in the language of a formal logic, such as first-order logic. Expressivity of first-order logic makes it convenient for encoding problems from diverse application domains. As a result, theorem provers for first-order logic have been used for automation in proof assistants, verification of programs, static analysis of networks, and other purposes. However, their efficient usage remains challenging. One of the challenges is the complexity of translating domain problems to the logic of theorem provers. Not only can such translation be cumbersome due to semantic differences between the domain and the logic, but it might inadvertently result in problems that provers cannot easily handle.

The work presented in the thesis addresses this challenge by developing an extension of first-order logic named FOOL. FOOL contains syntactical features of programming languages and more expressive logics, is friendly for translation of problems from various domains, and can be efficiently supported by existing theorem provers. We describe the syntax and semantics of FOOL and present a simple translation from FOOL to plain first-order logic. We describe a more efficient clausal normal form transformation algorithm for FOOL and based on it implement a support for FOOL in the Vampire theorem prover. We illustrate the efficient use of FOOL for program verification by describing a concise encoding of next state relations of imperative programs in FOOL. We demonstrate the efficiency of automated theorem proving in FOOL with an extensive set of experiments. In these experiments we compare the performance of Vampire on a large collection of problems from various sources translated to FOOL and ordinary first-order logic. Finally, we fix the syntax for FOOL in TPTP, the standard language of first-order theorem provers.

\tableofcontents

% \chapter*{Acknowledgements}
% I am indebted to many people who in different ways supported me during these past five years.

I would like to thank my supervisor Laura Kov\'acs for setting me on the path that eventually led to this thesis and her help along the way. I am grateful to my co-supervisor Andrei Voronkov for his guidance and for showing me how academic work should be approached. I want to thank my local co-supervisor Moa Johansson for always being ready to help and my examiner Koen Claessen for his good advice and timely support. To my collaborators Martin Suda and Giles Reger, thank you for the insightful discussions about automated reasoning that we had. I am grateful to Byron Cook for inviting me to visit Amazon Web Services and for the great time that I had there.

I was lucky to be surrounded by a lot of amazing colleagues and friends at Chalmers. Daniel~H., Daniel~S., Iulia, Jeff, Mauricio, Ra\'ul, Pablo, Simon and others, thank you for all the amazing time we spent together. Another thank you goes to Carlo, Grischa, Enzo, Marco and Pierguiseppe for sharing my interest in music. I hope we will get to play more gigs in the future.

Finally, a very special thank you goes to Lydia for her continuous support and patience.


\mainmatter

\chapter{Introduction}
\label{chap:intro}
% !TEX root = main.tex

This thesis studies automated theorem proving in first-order logic and its applications. The history of automated theorem proving in first-order logic dates back to the early 1950s (see e.g. \cite{bundy1999survey,davis2001early,harrison2007short} for a historical overview). Over the years proof search algorithms and implementations of automated theorem provers have matured and are now used for practical applications. Among these applications are static analysis and verification of software and hardware, automation for proof assistants, knowledge representation, natural language processing and others.

The efficient usage of first-order theorem provers might be challenging. One of the challenges is representation of application problems in first-order logic in a way that is efficient for automated reasoning. Systems that rely on first-order provers, such as program verification tools and proof assistants, usually do not deal with first-order logic natively. Instead, they translate problems in their respective domains (program properties or formulas in the logic of the proof assistant) to problems in first-order logic. There could be multiple ways of translating a problem because of the mismatch between the semantics of the domain and that of first-order logic. A theorem prover might succeed on the results of some of these translations and fail on the others. Users of a theorem prover might find designing a translation that is friendly to the prover to be a difficult task. Such translation might require solid knowledge of how theorem provers work and are implemented, something that the users of the prover might not have. Assessing whether a translation of a certain problem to first-order logic is good might be difficult as well. Such assessment can often only be made through tedious experiments with running theorem provers, configured with different settings, on the results of the translation. A perfect translation might not necessarily exist, because different translation might work better in different scenarios. Furthermore, for some types of problems, their translations to first-order logic cannot be efficiently handled by a theorem prover at all unless the prover is extended with specialised inference rules and heuristics.

The complexity of preparing problems for first-order theorem provers can be battled by extending the logic supported by the provers. Such extension should include theories and new syntactical features that are common in problems from application domains but sensitive to translations. The appropriate translation of these features to plain first-order logic therefore becomes the responsibility of the provers themselves. The right choice of new features and their efficient implementation in theorem provers facilitates applications of automated theorem proving. Firstly, users of theorem provers are relieved from the tedious translations and can express their problems closer to their original domains. Secondly, theorem provers are able to implement translations of these features that suit them best. Thirdly, theorem provers can try multiple different translations in the same proof attempt. Finally, theorem provers can enhance proof search for problems with specific features by implementing dedicated inference rules and preprocessing steps for these features.

This thesis addresses the following research question: \emph{which new extensions of first-order theorem provers are useful for applications and how can these extensions be efficiently implemented?} The thesis identifies that first class Boolean sort, \ITE\ and \LETIN\ expressions are useful for problems from program verification and automation of proof assistants and are generally not supported by first-order theorem provers. The thesis presents a modification of first-order logic named FOOL that contains these features and gives new techniques for reasoning in it and using it. The thesis describes implementation details and challenges in the Vampire theorem prover, however the described extensions and their implementation can be carried out in any other first-order prover.

This chapter describes the background of the thesis and is structured as follows. First, we overview the key concepts of automated theorem proving in first-order logic. Then, we explain how program verification tools and proof assistants benefit from extensions of theorem provers presented in the thesis. Finally, we detail the main contributions of the thesis and overview its structure.

% how to construct proofs automatically using a computer. The latter is the domain of automated theorem proving. It is one of the central and hardest areas of computer mathematics and artificial intelligence. Automated methods of proving theorems precede the existence of computers.

% Algorithms of automated theorem proving are implemented in computer programs called theorem provers. A theorem prover takes a logical conjecture as input and tries to either construct its proof or demonstrate that the conjecture is invalid. Theorem provers can be classified by the logic they support. Propositional, first-order and higher-order logic are among the logics that received the most attention in automated theorem proving.

%Theorem provers work by applying \emph{inference rules}. An inference rule is a $n$-ary ($n>0$) relation on clauses written as \[\infer[,]{B}{A_1 & \ldots & A_n}\] where $A_1,\ldots,A_n$ are premises and $B$ is the conclusion. An \emph{inference system} $\mathcal{I}$ is a collection of inference rules.

\section*{Automated Theorem Proving in First-Order Logic}
\addcontentsline{toc}{section}{Automated Theorem Proving in First-Order Logic}

First-order logic is not decidable, there is no algorithm that could in general determine whether a given first-order formula is valid or not. First-order logic is semi-decidable, an algorithm that enumerates all finite derivations in the logical system until a given first-order formula is found, terminates if the formula is valid, and may run forever otherwise. If a formula is satisfiable but not valid, there is no algorithm that could in general demonstrate that. A well studied and generally best performing class of algorithms that search for validity of first-order problems are those based on \emph{saturation} and the calculus of \emph{resolution} and \emph{superposition}. These algorithms are implemented in automated theorem provers such as E~\cite{E13}, Spass~\cite{Spass} and Vampire~\cite{Vampire13}.

First-order theorem provers work with first-order formulas represented as sets of \emph{clauses}. A first-order formula is in a clausal normal form (CNF) if it is a universally quantified conjunction of disjunctions of literals. An alternative representation of a CNF is as a set of first-order clauses, where each clause is a finite multiset of literals. A \emph{clausification} algorithm converts an arbitrary first-order formula to a set of first-order clauses, preserving satisfiability. Most first-order provers that support formulas in full first-order logic implement such algorithms as part of their preprocessing of the input.

First-order theorem provers construct proofs by \emph{refutation}. Given a first-order problem of the form $\mathit{Premises}\implies\mathit{Conjecture}$, a theorem prover first negates the conjecture, obtaining $\mathit{Premises}\wedge\neg\mathit{Conjecture}$, then converts this formula to a set of clauses $S$ and attempts to show that $S$ is unsatisfiable by deriving contradiction (the empty clause). To that end, the theorem prover \emph{saturates} the set $S$ with respect to some \emph{inference system} $\mathcal{I}$ which is a collection of \emph{inference rules}. An inference rule is a $n$-ary ($n\ge0$) relation on clauses written as \[\infer[,]{B}{A_1 & \ldots & A_{n-1}}\] where $A_1,\ldots,A_{n-1}$ are premises and $B$ is the conclusion. A set of clauses is called saturated with respect to $\mathcal{I}$ if for every inference of $\mathcal{I}$ with premises in this set, the conclusion of the inference also belongs to that set. To saturate the set $S$, the theorem prover systematically and exhaustively applies inference rules from $\mathcal{I}$ to premises from $S$ and adds the conclusion of each inference to $S$. If the empty clause is derived during this process, then the initial set $S$ is unsatisfiable and the input problem is valid. In such case the theorem prover returns the proof of the problem as a tree of inferences with clauses from the initial set $S$ as leafs and the empty clause as the root. If after applying all inferences between clauses in the saturated set $S$ the empty clause has not been derived and the inference system $\mathcal{I}$ is complete then the initial set $S$ is satisfiable and the problem is not valid. In such case the theorem prover returns the saturated set $S$. Saturation might not terminate on a satisfiable set of clauses, in such case the theorem prover sooner or later runs out of resources and fails. In practice, finite saturation is rare and theorem provers focus on deriving the empty clause by implementing various techniques and heuristics that make exploration of the search space of clauses more efficient.

Modern theorem provers employ inference systems that include refinements of the calculus of resolution, derived from the work of Robinson~\cite{Robinson65}, and superposition, derived from the work of Bachmair and Ganzinger~\cite{BG94} (see also \cite{Ganzinger01,NieuwenhuisRubio:HandbookAR:paramodulation:2001}). The inference rules in this calculus are guarded with side conditions which determine whether a rule can be applied. These conditions prevent the search space of clauses from growing too fast and are essential in practice. The key concepts used in these conditions are a \emph{simplification ordering} and a \emph{literal selection function}. They are understood as parameters of the calculus. A simplification ordering on terms $\succ$ captures the notion of simplicity (see e.g. \cite{DBLP:books/el/RV01/DershowitzP01}) i.e. $t_1 \succ t_2$ implies that $t_2$ is in some way simpler than $t_1$. There are direct extensions of simplification ordering to literals and clauses. A literal selection function determines for a given clause which literals should be used for inferences. Figure~\ref{fig:intro/calculus} shows the most important inference rules of the superposition and resolution calculus (selected literals are underlined). In this figure, $\mathrm{mgu}$ denotes a most general unifier of two first-order terms and $L[s]$ ($t[s]$) denotes that a term $s$ occurs in a literal $L$ (term $t$).

\begin{figure}[ht]
  \begin{equation*}
    \begin{aligned}
      &
      \begin{aligned}
        &
        \begin{aligned}
          \begin{aligned}
            &\text{\textbf{Resolution}}\\
            &\infer[,]{(C_1\vee C_2)\theta}{\underline{A} \vee C_1 & \underline{\neg A'} \vee C_2}
          \end{aligned}
          &\quad\quad
          \begin{aligned}
            &\text{\textbf{Factoring}}\\
            &\infer[,]{(A \vee C)\theta}{\underline{A} \vee A' \vee C}
          \end{aligned}
        \end{aligned}
        \\[0.25em]
        &\text{where, for both inferences, $\theta=\mathrm{mgu}(A,A')$ and $A$ is not an equality}
      \end{aligned}
      \\[1em]
      &
      \begin{aligned}
        &\text{\textbf{Superposition}}
        \\
        &
        \infer[, \begin{array}{l}\text{where $\theta=\mathrm{mgu}(l,s)$, $r\theta\not\succeq l\theta$}\\\text{and $L[r]$ is not an equality}\end{array}]{(L[r] \vee C_1 \vee C_2)\theta}{\underline{l \eql r} \vee C_1 & \underline{L[s]} \vee C_2}
        \\[0.5em]
        &
        \quad\quad\text{or}
        \\[0.4em]
        &
        \infer[, \begin{array}{l}\text{where $\theta=\mathrm{mgu}(l,s)$, $t\theta\not\succeq s\theta$, $t'\theta\not\succeq t[s]\theta$}\\\text{and $\otimes$ is either $\eql$ or $\not\eql$}\end{array}]{(t[r] \otimes t' \vee C_1 \vee C_2)\theta}{\underline{l \eql r} \vee C_1 & \underline{t[s] \otimes t'} \vee C_2}
      \end{aligned}
      \\[1em]
      &
      \begin{aligned}
        \begin{aligned}
          &\text{\textbf{Equality resolution}}\\
          &\infer[,]{C\theta}{\underline{s\eql t} \vee C}\\
          &\text{where $\theta=\mathrm{mgu}(s,t)$}
        \end{aligned}
        &\quad\quad
        \begin{aligned}
          &\text{\textbf{Equality factoring}}\\
          &\infer[,]{(t \not\eql t' \vee s' \eql t' \vee C)\theta}{\underline{s \eql t} \vee s' \eql t' \vee C}\\
          &\text{where $\theta=\mathrm{mgu}(s,t)$, $t\theta\not\succeq s\theta$ and $t'\theta\not\succeq s'\theta$}
        \end{aligned}
      \end{aligned}
    \end{aligned}
  \end{equation*}
  \caption{The inference rules of the superposition and resolution calculus.\label{fig:intro/calculus}}
\end{figure}

An important concept related to saturation in \emph{redundancy elimination}. A clause $C$ from a set $S$ is called redundant in $S$ if it is a logical consequence of clauses in $S$ strictly smaller than $C$ w.r.t. to a simplification ordering. Redundant clauses can be eliminated from the search space without compromising completeness. A powerful criterion of redundancy of a clause is \emph{subsumption}. A clause $A$ subsumes $B$ if some subclause of $B$ is an instance of $A$. If a clause $A$ from a set $S$ subsumes $B$, $B$ is redundant in $S$. \emph{Saturation up to redundancy}~\cite{NieuwenhuisRubio:HandbookAR:paramodulation:2001} terminates when the inference system cannot derive any new clauses that are not redundant in the search space. %The notion of redundancy improves the performance of theorem provers by constraining the growth of the search space.

Another powerful technique is \emph{splitting}~\cite{DBLP:conf/cade/HoderV13} of long clauses into smaller ones with disjoint sets of used variables so that the search space can be explored in smaller parts. This technique is motivated by the observation that long clauses slow down saturation based proof search. A recent improvement of splitting is the AVATAR architecture~\cite{DBLP:conf/cav/Voronkov14} that employs a SAT or SMT solver to guide splitting decisions.

%Another useful tool in reducing search space explosion is splitting [8] where clauses are split so that the search space can be explored in smaller parts.

% A new, highly successful, approach to splitting is found in the AVATAR architecture [23], which uses a Splitting module with a SAT solver at its core to make splitting decisions.

The aforementioned notions and methods and many other refinements of proof search, implemented in theorem provers, aim to constrain the growth of the search space and avoid unnecessary inferences. Ultimately, the behaviour of a theorem prover can be tuned in many different ways. Whether or not a theorem prover solves the input problem depends to a large degree on the choice of parameters of the proof search algorithm. Different combinations of these parameters can solve different problems. For that reason theorem provers such as E, iProver and Vampire implement \emph{portfolios} of proof search strategies. Based on certain characteristics of the input, theorem provers select the appropriate strategies and schedules for them, and then run these strategies one by one in a time-slicing fashion. Some of these strategies are designed to be refutationally incomplete~--- they cannot derive the empty clause from an arbitrary unsatisfiable set of clauses, but for some unsatisfiable sets of clauses they derive the empty clause very quickly. The usage of multiple proof search strategies in the same proof attempt allows theorem provers to succeed on a larger number of problems. Some provers also extend their portfolios with proof search techniques other than saturation. For example, Vampire includes in its portfolios an implementation of the Inst-Gen calculus~\cite{DBLP:conf/birthday/Korovin13} and a finite model builder~\cite{VampireFMB}.

Another contributing factor to the success of a theorem prover is how well the input problem is prepared to be processed by saturation. First-order theorem provers are known to be fragile with respect to the input. Multiple, often subtle, characteristics of a first-order problem might affect the performance of saturation based proof search. These characteristics include, for example, the number of clauses in the problem, the size of clauses and the size of the signature. Theorem provers implement elaborate preprocessing techniques, in particular improvements of clausification algorithms (see e.g. \cite{nonnengart2001computing,azmy2013computing,newcnf_fol}), that aim to produce good sets of clauses.

Some first-order formulas can be problematic for efficient proof search. A common technique employed by theorem provers is to replace such formulas with specialised inference rules. A well known example of this technique is handling of equality. Equality can be finitely axiomatised in first-order logic as a congruence relation. However, resolution and factoring with equality axioms are known to generate a lot of (mostly unnecessary) new clauses and thus is very inefficient. Rather than axiomatising equality, first-order provers consider it part of the logic and implement specialised inference rules for equality reasoning. These inference rules include refinements of the paramodulation rule~\cite{WRCS67,Robinson1969}. They are part of the standard arsenal of inference rules used by theorem provers. Another example is the extensionality resolution rule, implemented in Vampire~\cite{ATVA14}. This rule replaces difficult extensionality axioms that are routinely used in encodings of data collections and sets.

The performance of first-order theorem provers is evaluated empirically on large corpora of problems. Comparison of provers is mostly based on success rates and run times. The main corpus is the Thousands of Problems for Theorem Provers (TPTP) library~\cite{TPTP}. The problems in this corpus are written in a variety of languages, such as FOF for untyped first-order formulas, TFF0~\cite{tff0} for typed monomorphic first-order formulas and TFF1~\cite{tff1} for typed rank-1 polymorphic first-order formulas. The TPTP library is used as a basis for the annual CASC system competition~\cite{CASC}.

Many practical problems tackled by theorem provers are expressed in the combination of first-order logic and theories. For example, problems coming from program verification routinely use integer arithmetic, arrays and datatypes. Most interesting theories do not have a complete encoding in first-order logic and require dedicated support in theorem provers. Vampire handles the theory of integer arithmetic by (i) automatically adding incomplete relevant theory axioms to the search space; (ii) applying dedicated inference rules for ground evaluation of theory terms; and (iii) using AVATAR modulo theories~\cite{DBLP:conf/gcai/RegerB0V16}. Vampire supports the polymorphic theory or arrays by automatically instantiating theory axioms for each sort of arrays~\cite{VampireAndFOOL}. Finally, Vampire supports datatypes and co\-data\-types~\cite{BPR18}. Their underlying theory of term algebras cannot be finitely axiomatised in first-order logic, however complete reasoning with this theory was implemented using dedicated inference rules.

\section*{Extensions of First-Order Logic for Applications}
\addcontentsline{toc}{section}{Extensions of First-Order Logic for Applications}

%The two main areas of application of first-order theorem provers, considered in this thesis, are deductive program verification and automation of proof assistants. 

\paragraph{Deductive Program Verification.}
The task of a program verification tool is to check whether a given program satisfies its specification. A program specification can be expressed with logical formulas that annotate program statements and capture their properties. Typical examples of such properties are pre-conditions, post-conditions and loop invariants. These program properties are checked using various tools (see e.g. \cite{Bonacina10} for a detailed overview). Deductive program verification sees compliance with specification as a logical problem that can be checked by automated theorem provers. For that, program statements are first translated to logical formulas that capture the semantics of the statements. Then, a theorem is built with the translated formulas as premises and program properties as the conjecture. Validity of the theorem is interpreted as that the program statements have their annotated properties. Conversely, failure to show validity might indicate a bug in the program. Program verification frameworks such as Boogie~\cite{DBLP:conf/fmco/BarnettCDJL05}, Why3~\cite{DBLP:conf/esop/FilliatreP13} and Frama-C~\cite{FramaC} rely on automated theorem provers for checking program properties.

Theorem provers can be used not just for checking program properties, but also for generating them. Recent approaches in interpolation and loop invariant generation~\cite{McMillan08,fase2009,hoder2012popl} present initial results of using first-order theorem provers for generating quantified program properties. First-order theorem provers can also be used to generate program properties with quantifier alternations~\cite{fase2009}; such properties could not be generated fully automatically by any previously known method.

\paragraph{Automation of Proof Assistants.}
Proof assistants are software tools that assist users in constructing proofs of mathematical problems. Proof assistants use formalisations of mathematics based on higher-order logic (Isabelle/HOL~\cite{Isabelle}), type theory (Coq~\cite{Coq}), set theory (Mizar~\cite{Mizar}) and others. Many proof assistants enhance the workflow of their users by automatically filling in parts of the user's proof with the help of tactics. Tactics are specialised scripts that run a predefined collection of proof searching strategies. These strategies can be implemented inside the proof assistant itself or rely on third-party automated theorem provers~\cite{Sledgehammer,DBLP:conf/icms/UrbanHV10}. Automation using external theorem provers, including first-order ones, is implemented e.g. in the Sledgehammer extension~\cite{Sledgehammer} of Isabelle. Sledgehammer heuristically picks lemmas and definitions that might be necessary for the proof, translates them to the logic of automated theorem provers and hands over the resulting formulas to the provers. If one of the provers returns a proof, Sledgehammer uses this proof to reconstruct a proof in the calculus of Isabelle. The translation of Isabelle's lemmas and definitions might be incomplete because the logic of Isabelle is more expressive than that of automated provers.

The translation of the following features of programming languages and more expressive logics to plain first-order logic might be cumbersome and inefficient. This thesis presents features of FOOL that can be used for a more straightforward translation. Further, the thesis present methods of efficient support of these features and an implementation of these methods in Vampire.
\begin{enumerate}
  \item Boolean values in programming languages are used both as expressions in conditional and loop statements and as Boolean flags passed as arguments to functions. A natural way of translating program statements with Booleans into formulas is by translating conditions as formulas and function arguments as terms. Yet one cannot mix Boolean terms and formulas in the same way in plain first-order logic. FOOL contains the Boolean sort as its first class sort. Formulas in FOOL are indistinguishable from Boolean terms which coincides with the treatment of Booleans in programming languages.
  \item Properties expressed in higher-order logic routinely use quantification over the interpreted Boolean sort; this is not allowed in plain first-order logic. FOOL allows quantification over the first class Boolean sort. Besides proof assistants, the first class Boolean sort is useful to higher-order automated theorem provers such as Satallax~\cite{Satallax} and Leo-II~\cite{LeoII} that employ first-order provers for their proof search.
  \item Imperative programs are structured as sequences of variable updates. Standard techniques for translating such sequences to logic involve computing a static single assignment (SSA) form of the program. Computation of an SSA form introduces intermediate variables and their presence in the resulting formula can deteriorate the performance of a theorem prover. FOOL contains \LETIN\ expressions. One can concisely express sequences of variable assignments in FOOL as nested \LETIN\ and leave the decision of naming intermediate states of the program or not to the theorem prover.
  \item Both programming language and logics of proof assistants routinely use conditional expressions and local definitions of functions. The standard approaches for translating them are inlining and naming. Either one of these approaches can result in difficult first-order formulas. FOOL contains \ITE\ expressions and allows \LETIN\ expressions to define function and predicate symbols with arbitrary arity. The choice between inlining and naming is left to the theorem prover itself which is better equipped to make it.
\end{enumerate}

\section*{Contributions of the Thesis}
\label{sect:intro:contributions}
\addcontentsline{toc}{section}{Contributions of the Thesis}
%This thesis contributes to the area of automated reasoning by exploring which extensions of first-order theorem provers facilitate their practical applications, in particular program verification and automation for proof assistants. This section summarises the main contributions of the thesis.

In summary, the work presented in this thesis
\begin{enumerate}
  \item introduces the extension FOOL of first-order logic that contains useful syntactical constructs that are usually not supported by first-order provers, mentioned before;
  \item explores how reasoning in FOOL can be efficiently implemented in existing automated theorem provers for first-order logic;
  \item gives practical evidence of usefulness of FOOL for application through examples and developed translation techniques;
  \item gives practical evidence of efficiency of reasoning with FOOL through experimental results on large diverse collections of problems.
\end{enumerate}

\paragraph{FOOL.}
The thesis presents FOOL, standing for first-order logic (FOL) with Boolean sort. \folb{} extends ordinary many-sorted FOL with \begin{enumerate*}[label=(\roman*)]\item first class Boolean sort, \item Boolean variables used as formulas, \item formulas used as arguments to function and predicate symbols, \item \ITE\ expressions and \item \LETIN\ expressions.\end{enumerate*} \ITE\ and \LETIN\ expressions can occur as both terms and formulas. \LETIN\ expressions can use (multiple simultaneous) definitions of function symbols, predicate symbols, and tuples. The thesis presents the definition of FOOL, its semantics, and a simple model-preserving translation from \folb{} formulas to formulas of first-order logic. This translation can be used to support \folb{} in existing first-order provers.

\paragraph{Reasoning with FOOL.}
The thesis presents two approaches to an implementation of FOOL in first-order provers that improve over the simple translation of FOOL to FOL. The first approach is a new technique of dealing with the Boolean sort in superposition theorem provers. This technique includes replacement of one of the Boolean sort axioms with a specialised inference rule called FOOL paramodulation. The second approach is a new algorithm \nfcnf{} that transforms FOOL formulas directly to first-order clauses. The thesis presents an implementation of the simple translation from FOOL to FOL and both improved approaches in Vampire.

\paragraph{Applications of FOOL.}
The thesis presents an encoding of the next state relations of imperative programs in FOOL. Compared to similar methods, this encoding avoids introducing intermediate variables and results in FOOL formulas that concisely represent the structure of program fragments in logic.
%The thesis presents a translation of imperative programs annotated with their pre- and post-conditions to partial correctness properties of these programs.
The thesis presents a work on verification of virtual private cloud network configurations with Vampire. The encoding of verification problems in this work relies on first class Booleans, the theory of arrays and the theory of tuples.

\paragraph{Practical Evaluation.}
The thesis presents extensive experiments on running Vampire, other first-order theorem provers, higher-order theorem provers and SMT solvers on FOL and FOOL problems. These problems come from various sources: benchmarks from the TPTP and SMT-LIB library, proof obligations generated by the Isabelle proof assistant, and verification conditions generated by multiple different program verification tools. The experimental results obtained with these problems show in particular that \begin{enumerate}
  \item Vampire with FOOL paramodulation performs better than Vampire with the simple translation from FOOL to FOL;
  \item Vampire with \nfcnf{} performs better that Vampire with FOOL paramodulation;
  \item Vampire performs better on verification conditions translated to FOOL than the same verification conditions translated to FOL using methods implemented in state-of-the-art verification tools.
\end{enumerate}

\paragraph{Impact on TPTP.}
The language of FOOL is a superset of TFF0~--- the monomorphic first-order part of the TPTP language. The thesis describes a modification of the TPTP language needed to represent \folb{} formulas. This modification has been included in the TPTP standard as the TPTP Extended Typed First-Order Form (TFX).

\paragraph{Impact on Vampire.}
The language of FOOL is a superset of the core theory of the SMT-LIB language~\cite{SMT-LIB}, the standard language of SMT solvers. First-order provers that support \folb{} can therefore reason about some problems from the SMT-LIB library. This opens up an opportunity to evaluate first-order provers on problems that were previously only checked by SMT solvers. Vampire gained support for SMT-LIB based on its implementation of FOOL, and since 2016 has been participating in the SMT-COMP competition~\cite{DBLP:conf/cav/BarrettMS05} where it contends against SMT solvers.

The support of both FOOL and theories such as arithmetic, arrays and datatypes, makes Vampire a convenient and powerful tool for reasoning about properties of programs.


\section*{Structure of the Thesis}
\label{sect:intro:overview}
\addcontentsline{toc}{section}{Structure of the Thesis}

The work described in this thesis has been carried out in six papers, each contained in a separate chapter. Four papers (Chapters~\ref{chap:fool}, \ref{chap:implementation}, \ref{chap:cnf} and \ref{chap:boogie}) were published in peer-reviewed conferences, one (Chapter~\ref{chap:tfx}) was published in a peer-reviewed workshop, and one (Chapter~\ref{chap:aws}) is a technical report not yet submitted for publication. The references of the papers have been combined into a single bibliography at the end of the thesis. Other than that, the papers have only been edited for formatting purposes, and in general appear in their original form.

The chapters of this thesis are arranged in the order in which their correspondent papers were written. Chapter~\ref{chap:fool} presents the syntax and semantics of FOOL. Chapter~\ref{chap:implementation} presents the implementation of FOOL in Vampire. Chapter~\ref{chap:cnf} presents an efficient clausification algorithm for FOOL. Chapter~\ref{chap:boogie} describes an encoding of the next state relations of imperative programs in FOOL. Chapter~\ref{chap:aws} describes an approach to network verification based on automated reasoning in first-order logic, which uses features of FOOL. Finally, Chapter~\ref{chap:tfx} describes TFX, the extension of the TPTP language that contains the syntax for FOOL.

Each of the papers contained in this thesis has been written and presented separately. As a result, the introductory remarks and preliminaries of some of the chapters overlap. Another consequence is that some ideas presented in earlier chapters are revisited and developed in later chapters. One example of such idea is the encoding of the next state relations of imperative programs in FOOL. A sketch of this encoding first appears in Chapter~\ref{chap:implementation} and preliminary experimental results are discussed in Chapter~\ref{chap:cnf}. The precise formal description of the encoding and extensive evaluation is however given later in Chapter~\ref{chap:boogie}. Another example is the set of syntactical constructs available in FOOL. The original description of FOOL in Chapter~\ref{chap:fool} does not include \LETIN\ expressions with simultaneous definitions, definitions of tuples and tuple expressions. These constructs are included in later chapters.

The contributions of the thesis are the cumulative contributions of all six papers. The rest of this section details the main contributions of each individual paper.

\subsection*{\hyperref[chap:fool]{Chapter 1.} A First Class Boolean Sort in\\First-Order Theorem Proving and TPTP}
The paper presents the syntax and semantics of \folb. We show that \folb\ is a modification of FOL and reasoning in it reduces to reasoning in FOL. We give a model-preserving \iffalse(modulo introduced definitions)\fi translation of \folb\ to FOL that can be used for proving theorems in \folb\ in a first-order prover. We discuss a modification of superposition calculus that can reason efficiently in the presence of Boolean sort. This modification includes replacement of one of the Boolean sort axioms with a specialised inference rule that we called \folb\ paramodulation. We note that the TPTP language can be changed to support \folb, which will also simplify some parts of the TPTP syntax. 

\paragraph{Statement of contribution.} The paper is co-authored with Laura Kov\'{a}cs and Andrei Voronkov. Evgenii Kotelnikov contributed to the formalisation of \folb{} and its translation to FOL.

\paragraph{Bibliographic information.} The paper has been published in the proceedings of the 8th Conference on Intelligent Computer Mathematics (CICM) in 2015~\cite{FOOL}.

\subsection*{\hyperref[chap:implementation]{Chapter 2.} The Vampire and the \folb{}}
The paper describes the implementation of \folb\ in Vampire. We extend and simplify the TPTP language by providing more powerful and uniform representations of \ITE\ and \LETIN\ expressions. We demonstrate usability and high performance of our implementation on two collections of benchmarks, coming from the higher-order part of the TPTP library and from the Isabelle interactive theorem prover. We compare the results of running Vampire on the benchmarks with those of SMT solvers and higher-order provers. Moreover, we compare the performance of Vampire with and without \folb{} paramodulation. We give a simple extension of \folb, allowing to express the next state relation of a program as a Boolean formula which is linear in the size of the program.

\paragraph{Statement of contribution.} The paper is co-authored with Laura Kov\'{a}cs, Giles Reger and Andrei Voronkov. Evgenii Kotelnikov contributed with the implementation of \folb{} in Vampire and the experiments.

\paragraph{Bibliographic information.} The paper has been published in the proceedings of the 5th ACM SIGPLAN Conference on Certified Programs and Proofs (CPP) in 2016~\cite{VampireAndFOOL}.

\subsection*{\hyperref[chap:cnf]{Chapter 3.} A Clausal Normal Form Translation\\for \folb{}}
The paper presents a clausification algorithm that translates a FOOL formula to an equisatisfiable set of first-order clauses. This algorithm aims to minimise the number of clauses and the size of the resulting signature, especially on formulas with \ITE, \LETIN\ expressions and complex Boolean structure. We demonstrate by experiments that the implementation of this algorithm in Vampire increases performance of the prover on \folb{} problems compared to the earlier translation of \folb{} formulas to full first-order logic. We extended Vampire with new preprocessing options that can be used to strengthen its portfolios.

\paragraph{Statement of contribution.} The paper is co-authored with Laura Kov\'{a}cs, Martin Suda and Andrei Voronkov. Evgenii Kotelnikov contributed with the extension of \newcnf{} that supports \folb{}, the implementation of this extension in Vampire and the experiments.

\paragraph{Bibliographic information.} The paper has been published in the proceedings of the 2nd Global Conference on Artificial Intelligence (GCAI) in 2016~\cite{FOOLCNF}.

\subsection*{\hyperref[chap:boogie]{Chapter 4.} A FOOLish Encoding of the Next State Relations of Imperative Programs}
The paper describes an encoding of the next state relations of imperative programs with variable updates and \ITE\ statements in FOOL. Based on this encoding the paper presents a translation of imperative programs annotated with their pre- and post-conditions to partial correctness properties of these programs. We demonstrate by experiments that this translation results in formulas that are easier for Vampire than the formulas produced by program verification tool such Boogie and BLT.

\paragraph{Statement of contribution.} The paper is co-authored with Laura Kov\'{a}cs and Andrei Voronkov. Evgenii Kotelnikov contributed with the formalisation of the translation of imperative programs to FOOL and the experiments.

\paragraph{Bibliographic information.} The paper has been published in the proceedings of the 9th International Joint Conference on Automated Reasoning (IJCAR) in 2018~\cite{KKV18}.

\subsection*{\hyperref[chap:aws]{Chapter 5.} Checking Network Reachability Properties by Automated Reasoning in First-Order Logic}
The paper describes an approach for static verification of virtual private cloud networks using automated theorem proving for first-order logic. We model networks with Horn clauses and check first-order properties of these models using the Vampire theorem prover. We used Vampire both as a saturation-based theorem prover and a finite model builder for different kinds of checked properties.

\paragraph{Statement of contribution.} The chapter is co-authored with Pavle Suboti\'{c} and based on a joint work with Byron~Cook, Temesghen Kahsai and Sean~McLaughlin. Evgenii Kotelnikov contributed with the encoding of network reachability properties in first-order logic and the implementation of a checker for these problems based on Vampire.

\subsection*{\hyperref[chap:tfx]{Chapter 6.} TFX: The TPTP Extended Typed First-Order Form}
The paper presents the new language TFX that extends and simplifies the language of typed first-order formulas TFF. TFX includes the first class Boolean sort, \ITE\ expressions, \LETIN\ expressions and tuples. The inclusion of these syntactic constructs was motivated by the work on FOOL and FOOL formulas can be directly expressed in TFX. TFX has been included in the latest release of the TPTP library.

\paragraph{Statement of contribution.} The paper is co-authored with Geoff Sutcliffe. Evgenii Kotelnikov contributed with the discussion of the TFX syntax, the description of FOOL and examples of FOOL problems.

\paragraph{Bibliographic information.} The paper has been published in the proceedings of the 6th Workshop on Practical Aspects of Automated Reasoning (PAAR) in 2018~\cite{SutcliffeK18}.

%\section*{Conclusion}
%\addcontentsline{toc}{section}{Conclusion}
%FOOL can be used to express quantified Boolean formulas (QBF) thanks to its first class Boolean sort. The clausification algorithm \nfcnf{}, presented in Chapter~\ref{chap:cnf}, translates these formulas into a CNF in effectively propositional logic (EPR). Obtaining a formula in EPR is a desirable property to have since there are first-order proving methods known to be efficient for dealing with the fragment (see e.g.~\cite{DBLP:conf/birthday/Korovin13}).

%The thesis focuses on practical features extending first-order theorem provers for making them better suited for applications of program verification and proof automation for interactive theorem provers.

% Furthermore, some problems that previously required higher-order logic can now be expressed directly in FOOL. For example, the current version of the TPTP library contains over a hundred of such problems. One can check these problems with first-order provers that support FOOL rather than higher-order provers.

\def\paperOneContentsTitle{A First Class Boolean Sort in\\First-Order Theorem Proving and TPTP}
\def\paperOneChapterTitle{A First Class Boolean Sort in\\First-Order Theorem Proving\\and TPTP}
\def\paperOneAuthors{Evgenii~Kotelnikov, Laura~Kov\'acs and Andrei~Voronkov}
\def\paperOneAbstract{To support reasoning about properties of programs operating with boolean values one needs theorem provers to be able to natively deal with the boolean sort. This way, program properties can be translated to first-order logic and theorem provers can be used to prove program properties efficiently. However, in the TPTP language, the input language of automated first-order theorem provers, the use of the boolean sort is limited compared to other sorts, thus hindering the use of first-order theorem provers in program analysis and verification. In this paper, we present an extension \folb\ of many-sorted first-order logic, in which the boolean sort is treated as a first-class sort. Boolean terms are indistinguishable from formulas and can appear as arguments to functions. In addition, \folb\ contains \ITE\ and \LETIN\ constructs. We define the syntax and semantics of \folb\ and its model-preserving translation to first-order logic. We also introduce a new technique of dealing with boolean sorts in superposition-based theorem provers. Finally, we discuss how the TPTP language can be changed to support \folb.}
\def\paperOnePublication{Published in the \emph{Proceedings of the 8th Conference on Intelligent\\Computer Mathematics}, pages 71--86. Springer, 2015.}
\paperchapter{\paperOneContentsTitle}
             {\paperOneChapterTitle}
             {\paperOneAuthors}
             {\paperOneAbstract}
             {\paperOnePublication}
\label{chap:fool}
\section{Introduction}
\label{sec:cicm15/introduction}

%\EK{TODO: Automated reasoning is central and also the hardest part in computer mathematics.}

Automated program analysis and verification requires
discovering and proving program properties. Typical examples of such properties are loop invariants or Craig interpolants. These properties usually are expressed in combined theories of various data structures, such as integers and arrays, and hence require reasoning with both theories and quantifiers. Recent approaches in interpolation and loop invariant generation~\cite{McMillan08,fase2009,hoder2012popl} present initial results of using first-order theorem provers for generating quantified program properties. First-order theorem provers can also be used to generate program properties with quantifier alternations~\cite{fase2009}; such properties could not be generated fully automatically by any previously known method.
Using first-order theorem prover to generate, and not only prove program properties, opens new directions in analysis and verification of real-life programs.

First-order theorem provers, such as iProver~\cite{iProver}, E~\cite{E13}, and Vampire~\cite{Vampire13}, lack however various features that are crucial for program analysis. For example, first-order theorem provers do not yet efficiently handle (combinations of) theories;
nevertheless, sound but incomplete theory axiomatisations can be used in a first-order prover even for theories having no finite axiomatisation. Another difficulty in modelling properties arising in program analysis using theorem provers is the gap between the semantics of expressions used in programming languages and expressiveness of the logic used by the theorem prover. A similar gap exists between the language used in presenting mathematics. For example, a standard way to capture assignment in program analysis is to use a \LETIN\ expression, which introduces a local binding of a variable, or a function for array assignments, to a value. There is no local binding expression in first-order logic, which means that any modelling of imperative programs using first-order theorem provers at the backend, should implement a translation of \LETIN\ expressions. Similarly, mathematicians commonly use local definitions within definitions and proofs. Some functional programming languages also contain expressions introducing local bindings. In all three cases, to facilitate the use of first-order provers, one needs a theorem prover implementing \LETIN\ constructs natively.

Efficiency of reasoning-based program analysis largely depends on how programs are translated into a collection of logical formulas capturing the program semantics. The boolean structure of a program property that can be efficiently treated by a theorem prover is however very sensitive to the architecture of the reasoning engine of the prover. Deriving and expressing program properties in the ``right'' format therefore requires solid knowledge about how theorem provers work and are implemented~--- something that a user of a verification tool might not have. Moreover, it can be hard to efficiently reason about certain classes of program properties, unless special inference rules and heuristics are added to the theorem prover, see e.g.~\cite{ATVA14} when it comes to prove properties of data collections with extensionality axioms.

In order to increase the expressiveness of program properties generated by reasoning-based program analysis, the language of logical formulas accepted by a theorem prover needs to be extended with constructs of programming languages. This way, a straightforward translation of programs into first-order logic can be achieved, thus relieving users from designing translations which can be efficiently treated by the theorem prover.
One example of such an extension is recently added to the TPTP language~\cite{TPTP} of first-order theorem provers, resembling \ITE\ and \LETIN\ expressions that are common in programming languages. Namely, special functions \lstinline'$ite_t' and \lstinline'$ite_f' can respectively be used to express a conditional statement on the level of logical terms and formulas, and \lstinline'$let_tt', \lstinline'$let_tf', \lstinline'$let_ff' and \lstinline'$let_ft' can be used to express local variable bindings for all four possible combinations of logical terms (\lstinline't') and formulas (\lstinline'f'). While satisfiability modulo theory (SMT) solvers, such as Z3~\cite{Z3} and CVC4~\cite{CVC4}, integrate \ITE\ and \LETIN\ expressions, in the first-order theorem proving community so far only Vampire supports such expressions.

To illustrate the advantage of using \ITE\ and \LETIN\ expressions in automated provers, let us consider the following example. We are interested in verifying the partial correctness of the code fragment below:
% \pagebreak
\begin{lstlisting}[language=cpp]
if (r(a)) {
  a := a + 1
} else {
  a := a + q(a)
}
\end{lstlisting}
using the pre-condition $((\forall x) P(x) \Rightarrow x \ge 0) \wedge ((\forall x) \mathtt{q}(x) > 0) \wedge P(\mathtt{a})$ and the post-condition $\mathtt{a} > 0$.
Let $\mathtt{a1}$ denote the value of the program variable $\mathtt{a}$ after the execution of the \verb'if' statement. Using \ITE\ and \LETIN\ expressions, the next state function for $\mathtt{a}$ can naturally be expressed by the following formula:
\begin{lstlisting}[language=cpp]
a1 = if r(a) then let a = a + 1 in a
             else let a = a + q(a) in a
\end{lstlisting}

This formula can further be encoded in TPTP, and hence used by a theorem prover as a hypothesis in proving partial correctness of the above code snippet. We illustrate below the TPTP encoding of the first-order problem corresponding to the partial program correctness problem we consider.  Note that the pre-condition becomes a hypothesis in TPTP, whereas the proof obligation given by the post-condition is a TPTP conjecture. All formulas below are typed first-order formulas (\lstinline'tff') in TPTP that use the built-in integer sort (\lstinline'$int').
\begin{lstlisting}[language=tptp]
tff(1, type, p: $int > $o).
tff(2, type, q: $int > $int).
tff(3, type, r: $int > $o).
tff(4, type, a: $int).
tff(5, hypothesis, ![X: $int]: (p(X) => $greatereq(X, 0))).
tff(6, hypothesis, ![X: $int]: ($greatereq(q(X), 0))).
tff(7, hypothesis, p(a)).
tff(8, hypothesis,
    a1 = $ite_t(r(a), $let_tt(a, $sum(a, 1), a),
                      $let_tt(a, $sum(a, q(a)), a))).
tff(9, conjecture, $greater(a1, 0)).
\end{lstlisting}

Running a theorem prover that supports \lstinline'$ite_t' and \lstinline'$let_tt' on this TPTP problem would prove the partial correctness of the program we considered. Note that without the use of \ITE\ and \LETIN\ expressions, a more tedious translation is needed for expressing the next state function of the program variable $\mathtt{a}$ as a first-order formula. When considering more complex programs containing multiple conditional expressions assignments and composition,
computing the next state function of a program variable results in a formula of size exponential in the number of conditional expressions. This problem of computing the next state function of variables is well-known in the program analysis community, by computing so-called static single assignment (SSA) forms. Using the \ITE\ and \LETIN\ expressions recently introduced in TPTP and already implemented in Vampire \cite{PSI14}, one can have a linear-size translation instead.

Let us however note that the usage of conditional expressions in TPTP is somewhat limited. The first argument of \lstinline'$ite_t' and \lstinline'$ite_f' is a logical formula, which means that a boolean condition from the program definition should be translated as such. At the same time, the same condition can be treated as a value in the program, for example, in a form of a boolean flag, passed as an argument to a function. Yet we cannot mix terms and formulas in the same way in a logical statement.
A possible solution would be to map the boolean type of programs to a user-defined boolean sort, postulate axioms about its semantics, and manually convert boolean terms into formulas where needed. This approach, however, suffers the disadvantages mentioned earlier, namely the need to design a special translation and its possible inefficiency.

Handling boolean terms as formulas is needed not only in applications of reasoning-based program analysis, but also in various problems of formalisation of mathematics.
For example, if one looks at two largest kinds of attempts to formalise mathematics and proofs: those performed by interactive proof assistants, such as Isabelle~\cite{Isabelle},  and the Mizar project~\cite{Mizar}, one can see that first-order theorem provers are the main workhorses behind computer proofs in both cases~--- see e.g.~\cite{Sledgehammer,DBLP:conf/icms/UrbanHV10}.
Interactive theorem provers, such as Isabelle routinely use quantifiers over booleans.  Let us illustrate this by the
following examples, chosen among 490 properties about (co)algebraic datatypes, featuring quantifiers over booleans, generated by Isabelle and kindly found for us by Jasmin Blanchette. Consider the distributivity of a conditional expression (denoted by the $\mathrm{ite}$ function) over logical connectives, a pattern that is widely used in reasoning about properties of data structures. For lists and the $\mathtt{contains}$ function that checks that its second argument contains the first one, we have the following example:
\begin{gather}\label{formula:contains}
  \begin{aligned}
&(\forall\ofsort{p}{\bool})(\forall\ofsort{l}{list_A})(\forall\ofsort{x}{A})(\forall\ofsort{y}{A}) \\
&\quad\mathtt{contains}(l,\mathrm{ite}(p,x,y)) \doteq \\
&\quad\quad(p \Rightarrow \mathtt{contains}(l,x)) \wedge (\neg p \Rightarrow \mathtt{contains}(l,y))
 \end{aligned}
\end{gather}

A more complex example with a heavy use of booleans is the unsatisfiability of the definition of $\mathtt{subset\_sorted}$.
\begin{gather}\label{formula:subset-sorted}
\begin{aligned}
&(\forall\ofsort{l_1}{list_A})(\forall\ofsort{l_2}{list_A})(\forall\ofsort{p}{\bool}) \\
&\hspace{0.5em}\neg (\mathtt{subset\_sorted}(l_1,\,l_2) \doteq p ~\wedge \\
&\hspace{1.6em}      (\forall\ofsort{l_2'}{list_A})\neg (l_1 \doteq \mathtt{nil} \wedge l_2 \doteq l_2' \wedge p) ~\wedge \\
&\hspace{1.6em}      (\forall\ofsort{x_1}{A})(\forall\ofsort{l_1'}{list_A})\neg (l_1 \doteq \mathtt{cons}(x_1,\,l_1') \wedge l_2 \doteq \mathtt{nil} \wedge \neg p) ~\wedge \\
&\hspace{1.6em}      (\forall\ofsort{x_1}{A})(\forall\ofsort{l_1'}{list_A})(\forall\ofsort{x_2}{A})(\forall\ofsort{l_2'}{list_A}) \\
&\hspace{2.1em}       \neg (l_1 \doteq \mathtt{cons}(x_1,\,l_1') \wedge l_2 \doteq \mathtt{cons}(x_2,\,l_2') ~\wedge \\
&\hspace{3.3em}       p \doteq \mathrm{ite}(x_1 < x_2,\,\false,\\
&\hspace{6.7em}                             \mathrm{ite}(x_1 \doteq x_2,\mathtt{subset\_sorted}(l_1',\,l_2'), \\
&\hspace{12.1em}                                        \mathtt{subset\_sorted}(\mathtt{cons}(x_1,\,l_1'),\,l_2')))))
\end{aligned}
\end{gather}
The $\mathtt{subset\_sorted}$ function takes two sorted lists and checks that its second argument is a sublist of the first one.

Problems with boolean terms are also common in the SMT-LIB project~\cite{SMT-LIB}, the collection of benchmarks for SMT-solvers. Its core logic is a variant of first-order logic that treats boolean terms as formulas, in which logical connectives and conditional expressions are defined in the core theory.

%Note in particular that this formula employes quantification over boolean variables and passing boolean terms as arguments to logical connectives, that would not be admissible in ordinary first-order logic.

In this paper we propose a modification \folb\ of first-order logic, which includes a first-class boolean sort and \ITE\ and \LETIN\ expressions, aimed for being used in automated first-order theorem proving. It is the smallest logic that contains both the SMT-LIB core theory and the monomorphic first-order subset of TPTP. The syntax and semantics of the logic are given in Section~\ref{sec:folbool}.
%In this paper we propose a modification of first-order logic, similar to the SMT-LIB logic, aimed at being used for first-order theorem proving. The modification includes formalisation of \verb'if'-\verb'then'-\verb'else' and \verb'let'-\verb'in' expressions and treatment of the boolean sort as a first class sort. This way the translation of certain program fragments with boolean values into logical statements become straightforward. The syntax and semantics of the logic is given in Section~\ref{sec:folbool}.
We further describe how \folb\ can be translated to the ordinary many-sorted first-order logic in Section~\ref{sec:folb-to-fol}.
Section~\ref{sec:superposition} discusses superposition-based theorem proving and proposes a new way of dealing with the boolean sort in it.
In Section~\ref{sec:tptp} we discuss the support of the boolean sort in TPTP and propose changes to it required to support a first-class boolean sort. We point out that such changes can also partially simplify the syntax of TPTP.
Section~\ref{sec:cicm15/related} discusses related work and Section~\ref{sec:cicm15/conclusions} contains concluding remarks.

The main contributions of this paper are the following:

\begin{enumerate}
\item the definition of \folb\ and its semantics;
\item a translation from \folb\ to first-order logic, which can be used to support \folb\ in existing first-order theorem provers;
\item a new technique of dealing with the boolean sort in superposition theorem provers, allowing one to replace boolean sort axioms by special rules;
\item a proposal of a change to the TPTP language, intended to support \folb\ and also simplify \ITE\ and \LETIN\ expressions.
\end{enumerate}


%------------------------------------------------------------------------------
\section[First-Order Logic with Boolean Sort]{First-Order Logic with Boolean Sort}
\label{sec:folbool}

First-order logic with the boolean sort (\folb) extends many-sorted first-order logic (FOL) in two ways:
\begin{enumerate}
\item formulas can be treated as terms of the built-in boolean sort; and
\item one can use \ITE\ and \LETIN\ expressions defined below.
\end{enumerate}
\folb\ is the smallest logic containing both the SMT-LIB core theory and the monomorphic first-order part of the TPTP language. It extends the SMT-LIB core theory by adding \LETIN\ expressions defining functions and TPTP by the first-class boolean sort.


\subsection{Syntax}

We assume a countable infinite set of \emph{variables}.

\begin{definition}\label{def:folb-signature}\em
  A \emph{signature} of first-order logic with the boolean sort is a triple $\Sigma = (S, F, \context)$, where:

  \begin{enumerate}
  \item $S$ is a set of \emph{sorts}, which contains a special sort $\bool$. A \emph{type} is either a sort or a non-empty sequence $\sigma_1,\ldots,\sigma_n,\sigma$ of sorts, written as $\sigma_1 \times \ldots \times \sigma_n \to \sigma$. When $n = 0$, we will simply write $\sigma$ instead of $\to\sigma$. We call a \emph{type assignment} a mapping from a set of variables and function symbols to types, which maps variables to sorts.

    \item $F$ is a set of \emph{function symbols}. We require $F$ to contain binary function symbols $\vee$, $\wedge$, $\implies$ and $\liff$, used in infix form, a unary function symbol $\neg$, used in prefix form, and nullary function symbols $\true$, $\false$.

    \item $\context$ is a \emph{type assignment} which maps each function symbol $f$ into a type $\tau$. When the signature is clear from the context, we will write $\ofsort{f}{\tau}$ instead of $\context(f)=\tau$ and say that $f$ is of the type $\tau$.

    We require the symbols $\vee, \wedge, \implies, \liff$ to be of the type $\bool \times \bool \to \bool$, $\neg$ to be of the type $\bool \to \bool$ and $\true,\false$ to be of the type $\bool$. \QED
  \end{enumerate}
\end{definition}
In the sequel we assume that $\Sigma = (S,F,\context)$ is an arbitrary but fixed signature.

To define the semantics of \folb, we will have to extend the signature and also assign sorts to variables. Given a type assignment $\context$, we define $\context,x:\sigma$ to be the type assignment that maps a variable $x$ to $\sigma$ and coincides otherwise with $\context$. Likewise, we define $\context,f:\tau$ to be the type assignment that maps a function symbol $f$ to $\tau$ and coincides otherwise with $\context$.

Our next aim is to define the set of terms and their sorts with respect to a type assignment $\context$. This will be done using a relation $\context \vdash t:\sigma$, where $\sigma \in S$, terms can then be defined as all such expressions $t$.

\begin{definition}\label{def:folb-terms}\rm
  The relation $\context \vdash t:\sigma$, where $t$ is an expression and $\sigma \in S$ is defined inductively as follows. If $\context \vdash t:\sigma$, then we will say that $t$ is a \emph{term of the sort $\sigma$} w.r.t.\ $\context$.
%We will also write $\context \vdash t_1:\sigma_1,\ldots,$
  \begin{enumerate}
    \item If $\context(x) = \sigma$, then $\context \vdash x:\sigma$.

    \item If $\context(f) = \sigma_1 \times \ldots \times \sigma_n \to \sigma$, $\context \vdash t_1:\sigma_1$, \ldots, $\context \vdash t_n:\sigma_n$, then $\context \vdash  f(t_1, \ldots, t_n) : \sigma$.

    \item If $\context \vdash \phi:\bool$, $\context \vdash t_1:\sigma$ and $\context \vdash t_2:\sigma$, then $\context \vdash (\ite{\phi}{t_1}{t_2}):\sigma$.

    \item Let $f$ be a function symbol and $x_1,\ldots,x_n$ pairwise distinct variables. If $\context,x_1:\sigma_1,\ldots,x_n:\sigma_n \vdash s:\sigma$ and $\context,f:(\sigma_1\times \ldots \times\sigma_n \to\sigma) \vdash t : \tau$, then $\context \vdash (\letin{f(x_1:\sigma_1, \ldots, x_n:\sigma_n)}{s}{t}) : \tau$.

    \item If $\context \vdash  s:\sigma$ and $\context \vdash  t:\sigma$, then $\context \vdash (s \eql t) : \bool$.

    \item If $\context,x : \sigma \vdash \phi : \bool$, then $\context \vdash (\forall x : \sigma)\phi : \bool$ and $\context \vdash (\exists x:\sigma)\phi : \bool$. \QED
  \end{enumerate}
\end{definition}
We only defined a \LETIN\ expression for a single function symbol. It is not hard to extend it to a \LETIN\ expression that binds multiple pairwise distinct function symbols in parallel, the details of such an extension are straightforward.

When $\context$ is the type assignment function of $\Sigma$ and $\context \vdash t : \sigma$, we will say that $t$ is a \emph{$\Sigma$-term of the sort $\sigma$}, or simply that $t$ is \emph{a term of the sort $\sigma$}. It is not hard to argue that every $\Sigma$-term has a unique sort.

According to our definition, not every term-like expression has a sort. For example, if $x$ is a variable and $\context$ is not defined on $x$, then $x$ is a not a $term$ w.r.t.\ $\context$. To make the relation between term-like expressions and terms clear, we introduce a notion of free and bound occurrences of variables and function symbols. We call the following occurrences of variables and function symbols \emph{bound}:

\begin{enumerate}
\item any occurrence of $x$ in $(\forall x:\sigma) \phi$ or in $(\exists x:\sigma) \phi$;
\item in the term $\letin{f(x_1:\sigma_1, \ldots, x_n:\sigma_n)}{s}{t}$ any occurrence of a variable $x_i$ in $f(x_1:\sigma_1, \ldots, x_n:\sigma_n)$ or in $s$, where $i = 1,\ldots, n$.
\item in the term $\letin{f(x_1:\sigma_1, \ldots, x_n:\sigma_n)}{s}{t}$ any occurrence of the function symbol $f$ in $f(x_1:\sigma_1, \ldots, x_n:\sigma_n)$ or in $t$.
\end{enumerate}
All other occurrences are called \emph{free}. We say that a variable or a function symbol is \emph{free} in a term $t$ if it has at least one free occurrence in $t$. A term is called \emph{closed} if it has no occurrences of free variables.

\begin{theorem}\rm
  Suppose $\context \vdash t : \sigma$. Then
  \begin{enumerate}
    \item for every free variable $x$ of $t$, $\context$ is defined on $x$;
    \item for every free function symbol $f$ of $t$, $\context$ is defined on $f$;
    \item if $x$ is a variable not free in $t$, and $\sigma'$ is an arbitrary sort, then
      $\context, x : \sigma' \vdash t : \sigma$;
    \item if $f$ is a function symbol not free in $t$, and $\tau$ is an arbitrary type, then $\context, f : \tau \vdash t : \sigma$. \QED
  \end{enumerate}
\end{theorem}

\begin{definition}\rm
  A \emph{predicate symbol} is any function symbol of the type $\sigma_1 \times \ldots \times \sigma_n \to \bool$.
  A \emph{$\Sigma$-formula} is a $\Sigma$-term of the sort $\bool$. All $\Sigma$-terms that are not $\Sigma$-formulas are called \emph{non-boolean terms}. \QED
\end{definition}

Note that, in addition to the use of \LETIN\ and \ITE, \folb\ is a proper extension of first-order logic. For example, in \folb\ formulas can be used as arguments to terms and one can quantify over booleans. As a consequence, every quantified boolean formula is a formula in \folb.

\subsection{Semantics}

As usual, the semantics of \folb\ is defined by introducing a notion of \emph{interpretation} and defining how a term is evaluated in an interpretation.

\begin{definition}\label{def:folb-interpretation}\rm
  Let $\context$ be a type assignment.
  A \emph{$\context$-interpretation} $\intI$ is a map, defined as follows. Instead of $\intI(e)$ we will write $\interpret{e}{\intI}$, for every element $e$ in the domain of $\intI$.
  \begin{enumerate}
    \item Each sort $\sigma \in S$ is mapped to a nonempty domain $\interpret{\sigma}{\intI}$. We require $\interpret{\bool}{\intI} = \left\{0, 1\right\}$.

    \item If $\context \vdash x:\sigma$, then $\interpret{x}{\intI} \in \interpret{\sigma}{\intI}$.

    \item If $\context(f) = \sigma_1 \times \ldots \times \sigma_n \to \sigma$, then $\interpret{f}{\intI}$ is a function from $\interpret{\sigma_1}{\intI} \times \ldots \times \interpret{\sigma_n}{\intI}$ to $\interpret{\sigma}{\intI}$.

    \item We require $\interpret{\true}{\intI} = 1$ and $\interpret{\false}{\intI} = 0$. We require $\interpret{\wedge}{\intI}$, $\interpret{\vee}{\intI}$, $\interpret{\implies}{\intI}$, $\interpret{\liff}{\intI}$ and $\interpret{\neg}{\intI}$ respectively to be the logical conjunction, disjunction, implication, equivalence and negation, defined over $\{0,1\}$ in the standard way. \QED
  \end{enumerate}
%  We will call the symbols $\true$, $\false$, $\wedge$, $\vee$, $\implies$, $\liff$ and $\neg$ \emph{interpreted} and all other symbols \emph{uninterpreted}.\AV{don't know if this will be used}
\end{definition}

Given a $\context$-interpretation $\intI$ and a function symbol $f$, we define $\variant{\intI}{f}{g}$ to be the mapping that maps $f$ to $g$ and coincides otherwise with $\intI$.
Likewise, for a variable $x$ and value $a$ we define $\variant{\intI}{x}{a}$ to be the mapping that maps $x$ to $a$ and coincides otherwise with $\intI$.

\begin{definition}\label{def:folb-term-evaluation}\rm
  Let $\intI$ be a $\context$-interpretation, and $\context \vdash t:\sigma$. The \emph{value of $t$ in $\intI$}, denoted as $\eval{t}{\intI}$, is a value in $\interpret{\sigma}{\intI}$ inductively defined as follows:
  \[
    \begin{array}{rcl}
      \eval{x}{\intI}&=&\interpret{x}{\intI}.
      \\*[1ex]
      \eval{f(t_1, \ldots, t_n)}{\intI}&=&\interpret{f}{\intI}(\eval{t_1}{\intI}, \ldots, \eval{t_n}{\intI}).
      \\*[1ex]
      \eval{s \eql t}{\intI} & = &
        \left\{ \begin{array}{ll}
                  1, & \text{if } \eval{s}{\intI} = \eval{t}{\intI}; \\*[1ex]
                  0, & \text{otherwise.}
                \end{array}\right.
      \\*[1ex]
      \eval{(\forall x : \sigma)\phi}{\intI} & = &
        \left\{ \begin{array}{ll}
                  1, & \text{if } \eval{\phi}{\replacement{\intI}{x}{a}} = 1\\*[1ex]
                     & \text{~~~for all } a \in \interpret{\sigma}{\intI}; \\*[1ex]
                  0, & \text{otherwise.}
                \end{array}\right.
      \\*[1ex]
      \eval{(\exists x : \sigma)\phi}{\intI} & = &
        \left\{ \begin{array}{ll}
                  1, & \text{if } \eval{\phi}{\replacement{\intI}{x}{a}} = 1\\*[1ex]
                     & \text{~~~for some } a \in \interpret{\sigma}{\intI}; \\*[1ex]
                  0, & \text{otherwise.}
                \end{array}\right.
      \\*[1ex]
      \eval{\ite{\phi}{s}{t}}{\intI}&=&\left\{ \begin{array}{ll}
        \eval{s}{\intI},&\text{if $\eval{\phi}{\intI} = 1$;} \\*[1ex]
        \eval{t}{\intI},&\text{otherwise.}
      \end{array}\right.
    \end{array}
  \]
  \[
    \begin{array}{rcl}
      \eval{\letin{f(x_1:\sigma_1,\ldots,x_n:\sigma_n)}{s}{t}}{\intI}&=&\eval{t}{\replacement{\intI}{f}{g}},
    \end{array}
  \]
  where $g$ is such that for all $i = 1, \ldots, n$ and $a_i \in \interpret{\sigma_i}{\intI}$, we have $g(a_1, \ldots, a_n) = \eval{s}{\replacement{\intI}{x_1 \ldots x_n}{a_1 \ldots a_n}}$. \QED
\end{definition}

\begin{theorem}\label{thm:semantics}\rm
  Let $\context \vdash \phi : \bool$ and $\intI$ be a $\context$-interpretation. Then
  \begin{enumerate}
    \item for every free variable $x$ of $\phi$, $\intI$ is defined on $x$;
    \item for every free function symbol $f$ of $\phi$, $\intI$ is defined on $f$;
    \item if $x$ is a variable not free in $\phi$, $\sigma$ is an arbitrary sort, and $a \in \interpret{\sigma}{\intI}$ then $\eval{\phi}{\intI} = \eval{\phi}{\replacement{\intI}{x}{a}}$;
    \item if $f$ is a function symbol not free in $\phi$, $\sigma_1,\ldots,\sigma_n,\sigma$ are arbitrary sorts and $g \in \interpret{\sigma_1}{\intI} \times \ldots \times \interpret{\sigma_n}{\intI} \to \interpret{\sigma}{\intI}$, then $\eval{\phi}{\intI} = \eval{\phi}{\replacement{\intI}{f}{g}}$. \QED
  \end{enumerate}
\end{theorem}

Let $\context \vdash \phi : \bool$. A $\context$-interpretation $\intI$ is called a \emph{model} of $\phi$, denoted by $\intI \models \phi$, if $\eval{\phi}{\intI} = 1$. If $\intI \models \phi$, we also say that $\intI$ \emph{satisfies} $\phi$. We say that $\phi$ is \emph{valid}, if $\intI \models \phi$ for all $\context$-interpretations $\intI$, and \emph{satisfiable}, if $\intI \models \phi$ for at least one $\context$-interpretation $\intI$. Note that Theorem~\ref{thm:semantics} implies that any interpretation, which coincides with $\intI$ on free variables and free function symbols of $\phi$ is also a model of $\phi$.


%------------------------------------------------------------------------------
\section{Translation of \folb{} to FOL}
\label{sec:folb-to-fol}

% !TEX root = ../main.tex
\folb\ is a modification of FOL. Every FOL formula is syntactically a \folb\ formula and has the same models, but not the other way around. In this section we present a translation from \folb\ to FOL, which preserves models. This translation can be used for proving theorems of \folb\ using a first-order theorem prover. We do not claim that this translation is efficient -- more research is required on designing translations friendly for first-order theorem provers.

We do not formally define many-sorted FOL with equality here, since FOL is essentially a subset of \folb, which we will discuss now.  

We say that an occurrence of a subterm $s$ of the sort $\bool$ in a term $t$ is in a \emph{formula context} if it is an argument of a logical connective or the occurrence in either $(\forall x:\sigma)s$ or $(\exists x:\sigma)s$. We say that an occurrence of $s$ in $t$ is in a \emph{term context} if this occurrence is an argument of a function symbol, different from a logical connective, or an equality. We say that a formula of \folb\ is \emph{syntactically first order} if it contains no \ITE\ and \LETIN\ expressions, no variables occurring in a formula context and no formulas occurring in a term context. By restricting the definition of terms to the subset of syntactically first-order formulas, we obtain the standard definition of many-sorted first-order logic, with the only exception of having a distinguished boolean sort and constants $\true$ and $\false$ occurring in a formula context.

Let $\phi$ be a closed $\Sigma$-formula of \folb{}. We will perform the following steps to translate $\phi$ into a first-order formula. During the translation we will maintain a set of formulas $D$, which initially is empty. The purpose of $D$ is to collect a set of formulas (definitions of new symbols), which guarantee that the transformation preserves models.

\begin{enumerate}
\item Make a sequence of translation steps obtaining a syntactically first order formula $\phi'$. During this translation we will introduce new function symbols and add their types to the type assignment $\context$. We will also add formulas describing properties of these symbols to $D$. The translation will guarantee that the formulas $\phi$ and $\bigwedge_{\psi \in D}\psi \wedge \phi'$ are equivalent, that is, have the same models restricted to $\Sigma$.

\item Replace the constants $\true$ and $\false$, standing in a formula context, by nullary predicates $\top$ and $\bot$ respectively, obtaining a first-order formula.

\item Add special boolean sort axioms.
\end{enumerate}
During the translation, we will say that a function symbol or a variable is \emph{fresh} if it neither appears in $\phi$ nor in any of the definitions, nor in the domain of $\context$.

We also need the following definition. Let $\context \vdash t:\sigma$, and $x$ be a variable occurrence in $t$. The \emph{sort of this occurrence of $x$} is defined as follows:

\begin{enumerate}
\item any free occurrence of $x$ in a subterm $s$ in the scope of $(\forall x:\sigma')s$ or $(\exists x:\sigma')s$ has the sort $\sigma'$.
\item any free occurrence of $x_i$ in a subterm $s_1$ in the scope of \\$\letin{f(x_1:\sigma_1, \ldots, x_n:\sigma_n)}{s_1}{s_2}$ has the sort $\sigma_i$, where $i = 1,\ldots,n$. 
\item a free occurrence of $x$ in $t$ has the sort $\context(x)$.
\end{enumerate}
If $\context \vdash t:\sigma$, $s$ is a subterm of $t$ and $x$ a free variable in $s$, we say that $x$ has a sort $\sigma'$ in $s$ if its free occurrences in $s$ have this sort.

The translation steps are defined below. We start with an empty set $D$ and an initial \folb\ formula $\phi$, which we would like to change into a syntactically first-order formula. At every translation step we will select a formula $\chi$, which is either $\phi$ or a formula in $D$, which is not syntactically first-order, replace a subterm in $\chi$ it by another subterm, and maybe add a formula to $D$. The translation steps can be applied in any order.

\begin{enumerate}
  \item Replace a boolean variable $x$ occurring in a formula context, by $x \eql \true$.

  \item Suppose that $\psi$ is a formula occurring in a term context such that (i) $\psi$ is different from $\true$ and $\false$, (ii) $\psi$ is not a variable, and (iii) $\psi$ contains no free occurrences of function symbols bound in $\chi$. Let $x_1,\ldots,x_n$ be all free variables of $\psi$ and $\sigma_1,\ldots,\sigma_n$ be their sorts. Take a fresh function symbol $g$, add the formula $(\forall x_1:\sigma_1)\ldots(\forall x_n:\sigma_n) (\psi \liff g(x_1,\ldots,x_n) \eql \true)$ to $D$ and replace $\psi$ by $g(x_1,\ldots,x_n)$. Finally, change $\context$ to $\context,g : \sigma_1 \times \ldots \times \sigma_n \to \bool$.

  \item Suppose that $\ite{\psi}{s}{t}$ is a term containing no free occurrences of function symbols bound in $\chi$. Let $x_1,\ldots,x_n$ be all free variables of this term and $\sigma_1,\ldots,\sigma_n$ be their sorts. Take a fresh function symbol $g$, add the formulas $(\forall x_1:\sigma_1)\ldots(\forall x_n:\sigma_n) (\psi \implies g(x_1,\ldots,x_n) \eql s)$ and $(\forall x_1:\sigma_1)\ldots(\forall x_n:\sigma_n) (\neg\psi \implies g(x_1,\ldots,x_n) \eql t)$ to $D$ and replace this term by $g(x_1,\ldots,x_n)$. Finally, change $\context$ to $\context,g : \sigma_1 \times \ldots \times \sigma_n \to \sigma_0$, where $\sigma_0$ is such that $\context,x_1:\sigma_1,\ldots,x_n:\sigma_n \vdash s : \sigma_0$.

  \item Suppose that $\letin{f(x_1:\sigma_1, \ldots, x_n:\sigma_n)}{s}{t}$ is a term containing no free occurrences of function symbols bound in $\chi$. Let $y_1,\ldots,y_m$ be all free variables of this term and $\tau_1,\ldots,\tau_m$ be their sorts. Note that the variables in $x_1,\ldots,x_n$ are not necessarily disjoint from the variables in $y_1,\ldots,y_m$. 

Take a fresh function symbol $g$ and fresh sequence of variables $z_1,\ldots,z_n$. Let the term $s'$ be obtained from $s$ by replacing all free occurrences of $x_1,\ldots,x_n$ by $z_1,\ldots,z_n$, respectively. Add the formula $(\forall z_1:\sigma_1)\ldots(\forall z_n:\sigma_n) (\forall y_1:\tau_1)\ldots(\forall y_m:\tau_m) (g(z_1,\ldots,z_n,y_1,\ldots,\allowbreak y_m) \eql s')$ to $D$. Let the term $t'$ be obtained from $t$ by replacing all bound occurrences of $y_1,\ldots,y_m$ by fresh variables and each application $f(t_1, \ldots, t_n)$ of a free occurrence of $f$ in $t$ by $g(t_1, \ldots, t_n,\allowbreak y_1, \ldots, y_m)$. Then replace $\letin{f(x_1:\sigma_1, \ldots, x_n:\sigma_n)}{s}{t}$ by $t'$. Finally, change $\context$ to $\context,g : \sigma_1 \times \ldots \times \sigma_n \times \tau_1 \times \ldots \times \tau_m \to \sigma_0$, where $\sigma_0$ is such that $\context,x_1:\sigma_1,\ldots,x_n:\sigma_n,y_1:\tau_1,\ldots,y_m:\tau_m \vdash s : \sigma_0$. 
\end{enumerate}
The translation terminates when none of the above rules apply.

We will now formulate several of properties of this translation, which will imply that, in a way, it preserves models. These properties are not hard to prove, we do not include proofs in this paper.

\begin{lemma}\label{lemma:step-preserves-equivalence}\rm
  Suppose that a single step of the translation changes a formula $\phi_1$ into $\phi_2$, $\delta$ is the formula added at this step (for step 1 we can assume $\true=\true$ is added), $\context$ is the type assignment before this step and $\context'$ is the type assignment after. Then for every $\context'$-interpretation $\intI$ we have $\intI \models \delta \implies (\phi_1 \liff \phi_2)$. \QED
\end{lemma}

By repeated applications of this lemma we obtain the following result.

\begin{lemma}\label{lemma:definitions-preserve-models}\rm
  Suppose that the translation above changes a formula $\phi$ into $\phi'$, $D$ is the set of definitions obtained during the translation, $\context$ is the initial type assignment and $\context'$ is the final type assignment of the translation. Let $I'$ be any interpretation of $\context'$. Then $I' \models \bigwedge_{\psi \in D} \psi \implies (\phi \Leftrightarrow \phi')$. \QED
\end{lemma}

We also need the following result.

\begin{lemma}\label{lem:termination}\rm
  Any sequence of applications of the translation rules terminates. \QED
\end{lemma}

The lemmas proved so far imply that the translation terminates and the final formula is equivalent to the initial formula in every interpretation satisfying all definitions in $D$. To prove model preservation, we also need to prove some properties of the introduced definitions. 

\begin{lemma}\label{lem:satisfy:definitions}\rm
  Suppose that one of the steps 2--4 of the translation translates a formula $\phi_1$ into $\phi_2$, $\delta$ is the formula added at this step, $\context$ is the type assignment before this step, $\context'$ is the type assignment after, and $g$ is the fresh function symbol introduced at this step. Let also $\intI$ be $\context$-interpretation. Then there exists a function $h$ such that $\replacement{\intI}{g}{h} \models \delta$. \QED
\end{lemma}

These properties imply the following result on model preservation.

\begin{theorem}\label{thm:model:preservation}\rm
  Suppose that the translation above translates a formula $\phi$ into $\phi'$, $D$ is the set of definitions obtained during the translation, $\context$ is the initial type assignment and $\context'$ is the final type assignment of the translation. 
  \begin{enumerate}
    \item Let $\intI$ be any $\context$-interpretation. Then there is a $\context'$-interpretation $I'$ such that $\intI'$ is an extension of $\intI$ and $\intI' \models \bigwedge_{\psi \in D} \psi \wedge \phi'$.
    \item Let $\intI'$ be a $\context'$-interpretation and $\intI' \models \bigwedge_{\psi \in D} \psi \wedge \phi'$. Then $\intI' \models \phi$. \QED
  \end{enumerate}
\end{theorem}
This theorem implies that $\phi$ and $\bigwedge_{\psi \in D} \psi \wedge \phi'$ have the same models, as far as the original type assignment (the type assignment of $\Sigma$) is concerned. The formula $\bigwedge_{\psi \in D} \psi \wedge \phi'$ in this theorem is syntactically first-order. Denote this formula by $\gamma$. Our next step is to define a model-preserving translation from syntactically first-order formulas to first-order formulas.

To make $\gamma$ into a first-order formula, we should get rid of $\true$ and $\false$ occurring in a formula context. To preserve the semantics, we should also add axioms for the boolean sort, since in first-order logic all sorts are uninterpreted, while in \folb\ the interpretations of the boolean sort and constants $\true$ and $\false$ are fixed. 

To fix the problem, we will add axioms expressing that the boolean sort has two elements and that $\true$ and $\false$ represent the two distinct elements of this sort.
\begin{equation}\label{axiom:bool}
  \forall (x:\bool)(x \eql \true \vee x \eql \false) \wedge \true \not\eql \false.
\end{equation}
Note that this formula is a tautology in \folb, but not in FOL.

Given a syntactically first-order formula $\gamma$, we denote by $\toFOL{\gamma}$ the formula obtained from $\gamma$ by replacing all occurrences of $\true$ and $\false$ in a formula context by logical constants $\top$ and $\bot$ (interpreted as always true and always false), respectively and adding formula \eqref{axiom:bool}.

\begin{theorem}\label{thm:model:preservation:2}\rm
  Let $\context$ is a type assignment and $\gamma$ be a syntactically first-order formula such that $\context \vdash \gamma:\bool$.
  \begin{enumerate}
  \item Suppose that $\intI$ is a $\context$-interpretation and $\intI \models \gamma$ in \folb. Then $\intI \models \toFOL{\gamma}$ in first-order logic.
  \item Suppose that $\intI$ is a $\context$-interpretation and $\intI \models \toFOL{\gamma}$ in first-order logic. Consider the \folb-interpretation $\intI'$ that is obtained from $\intI$ by changing the interpretation of the boolean sort $\bool$ by $\{0,1\}$ and the interpretations of $\true$ and $\false$ by the elements $1$ and $0$, respectively, of this sort. Then $\intI' \models \gamma$ in \folb. \QED
  \end{enumerate}
\end{theorem}

Theorems~\ref{thm:model:preservation} and~\ref{thm:model:preservation:2} show that our translation preserves models. Every model of the original formula can be extended to a model of the translated formulas by adding values of the function symbols introduced during the translation. Likewise, any first-order model of the translated formula becomes a model of the original formula after changing the interpretation of the boolean sort to coincide with its interpretation in \folb.

%------------------------------------------------------------------------------
\section{Superposition for \folb{}}
\label{sec:superposition}

% !TEX root = ../main.tex
In
Section~\ref{sec:folb-to-fol} we presented a model-preserving
syntactic translation of \folb{} to FOL.
Based on this translation, automated reasoning about \folb{} formulas
can be done by translating a \folb{} formula into a FOL
formula, and using an automated first-order theorem prover on the resulting FOL formula.
State-of-the-art first-order theorem provers, such as Vampire~\cite{Vampire13}, E~\cite{E13} and
Spass~\cite{Spass}, implement superposition calculus for proving first-order formulas. Naturally, we would like to have a translation exploiting such provers in an efficient manner.

Note however that our translation adds the two-element domain axiom
$\forall (x:\bool)\allowbreak(x \eql \true \vee x \eql \false)$ for the boolean sort. This axioms will be converted to the clause
\begin{equation}\label{clause:T|F}
  x \eql \true \vee x \eql \false,
\end{equation}
where $x$ is a boolean variable. In this section we
explain why this axiom requires a special treatment and propose a solution to overcome problems caused by its presence.
%
%ththerefore be reduced to reasoning about the translated FOL formula
%using established technics such as superposition calculi. The extra
%axioms, added to the set of definitions at the last step of the
%translation, however, might not be treated efficiently by a
%superposition inference system. In this section
%we will explain the difficulties raised by the presence of these axioms and formulate a property that must be satisfied by a superposition inference system in order to be able to reason in \folb{} efficiently.

We assume some basic understanding of first-order theorem proving and superposition calculus, see, e.g.~\cite{Ganzinger01,NieuwenhuisRubio:HandbookAR:paramodulation:2001}. We fix a superposition inference system for first-order logic with equality, parametrised by a simplification ordering $\succ$ on literals and a well-behaved literal selection function \cite{Vampire13}, that is a function that guarantees completeness of the calculus. We denote selected literals by underlining them. We assume that equality literals are treated by a dedicated inference rule, namely, the ordered paramodulation rule~\cite{Robinson1969}:
\[
\infer[\quad\text{if}\ \theta = \mathrm{mgu}(l, s),]{(L[r] \vee C \vee D)\theta}%
{\underline{l \eql r} \vee C & \underline{L[s]} \vee D}
\]
where $C,D$ are clauses, $L$ is a literal, $l,r,s$ are terms, $\mathrm{mgu}(l, s)$ is a most general unifier of $l$ and $s$, and $r\theta \not\succeq l\theta$.
The notation $L[s]$ denotes that $s$ is a subterm of $L$, then $L[r]$ denotes the result of replacement of $s$ by $r$.

Suppose now that we use an off-the-shelf superposition theorem prover to reason about FOL formulas obtained by our translation. W.l.o.g, we assume that $\true \succ \false$ in the term ordering used by the prover. Then self-paramodulation (from $\true$ to $\true$) can be applied to clause~\eqref{clause:T|F} as follows:
\[
\infer{x \eql y \vee x \eql \false \vee y \eql \false}%
{\underline{x \eql \true} \vee x \eql \false & \underline{y \eql \true} \vee y \eql \false}
\]

The derived clause $x \eql y \vee x \eql false \vee y \eql \false$ is a recipe for disaster, since the literal $x \eql y$ must be selected and can be used for paramodulation into every non-variable term of a boolean sort. Very soon the search space will contain many clauses obtained as logical consequences of clause \eqref{clause:T|F} and results of paramodulation from variables applied to them. This will cause a rapid degradation of performance of superposition provers.

To get around this problem, we propose the following solution. First, we will choose term
orderings $\succ$ having the following properties: $\true\succ\false$ and $\true$ and
$\false$ are the smallest ground terms w.r.t.\ $\succ$. Consider now all ground instances of \eqref{clause:T|F}. They have the form $s \eql \true \vee s \eql \false$, where $s$ is a ground term. When $s$ is either $\true$ or $\false$, this instance is a tautology, and hence redundant. Therefore, we should only consider instances for which $s \succ \true$. This prevents self-paramodulation of \eqref{clause:T|F}.

Now the only possible inferences with \eqref{clause:T|F} are inferences of the form
\[
\infer[,]{C[\true] \vee s \eql \false}%
{\underline{x \eql \true} \vee x \eql \false & C[s]}
\]
where $s$ is a non-variable term of the sort $\bool$.
To implement this, we can remove clause \eqref{clause:T|F} and add as an extra inference rule to the superposition calculus the following rule:
\[
\infer[,]{C[\true] \vee s \eql \false}%
{C[s]}
\]
where $s$ is a non-variable term of the sort $\bool$ other than $\true$ and $\false$.


%------------------------------------------------------------------------------
\section{TPTP Support for \folb{}}
\label{sec:tptp}

% !TEX root = ../main.tex
The typed monomorphic first-order formulas subset, called TFF0, of the TPTP language~\cite{TPTP}, is a representation language for many-sorted first-order logic. It contains \ITE\ and \LETIN\ constructs (see below), which is useful for applications, but is inconsistent in its treatment of the boolean sort. It has a predefined atomic sort symbol \lstinline'$o' denoting the boolean sort. However, unlike all other sort symbols, \lstinline'$o' can only be used to declare the return type of predicate symbols. This means that one cannot define a function having a boolean argument, use boolean variables or equality between booleans. 

Such an inconsistent use of the boolean sort results in having two kinds of \ITE\ expressions and four kinds of \LETIN\ expressions. For example, a \folb-term $\letin{f(x_1:\sigma_1, \ldots, x_n:\sigma_n)}{s}{t}$ can be represented using one of the four TPTP alternatives \lstinline'$let_tt', \lstinline'$let_tf', \lstinline'$let_ft' or \lstinline'$let_ff', depending on whether $s$ and $t$ are terms or formulas. 

Since the boolean type is second-class in TPTP, one cannot directly represent formulas coming from program analysis and interactive theorem provers, such as formulas \eqref{formula:contains} and \eqref{formula:subset-sorted} of Section~\ref{sec:cicm15/introduction}.

We propose to modify the TFF0 language of TPTP to coincide with \folb. It is not late to do so, since there is no general support for \ITE\ and \LETIN. To the best of our knowledge, Vampire is currently the only theorem prover supporting full TFF0. Note that such a modification of TPTP would make multiple forms of \ITE\ and \LETIN\ redundant. It will also make it possible to directly represent the SMT-LIB core theory.

We note that our changes and modifications on TFF0 can also be applied to the TFF1 language of TPTP~\cite{tff1}. TFF1 is  a polymorphic extension of TFF0 and its formalisation  does not treat the boolean sort. Extending our work to TFF1 should not be hard but has to be done in detail.

%------------------------------------------------------------------------------
\section{Related Work}
\label{sec:cicm15/related}

% !TEX root = ../main.tex
Handling boolean terms as formulas is common in the SMT community. The SMT-LIB project~\cite{SMT-LIB} defines its core logic as first-order logic extended with the distinguished first-class boolean sort and the \verb'let'-\verb'in' expression used for local bindings of variables. The core theory of SMT-LIB defines logical connectives as boolean functions and the ad-hoc polymorphic \verb'if'-\verb'then'-\verb'else' ($ite$) function, used for conditional expressions. 
% SMT-solvers do not reason in the core logic, but use quantifier-free fragments of it with theories. 
The language \folb\ defined here extends the SMT-LIB core language with local function definitions,
using \verb'let'-\verb'in' expressions defining functions of arbitrary, and not just zero, arity. This, \folb\ contains both this language and the TFF0 subset of TPTP. Further, we present a translation of \folb\ to FOL and show how one can improve superposition theorem provers to reason with the boolean sort. 

% Unlike SMT-LIB, \folb{} defines logical connectives as interpreted functions and not as part of a theory, and the \verb'if'-\verb'then'-\verb'else' construct as part of the logic language.

Efficient superposition theorem proving in finite domains, such as the boolean domain, is also discussed in~\cite{HillenbrandWeidenbach13}. The approach of~\cite{HillenbrandWeidenbach13} sometimes falls back to enumerating instances of a clause by instantiating finite domain variables with all elements of the corresponding domains. We point out here that for the boolean (i.e., two-element) domain there is a simpler solution. However, the approach of~\cite{HillenbrandWeidenbach13} also allows one to handle domains with more than two elements. One can also generalise our approach to arbitrary finite domains by using binary encodings of finite domains, however, this will necessarily result in loss of efficiency, since a single variable over a domain with $2^k$ elements will become $k$ variables in our approach, and similarly for function arguments.


%------------------------------------------------------------------------------
\section{Conclusion}
\label{sec:cicm15/conclusions}

We defined first-order logic with the first class boolean sort (\folb{}). It extends ordinary many-sorted first-order logic (FOL) with (i) the boolean sort such that terms of this sort are indistinguishable from formulas and (ii) \ITE\ and \LETIN\ expressions. The semantics of \LETIN\ expressions in \folb{} is essentially their semantics in functional programming languages, when they are not used for recursive definitions. In particular, non-recursive local functions can be defined and function symbols can be bound to a different sort in nested \verb'let'-\verb'in' expressions.

We argued that these extensions are useful in reasoning about problems coming from program analysis and interactive theorem proving. The extraction of properties from certain program definitions (especially in functional programming languages) into \folb{} formulas is more straightforward than into ordinary FOL formulas and potentially more efficient. In a similar way, a more straightforward translation of certain higher-order formulas into \folb{} can facilitate proof automation in interactive theorem provers.

\folb{} is a modification of FOL and reasoning in it reduces to reasoning in FOL. We gave a translation of \folb{} to FOL that can be used for proving theorems in \folb{} in a first-order theorem prover. We further discussed a modification of superposition calculus that can reason efficiently in presence of the boolean sort. Finally, we pointed out that the TPTP language can be changed to support \folb{}, which will also simplify some parts of the TPTP syntax.

Implementation of theorem proving support for \folb{}, including its super\-po\-sition-friendly translation to CNF, is an important task for future work. Further, we are also interested in extending \folb{} with theories, such as the theory of integer linear arithmetic and arrays.

%------------------------------------------------------------------------------
\section*{Acknowledgements}
\label{sec:cicm15/acknowledgements}

% !TEX root = ../main.tex
The first two authors were partially supported by the Wallenberg Academy Fellowship 2014, the Swedish VR grant D0497701, and the Austrian research project FWF S11409-N23. The third author was Partially supported by the EPSRC grant ``Reasoning in Verification and Security''.


\def\paperTwoContentsTitle{The Vampire and the FOOL}
\def\paperTwoChapterTitle{The Vampire and the FOOL}
\def\paperTwoAuthors{Evgenii~Kotelnikov, Laura~Kov\'{a}cs,\\Giles~Reger and Andrei~Voronkov}
\def\paperTwoAbstract{This paper presents new features recently implemented in the theorem prover Vampire, namely support for first-order logic with a first class boolean sort (\folb{}) and polymorphic arrays. In addition to having a first class boolean sort, \folb{} also contains \ITE\ and \LETIN\ expressions. We argue that presented extensions facilitate reasoning-based program analysis, both by increasing the expressivity of first-order reasoners and by gains in efficiency.}
\def\paperTwoPublication{Published the \emph{Proceedings of the 5th ACM SIGPLAN Conference on Certified Programs and Proofs}, pages 37--48. ACM New York, 2016.}
\paperchapter{\paperTwoContentsTitle}
             {\paperTwoChapterTitle}
             {\paperTwoAuthors}
             {\paperTwoAbstract}
             {\paperTwoPublication}
\label{chap:implementation}
% !TEX root = ../main.tex
\section{Introduction}
\label{sect:introduction}

Automated program analysis and verification requires discovering and proving program properties. These program properties are checked using various tools, including theorem provers. The translation of program properties into formulas accepted by a theorem prover is not straightforward because of a mismatch between the semantics of the programming language constructs and that of the input language of the theorem prover. If program properties are not directly expressible in the input language, one should implement a translation of such program properties to the language. Such translations can be very complex and thus error prone.

The performance of a theorem prover on the result of a translation crucially depends on whether the translation introduces formulas potentially making the prover inefficient. Theorem provers, especially first-order ones, are known to be very fragile with respect to the input. Expressing program properties in the ``right'' format therefore requires solid knowledge about how theorem provers work and are implemented~--- something that a user of a verification tool might not have. Moreover, it can be hard to efficiently reason about certain classes of program properties, unless special inference rules and heuristics are added to the theorem prover. For example, \cite{ATVA14} shows a considerable gain in performance on proving properties of data collections by using a specially designed extensionality resolution rule.

If a theorem prover natively supports expressions that mirror the semantics of programming language constructs, we solve both above mentioned problems. First, the users do not have to design translations of such constructs. Second, the users do not have to possess a deep knowledge of how the theorem prover works~--- the efficiency becomes the responsibility of the prover itself.

In this work we present new features recently developed and implemented in the theorem prover Vampire~\cite{Vampire13} to natively support mirroring programming language constructs in its input language. They include (i) FOOL~\cite{FOOL}, that is the extension of first-order logic by a first-class boolean sort, \ITE\ and \LETIN\ expressions, and (ii)  polymorphic arrays.

This paper is structured as follows. Section~\ref{sect:fool} presents how FOOL is implemented in Vampire and focuses on new extensions to the TPTP input language~\cite{TPTP} of first-order provers. Section~\ref{sect:fool}  extends the TPTP language of monomorphic many-sorted first-order formulas, called TFF0~\cite{tff0}, and allows users to treat the built-in boolean sort \tptpo\ as a first class sort. Moreover, it introduces expressions \dite\ and \dlet, which unify various TPTP \ITE\ and \LETIN\ expressions.

Section~\ref{sect:arrays} presents a formalisation of a polymorphic theory of arrays in TPTP and its implementation in Vampire. It extends TPTP with features of the TFF1 language~\cite{tff1} of rank-1 polymorphic  first-order formulas, namely, sort arguments for the built-in array sort constructor \darraySymb. Sort variables however are not supported.

We argue that these extensions make the translation of properties of some programs to TPTP easier. To support this claim, in Section~\ref{sect:example} we discuss representation of various programming and other constructs in the extended TPTP language. We also give a linear translation of  the next state relation for any program with assignments, \ITE, and sequential composition.

Experiments with theorem proving with FOOL formulas are described in Section~\ref{sect:experiments}. In particular, we show that the implementation of a new inference rule, called FOOL paramodulation, improves performance of theorem provers using superposition calculus.

Finally, Section~\ref{sect:related} discusses related work and Section~\ref{sect:future} outlines future work.

% \LK{added the text below, addressing the CFP}

\noindent\paragraph{Summary of the main results.}
\begin{itemize}
\item We describe an implementation of first-order logic with a first-class boolean sort. This bridges the gap between input languages for theorem provers and logics and tools used in program analysis. We believe it is a first ever implementation of first-class boolean sorts in superposition theorem provers.

\item We extend and simplify the TPTP language~\cite{TPTP}, by providing more powerful and more uniform representations of \ITE\ and \LETIN\ expressions. To the best of our knowledge, Vampire is the only superposition theorem prover implementing these constructs.

\item We formalise and describe an implementation in Vampire of a polymorphic theory of arrays. Again, we believe that Vampire is the only superposition theorem prover implementing this theory.

\item We give a simple extension of FOOL, allowing to express the next state relation of a program as a boolean formula which is linear in the size of the program. This  boolean formula captures the exact semantics of the program and can be used by a first-order theorem prover. We are not aware of any other work on extending theorem provers with support for representing fragments of imperative programs.

\item We demonstrate usability and high performance of our implementation on two collections of examples, coming from the higher-order part of the TPTP library and from the Isabelle interactive theorem prover~\cite{Isabelle}. Our experimental results show that Vampire outperforms systems which could previously be used to solve such problems:
higher-order theorem provers and satisfiability modulo theory (SMT) solvers.
\end{itemize}

The paper focuses on new, practical features extending first-order theorem provers for making them better suited for applications of reasoning in various theories, program analysis and verification. While the paper describes implementation details and challenges in the Vampire theorem prover, the described features and their implementation can be carried out in any other first-order prover.

Summarising, we believe that our paper advances the state-of-the-art in formal certification of programs and proofs. With the use of FOOL and polymorphic arrays, we bring first-order theorem proving closer to program logics and make first-order theorem proving better suited for program analysis and verification. We also believe that an implementation of FOOL advances automation of mathematics, making many problems using the boolean type directly understood by a first-order theorem prover, while they previously were treated as higher-order problems.


%------------------------------------------------------------------------------
\section{First Class Boolean Sort}
\label{sect:fool}

Our recent work~\cite{FOOL} presented a modification of many-sorted first-order logic that contains a boolean sort with a fixed interpretation and treats terms of the boolean sort as formulas. We called this logic FOOL, standing for first-order logic (FOL) + boolean sort. FOOL extends FOL by (i) treating boolean terms as formulas; (ii) \ITE\ expressions; and (iii) \LETIN\ expressions. There is a model-preserving transformation of FOOL formulas to FOL formulas, hence an implementation of this transformation makes it possible to prove FOOL formulas using a first-order theorem prover.

The language of FOOL is, essentially, a superset of the core language of SMT-LIB~2~\cite{SMT-LIB}, the library of problems for SMT solvers. The difference between FOOL and the core language is that the former has richer \LETIN\ expressions, which support local definitions of functions symbols of arbitrary arity, while the latter only supports local binding of variables.

FOOL can be regarded as the smallest superset of the SMT-LIB~2 Core language and TFF0. An implementation of a translation of FOOL to FOL thus also makes it possible to translate SMT-LIB problems to TPTP. Consider, for example, the following tautology, written in the SMT-LIB syntax: \verb'(exists ((x Bool)) x)'. It quantifies over boolean variables and uses a boolean variable as a formula. Neither is allowed in the standard TPTP language, but can be directly expressed in an extended TPTP that represents FOOL.

The rest of this section presents features of FOOL not included in FOL, explains how they are implemented in Vampire and how they can be represented in an extended TPTP syntax understood by Vampire.

\subsection{Proving with the Boolean Sort}

Vampire supports many-sorted predicate logic and the TFF0 syntax for this logic. In many-sorted predicate logic all sorts are uninterpreted, while the boolean sort should be interpreted as a two-element set. There are several ways to support the boolean sort in a first-order theorem prover, for example, one can axiomatise it by adding two constants $\true$ and $\false$ of this sort and two axioms: $(\forall x:\bool)(x \eql \true \lor x \eql \false)$ and $\true \neql \false$. However, as we discuss in \cite{FOOL}, using this axiomatisation in a superposition theorem prover may result in performance problems caused by self-paramodulation of $x \eql \true \lor x \eql \false$.

To overcome this problem, in \cite{FOOL} we proposed the following modification of the superposition calculus.
\begin{enumerate}
  \item Use a special simplification ordering that makes the constants $\true$ and $\false$ smallest terms of the sort $\bool$ and also makes $\true$ greater than $\false$.

\item Add the axiom $\true \neql \false$.

\item Add a special inference rule, called \emph{FOOL paramodulation}, of the form
  \[
    \infer[,]{C[\mathtt{\true}] \lor s \eql \mathtt{\false}}{C[s]}
  \]
where
\begin{enumerate}
\item $s$ is a term of the sort $\bool$ other than $\true$ and $\false$;
\item $s$ is not a variable;
\end{enumerate}
\end{enumerate}

Both ways of dealing with the boolean sort are supported in Vampire. The option \verb|--fool_paramodulation|, which can be set to \verb|on| or \verb|off|, chooses one of them. The default value is \verb|on|, which enables the modification.

Vampire uses the TFF0 subset of the TPTP syntax, which does not fully support FOOL. To write FOOL formulas in the input, one uses the standard TPTP notation: \tptpo\ for the boolean sort, \dtrue\ for $\true$ and \dfalse\ for $\false$. There are, however, two ways to output the boolean sort and the constants. One way will use the same notation as in the input and is the default, which is sufficient for most applications. The other way can be activated by the option \verb'--show_fool on', it will
\begin{enumerate}
  \item denote as \dbool\ every occurrence of $\bool$ as a sort of a variable or an argument (to a function or a predicate symbol);
  \item denote as \ddtrue\ every occurrence of $\true$ as an argument; and
  \item denote as \ddfalse\ every occurrence of $\false$ as an argument.
\end{enumerate}
Note that an occurrence of any of the symbols \dbool, \ddtrue\ or \ddfalse\ anywhere in an input problem is not recognised as syntactically correct by Vampire.

Setting \verb'--show_fool' to \verb'on' might be necessary if Vampire is used as a front-end to other reasoning tools. For example, one can use Vampire not only for proving, but also for preprocessing the input problem or converting it to clausal normal form. To do so, one uses the options \verb|--mode preprocess| and \verb|--mode clausify|, respectively. The output of Vampire can then be passed to other theorem provers, that either only deal with clauses or do not have sophisticated preprocessing. Setting \verb'--show_fool' to \verb'on' appends a definition of a sort denoted by \dbool\ and constants denoted by \ddtrue\ and \ddfalse\ of this sort to the output. That way the output will always contain syntactically correct TFF0 formulas, which might not be true if the option is set to \verb'off' (the default value).

Every formula of the standard FOL is syntactically a FOOL formula and has the same models. Vampire does not reason in FOOL natively, but rather translates the input FOOL formulas into FOL formulas in a way that preserves models. This is done at the first stage of preprocessing of the input problem.

Vampire implements the translation of FOOL formulas to FOL given in~\cite{FOOL}. It involves replacing parts of the problem that are not syntactically correct in the standard FOL by applications of fresh function and predicate symbols. The set of assumptions is then extended by formulas that define these symbols. Individual steps of the translation are displayed when the \verb'--show_preprocessing' option is set to \verb'on'.

In the next subsections we present the features of FOOL that are not present in FOL together with their syntax in the extended TFF0 and their implementation in Vampire.

\subsection{Quantifiers over the Boolean Sort}

FOOL allows quantification over $\bool$ and usage of boolean variables as formulas. For example, the formula $(\forall x:\bool)(x \lor \neg x)$ is a syntactically correct tautology in FOOL. It is not however syntactically correct in the standard FOL where variables can only occur as arguments.

Vampire translates boolean variables to FOL in the following way. First, every formula of the form $x \liff y$, where $x$ and $y$ are boolean variables, is replaced by $x \eql y$. Then, every occurrence of a boolean variable $x$ anywhere other than in an argument is replaced by $x \eql \true$. For example, the tautology $(\forall x:\bool)(x \lor \neg x)$ will be converted to the FOL formula $(\forall x:\bool)(x \eql \true \lor x \neql \true)$ during preprocessing.

Note that it is possible to directly express quantified boolean formulas (QBF) in FOOL, and use Vampire to reason about them.

TFF0 does not support quantification over booleans. Vampire supports an extended version of TFF0 where the sort symbol \tptpo\ is allowed to occur as the sort of a quantifier and boolean variables are allowed to occur as formulas. The formula $(\forall x:\bool)(x \lor \neg x)$ can be expressed in this syntax as \lstinline'![X:$o]: (X | ~X)'. %$

\subsection[Functions and Predicates with Boolean Arguments]{Functions and Predicates with\\Boolean Arguments}

Functions and predicates in FOOL are allowed to take booleans as arguments. For example, one can define the logical implication as a binary function $\mathit{impl}$ of the type $\bool \times \bool \to \bool$ using the following axiom:
\[
  (\forall x: \bool)(\forall y: \bool)(\mathit{impl}(x, y) \liff \neg x \lor y).
\]

Since Vampire supports many-sorted logic, this feature requires no additional implementation, apart from changes in the parser.

In TFF0, functions and predicates cannot have arguments of the sort \tptpo. In the version of TFF0, supported by Vampire, this restriction is removed. Thus, the definition of $\mathit{impl}$ can be expressed in the following way.
\begin{lstlisting}
tff(impl, type, impl: ($o * $o) > $o).
tff(impl_definition, axiom,
    ![X:$o, Y:$o]: (impl(X, Y) <=> (~X | Y))).
\end{lstlisting}%$

\subsection{Formulas as Arguments}

Unlike the standard FOL, FOOL does not make a distinction between formulas and boolean terms. It means that a function or a predicate can take a formula as a boolean argument, and formulas can be used as arguments to equality between booleans. For example, with the definition of $\mathit{impl}$, given earlier, we can express in FOOL that
$P$ is a graph of a (partial) function of the type $\sigma \to \tau$ as follows:
\begin{equation}\label{eq:bool-arg-example}
  (\forall x:\sigma)(\forall y:\tau)(\forall z:\tau)\mathit{impl}(P(x,y) \land P(x,z), y \eql z).
\end{equation}

Note that the definition of $\mathit{impl}$ could as well use equality instead of equivalence.

In order to support formulas occurring as arguments, Vampire does the following. First, every expression of the form $\varphi \eql \psi$ is replaced by $\varphi \liff \psi$. Then, for each formula $\psi$ occurring as an argument the following translation is applied. If $\psi$ is a nullary predicate $\top$ or $\bot$, it is replaced by $\true$ or $\false$, respectively. If $\psi$ is a boolean variable, it is left as is. Otherwise, the translation is done in several steps. Let $x_1,\ldots,x_n$ be all free variables of $\psi$ and $\sigma_1,\ldots,\sigma_n$ be their sorts. Then Vampire
\begin{enumerate}
  \item introduces a fresh function symbol $g$ of the type $$\sigma_1 \times \ldots \times \sigma_n \to \bool;$$
  \item adds the definition $$(\forall x_1:\sigma_1)\ldots(\forall x_n:\sigma_n)(\psi \liff g(x_1,\ldots,x_n) \eql \true)$$ to its set of assumptions;
  \item replaces $\psi$ by $g(x_1,\ldots,x_n)$.
\end{enumerate}

For example, after this translation has been applied for both arguments of $\mathit{impl}$, \eqref{eq:bool-arg-example} becomes $$(\forall x:\sigma)(\forall y:\sigma)(\forall z:\sigma)\mathit{impl}(g_1(x, y, z), g_2(y, z)),$$ where $g_1$ and $g_2$ are fresh function symbol of the types $\sigma \times \tau \times \tau \to \bool$ and $\tau \times \tau \to \bool$, respectively, defined by the following formulas:
\begin{enumerate}
  \item $(\forall x:\sigma)(\forall y:\tau)(\forall z:\tau)(P(x,y) \land P(x,z) \liff g_1(x,y,z) \eql \true)$;
  \item $(\forall y:\tau)(\forall z:\tau)(y \eql z \liff g_2(y,z) \eql \true)$.
\end{enumerate}

TFF0 does not allow formulas to occur as arguments. The extended version of TFF0, supported by Vampire, removes this restriction for arguments of the boolean sort. Formula~\eqref{eq:bool-arg-example} can be expressed in this syntax as follows:
\begin{lstlisting}
![X:s, Y:t, Z:t]: impl(p(X, Y) & p(X, Z), Y = Z)
\end{lstlisting}

For a more interesting example, consider the following logical puzzle taken from the TPTP problem \mbox{PUZ081}:
\begin{quote}
  A very special island is inhabited only by knights and knaves. Knights always tell the truth, and knaves always lie. You meet two inhabitants: Zoey and Mel. Zoey tells you that Mel is a knave. Mel says, `Neither Zoey nor I are knaves'. Who is a knight and who is a knave?
\end{quote}

\newcommand{\knight}{\mathit{Knight}}
\newcommand{\knave}{\mathit{Knave}}
\newcommand{\says}{\mathit{Says}}
\newcommand{\statement}{\mathit{statement}}
\newcommand{\person}{\mathit{person}}
\newcommand{\zoye}{\mathit{zoye}}
\newcommand{\mel}{\mathit{mel}}
To solve the puzzle, one can formalise it as a problem in FOOL and give a corresponding extended TFF0 representation to Vampire. Let $\zoye$ and $\mel$ be terms of a fixed sort $\person$ that represent Zoye and Mel, respectively. Let $\says$ be a predicate that takes a term of the sort $\person$ and a boolean term. We will write $\says(p, s)$ to denote that a person $p$ made a logical statement $s$. Let $\knight$ and $\knave$ be predicates that take a term of the sort $\person$. We will write $\knight(p)$ or $\knave(p)$ to denote that a person $p$ is a knight or a knave, respectively. We will express the fact that knights only tell the truth and knaves only lie by axioms $(\forall p:\person)(\forall s:\bool)(\knight(p) \land \says(p, s) \implies s)$ and $(\forall p:\person)(\forall s:\bool)(\knave(p) \land \says(p, s) \implies \neg s)$, respectively. We will express the fact that every person is either a knight or a knave by the axiom $(\forall p:\person)(\knight(p) \oplus \knave(p))$, where $\oplus$ is the ``exclusive or'' connective. Finally, we will express the statements that Zoye and Mel make in the puzzle by axioms $\says(\zoye, \knave(\mel))$ and $\says(\mel, \neg\knave(\zoye) \land \neg\knave(\mel))$, respectively.

The axioms and definitions, given above, can be written in the extended TFF0 syntax in the following way.
\begin{lstlisting}
tff(person, type, person: $tType).
tff(says, type, says: (person * $o) > $o).

tff(knight, type, knight: person > $o).
tff(knights_always_tell_truth, axiom,
    ![P:person, S:$o]:
      (knight(P) & says(P, S) => S)).

tff(knave, type, knave: person > $o).
tff(knaves_always_lie, axiom,
    ![P:person, S:$o]:
      (knave(P) & says(P, S) => ~S)).

tff(very_special_island, axiom,
    ![P:person]: (knight(P) <~> knave(P))).

tff(zoey, type, zoey: person).
tff(mel,  type, mel:  person).

tff(zoye_says, hypothesis,
    says(zoey, knave(mel))).

tff(mel_says, hypothesis,
    says(mel, ~knave(zoey) & ~knave(mel))).
\end{lstlisting}%$

Vampire accepts this code, finds that the problem is satisfiable and outputs the saturated set of clauses. There one can see that Zoey is a knight and Mel is a knave. Note that the existing formalisations of this puzzle in TPTP (files \verb'PUZ081^1.p', \verb'PUZ081^2.p' and \verb'PUZ081^3.p') employ the language of higher-order logic (THF)~\cite{THF}. However, as we have just shown, one does not need to resort to reasoning in higher-order logic for this problem, and can enjoy the efficiency of reasoning in first-order logic.

%Running Vampire with the options \verb'--mode preprocess' and \verb'--show_fool on' produces the following output.
%\begin{lstlisting}
%tff(type_def_0, type, $bool: $tType).
%tff(func_def_0, type, $$false: $bool).
%tff(func_def_1, type, $$true: $bool).
%tff(func_def_4, type, bG0: $bool).
%tff(func_def_5, type, bG1: $bool).
%tff(pred_def_1, type, says: ($i * $bool) > $o).
%
%tff(u4, axiom,
%    $$false != $$true).
%
%tff(u5, axiom,
%    ![X0: $bool]: (($$true = X0) | ($$false = X0))).
%
%tff(u6,axiom,
%    ![X0]: ((says(X0, $$true) <~> says(X0, $$false)))).
%
%tff(u7, axiom,
%    says(mel, $$false) <=> ($$true = bG0)).
%
%tff(u8, hypothesis,
%    says(zoey, bG0)).
%
%tff(u9, axiom,
%    (~(says(zoey, $$false) | says(mel, $$false))) <=> ($$true = bG1)).
%
%tff(u10, hypothesis,
%    says(mel, bG1)).
%\end{lstlisting}
%
%Note that definitions \lstinline'type_def_0', \lstinline'func_def_0' and \lstinline'func_def_1', and units \lstinline'u4' and \lstinline'u5' axiomatise the theory of booleans. Units \verb'func_def_4' and \verb'func_def_5' introduce fresh function symbols, and units \verb'u7' and \verb'u9' are their definitions.

This example makes one think about representing sentences in various epistemic or first-order modal logics in FOOL.

\subsection{\ITE}

FOOL contains expressions of the form $\ite{\psi}{s}{t}$, where $\psi$ is a boolean term, and $s$ and $t$ are terms of the same sort. The semantics of such expressions mirrors the semantics of conditional expressions in programming languages.

\ITE\ expressions are convenient for expressing formulas coming from program analysis and interactive theorem provers. For example, consider the $\mathit{max}$ function of the type $\Z \times \Z \to \Z$ that returns the maximum of its arguments. Its definition can be expressed in FOOL as
\begin{equation}\label{eq:ite-t-example}
  (\forall x:\Z)(\forall y:\Z)(\mathit{max}(x, y) \eql \ite{x \geq y}{x}{y}).
\end{equation}

To handle such expressions, Vampire translates them to FOL. This translation is done in several steps. Let $x_1,\ldots,x_n$ be all free variables of $\psi$, $s$ and $t$, and $\sigma_1,\ldots,\sigma_n$ be their sorts. Let $\tau$ be the sort of both $s$ and $t$. The steps of translation depend on whether $\tau$ is $\bool$ or a different sort. If $\tau$ is not $\bool$, Vampire
\begin{enumerate}
  \item introduces a fresh function symbol $g$ of the type $$\sigma_1 \times \ldots \times \sigma_n \to \tau;$$
  \item adds the definitions
\begin{equation*}
\begin{aligned}
(\forall x_1:\sigma_1)\ldots(\forall x_n:\sigma_n) (\psi &\implies g(x_1,\ldots,x_n) \eql s),\\
(\forall x_1:\sigma_1)\ldots(\forall x_n:\sigma_n) (\neg\psi &\implies g(x_1,\ldots,x_n) \eql t)
\end{aligned}
\end{equation*} to its set of assumptions;
  \item replaces $\ite{\psi}{s}{t}$ by $g(x_1,\ldots,x_n)$.
\end{enumerate}

If $\tau$ is $\bool$, the following is different in the steps of translation:
\begin{enumerate}
  \item a fresh predicate symbol $g$ of the type $\sigma_1 \times \ldots \times \sigma_n$ is introduced instead; and
  \item the added definitions use equivalence instead of equality.
\end{enumerate}
\noindent
For example, after this translation \eqref{eq:ite-t-example} becomes $$(\forall x:\Z)(\forall y:\Z)(\mathit{max}(x, y) \eql g(x, y)),$$ where $g$ is a fresh function symbol of the type $\Z \times \Z \to \Z$ defined by the following formulas:
\begin{enumerate}
  \item $(\forall x:\Z)(\forall y:\Z)(x \geq y \implies g(x, y) \eql x)$;
  \item $(\forall x:\Z)(\forall y:\Z)(x \not\geq y \implies g(x, y) \eql y).$
\end{enumerate}

TPTP has two different expressions for \ITE: \ditet\ for constructing terms and \ditef\ for constructing formulas. \ditet\ takes a formula and two terms of the same sort as arguments. \ditef\ takes three formulas as arguments.

Since FOOL does not distinguish formulas and boolean terms, it does not require separate expressions for the formula-level and term-level \ITE. The extended version of TFF0, supported by Vampire, uses a new expression \dite, that unifies \ditet\ and \ditef. \dite\ takes a formula and two terms of the same sort as arguments. If the second and the third arguments are boolean, such \dite\  expression is equivalent to \ditef, otherwise it is equivalent to \ditet.

Consider, for example, the above definition of $\mathit{max}$. It can be encoded in the extended TFF0 as follows.
\begin{lstlisting}
tff(max, type, max: ($int * $int) > $int).
tff(max_definition, axiom,
    ![X:$int, Y:$int]:
      (max(X, Y) = $ite($greatereq(X, Y), X, Y))).
\end{lstlisting}%$
It uses the TPTP notation \dint\ for the sort of integers and \dgreatereq\ for the greater-than-or-equal-to comparison of two numbers.

% This is working example for TPTP, should we use it instead?
%\begin{lstlisting}
%![X:$int, Y:$int]: (max(X, Y) = $ite($greatereq(X, Y), X, Y))
%\end{lstlisting}

Consider now the following valid property of $\mathit{max}$:
\begin{equation}\label{eq:ite-f-example}
  (\forall x:\Z)(\forall y:\Z)(\ite{\mathit{max}(x, y) \eql x}{x \geq y}{y \geq x}).
\end{equation}

Its encoding in the extended TFF0 can use the same \dite\ expression:
\begin{lstlisting}
![X:$int, Y:$int]:
  $ite(max(X, Y) = X, $greatereq(X, Y), $greatereq(Y, X)).
\end{lstlisting}%$

Note that TFF0 without \dite\ has to differentiate between terms and formulas, and so requires to use \ditet\ in~\eqref{eq:ite-t-example} and \ditef\ in~\eqref{eq:ite-f-example}.

\subsection{\LETIN}

FOOL contains \LETIN\ expressions that can be used to introduce local function definitions. They have the form
\begin{equation}\label{eq:let}
\begin{aligned}
\mathtt{let}\;&\binding{f_1(x^1_1:\sigma^1_1,\ldots,x^1_{n_1}:\sigma^1_{n_1})}{s_1};\\
              &\ldots\\
              &\binding{f_m(x^m_1:\sigma^m_1,\ldots,x^m_{n_m}:\sigma^m_{n_m})}{s_m}\\
 \mathtt{in}\;&t,
\end{aligned}
\end{equation}
where
\begin{enumerate}
  \item $m \geq 1$;
  \item $f_1,\ldots,f_m$ are pairwise distinct function symbols;
  \item $n_i \geq 0$ for each $1 \leq i \leq m$;
  \item $x^i_1\ldots,x^i_{n_i}$ are pairwise distinct variables for each $1 \leq i \leq m$; and
  \item $s_1,\ldots,s_m$ and $t$ are terms.
\end{enumerate}

%We will write a \LETIN\ expression simply as $\letin{f(x_1:\sigma_1,\ldots,x_n:\sigma_n)}{s}{t}$ if $m = 1$.

The semantics of \LETIN\ expressions in FOOL mirrors the semantics of simultaneous non-recursive local definitions in programming languages. That is, $s_1,\ldots,s_m$ do not use the bindings of $f_1,\ldots,f_m$ created by this definition.

Note that an expression of the form \eqref{eq:let} is not in general equivalent to $m$ nested \LETIN s
\begin{equation}\label{eq:let-singles}
\begin{aligned}
&\mathtt{let}\;\binding{f_1(x^1_1:\sigma^1_1,\ldots,x^1_{n_1}:\sigma^1_{n_1})}{s_1}\;\mathtt{in}\\
&\quad\;\ddots\\
&\quad\quad\;\mathtt{let}\;\binding{f_m(x^m_1:\sigma^m_1,\ldots,x^m_{n_m}:\sigma^m_{n_m})}{s_m}\;\mathtt{in}\\
&\quad\quad\quad t.
\end{aligned}
\end{equation}
The main application of \LETIN\ expressions is in problems coming from program analysis, namely modelling of assignments. Consider for example the following code snippet featuring operations over an integer \verb'array'.
\begin{verbatim}
array[3] := 5;
array[2] + array[3];
\end{verbatim}
It can be translated to FOOL in the following way. We represent the integer array as an uninterpreted function $\arrayt$ of the type $\Z \to \Z$ that maps an index to the array element at that index. The assignment of an array element can be translated to a combination of \LETIN\ and \ITE.
\begin{equation}\label{eq:let-function-example}
\begin{aligned}
  &\mathtt{let}\;\binding{\arrayt(i:\Z)}{\ite{i \eql 3}{5}{\arrayt(i)}}\;\mathtt{in}\\
  &\quad\arrayt(2) + \arrayt(3)
\end{aligned}
\end{equation}

%Let $\context$ be a type assignment. Let $f_1,\ldots,f_m$ be pairwise distinct function symbols, and %$x_{i,1}\ldots,x_{i,n}$ be pairwise distinct variables for each $1 \leq i \leq n$. If %$$\context,x_{i,1}:\sigma_{i,1}\ldots,x_{i,n}:\sigma_{i,n} \vdash s_i:\sigma_i$$ for each $1 \leq i \leq n$, and
%\begin{align*}
%\context,\;&f_1:(\sigma_{1,1} \times \ldots \times \sigma_{n,1} \to \sigma_1),\\
%           &\ldots\\
%           &f_m:(\sigma_{m,1} \times \ldots \times \sigma_{m,n} \to \sigma_m)\vdash t:\tau,
%\end{align*}
%then
%\begin{align*}
%\context \vdash (\mathtt{let}\;&\binding{f_1(x_{1,1}:\sigma_{1,1},\ldots,x_{1,n}:\sigma_{1,n})}{s_1};\\
%                               &\ldots\\
%                               &\binding{f_m(x_{m,1}:\sigma_{m,1},\ldots,x_{m,n}:\sigma_{m,n})}{s_m}\\
%                  \mathtt{in}\;&t):\tau.
%\end{align*}
%In other words, it is a term of the sort $\tau$. Note that it is equivalent to a \LETIN\ expression with a single function binding when $m = 1$.

Multiple bindings in a \LETIN\ expression can be used to concisely express simultaneous assignments that otherwise would require renaming. In the following example, constants $a$ and $b$ are swapped by a \LETIN\ expression. The resulting formula is equivalent to $f(b, a)$.
\begin{equation}\label{eq:parallel-let-example}
\letinpar{a}{b}{b}{a}{f(a, b)}
%\begin{aligned}
%  &\mathtt{let}\;\binding{a}{1};\;\binding{b}{2}\;\mathtt{in}\\
%  &\quad\mathtt{let}\;\binding{a}{b};\;\binding{b}{a}\;\mathtt{in}\\
%  &\quad\quad f(a, b)
%\end{aligned}
\end{equation}

In order to handle \LETIN\ expressions Vampire translates them to FOL. This is done in three stages for each expression in \eqref{eq:let}.
\begin{enumerate}
  \item For each function symbol $f_i$ where $0 \leq i < m$ that occurs freely in any of $s_{i+1},\ldots,s_m$, introduce a fresh function symbol $g_i$. Replace all free occurrences of $f_i$ in $t$ by $g_i$.
  \item Replace the \LETIN\ expression by an equivalent one of the form \eqref{eq:let-singles}. This is possible because the necessary condition was satisfied by the previous step.
  \item Apply a translation to each of the \LETIN\ expression with a single binding, starting with the innermost one.
\end{enumerate}

The translation of an expression of the form $$\letin{f(x_1:\sigma_1,\ldots,x_n:\sigma_n)}{s}{t}$$ is done by the following sequence of steps. Let $y_1,\ldots,y_m$ be all free variables of $s$ and $t$, and $\tau_1,\ldots,\tau_m$ be their sorts. Note that the variables in $x_1,\ldots,x_n$ are not necessarily disjoint from the variables in $y_1,\ldots,y_m$. Let $\sigma_0$ be the sort of $s$. The steps of translation depend on whether $\sigma_0$ is $\bool$ and not. If $\sigma_0$ is not $\bool$, Vampire
\begin{enumerate}
  \item introduces a fresh function symbol $g$ of the type $$\sigma_1 \times \ldots \times \sigma_n \times \tau_1 \times \ldots \times \tau_m \to \sigma_0;$$
  \item adds to the set of assumptions the definition
  \begin{align*}
    &(\forall z_1:\sigma_1)\ldots(\forall z_n:\sigma_n) (\forall y_1:\tau_1)\ldots(\forall y_m:\tau_m)\\
    &\quad(g(z_1,\ldots,z_n,y_1,\ldots,y_m) \eql s'),
  \end{align*} where $z_1,\ldots,z_n$ is a fresh sequence of variables and $s'$ is  obtained from $s$ by replacing all free occurrences of $x_1,\ldots,x_n$ by $z_1,\ldots,z_n$, respectively; and
  \item replaces $\letin{f(x_1:\sigma_1,\ldots,x_n:\sigma_n)}{s}{t}$ by $t'$, where $t'$ is obtained from $t$ by replacing all bound occurrences of $y_1,\ldots,y_m$ by fresh variables and each application $f(t_1, \ldots, t_n)$ of a free occurrence of $f$ by $g(t_1, \ldots, t_n,\allowbreak y_1, \ldots, y_m)$.
\end{enumerate}

If $\sigma_0$ is $\bool$, the steps of translation are different:
\begin{enumerate}
  \item a fresh predicate symbol of the type \[\sigma_1 \times \ldots \times \sigma_n \times \tau_1 \times \ldots \times \tau_m\] is introduced instead;
  \item the added definition uses equivalence instead of equality.
\end{enumerate}

For example, after this translation \eqref{eq:let-function-example} becomes $g(2) + g(3)$, where $g$ is a fresh function symbol of the type $\Z \to \Z$ defined by the following formula: $$(\forall i:\Z)(g(i) \eql \ite{i \eql 3}{5}{\arrayt(i)}).$$

The example~\eqref{eq:parallel-let-example} is translated in the following way. First, the \LETIN\ expression is translated to the form~\eqref{eq:let-singles}. The constant $a$ has a free occurrence in the body of $b$, therefore it is replaced by a fresh constant $a'$. The formula \eqref{eq:parallel-let-example} becomes
\begin{equation*}
\begin{aligned}
  &\mathtt{let}\;\binding{a'}{b}\;\mathtt{in}\\
  &\quad\mathtt{let}\;\binding{b}{a}\;\mathtt{in}\\
  &\quad\quad\ f(a', b).
\end{aligned}
\end{equation*}
Then, the translation is applied to both \LETIN\ expressions with a single binding and the resulting formula becomes $f(a'', b')$, where $a''$ and $b'$ are fresh constants, defined by formulas $a'' \eql b$ and $b' \eql a$.

TPTP has four different expressions for \LETIN: \dlettt\ and \dletft\ for constructing terms, and \dlettf\  and \dletff\  for constructing formulas. All of them denote a single binding. \dlettt\ and \dlettf\ denote a binding of a function symbol, whereas \dletft\ and \dletff\ denote a binding of a predicate symbol. All four expressions take a (possibly universally quantified) equation as the first argument and a term (in case of \dlettt\ and \dletft) or a formula (in case of \dlettf\ and \dletff) as the second argument. TPTP does not provide any notation for \LETIN\  expressions with multiple bindings.

Similarly to \ITE, \LETIN\ expressions in FOOL do not need different notation for terms and formulas. The modification of TFF0 supported by Vampire introduces a new \dlet\ expression, that unifies \dlettt, \dletft, \dlettf\ and \dletff, and extends them to support multiple bindings. Depending on whether the binding is of a function or predicate symbol and whether the second argument of the expression is term or formula, a \dlet\ expression is equivalent to one of \dlettt, \dletft, \dlettf\ and \dletff.

The new \dlet\ expressions use different syntax for bindings. Instead of a quantified equation, they use the following syntax: a function symbol possibly followed by a list of variable arguments in parenthesis, followed by the \lstinline':=' operator and the body of the binding. Similarly to quantified variables, variable arguments are separated with commas and each variable might include a sort declaration. A sort declaration can be omitted, in which case the variable is assumed to the be of the sort of individuals (\verb|$i|).

Formula \eqref{eq:let-function-example} can be written in the extended TFF0 with the TPTP interpreted function \dsum, representing integer addition, as follows:
\begin{lstlisting}
$let(array(I:$int) := $ite(I = 3, 5, array(I)),
     $sum(array(2), array(3))).
\end{lstlisting}

%A \LETIN\ defintion of a constant \eqref{eq:let-constant-example} can be expressed as follows.
%\begin{lstlisting}
%$let(d := f(f(c)), g(d, d))
%\end{lstlisting}

The same \dlet\ expression can be used for multiple bindings. For that, the bindings should be separated by a semicolon and passed as the first argument. The formula~\eqref{eq:parallel-let-example} can be written using \dlet\ as follows.
\begin{lstlisting}
$let(a := b; b := a, f(a, b)))
\end{lstlisting}

Overall, \dite\ and \dlet\ expressions provide a more concise syntax for TPTP formulas than the TFF0 variations of \ITE\ and \LETIN\  expressions. To illustrate this point, consider the following snippet of TPTP code, taken from the TPTP problem \mbox{SYN000\_2}.
\begin{lstlisting}
tff(let_binders, axiom, ![X:$i]:
    $let_ff(![Y1:$i, Y2:$i]: (q(Y1, Y2) <=> p(Y1)),
      q($let_tt(![Z1:$i]:
          (f(Z1) = g(Z1, b)), f(a)), X) &
      p($let_ft(![Y3:$i, Y4:$i]: (q(Y3, Y4) <=>
          $ite_f(Y3 = Y4, q(a, a), q(Y3, Y4))),
          $ite_t(q(b, b), f(a), f(X)))))).
\end{lstlisting}

It uses both of the TFF0 variations of \ITE\ and three different variations of \LETIN. The same snippet can be expressed more concisely using \dite\ and \dlet\ expressions.
\begin{lstlisting}
tff(let_binders, axiom, ![X:$i]:
    $let(q(Y1, Y2) := p(Y1),
      q($let(f(Z1) := g(Z1, b), f(a)), X) &
      p($let(q(Y3, Y4) := $ite(Y3 = Y4, q(a, a), q(Y3,Y4))),
          $ite(q(b, b), f(a), f(X)))))).
\end{lstlisting}


%------------------------------------------------------------------------------
\section{Polymorphic Theory of Arrays}
\label{sect:arrays}

Using built-in arrays and reasoning in the first-order theory of arrays are common in program analysis, for example for finding loop invariants in programs using arrays~\cite{fase2009}. Previous versions of Vampire supported theories of integer arrays and arrays of integer arrays~\cite{Vampire13}. No other array sorts were supported and in order to implement one it would be necessary to hardcode a new sort and add the theory axioms corresponding to that sort. In this section we describe a polymorphic theory of arrays implemented in Vampire.

\subsection{Definition}
The polymorphic theory of arrays is the union of theories of arrays parametrised by two sorts: sort $\tau$ of indexes and sort $\sigma$ of values. It would have been proper to call these theories the theories of maps from $\tau$ to $\sigma$, however we decided to call them arrays for the sake of compatibility with arrays as defined in SMT-LIB.

A theory of arrays is a first-order theory that contains a sort
$\arrayt(\tau,\sigma)$, function symbols $\selectf :
\arrayt(\tau,\sigma) \times \tau \to \sigma$ and $\storef :
\arrayt(\tau,\sigma) \times \tau \times \sigma \to
\arrayt(\tau,\sigma)$, and three axioms.
The function symbol $\selectf$ represents a binary operation of
extracting an array element by its index.
The function symbol $\storef$ represents a ternary operation of updating an array at a given index with a given value. The array axioms are:
\begin{enumerate}
%  \item array congruence $$(\forall a:\arrayt(\tau,\sigma))(\forall i:\tau)(\forall j:\tau)(i \eql j \implies \select{a}{i} \eql \select{a}{j});$$
  \item read-over-write 1
        \begin{align*}
          &(\forall a:\arrayt(\tau,\sigma))(\forall v:\sigma)(\forall i:\tau)(\forall j:\tau)\\
          &\quad(i \eql j \implies \select{\store{a}{i}{v}}{j} \eql v);
        \end{align*}
  \item read-over-write 2
        \begin{align*}
          &(\forall a:\arrayt(\tau,\sigma))(\forall v:\sigma)(\forall i:\tau)(\forall j:\tau)\\
          &\quad(i \neql j \implies \select{\store{a}{i}{v}}{j} \eql \select{a}{j});
        \end{align*}
%  \item extensionality $$(\forall a:\arrayt(\tau,\sigma))(\forall b:\arrayt(\tau,\sigma))((\forall i:\tau)(\select{a}{i} \eql \select{b}{i}) \liff a\eql b).$$
  \item extensionality
        \begin{align*}
          &(\forall a:\arrayt(\tau,\sigma))(\forall b:\arrayt(\tau,\sigma))\\
          &\quad((\forall i:\tau)(\select{a}{i} \eql \select{b}{i}) \implies a\eql b).
        \end{align*}
\end{enumerate}
We will call every concrete instance of the theory of arrays for
concrete sorts $\tau$ and $\sigma$ the \emph{$(\tau,\sigma)$-instance}.

%One can use the polymorphic theory of arrays to express program properties. For example, the following formula expresses the fact that an integer array $a$ of size $n$ is sorted.
%\begin{equation}\label{eq:arrays-example}
%  (\forall i:\tau)(i \geq 0 \land i < n \implies \select{a}{i} \leq \select{a}{i + 1})
%\end{equation}

One can use the polymorphic theory of arrays to express program properties. Recall the code snippet involving arrays mentioned in Section~\ref{sect:fool}:
\begin{lstlisting}[language=cpp]
array[3] := 5;
array[2] + array[3];
\end{lstlisting}
Formula~\eqref{eq:let-function-example} used an interpreted function to represent the array in this code. We can alternatively use arrays to represent it as follows
\begin{equation}\label{eq:arrays-example}
\begin{aligned}
&\mathtt{let}\;\binding{\arrayt}{\store{\arrayt}{3}{5}}\;\mathtt{in}\\
&\quad\select{\arrayt}{2} + \select{\arrayt}{3}
\end{aligned}
\end{equation}

\subsection{Implementation in Vampire}

Vampire implements reasoning in the polymorphic theory of arrays by adding corresponding sorts axioms when the input uses array sorts and/or functions.

Whenever the input problem uses a sort $\arrayt(\tau,\sigma)$, Vampire adds this sort and function symbols $\selectf$ and $\storef$ of the types $\arrayt(\tau,\sigma) \times \tau \to \sigma$ and $\arrayt(\tau,\sigma) \times \tau \times \sigma \to \arrayt(\tau,\sigma)$, respectively.

If the input problem contains $\storef$, Vampire adds the following axioms for the sorts $\tau$ and $\sigma$ used in the corresponding array theory instance:

\begin{equation}\label{eq:array-axiom-1}
  \begin{aligned}
    &(\forall a:\arrayt(\tau,\sigma))(\forall i:\tau)(\forall v:\sigma)\\
    &\quad(\select{\store{a}{i}{v}}{i} \eql v)
  \end{aligned}
\end{equation}
\begin{equation}\label{eq:array-axiom-2}
  \begin{aligned}
    &(\forall a:\arrayt(\tau,\sigma))(\forall i:\tau)(\forall j:\tau)(\forall v:\sigma)\\
    &\quad(i \neql j \implies \select{\store{a}{i}{v}}{j} \eql \select{a}{j})
  \end{aligned}
\end{equation}
\begin{equation}\label{eq:array-axiom-3}
  \begin{aligned}
    &(\forall a:\arrayt(\tau,\sigma))(\forall b:\arrayt(\tau,\sigma))\\
    &\quad(a \not\eql b \implies (\exists i:\tau)(\select{a}{i} \neql \select{b}{i}))
  \end{aligned}
\end{equation}
These axioms are equivalent to the axioms read-over-write~1, read-over-write~2 and extensionality.

If the input contains only $\selectf$ but not $\storef$ for this instance, then only extensionality \eqref{eq:array-axiom-3} is added.

Theory axioms are not added when the \verb'--theory_axioms' option is set to \verb'off' (the default value is \verb'on'), which leaves an option for the user to try her or his own axiomatisation of arrays.

Vampire uses the extensionality resolution rule~\cite{ATVA14} to efficiently reason with the extensionality axiom.

To express arrays, the TPTP syntax extension supported by Vampire
allows, for every pair of sorts $\tau$ and $\sigma$, denoted by
\lstinline't' and \lstinline's' in the TFF0 syntax, to denote the sort
$\arrayt(\tau,\sigma)$ by \darray{\lstinline's'}{\lstinline't'}. Function symbols $\selectf$
and $\storef$ can be expressed as ad-hoc polymorphic \dselect\ and
\texttt{\$store}, respectively for every pairs of sorts
$\tau,\sigma$. Previously,
the theories of integer arrays and arrays of integer arrays were
represented as sorts \darrayone\ and \darraytwo\ in Vampire,
with the corresponding sort-specific function symbols \dselectone, \dselecttwo, \dstoreone\  and
\dstoretwo. Our new implementation in Vampire, with
support for the polymorphic theory of arrays, deprecates these
two concrete array theories. Instead, one can now use the sorts
\darray{\dint}{\dint} and \darray{\dint}{\darray{\dint}{\dint}}.
%The example~\eqref{eq:arrays-example} can be expressed in the extended TFF0 using the built-in theory of integers. Here \lstinline'$greatereq', \lstinline'$less' and \lstinline'$sum' represent operations of greater-or-equal comparison, less comparison and integer addition, respectively.
%\begin{lstlisting}
%![I : $int]: (($greatereq(i, 0) & $less(i, n)) =>
%                 $lesseq($select(a, i), $select(a, $sum(i, 1))))
%\end{lstlisting}
For example, formula~\eqref{eq:arrays-example} can be written in the extended TFF0 syntax as follows:
\begin{lstlisting}
$let(array := $store(array,3,5),
     $sum($select(array,2), $select(array,3))).
\end{lstlisting}%$

%------------------------------------------------------------------------------
\subsection{Theory of Boolean Arrays}

An interesting special case of the polymorphic theory of arrays is the theory of boolean arrays. In that theory the $\selectf$ function has the type $\arrayt(\tau,\bool) \times \tau \to \bool$ and the $\storef$ function has the type $\arrayt(\tau,\bool) \times \tau \times \bool \to \arrayt(\tau,\bool)$. This means that applications of $\selectf$ can be used as formulas and $\storef$ can have a formula as the third argument.

Vampire implements the theory of booleans arrays similarly to other sorts, by adding theory axioms when the option \verb'--theory_axioms' is enabled. However, the theory axioms are different for the following reason. The axioms of the theory of boolean arrays are syntactically correct in FOOL but not in FOL, because they use quantification over booleans. However, Vampire adds theory axioms only after a translation of FOOL to FOL. For this reason, Vampire uses the following set of axioms for boolean arrays:
\begin{equation*}
  \begin{aligned}
    &(\forall a:\arrayt(\tau,\bool))(\forall i:\tau)(\forall v:\bool)\\
    &\quad(\select{\store{a}{i}{v}}{i} \liff (v \eql \true))
  \end{aligned}
\end{equation*}
\begin{equation*}
  \begin{aligned}
    &(\forall a:\arrayt(\tau,\bool))(\forall i:\tau)(\forall j:\tau)(\forall v:\bool)\\
    &\quad(i \neql j \implies \select{\store{a}{i}{v}}{j} \liff \select{a}{j})
  \end{aligned}
\end{equation*}
\begin{equation*}
  \begin{aligned}
    &(\forall a:\arrayt(\tau,\bool))(\forall b:\arrayt(\tau,\bool))\\
    &\quad(a \not\eql b \implies (\exists i:\tau)(\select{a}{i} \oplus \select{b}{i}))
  \end{aligned}
\end{equation*}
where $\oplus$ is the ``exclusive or'' connective.

\newcommand{\encrypt}{\mathit{encrypt}}
\newcommand{\key}{\mathit{key}}
\newcommand{\msg}{\mathit{message}}
\newcommand{\plaintext}{\mathit{plaintext}}
\newcommand{\cipher}{\mathit{cipher}}

One can use the theory of boolean arrays, for example, to express properties of bit vectors. In the following example we give a formalisation of a basic property of XOR encryption, where the key, the message and the cipher are bit vectors. Let $\encrypt$ be a function of the type $\arrayt(\Z,\bool) \times \arrayt(\Z,\bool) \to \arrayt(\Z,\bool)$. We will write $\encrypt(\msg, \key)$ to denote the result of bit-wise application of the XOR operation to $\msg$ and $\key$. For simplicity we will assume that the message and the key are of equal length. The definition of $\encrypt$ can be expressed with the following axiom:
\begin{align*}
  &(\forall \msg:\arrayt(\Z,\bool))(\forall \key:\arrayt(\Z,\bool))(\forall i:\Z)\\
  &\quad(\select{\encrypt(\msg,\key)}{i} \eql \\
  &\quad\quad\select{\msg}{i} \oplus \select{\key}{i}).
\end{align*}

An important property of XOR encryption is its vulnerability to the known plaintext attack. It means that knowing a message and its cipher, one can obtain the key that was used to encrypt the message by encrypting the message with the cipher. This property can be expressed by the following formula.
\begin{align*}
  &(\forall \plaintext:\arrayt(\Z,\bool))(\forall \cipher:\arrayt(\Z,\bool))\\
  &\quad(\forall \key:\arrayt(\Z,\bool))(\cipher \eql \encrypt(\plaintext,\key) \implies\\
  &\quad\quad\key \eql \encrypt(\plaintext,\cipher))
\end{align*}

The sort $\arrayt(\Z,\bool)$ is represented in the extended TFF0 syntax as \darray{\dint}{\dbool}. The presented property of XOR encryption can be expressed in the extended TFF0 in the following way.
\begin{lstlisting}[language=tptp]
tff(encrypt, type, encrypt: ($array($int,$o) *
    $array($int,$o)) > $array($int,$o)).

tff(xor_encryption, axiom,
    ![Message:$array($int,$o),
      Key:$array($int,$o), I:$int]:
      ($select(encrypt(Message, Key), I) =
        ($select(Message, I) <~> $select(Key,I)))).

tff(known_plaintext_attack, conjecture,
    ![Plaintext:$array($int,$o),
      Cipher:$array($int,$o), Key:$array($int,$o)]:
        ((Cipher = encrypt(Plaintext, Key)) =>
          (Key = encrypt(Plaintext, Cipher)))).
\end{lstlisting}


%------------------------------------------------------------------------------
\section{Program Analysis with the~New~Extensions}
\label{sect:example}

% !TEX root = ../main.tex
\begin{figure*}[b]
  \vspace{-1em}
  \begin{center}
    \parbox{0cm}{
      \begin{tabbing}
        res \ass\ x;\\
        \IF\ (x $>$ y) \\\inc
        \THEN\ max \ass\ x;\\
        \ELSE\ max \ass\ y;\\\dec
        \IF\ (max $>$ 0) \\\inc
        \THEN\ res \ass\ res $+$ max;\\
        \ELSE\ res \ass\ res $-$ max;\\[.5em]\dec
        \reserved{assert} res $\geq$ x
      \end{tabbing}
    }
  \end{center}
  \vspace{-2em}
  \caption{Sequence of conditionals.\label{fig:seqITE}}
\end{figure*}

\begin{figure*}[tb]

\begin{lstlisting}[language=tptp]
tff(x, type, x: $int).
tff(y, type, y: $int).
tff(max, type, max: $int).
tff(res, type, res: $int).
tff(res1, type, res1: $int).

tff(transition_relation, hypothesis,
    res1 = $let(res := x,
           $let(max := $ite($greater(x, y),
                            $let(max := x, max),
                            $let(max := y, max)),
           $let(res := $ite($greater(max, 0),
                            $let(res := $sum(res, max),
                                 res),
                            $let(res := $diff(res, max),
                                 res)),
                res)))).

tff(safety_property, conjecture, $greatereq(res1, x)).
\end{lstlisting}
\caption{Representation of the partial correctness statement of the code on Figure~\ref{fig:seqITE} in Vampire\label{fig:VampireITE}.}
\end{figure*}

In this section we illustrate how FOOL makes first-order theorem
provers better suited to applications in program analysis and
verification.
Firstly,  we give concrete examples of the use of FOOL for
expressing program properties. We avoid various
program analysis steps, such as SSA form computations and renaming
program variables; instead we show how program properties can directly
be expressed in FOOL.
 We also present a technique for
automatically generating the next state relation of any program with
assignments, \ITE, and sequential composition.
For doing so,  we introduce a simple extension of FOOL,
allowing for a general translation that is linear in the size of the
program.
This is a new result intended to understand which extensions of
first-order logic are adequate for naturally representing fragments of
imperative programs.

%%%%%%%%%%%%%%%%%%%%%%%%%%%%%%%%%%%%%%%%%%%%%%%%
\subsection{Encoding the Next State Relation}\label{sec:foolp}

Consider the program given in
Figure~\ref{fig:seqITE}, written in a C-like syntax, using a sequence
of two conditional statements.
The program first computes the maximal value $\mathit{max}$ of two integers $x$ and
$y$ and then adds the absolute value of $\mathit{max}$ to $x$. A safety assertion,
in FOL, is specified at the end of the loop, using the
{\bf assert} construct. This program is clearly safe, the
assertion is satisfied. To prove program safety, one needs to reason
about the program's transition relation, in particular reason about
conditional statements, and express the final value of
the program variable $\mathit{res}$. The partial correctness of the program
of Figure~\ref{fig:seqITE} can be \emph{automatically} expressed in FOOL,
and then Vampire can be used to prove program safety.
This requires us to encode (i)
the next state value of $\mathit{res}$ (and $\mathit{max}$) as a hypothesis
in the extended TFF0 syntax of FOOL,
by using the \ITE\ ({\tt \$ite}) and \LETIN\ ({\tt \$let})
constructs, and (ii)
the safety property as the conjecture to be proven by Vampire.

Figure~\ref{fig:VampireITE} shows this extended TFF0 encoding.
The use of \ITE\ and \LETIN\ constructs allows us to have a
direct  encoding of the  transition relation of
Figure~\ref{fig:seqITE} in FOOL. Note that each expression from the program appears only once in the encoding.

We now explain how the encoding of the next state values of program
variables can be generated automatically.
We consider programs using assignments $\coloneqq$,
\ITE\ and sequential composition $;$.
We begin by making an assumption about the structure of programs (which we relax later). A program $P$ is in \emph{restricted form} if for any subprogram of the form \IF\ e \THEN\ $P_1$ \ELSE\ $P_2$ the subprograms $P_1$ and $P_2$ only make assignments to the same single variable. Given a program $P$ in restricted form let us define its translation $[P]$ inductively as follows:
%
\begin{itemize}
	\item $[$x\ASS e$]$ is $\letin{x}{e}{x}$;
	\item $[$\IF\ e \THEN\ $P_1$ \ELSE\ $P_2]$, where $P_1$ and
          $P_2$ update $x$,  is $\letin{x}{\ite{e}{[P_1]}{[P_2]}}{x}$;
	\item $[P_1$;\;$P_2]$ is $\mathtt{let}~D~\mathtt{in}~[P_2]$ where $[P_1]$ is $\mathtt{let}~D~\mathtt{in}~x$.
\end{itemize}
%
Given a program $P$, the next state value for variable $x$ can be
given by $[P$; x\ASS x$]$,
i.e. by ensuring the final statement of the program updates the
variable of interest.
The restricted form is required as conditionals must be viewed
as assignments in the translation and assignments can only be made to single variables.

\begin{figure*}[tb]
  \vspace{-1em}
  \begin{center}
    \parbox{0cm}{
      \begin{tabbing}
        \IF\ (x $>$ y) \\\inc
        \THEN\ t \ass\ x; x \ass\ y; y \ass\ t;\\\dec
        \reserved{assert} y $\geq$ x
      \end{tabbing}
    }
  \end{center}
  \vspace*{-2em}
  \caption{Updating multiple variables.\label{fig:tmpSwap}}
\end{figure*}

To demonstrate the limitations of this restriction let us consider the simple program in Figure~\ref{fig:tmpSwap} that ensures that x is not larger than y. We cannot apply the translation as the conditional updates three variables. To generalise the approach we can extend FOOL with \emph{tuple expressions}, let us call this extension \foolp. In this extended logic the next state values for Figure~\ref{fig:tmpSwap} can be encoded as follows:
%
\[
  \letnl{(x,y,t)}{\itenll{x > y}
                 {\letinnl{(x,y,t)}{(x,y,x)}
                          {\letinnl{(x,y,t)}{(y,y,t)}
                                   {\letin{(x,y,t)}{(x,t,t)}
                                            {(x,y,t)}}}}
                 {(x,y,t)}}
        {(x,y,t)}
\]
%
We now give a brief sketch of the extended logic \foolp\ and the associated translation. We omit details since its full definition and semantics would require essentially repeating definitions from \cite{FOOL}.  \foolp\ extends \fool\ by tuples; for all expressions $t_i$ of type $\sigma_i$ we can use a \emph{tuple expression} $(t_1,\ldots,t_n)$ of type $(\sigma_1,\ldots,\sigma_n)$. The logic should also include a suitable tuple projection function, which we do not discuss here.

This extension allows for a more general translation in two senses:
first, the previous restricted form is lifted; and second, it now
gives the next state values of {\it all} variables updated by the program. Given a program $P$ its translation $[P]$ will have the form $\letin{(x_1,\ldots,x_n)}{E}{(x_1,\ldots,x_n)}$, where $x_1,\ldots,x_n$ are all variables updated by $P$, that is, all variables used in the left-hand-side of an assignment. We inductively define $[P]$ as follows:
\begin{itemize}
	\item $[\text{x}_i \text{\ASS} e]$ is $\letin{(\ldots,x_i,\ldots)}{(\ldots,e,\ldots)}{(x_1,\ldots,x_n)}$,
	\item $[$\IF\ e \THEN\ $P_1$ \ELSE\ $P_2]$ is $\letin{(x_1,\ldots,x_n)}{\ite{e}{[P_1]}{[P_2]}}{(x_1,\ldots,x_n)},$
	\item $[P_1$;\;$P_2]$ is $\mathtt{let}~D~\mathtt{in}~[P_2]$ where $[P_1]$ is $\mathtt{let}~D~\mathtt{in}~(x_1,\ldots,x_n)$.
\end{itemize}
This translation is bounded by $O(v\cdot n)$, where $v$ is the number
of variables in the program and $n$ is the program size (number of
statements) as each program statement is used once with one or two
instances of $(x_1,\ldots,x_n)$.
This becomes $O(n)$ if we assume that the number of
variables is fixed. The translation could be refined so that some introduced  \LETIN\
expressions only use a subset of program variables.
Finally, this translation preserves the semantics
of the program.

\begin{theorem}\rm
  Let $P$ be a program with variables $(x_1,\ldots,x_n)$ and let $u_1,\ldots,u_n, v_1, \ldots, v_n$ be values (where $u_i$ and $v_i$ are of the same type as $x_i$). If $P$ changes the state $\{x_1\to u_1,\ldots,x_n\to u_n\}$ to $\{x_1\to v_1,\ldots,x_n\to v_n\}$ then the value of $[P]$ in $\{x_1\to u_1,\ldots,x_n\to u_n\}$ is $(v_1,\ldots,v_n)$.
\end{theorem}

This translation encodes the next state values of program variables by
directly following the structure of the program. This leads to a
succinct representation that, importantly, does not lose any
information or attempt to translate the program too early. This allows
the theorem prover to apply its own translation to FOL that it can
handle efficiently.   While \foolp{} is not yet fully supported in
Vampire, we believe experimenting with \foolp{} on
examples coming from program analysis and verification is an
interesting task for future work.


%%%%%%%%%%%%%%%%%%%%%%%%%%%%%%%%%%%%%%%%%%%%%%%%
\subsection{A Program with a Loop and Arrays}

\begin{figure*}[bt]
{
  \begin{center}
    \parbox{0cm}{
  \begin{tabbing}
    $a$ \ass\ $0$; $b$ \ass\ $0$; $c$ \ass\ $0$; \\[.5em]
    \reserved{invariant} a = b + c $\wedge$ \\
    {\color{white}\reserved{invariant}} a $\geq$ 0 $\wedge$ b $\geq$ 0 $\wedge$ c $\geq$
    0 $\wedge$ a $\leq$ k $\wedge$ \\
    {\color{white}\reserved{invariant}} $(\forall p) (0\leq p<b \implies
    (\exists i) (0 \leq i < a \wedge A[i] > 0 \wedge B[p] = A[i]))$\\[1em]
    \WHILE\ ($a \leq k$) \DO \\ \inc
      \IF\ ($A[a] > 0$) \\ \inc
        \THEN\ \=\+ $B[b]$ \ass\ $A[a]$\semicol\; $b$ \ass\ $b+1$\semicol \\ \dec
        \ELSE\ \=\+ $C[c]$ \ass\ $A[a]$\semicol\; $c$ \ass\ $c+1$\semicol \\ \dec \dec
      $a$ \ass\ $a+1$\semicol \\ \dec
    \OD\\[.5em]
    \reserved{assert} $(\forall p)(0 \leq p<b \implies B[p]> 0)$
  \end{tabbing}
    }
  \end{center}
  \caption{Array partition.\label{fig:partition}}
}
\end{figure*}

\begin{figure*}[tb]
\begin{lstlisting}[language=tptp]
tff(a, type, a: $int).
tff(b, type, b: $int).
tff(c, type, c: $int).
tff(k, type, k: $int).
tff(arrayA, type, arrayA: $array($int, $int)).
tff(arrayB, type, arrayB: $array($int, $int)).
tff(arrayC, type, arrayC: $array($int, $int)).

tff(invariant_property, hypothesis, inv <=>
    ((a = $sum(b, c)) &
     $greatereq(a, 0) & $greatereq(b, 0) &
     $greatereq(c, 0) & $lesseq(a, k) &
     ![P: $int]: ($lesseq(0, P) & $less(P, b) =>
       (?[I: $int]: ($lesseq(0, I) & $less(I, a) &
          $greater($select(arrayA, I), 0) &
          $select(arrayB, P) = $select(arrayA, I)))))).

tff(safety_property, conjecture,
    (inv & ~$lesseq(a, k)) =>
      (![P: $int]: ($lesseq(0, P) & $less(P, b) =>
                    $greater($select(arrayB, P), 0)))).
\end{lstlisting}
\caption{Representation of the partial correctness statement of the code on Figure~\ref{fig:partition} in Vampire\label{fig:loop_safety_Vampire}.}
\end{figure*}

Let us now show the use of FOOL in Vampire for reasoning about
programs with loops. Consider the program given in
Figure~\ref{fig:partition}, written in a C-like syntax.  The program
fills an integer-valued array $B$ by the strictly positive values
of a source array $A$, and an integer-valued array $C$ with
the non-positive values of $A$. A safety assertion, in FOL, is
specified at the end of the loop, using the {\bf assert}
construct. The program of Figure~\ref{fig:partition} is clearly safe
as the assertion is satisfied when the loop is exited.
However, to prove program safety we need additional
loop properties, that is loop invariants, that hold at any loop
iteration. These can be automatically generated using existing approaches, for
example the symbol elimination method for invariant generation in
Vampire~\cite{fase2009}. In this case we use the FOL property
specified in the {\bf invariant} construct of  Figure~\ref{fig:partition}. This invariant
property states that at any loop iteration, (i) the  sum of visited
array elements in $A$ is the sum of visited elements in $B$ and $C$
(that is, $a = b + c$), (ii) the number of visited array
elements in $A$, $B$, $C$ is positive (that is, $a\geq 0$, $b\geq 0$,
and $c\geq 0$), with $a\leq k$, and (iii) each array element
$B[0],\ldots,B[b-1]$ is a strictly positive element in
$A$. Formulating the latter property requires quantifier alternation
in FOL, resulting in the quantified property with $\forall\exists$
listed in the invariant of  Figure~\ref{fig:partition}.
We can verify the safety of the program using Hoare-style reasoning in Vampire.
The partial correctness property is that the invariant and the negation of the loop condition implies the safety assertion.
This is the conjecture to be proven by Vampire.
Figure~\ref{fig:loop_safety_Vampire} shows the encoding in the
extended TFF0 syntax of this partial
correctness statement; note that this uses the built-in theory of
polymorphic arrays in Vampire, where $arrayA$, $arrayB$ and $arrayC$
correspond respectively to the arrays $A$, $B$ and $C$.

So far, we assumed that the given invariant in
Figure~\ref{fig:partition} is
indeed an invariant. Using \foolp{} described in
Section~\ref{sec:foolp}, we can verify the inductiveness
property of the invariant, as follows: (i) express the the transition
relation of the loop in \foolp, and (ii) prove that, if the invariant
holds at an arbitrary loop iteration $i$, then it also holds at loop
iteration $i+1$. For proving this, we can again use \foolp\ to
formulate
the next state values of loop variables in the invariant at loop
iteration $i+1$.
Moreover, \foolp{} can also be used to express formulas as
inputs to the symbol elimination method for invariant generation in
Vampire. We leave the task of using \foolp{} for invariant generation
as further work.


%------------------------------------------------------------------------------
\section{Experimental Results}
\label{sect:experiments}

The extension of Vampire to support FOOL and the polymorphic theory of
arrays comprises about 3,100 lines of C++ code, of which the
translation of FOOL to FOL and FOOL paramodulation takes about 2,000
lines, changes in the parser about 500 lines and
the implementation of the polymorphic theory of arrays about 600 lines.
Our implementation is available at \url{www.cse.chalmers.se/~evgenyk/fool-experiments/} and will be included
in the forthcoming official release of Vampire.

In the sequel, by Vampire we mean its version including support for
FOOL and the polymorphic theory of arrays. We write \nofoolVampire\ for
its version with FOOL paramodulation turned off.

In this section we present experimental results obtained by running Vampire on FOOL problems. Unfortunately, no large collections of such problems are available, because FOOL was not so far supported by any first-order theorem prover. What we did was to extract such benchmarks from other collections.

\begin{enumerate}
\item We noted that many problems in the higher-order part of the TPTP library~\cite{TPTP} are FOOL problems, containing no real higher-order features. We converted them to FOOL problems.

\item We used a collection of first-order problems about (co)al\-ge\-braic datatypes, generated by the Isabelle theorem prover~\cite{Isabelle}, see Subsection~\ref{subsec:Isabelle} for more details.
\end{enumerate}
Our results are summarised in Tables~\ref{table:thf-results}--\ref{table:smt-lib-nontrivial} and discussed below. These results were obtained on a MacBook Pro with a 2,9 GHz Intel Core i5 and 8 Gb RAM, and using the time limit of 60 seconds per problem. Both the benchmarks and the results are available at \url{www.cse.chalmers.se/~evgenyk/fool-experiments/}.

\subsection{Experiments with TPTP Problems}
The higher-order part of the TPTP library contains 3036 problems. Among these problems, 134 contain either boolean arguments in function applications or quantification over booleans, but contain no lambda abstraction, higher-order sorts or higher-order equality. We used these 134 problems, since they belong to FOOL but not to FOL. We translated these problems from THF0 to the modification of TFF0, supported by Vampire using the following syntactic transformation: \begin{enumerate*}[label=(\alph*)]
\item every occurrence of the keyword \verb'thf' was replaced by \verb'tff';
\item every occurrence of a sort definition of the form \verb's_1 >  ... > s_n > s' was replaced by \verb's_1 * ... * s_n > s';
\item every occurrence of a function application of the form \verb'f @  t_1 @ ... @ t_n' was replaced by \verb'f(t_1, ..., t_n)'.
\end{enumerate*}

Out of 134 problems, 123 were marked as Theorem and 5 as
Unsatisfiable, 5 as CounterSatisfiable, and 1 as Satisfiable, using
the SZS status of TPTP. Essentially, this means that among their
satisfiability-checking analogues, 128 are unsatisfiable and 6 are
satisfiable. Vampire was run with the \verb'--mode casc' option for
unsatisfiable (Theorem and Unsatisfiable) problems and with \verb'--mode casc_sat' for satisfiable (CounterSatisfiable and Satisfiable) problems. These options correspond to the CASC competition modes of
Vampire for respectively proving validity (i.e. unsatisfiability) and
satisfiability of an input problem.

For this experiment, we compared the performance of Vampire with those of the higher-order theorem provers used in the the latest edition of CASC \cite{CASC25}:
Satallax~\cite{Satallax}, Leo-II~\cite{LeoII}, and Isabelle~\cite{Isabelle}. We note that all of them used the first-order theorem prover E~\cite{E13} for first-order reasoning (Isabelle also used several other provers).

\begin{table}[t]
  \caption{Runtimes in seconds of provers on the set of 134 higher-order TPTP problems.}
  \begin{center}
  \begin{tabular}{lrr}
    \hline Prover & Solved & Total time on solved problems \\ \hline
    Vampire & 134 & 3.59 \\
    \nofoolVampire & 134 & 7.28 \\
    Satallax & 134 & 23.93 \\
    Leo-II & 127 & 27.42 \\
    Isabelle & 128 & 893.80
  \end{tabular}
  \end{center}
  \label{table:thf-results}
\end{table}

Table~\ref{table:thf-results} summarises our results on these problems. Only Vampire, \nofoolVampire\ and Satallax were able to solve all of them, while
Vampire was the fastest among all provers. We believe these results
are significant for two reasons. First, for solving these problems
previously one 
needed higher-order theorem provers, but now can they be proven using first-order reasoners. Moreover, even on such simple problems there is a clear gain from using FOOL paramodulation.

% \LK{should we add table with some non-trivial HOL problems?}
% AV: no, they are all trivial, alas

\subsection[Experiments with Algebraic Datatypes Problems]{Experiments with Algebraic\\Datatypes Problems}
\label{subsec:Isabelle}

For this experiment, we used 152 problems generated by the Isabelle theorem prover. These
problems express various properties of (co)algebraic datatypes and are written in the SMT-LIB~2 syntax~\cite{SMT-LIB}. All 152 problems contain quantification over booleans, boolean arguments in function/predicate applications and \ITE\ expressions. These examples were generated and given to us by Jasmin Blanchette, following the recent work on reasoning about (co)datatypes~\cite{Blanchette15}. To run the benchmark we first translated the SMT-LIB files to the TPTP syntax using the SMTtoTPTP translator~\cite{SMTLIB2TPTP} version 0.9.2.
Let us note that this version of SMTtoTPTP does not fully support the
boolean type in SMT-LIB. However, by setting the option
\verb'--keepBool' in SMTtoTPTP, we managed to translate these 152
problems into an extension of TFF0, which Vampire can read.
We also modified the source code of  SMTtoTPTP so that  \ITE\
expressions in the SMT-LIB files are not expanded but translated to \dite\
in FOOL. A similar modification would have been needed for translating
\LETIN\ expressions; however, none of our 152 examples used \LETIN.

After translating these 152 problems into an extended TFF0 syntax
supporting FOOL, we ran Vampire twice on each benchmark: once using the
option \verb'--mode casc', and once using
\verb'--mode' \verb'casc_sat'.  For each problem, we recorded the
fastest successful run of Vampire. We used a similar setting for
evaluating \nofoolVampire.
In this experiment, we then compared Vampire with
the best available SMT solvers, namely with CVC4~\cite{CVC4} and
Z3~\cite{Z3}.

\begin{table}[tb]
  \caption{Runtimes in seconds of provers on the set of 152 algebraic datatypes problems.}%  \nofoolVampire\ denotes Vampire with disabled FOOL paramodulation.}
  \begin{center}
  \begin{tabular}{lrr}
    \hline Prover & Solved & Total time on solved problems \\ \hline
    Vampire & 59 & 26.580 \\
    Z3 & 57 & 4.291 \\
    \nofoolVampire & 56 & 26.095 \\
    CVC4 & 53 & 25.480
  \end{tabular}
  \end{center}
  \label{table:smt-lib-results}
\end{table}

Table~\ref{table:smt-lib-results} summarises the results of our experiments on these 152 problems. Vampire solved the largest number of problems, and all problems solved by \nofoolVampire\ were also solved by Vampire.
Figure~\ref{fig:smt-lib-diagram} shows the Venn diagram of the sets of
problems solved by Vampire, CVC4 and Z3, where the numbers denote the numbers of solved problems.
All problems apart from 11 were either solved by all systems or not solved by all systems. Table~\ref{table:smt-lib-nontrivial} details performance results on these 11 problems.

\begin{figure}[tb]
  %\vspace{-0.3em}
  \centering
  \begin{tikzpicture}
    \draw (0,0) circle (1.5cm);
    \draw (50:1cm) circle (1.55cm);
    \draw (0cm:0.8cm) circle (1.4cm);
    \node at (0.8cm:0.5cm) {$51$}; %
    \node at (-2.2cm:1.2cm) {$1$};
    \node at (1cm:-1.1cm) {$2$};
    \node at (-2cm:-1.1cm) {$3$};
    \node at (-3.8cm:-1.9cm) {$4$}; %
    \node at (-0.95cm:1.775cm) {$0$}; %
    \node at (0.45cm:1.775cm) {$1$}; %
    \node at (2.7cm:2.65cm) {Vampire};
    \node at (-3.7cm:1.8cm) {Z3};
    \node at (-1.6cm:2.3cm) {CVC4};
  \end{tikzpicture}
  \vspace{-0.3em}
  \caption{Venn diagram of the subsets of the algebraic datatypes problems, solved by Vampire, CVC4 and Z3.}
  \label{fig:smt-lib-diagram}
\end{figure}

\newcommand{\timeout}{---}
\newcommand{\gaveup}{---}
\begin{table}[tb]
  \caption{Runtimes in seconds of provers on selected algebraic datatypes problems. Dashes mean the solver failed to find a solution.}% \nofoolVampire\ denotes Vampire with disabled FOOL paramodulation.}
  \begin{center}
  \begin{tabular}{lrrr}
    \hline Problem & Vampire & CVC4 & Z3 \\ \hline
    \verb'afp/abstract_completeness/1830522' & \timeout & \timeout & 0.172 \\
    \verb'afp/bindag/2193162' & \timeout & \gaveup & 0.388 \\
    \verb'afp/coinductive_stream/2123602' & \timeout & 0.373 & 0.101 \\
    \verb'afp/coinductive_stream/2418361' & 3.392 & \timeout & \timeout \\
    \verb'afp/huffman/1811490' & 0.023 & \gaveup & \timeout \\
    \verb'afp/huffman/1894268' & 0.025 & \gaveup & 0.052 \\
    \verb'distro/gram_lang/3158791' & 0.047 & 0.179 & \timeout \\
    \verb'distro/koenig/1759255' & 0.070 & \timeout & \timeout \\
    \verb'distro/rbt_impl/1721121' & 4.523 & \timeout & \timeout \\
    \verb'distro/rbt_impl/2522528' & 0.853 & \gaveup & 0.064 \\
    \verb'gandl/bird_bnf/1920088' & 0.037 & \timeout & 0.077
  \end{tabular}
  \end{center}
  \label{table:smt-lib-nontrivial}
\end{table}

Based on our experimental results shown in Tables~\ref{table:smt-lib-results} and \ref{table:smt-lib-nontrivial}, we make the following observations. On the given set of problems the implementation of FOOL reasoning in Vampire was efficient enough to compete with state-of-the-art SMT solvers. This is significant because the problems were tailored for SMT reasoning. Vampire not only solved the largest number of problems, but also yielded runtime results that are comparable with those of CVC4. Whenever successful, Z3 turned out to be faster than Vampire; we believe this is because of the sophisticated preprocessing steps in Z3. Improving FOOL preprocessing in Vampire, for example for more efficient CNF translation of FOOL formulas, is an interesting task for further research. We note that the usage of FOOL paramodulation showed improvement.

%------------------------------------------------------------------------------
\section{Related Work}
\label{sect:related}

FOOL was introduced in our previous work~\cite{FOOL}. This also presented a translation from FOOL to the ordinary first-order logic, and FOOL paramodulation. In this paper we describe the first practical implementation of FOOL and FOOL paramodulation.

Superposition theorem proving in finite domains, such as the boolean domain, is also discussed in~\cite{HillenbrandWeidenbach13}. The approach of~\cite{HillenbrandWeidenbach13} sometimes falls back to enumerating instances of a clause by instantiating finite domain variables with all elements of the corresponding domains. Nevertheless, it allows one to also handle finite domains with more than two elements. One can also generalise our approach to arbitrary finite domains by using binary encodings of finite domains. However, this will necessarily result in loss of efficiency, since a single variable over a domain with $2^k$ elements will become $k$ variables in our approach, and similarly for function arguments.
Although \cite{HillenbrandWeidenbach13} reports preliminary results with the theorem prover SPASS, we could not make an experimental comparison since the SPASS implementation has not yet been made public.

Handling boolean terms as formulas is common in the SMT community. The SMT-LIB project~\cite{SMT-LIB} defines its core logic as first-order logic extended with the distinguished first-class boolean sort and the \LETIN\ expression used for local bindings of variables. The language of FOOL extends the SMT-LIB core language with local function definitions, using \LETIN\ expressions defining functions of arbitrary, and not just zero, arity.

A recent work \cite{SMTLIB2TPTP} presents SMTtoTPTP, a translator from SMT-LIB to TPTP. SMTtoTPTP does not fully support boolean sort, however one can use SMTtoTPTP with the \verb'--keepBool' option to translate SMT-LIB problems to the extended TFF0 syntax, supported by Vampire.

Our implementation of the polymorphic theory of arrays uses a syntax that coincides with the TPTP's own syntax for polymorphically typed first-order logic TFF1~\cite{tff1}.


%------------------------------------------------------------------------------
\section{Conclusion and Future Work}
\label{sect:future}

We presented new features recently implemented in Vampire. They include FOOL: the extension of first-order logic by a first-class boolean sort, \ITE\ and \LETIN\ expressions, and polymorphic arrays. Vampire implements FOOL by translating FOOL formulas into FOL formulas. We described how this translation is done for each of the new features. Furthermore, we described a modification of the superposition calculus by FOOL paramodulation that makes Vampire reasoning in FOOL more efficient. 
We also gave a simple extension to FOOL that allows one to express the next state relation of a program as a boolean formula which is linear in the size of the program.

Neither FOOL nor polymorphic arrays can be expressed in TFF0. In order to support them Vampire uses a modification of the TFF0 syntax with the following features:

\begin{enumerate}
  \item the boolean sort \tptpo\ can be used as the sort of arguments and quantifiers;
  \item boolean variables can be used as formulas, and formulas can be used as boolean arguments;
  \item \ITE\ expressions are represented using a single keyword \dite\ rather than two different keywords \ditet\ and \ditef;
  \item \LETIN\ expressions are represented using a single keyword \dlet\ rather than four different keywords \dlettt, \mbox{\dlettf,} \dletft\ and \dletff;
  \item \darraySymb, \dselect\ and \dstore\ are used to represent arrays of arbitrary types.
\end{enumerate}

Our experimental results have shown that our implementation, and especially FOOL paramodulation, are efficient and can be used to solve hard problems.

Many program analysis problems, problems used in the SMT community, and problems generated by interactive provers, which previously required (sometimes complex) ad hoc translations to first-order logic, can now be understood by Vampire without any translation. Furthermore, Vampire can be used to translate them to the standard TPTP without \ITE\ and \LETIN\ expressions, that is, the format understood by essentially all modern first-order theorem provers and used at recent CASC competitions. One should simply use \texttt{--mode preprocess} and Vampire will output the translated problem to \texttt{stdout} in the TPTP syntax. 

The translation to FOL described here is only the first step to the efficient handling of FOOL. It can be considerably improved. For example, the translation of \LETIN\ expressions always introduces a fresh function symbol together with a definition for it, whereas in some cases inlining the function would produce smaller clauses. Development of a better translation of FOOL is an important future work.

FOOL can be regarded as the smallest superset of the SMT-LIB~2 Core language and TFF0. A native implementation of an SMT-LIB parser in Vampire is an interesting future work. Note that such an implementation can also be used to translate SMT-LIB to FOOL or to FOL.

Another interesting future work is extending FOOL to handle polymorphism and implementing it in Vampire. This would allow us to parse and prove problems expressed in the TFF1~\cite{tff1} syntax. Note that the current usage of \darraySymb\ conforms with the TFF1 syntax for type constructors.


\section*{Acknowledgements}
%\acks
We acknowledge funding from the Austrian FWF National Research Network RiSE
S11409-N23, the Swedish VR grant D049770~--- GenPro, the
Wallenberg Academy Fellowship 2014, and the EPSRC grant ``Reasoning in Verification and Security''.

\def\paperThreeContentsTitle{A Clausal Normal Form Translation for \folb{}}
\def\paperThreeChapterTitle{A Clausal Normal Form\\Translation for \folb{}}
\def\paperThreeAuthors{Evgenii~Kotelnikov, Laura~Kov\'{a}cs,\\Martin~Suda and Andrei~Voronkov}
\def\paperThreeAbstract{Superposition-based theorem provers for first-order logic usually operate on sets of first-order clauses. It is well-known that the translation of a first-order formula to clausal normal form (CNF) can crucially affect the performance of a prover. This paper presents a superposition-friendly translation of \folb{} formulas to first-order clauses. Compared to our previous approach, the new translation can produce a smaller set of clauses with fewer introduced symbols. \iffalse initial results of designing an efficient CNF translation for formulas of the \folb{} logic that we described earlier. \folb{} extends many-sorted first-order logic with a first class boolean sort and \ITE\ and \LETIN\ expressions. Our CNF translation extends \newcnf{}, a top-down clausification algorithm.\fi }
\paperchapter{\paperThreeContentsTitle}
             {\paperThreeChapterTitle}
             {\paperThreeAuthors}
             {\paperThreeAbstract}
             {}
\label{chap:cnf}
\section{Introduction}
\label{sec:newcnf/introduction}
% !TEX root = ../main.tex

Automated theorem provers for first-order logic usually operate on sets of first-order clauses. In order to check a formula in full first-order logic, theorem provers first translate it to clausal normal form (CNF). It is well-known that the quality of this translation affects the performance of the theorem prover. While there is no absolute criterion of what the best CNF for a formula is, theorem provers usually try to make the CNF smaller according to some measure. This measure can include the number of clauses, the number of literals, the lengths of the clauses and the size of the resulting signature, i.e.~the number of function and predicate symbols. Implementors of CNF translations commonly employ formula simplification~\cite{nonnengart2001computing}, (generalised) formula naming~\cite{nonnengart2001computing,azmy2013computing}, and other clausification techniques, aimed to make the CNF smaller.

Our recent work~\cite{FOOL} presented a modification of many-sorted first-order with first-class boolean sort. We called this logic \folb{}, standing for first-order logic (FOL) with boolean sort. \folb{} extends FOL by (i) treating boolean terms as formulas; (ii) \ITE\ expressions; and (iii) \LETIN\ expressions. There is a model-preserving translation of \folb{} formulas to FOL that works by replacing parts of a \folb{} formula with applications of fresh function and predicate symbols and extending the set of assumptions with definitions of these symbols. We implemented~\cite{VampireAndFOOL} this translation in the Vampire theorem prover~\cite{Vampire13}. To check a \folb{} problem Vampire first translates it to first-order logic, then converts the resulting first-order formulas to a set of clauses.

%This translation can be used in first-order provers to support reasoning about \folb\ problems. We implemented~\cite{VampireAndFOOL} it in the Vampire theorem prover~\cite{Vampire13}. In that implementation, a \folb\ formula is first translated to first-order logic, then Vampire translates the resulting first-order formulas to first-order clauses and checks them.

%While the translation from~\cite{FOOL} provides an easy way to support \folb{} in existing first-order provers, it is not necessarily friendly to superposition. A \folb{} formula is translated to a set of first-order clauses in two steps rather than directly. That way, we miss out on the opportunity to specialise the translation and integrate existing clausification techniques into it. By Some \folb{} formulas can be translated to smaller CNFs than we obtain

While the translation from~\cite{FOOL} provides an easy way to support \folb{} in existing first-order provers, it is not necessarily efficient. A more efficient translation can convert a \folb{} formula directly to a set of first-order clauses, skipping the intermediate step of converting it to full first-order logic. This way, the translation can integrate known clausification techniques and improve the quality of the resulting clausal normal form. % We observed that the CNFs that we obtain from some \folb{} formulas with this translation are unnecessarily large, and it can damage the performance of the prover. We believe that this is due to the fact that \folb{} formulas are translated to CNF in two steps rather than directly. This way, we miss out on the opportunity to specialise the translation and integrate existing clausification techniques into it.


%While the translation from~\cite{FOOL} provides an easy way to support \folb{} in existing first-order provers, it is not necessarily friendly to superposition. We observed that the CNFs that we obtain from some \folb{} formulas with this translation are unnecessarily large, and it can damage the performance of the prover. We believe that this is due to the fact that \folb{} formulas are translated to CNF in two steps rather than directly. This way, we miss out on the opportunity to specialise the translation and integrate existing clausification techniques into it.

In this work we present a clausification algorithm that translates a \folb{} formula to an equisatisfiable set of first-order clauses. This algorithm aims to minimise the number of clauses and the size of the resuting signature, especially on formulas with \ITE, \LETIN\ expressions and complex boolean structure. This ultimately leads to an increase of performance of a theorem prover that implements it compared e.g. to the use of the translation to full first-order logic from~\cite{FOOL}.

Our algorithm is an extension of the \newcnf~\cite{newcnf_fol} clausification algorithm\footnote{The name \newcnf{} is tentative and is likely to be changed before the publication of~\cite{newcnf_fol}.} for first-order logic that enables it to translate \folb{} formulas. Section~\ref{sec:newcnf/cnf} revisits the essentials of \newcnf{} which are required for our extension presented in Section~\ref{sec:newcnf/fool}. Our algorithm combines translation of \folb{} formulas to first-order logic and clausification. In Section~\ref{sec:newcnf/comparison} we discuss how integrating clausification techniques help to produce smaller clausal normal forms. Section~\ref{sec:newcnf/experiments} describes the experiments on theorem proving with FOOL formulas using different translations. Finally, Section~\ref{sec:newcnf/conclusions} outlines future work.

The main contributions of this paper are the following:
\begin{enumerate}
  \item a clausification algorithm that translates a \folb{} formula to an equisatisfiable set of first-order clauses;
  \item an implementation of this algorithm in the Vampire theorem prover;
  \item experimental results that demonstrate an increase of performance of Vampire on \folb{} problems compared to its version with the translation of \folb{} formulas to first-order logic presented in~\cite{FOOL}.
\end{enumerate}

\section[Clausal Normal Form for First-Order Logic]{Clausal Normal Form for\\First-Order Logic}
\label{sec:newcnf/cnf}
% !TEX root = ../main.tex

% nonnengart2001computing also have (in the simple approach): 
% elimination of equivalences as part of NNF transform 
% (but one wants to decide about equivalences during naming)
% miniscoping and variable renaming just before skolemization
% (but let's ignore miniscoping and assume nice variables rightaway or a detail below the level of this presentation) 
%
%???? polarity dependent elimination of equivalences section has an argument about ugly invisible tautologies 
% (like the ones we mention below)
%

Traditional approaches to clausification~\cite{nonnengart2001computing} produce a clausal normal form of a given first-order formula in several stages, where each stage represents a single pass through the formula tree. These stages usually include (in this order): formula simplification, translation into negation normal form, formula naming, elimination of equivalences, skolemisation, and distribution of disjunctions over conjunctions. \newcnf{} takes a different approach that employs a single top-down traversal of the formula in which these stages are combined. This approach enables optimisations that are not available if the stages of clausification are independent. For example, compared the traditional staged approach \newcnf{} can introduce fewer Skolem functions on formulas with a complex nesting of equivalences and quantifiers. %Another example is an easy detection of intermediate tautologies, which are discarded on the fly. \newcnf{} thus maintains a more accurate count of sub-formula occurrences, on which the decision whether to name a sub-formula is based.

The main advantage of \newcnf{} for this work, however, is that its top-down traversal provides a suitable context not only for clausification of first-order formulas, but also of the extension of first-order logic with \folb{} features.

\newcnf{} works with the input first-order formula represented as a set of \emph{intermediate clauses}. An intermediate clause $\genclause{C}{\subst}$ is a pair of a multiset $C$ of signed first-order formulas and a substitution $\subst$ that maps variables to terms. We denote by $\genlit{\psi}{\sign}$ a first-order formula $\psi$, signed with $\sign \in \{\possign,\negsign\}$. Signs are used for polarity dependent elimination of equalities~\cite{nonnengart2001computing}. Substitution $\subst$ is used for skolemisation. During the trans\-la\-tion, $\subst$ is extended by skolemised variables and their corresponding Skolem terms. %, and at the end of the translation applied to each formula in $C$.

\newcnf{} starts with the input first-order formula $\phi$ and a set $\GC$ of intermediate clauses that contains a single intermediate clause $\genclause{\{\genlit{\phi}{\possign}\}}{\emptySubst}$, where $\emptySubst$ is an empty substitution. Then it makes a series of replacements of in\-ter\-me\-di\-ate clauses in $\GC$ until all intermediate clauses in $\GC$ contain only signed atomic formulas. A replacement of an intermediate clause might introduce Skolem functions and names of subformulas. Each replacement preserves the following invariant: the input formula is equivalent with respect to the original signature to the conjunction of universally quantified formulas of the form $\bigwedge_{\genlit{\psi}{\sign} \in C} \psi'$ for every $\genclause{C}{\subst}$ in $\GC$, where every $\psi'$ is $\psi \subst$ if $\sign = \possign$ and $\neg\psi \subst$ if $\sign = \negsign$. When every $\psi$ in each intermediate clause is atomic, $\GC$ contains the representation of a clausal normal form of the input formula.

% the conjunction of universally quantified formulas of the form $$\bigvee_{\genlit{\phi}{\possign} \in G} \phi\subst \vee \bigvee_{\genlit{\phi}{\negsign} \in G} \neg \phi\subst$$ for every $\genclause{C}{\subst}$ in $\GC$.

%\newcnf{} implements polarity dependent elimination of equalities~\cite{nonnengart2001computing}. For that, \newcnf{} signs formulas with a positive or negative sign. We denote by $\genlit{\phi}{\sign}$ a first-order formula $\phi$, signed with $\sign \in \{\possign,\negsign\}$. The substitution is used for skolemisation. During the translation $\subst$ is extended by skolemised variables and their corresponding Skolem terms.

%\newcnf{} maintains a set of \emph{intermediate clauses}. An intermediate clause $\genclause{C}{\subst}$ is a pair of a multiset $C$ of signed formulas and a substitution $\subst$. The substitution is used for skolemisation. During the translation $\subst$ is extended by skolemised variables and their corresponding Skolem terms. At the end of the translation $\subst$ is applied to every signed formulas of the intermediate clause. An intermediate clause $\genclause{C}{\subst}$ that only contains atomic signed formulas can be translated to a first-order clause by computing the set of literals \[ \{ A\subst \ |\ \genlit{A}{\possign} \in G \} \cup \{ \neg A\subst \ |\ \genlit{A}{\negsign} \in G \}.\] 

%\newcnf{} translates a first-order formula $\phi$ to its set of first-order clauses. It starts with a set $\GC$ of intermediate clauses that contains a single intermediate clause $\genclause{\{\genlit{\phi}{\possign}\}}{\emptySubst}$, where $\emptySubst$ is an empty substitution. Then it makes a series of replacements of intermediate clauses in $\GC$ until all the intermediate clauses in $\GC$ contain only atomic formulas. Finally, it builds the resulting set of first-order clauses by translating each intermediate clause in $\GC$ to a first-order clause.

For every subformula of $\phi$, \newcnf{} maintains its list of occurrences in the intermediate clauses of $\GC$. These occurrences are used for naming of formulas and are updated whenever intermediate clauses are added or removed from $\GC$. 

The replacements of intermediate clauses are guided by the structure of $\phi$. \newcnf{} traverses $\phi$ top-down, visiting every non-atomic subformula of $\phi$ exactly once in an order that respects the subformula relation. It means that for each distinct subformulas $\psi_1$ and $\psi_2$ of $\phi$ such that $\psi_1$ is a subformula of $\psi_2$, $\psi_2$ is visited before $\psi_1$.

For every subformula $\psi$, \newcnf{} computes its number of occurrences in intermediate clauses in $\GC$. If this number exceeds a pre-specified naming threshold, the formula $\psi$ is named as follows. Let $y_1,\ldots,y_n$ be free variables of $\psi$ and $\tau_1,\ldots,\tau_n$ be their sorts. \newcnf{} introduces a new predicate symbol $P$ of the sort $\sigma_1\times\ldots\times\sigma_n$. Then, each occurrence $\genlit{\psi}{\sign}$ in intermediate clauses in $\GC$ is replaced by $\genlit{P(y_1,\ldots,y_n)}{\sign}$. Finally, two intermediate clauses $\genclause{\{\genlit{P(y_1,\ldots,y_n)}{\negsign},\genlit{\psi}{\possign}\}}{\emptySubst}$ and $\genclause{\{\genlit{P(y_1,\ldots,y_n)}{\possign},\genlit{\psi}{\negsign}\}}{\emptySubst}$ are added to $\GC$. If the number of occurrences of $\psi$ does not exceed the naming threshold, each of the intermediate clauses that have an occurrence of $\genlit{\psi}{\possign}$ or $\genlit{\psi}{\negsign}$ is replaced with one or more new intermediate clauses according to the rules, described below.

Let $\psi$ be a subformula of $\phi$ and $\genclause{C}{\subst}$ be an intermediate clause such that $C$ has an occurrence of $\genlit{\psi}{\sign}$. The intermediate clauses that are added to $\GC$ depend on the top-level connective of $\psi$. For $\sign = \possign$ we have the following rules. The rules for $\sign = \negsign$ are dual.
\begin{itemize}
\item
	Suppose that $\psi$ is of the form $\neg \gamma$. Add an intermediate clause to $\GC$ obtained from $C$ by replacing the occurrence of $\genlit{\psi}{\possign}$ with $\genlit{\gamma}{\negsign}$.

\item
	Suppose that $\psi$ is of the form $\gamma_1 \lor \gamma_2$. Add an intermediate clause to $\GC$ obtained from $C$ by replacing the occurrence of $\genlit{\psi}{\possign}$ with $\genlit{\gamma_1}{\possign}, \genlit{\gamma_2}{\possign}$.
	
\item
	Suppose that $\psi$ is of the form $\gamma_1 \land \gamma_2$. Add two intermediate clauses to $\GC$ obtained from $C$ by replacing the occurrence of $\genlit{\psi}{\possign}$ with $\genlit{\gamma_1}{\possign}$ and $\genlit{\gamma_2}{\possign}$, respectively.

\item
	Suppose that $\psi$ in of the form $\gamma_1 \liff \gamma_2$. Add two intermediate clauses to $\GC$ obtained from $C$ by replacing the occurrence of $\genlit{\psi}{\possign}$ with $\genlit{\gamma_1}{\possign}, \genlit{\gamma_2}{\negsign}$ and $\genlit{\gamma_1}{\negsign}, \genlit{\gamma_2}{\possign}$, respectively.

\item
	Suppose that $\psi$ in of the form $\gamma_1 \lniff \gamma_2$. Add two intermediate clauses to $\GC$ obtained from $C$ by replacing the occurrence of $\genlit{\psi}{\possign}$ with $\genlit{\gamma_1}{\possign}, \genlit{\gamma_2}{\possign}$ and $\genlit{\gamma_1}{\negsign}, \genlit{\gamma_2}{\negsign}$, respectively.

\item
	Suppose that $\psi$ is of the form $(\forall x:\tau)\gamma$. Add an intermediate clause obtained from $C$ by replacing the occurrence of $\genlit{\psi}{\possign}$ with $\genlit{\gamma}{\possign}$.

\item
	Suppose that $\psi$ is of the form $(\exists x:\tau)\gamma$. Let $y_1,\ldots,y_n$ be all free variables of $\psi$ and $\tau_1,\ldots,\tau_n$ be their sorts. Introduce a fresh Skolem function symbol $\sk$ of the sort $\tau_1,\ldots,\tau_n\to\tau$. Add an intermediate clause $\genclause{C'}{\subst'}$, where $C'$ is obtained from $C$ by replacing the occurrence of $\genlit{\psi}{\possign}$ with $\genlit{\gamma}{\possign}$, and $\subst'$ extends $\subst$ with $x \mapsto \sk(y_1,\ldots,y_n)$.
\end{itemize}

When all subformulas of $\phi$ are traversed and the respective rules of replacing intermediate clauses are applied, the set $\GC$ only contains intermediate clauses with signed atomic formulas. $D$ is then converted to a set of first-order clauses by applying the substitution of each intermediate clause to its respective formulas.

Whenever an intermediate clause $\genclause{C}{\subst}$ is constructed, \newcnf{} eliminates immediate tautologies and redundant formulas. It means that
\begin{enumerate}
  \item if $C$ contains both $\genlit{\psi}{\possign}$ and $\genlit{\psi}{\negsign}$, $\genclause{C}{\subst}$ is not added to $\GC$;
  \item if $C$ contains multiple occurrences of a formula with the same sign, only one occurrence is kept in $C$;
  \item if $C$ contains $\genlit{\top}{\possign}$ or $\genlit{\bot}{\negsign}$, $\genclause{C}{\subst}$ is not added to $\GC$;
  \item if $C$ contains $\genlit{\bot}{\possign}$ or $\genlit{\top}{\negsign}$, it is not kept in $C$.
\end{enumerate}
These rules are not required for replacing intermediate clauses, however they simplify formulas and make the resulting set of clauses smaller.

\section{Clausal Normal Form for \folb}
\label{sec:newcnf/fool}
This section presents a clausification algorithm for \folb{}. This algorithm takes a \folb{} formula as input and produces a set of first-order clauses. The conjunction of these clauses is equisatisfiable to the input formula.

\folb{} extends many-sorted first-order logic with an interpreted boolean sort and the following syntactical constructs:
\begin{enumerate}
  \item boolean variables used as formulas;
  \item formulas used as arguments to function and predicate symbols;
  \item \ITE\ expressions that can occur as terms and formulas;
  \item \LETIN\ expressions that can occur as terms and formulas and can define an arbitrary number of function and predicate symbols.
\end{enumerate}

There are several ways to support the interpreted boolean sort in a first-order logic. The approach taken in~\cite{FOOL} proposes to axiomatise it by adding two constants $\true$ and $\false$ of this sorts and two axioms: $\true \neql \false$ and $(\forall x:\bool)(x \eql \true \lor x \eql \false)$. Furthermore, \cite{FOOL} proposed a modification of superposition calculus that included a replacement of the second axiom with the specialised \folb{} paramodulation rule. This modification prevents possible performance problems of a superposition theorem prover caused by self-paramodulation of $x \eql \true \lor x \eql \false$. The translation to first-order clauses presented in this section does not require boolean axioms or modifications of superposition calculus to correctly support the boolean sort. This property is explained at the end of this section.

Our algorithm is an extension of \newcnf{} that adds support for \folb{} formulas. In order to enable \newcnf{} to translate \folb{} and not just first-order formula we make the following changes to it.
\begin{itemize}
  \item We allow intermediate clauses to contain signed \folb{} formulas, and not just first-order formulas.
  \item We extend the \newcnf{} tautology elimination with the support for boolean variables. Whenever a boolean variable occurs in an intermediate clause twice with the opposite signs, that intermediate clause is not added to $\GC$. Whenever a boolean variable occurs in an intermediate clause multiple times with the same sign, only one occurrence is kept in the intermediate clause.
  \item We add extra rules that guide how intermediate clauses are replaced in the set $\GC$, detailed below. These rules correspond to syntactical constructs available in \folb{} but not in ordinary first-order logic.
  \item We change the rule that translates existentially quantified formulas to skolemise boolean variables using Skolem predicates and not Skolem functions. For that, we also allow substitutions to map boolean variables to Skolem literals. 
  \item We add an extra step of translation. After the input formula has been traversed, we apply substitutions of boolean variables to every formula in each respective intermediate clause. The resulting set of intermediate clauses might have Skolem literals occurring as terms. We run the clausification algorithm again on this set of intermediate clauses. The second run does not introduce new substitutions and results with a set of intermediate clauses that only contains atomic formulas and substitutions of non-boolean variables.
\end{itemize}

We extend the rules of replacing intermediate clauses with the cases detailed below. We will not distinguish formulas used as arguments as a separate syntactical construct, but rather treat each such formula $\phi$ as an \ITE\ expression of the form $\ite{\phi}{\true}{\false}$. We will assume that every \LETIN\ expression defines exactly one function or predicate symbol. Every \LETIN\ expression that defines more that one symbol can be transformed to multiple nested \LETIN\ expressions, each defining a single symbol, possibly by renaming some of the symbols. Moreover, we will assume that \LETIN\ expressions only occur as formulas. Every formula that contains a \LETIN\ expression that occurs as non-boolean term can be transformed to a \LETIN\ expression that defines the same symbol and occurs as formula. % EK: Should be careful with let inside let bindings!

Let $\psi$ be a subformula of the input formula $\phi$ and $\genclause{C}{\subst}$ be an intermediate clause such that $C$ has an occurrence of $\genlit{\psi}{\sign}$.
\begin{itemize}
  \item
    Suppose that $\psi$ is a boolean variable $x$. If $\subst$ does not map $x$, add the intermediate clause $\genclause{C'}{\subst'}$ to $\GC$, where $C'$ is obtained from $C$ by removing the occurrence of $\genlit{\psi}{\sign}$ and $\subst'$ extends $\subst$ with $x \mapsto \false$ if $\sign=\possign$, and $x \mapsto \true$ if $\sign=\negsign$. If $\subst$ does map $x$, add $\genclause{C}{\subst}$ to $\GC$. 

  \item
    Suppose that $\psi$ is $\gamma_1 \eql \gamma_2$, where $\gamma_1$ and $\gamma_2$ are formulas. Add two intermediate clauses to $\GC$ obtained from $C$ by replacing the occurrence of $\psi$ with $\genlit{\gamma_1}{-\sign}$, $\genlit{\gamma_2}{\possign}$ and $\genlit{\gamma_1}{\sign}$, $\genlit{\gamma_2}{\negsign}$, respectively.

  \item
    Suppose that $\psi$ is $\ite{\chi}{\gamma_1}{\gamma_2}$. Add two intermediate clauses to $\GC$ obtained from $C$ by replacing the occurrence of $\genlit{\psi}{\sign}$ with $\genlit{\chi}{\negsign}$, $\genlit{\gamma_1}{\sign}$ and $\genlit{\chi}{\possign}$, $\genlit{\gamma_2}{\sign}$, respectively.

  \item
    Suppose that $\psi$ is an atomic formula that contains one or more \ITE\ expressions occurring as terms. Each of the \ITE\ expressions is translated in one of two ways, either by expanding or by naming. We will describe both ways for a single \ITE\ expressions and then generalise for an arbitrary number of \ITE\ expressions. Suppose that $\psi$ is an atomic formula $L[\ite{\gamma}{s}{t}]$.

    \paragraph{Expanding.} Add two intermediate clauses to $\GC$ obtained from $C$ by replacing the occurrence of $\genlit{\phi}{\sign}$ with $\genlit{\gamma}{\negsign}$, $\genlit{L[s]}{\sign}$ and $\genlit{\gamma}{\possign}$, $\genlit{L[t]}{\sign}$, respectively.
    
    \paragraph{Naming.} Let $x_1,\ldots,x_n$ be all free variables of $\phi$, and $\tau_1,\ldots,\tau_n$ be their sorts. Let $\tau$ be be the sort of both $s$ and $t$. Then,
    \begin{enumerate}
      \item introduce a fresh predicate symbol $P$ of the sort $\tau\times\tau_1\times\ldots\times\tau_n$;
      \item introduce a fresh variable $y$ of the sort $\tau$;
      \item add an intermediate clause to $\GC$ that is obtained from $C$ by replacing the occurrence of $\genlit{\psi}{\sign}$ with $\genlit{L[y]}{\sign}$, $\genlit{P(y,x_1,\ldots,x_n)}{\negsign}$;
      \item add intermediate clauses $\genclause{\{\genlit{\gamma}{\negsign},\genlit{P(s,x_1,\ldots,x_n)}{\possign}\}}{\emptySubst}$ and\\$\genclause{\{\genlit{\gamma}{\possign},\allowbreak\genlit{P(t,\allowbreak x_1,\allowbreak \ldots,x_n)}{\possign}\}}{\emptySubst}$ to $\GC$.
    \end{enumerate}

    In order to eliminate all \ITE\ expressions we apply either expanding or naming to each of the \ITE\ expressions. We assume that a pre-specified expansion threshold limits the maximal number of expanded \ITE\ expressions inside one atomic formula. We start by expanding all \ITE\ expression and once the expansion threshold is reached, name the remaining \ITE\ expressions.

  \item
    Suppose that $\psi$ is $\letin{f(x_1:\sigma_1,\ldots,x_n:\sigma_n)}{t}{\gamma}$. It is translated in one of two ways, either by inlining or by naming. The choice of inlining or naming of \LETIN\ expressions in the problem is determined by a pre-specified boolean option. % provided by the user of the algorithm.
    
    \paragraph{Inlining.} Add an intermediate clause to $\GC$ that is obtained from $C$ by replacing the occurrence of $\genlit{\psi}{\sign}$ with $\genlit{\gamma'}{\sign}$. $\gamma'$ is obtained from $\gamma$ by replacing each application $f(t_1,\ldots,t_n)$ of a free occurrence of $f$ in $\gamma$ with $t'$, that is obtained from $t$ by replacing each free occurrence of $x_1,\ldots,x_n$ in $t$ with $t_1,\ldots,t_n$, respectively. We point out that inlining predicate symbols of zero arity does not hinder identification of tautologies thanks to intermediate tautology removal inside intermediate clauses.

    \paragraph{Naming.} Add an intermediate clause to $\GC$ that is obtained from $C$ by replacing the occurrence of $\genlit{\psi}{\sign}$ with $\genlit{\gamma}{\sign}$. Let $\tau$ be the sort of $t$. If $\tau$ is $\bool$, add intermediate clauses $\genclause{\{\genlit{f(x_1,\ldots,x_n)}{\negsign},\genlit{t}{\possign}\}}{\emptySubst}$ and $\genclause{\{\genlit{f(x_1,\ldots,x_n)}{\possign},\genlit{t}{\negsign}\}}{\emptySubst}$ to $\GC$. Otherwise, add an intermediate clause $\genclause{\{f(x_1,\ldots,x_n) \eql t\}}{\emptySubst}$ to $\GC$.
\end{itemize}

The extra step of translation that eliminates Skolem literals occurring as terms amounts to application of the expansion threshold-based procedure for \ITE\ expressions.

The extended \newcnf{} algorithm produces a set of first-order clauses. This set does not require boolean axioms to be equisatisfiable to the original \folb{} formula. The resulting set of clauses has the following two properties.
\begin{enumerate}
  \item It can only contain boolean variables and constants $\true$ and $\false$ as boolean terms. Every boolean term that occurs in $\phi$ is translated as formula and no boolean terms other than variables, $\true$ and $\false$ are introduced. 
  \item It does not contain equalities between boolean terms. Every boolean equality occurring in the input is translated as equivalence between its arguments, and no new boolean equalities are introduced.
\end{enumerate}
These two properties ensure that boolean variables will only be unified with $\true$ and $\false$ during superposition. Constants $\true$ and $\false$ cannot be unified with each other, therefore no logical inference can violate the properties of the boolean sort. % TODO: A more accurate argument is needed here!

\section{Discussion of the Translation}
\label{sec:newcnf/comparison}
Our extended \newcnf{} algorithm translates \folb{} formulas to sets of first-order clauses. It can be used in first-order theorem provers to support reasoning in \folb{}. This algorithm combines translation of \folb{} to first-order logic and clausification. This allowed us to enhance the translation by integrating clausification techniques into it. In particular, our extension integrates skolemisation, formula naming and tautology elimination. 

In what follows we look at the translation of different features of \folb{} done by the extended \newcnf{} and point out how the integrated clausification techniques help to obtain smaller clausal normal forms. We compare the extended \newcnf{} with our translation of \folb{} formulas to full first-order logic presented in~\cite{FOOL}.

%In the rest of this section we give specific simple examples of \folb{} formulas featuring different \folb{} constructs that are converted to a smaller CNF by the new translation compared to the old one. This comparison is merely illustrative. We provide a thorough evaluation of the new translation on a large set of benchmarks in Section~\ref{sec:newcnf/experiments}.

\subsection{Boolean Variables}
Our translation of \folb{} formulas to full first-order logic replaces each boolean variable $x$ occurring as formula with $x \eql \true$ and skolemises boolean variables using boolean Skolem functions. The extended \newcnf{} skolemises boolean variables using Skolem predicates and substitutes boolean variables that do not need skolemisation with constants $\true$ and $\false$. The approach taken in \newcnf{} is superior in two regards.
\begin{enumerate}
  \item The translation of \folb{} to full first-order logic converts each skolemised boolean variable $x$ occurring as formula to an equality between Skolem terms and $\true$. This translation requires a modification of superposition calculus presented in~\cite{FOOL} in order to avoid possible performance problems during superposition. \newcnf{} converts $x$ to a Skolem literal, that can be efficiently handled by standard superposition.  
  \item Substitution of a universally quantified boolean variable with $\true$ and $\false$ can decrease the size of the translation. If the variable occurs as formula, after applying the substitution, either the occurrence or the intermediate clause altogether will be discarded by tautology elimination.
\end{enumerate}

We note that the extended \newcnf{} translates formulas in quantified boolean form (QBF) to a clausal normal form of effectively propositional logic (EPR). Every literal in this translation is a Skolem predicate applied to boolean variables and constants $\true$ and $\false$.

% Consider a \folb\ formula $(\forall x:\bool)(x \lor P(x)),$ where $P$ is a predicate symbol of the sort $\bool \to \bool$. The old translation converts it to $(\forall x:\bool)(x \eql true \lor P(x)),$ that yields a clause $\{ x \eql \true, P(x) \}.$ The new translation eliminate the occurrence of $x$ as formula and extends the subsitution with a binding between $x$ and $\false$. The application of the substitution results with the CNF with one clause and one literal $\{ P(\false) \}$.

\subsection{\ITE}
The extended \newcnf{} translates \ITE\ expressions occurring as formulas and as terms differently.

An \ITE\ expression that occurs as formula is translated by introducing two intermediate clauses with a copy of the condition in each one. Translation of a nesting of such \ITE\ expressions easily leads to an exponential increase in the number of intermediate clauses. This is however averted by the formula naming mechanism of \newcnf{}.

An \ITE\ expression that occurs as term is translated either by expansion or naming. Expansion doubles the number of intermediate clauses with an occurrence of the condition of \ITE, and does not introduce fresh symbols. Naming adds exactly two new intermediate clauses but introduces a fresh symbol. The expansion threshold provides a trade-off between the increase of the number of intermediate clauses and the number of introduced symbols. For a large number of \ITE\ expressions it avoids the exponential increase in the number of intermediate clauses. For a small number of \ITE\ expressions inside an atomic formula it avoids growing the signature.

Formula naming averts the exponential increase in the number of intermediate clauses caused by expansion of nested \ITE\ expressions that occurs as terms. Consider for example the TPTP problem \verb'SYO500^1.003' that contains a conjecture of the form $$f_0(f_1(f_1(f_1(f_2(x))))) \eql f_0(f_0(f_0(f_1(f_2(f_2(f_2(x))))))),$$ where $f_0$, $f_1$ and $f_2$ are unary predicates that take a boolean argument and $x$ is a boolean constant. The extended \newcnf{} translates as an \ITE\ expression each application of $f_i$ that occurs as argument. Expansion of every \ITE\ expression doubles the number of intermediate clauses. However, the growth stops once the naming threshold is reached.

Our translation of FOOL formulas to full first-order logic replaces each non-boolean \ITE\ expression with an application of a fresh function symbol and adds the definition of the symbol to the set of assumptions. The definition is expressed as equality. The extended \newcnf{} avoid introducing new equalities and uses predicate guards for naming. This avoid possible performance problems caused by self-paramodulation similar to the ones described in~\cite{FOOL}.

%\paragraph{Formulas inside terms.} Consider a \folb\ formula $(\forall x:\sigma)P((\forall y:\tau)Q(x,y)),$ where $P$ and $Q$ are predicate symbols of the sorts $\bool \to \bool$ and $\sigma \times \tau \to \bool$, respectively. The old translation converts it to $(\forall x:\sigma)P(g(x)),$ where $g$ is a fresh function symbol with the following definition: $$(\forall x:\sigma)(g(x) \eql \true \liff (\forall y:\tau)Q(x,y)).$$
%
%These two formulas are then translated to the following set of clauses $$\{P(g(x))\}, \{g(x) \not\eql \true, Q(x,y\}, \{\neg Q(x,\mathit{sk}_y(x)), g(x) \eql \true\},$$ where $\mathit{sk}_y$ is a Skolem function introduced for the variable $y$.
% in one of the two implications which the equivalence represents 

%The new translation inlines $(\forall y:\tau)Q(x,y)$, yielding two intermediate clauses that are then converted to the following CNF $$\{\neg Q(x,\mathit{sk}_y(x)), P(\true)\}, \{Q(x,y), P(\false)\},$$ where $\mathit{sk}_y$ is again a Skolem function for the variables $y$.

%\paragraph{if-then-else expressions.} \cite{VampireAndFOOL} gives a definition of a $\mathit{max}$ function using an \ITE\ expression $$(\forall x:\Z)(\forall y:\Z)(\mathit{max}(x, y) \eql \ite{x \geq y}{x}{y}),$$ that is converted by the old translation to $$(\forall x:\Z)(\forall y:\Z)(\mathit{max}(x, y) \eql g(x, y)),$$ where $g$ is a fresh function symbol defined by the following formulas:
%\begin{enumerate}
%  \item $(\forall x:\Z)(\forall y:\Z)(x \geq y \implies g(x, y) \eql x)$;
%  \item $(\forall x:\Z)(\forall y:\Z)(x \not\geq y \implies g(x, y) \eql y).$
%\end{enumerate}
%This translation ultimately yields the set of three clauses $$\{\mathit{max}(x,y) \eql g(x,y)\}, \{x \not\geq y, g(x,y) \eql x\}, \{x \geq y, g(x,y) \eql y\}.$$
%
%The new translation inlines the \ITE\ expression and produces the following CNF $$\{x \not\geq y, \mathit{max}(x,y) \eql x\}, \{x \geq y, \mathit{max}(x,y) \eql y\}.$$

\subsection{\LETIN}
Our translation of \folb{} formulas to full first-order logic always name \LETIN\ expressions. The extended \newcnf{} provides the option to either name or inline \LETIN\ expressions. Naming introduces a fresh function or predicate symbol and does not multiply the resulting clauses. Inlining, on the other hand, does not introduce any symbols, but can drastically increase the number of clauses. Either of the translation might make a theorem prover inefficient. We point out that the number of clauses and the size of the resulting signature are not the only factors in that. For example, consider inlining of a \LETIN\ expression that defines a term. It does not introduce a fresh function symbol and does not increase the number of clauses. However, the inlined definition might increase the size of the term with respect to the simplification ordering. This affects the order in which literals will be selected during superposition, and ultimately the performance of the prover.

Designing a syntactical criteria for choosing between naming and inlining is an interesting task for future work.

% \cite{VampireAndFOOL} gives an example of a formula that swaps two constants in a \LETIN\ expression $\letinpar{a}{b}{b}{a}{f(a, b)}.$ The old translation converts it to $f(a'',b')$, where $a''$ and $b$ are fresh symbols with definitions $a'' \eql b$ and $b' \eql a$. The new translation simply inlines both \verb'let'-bindings, yilding $f(b,a)$.

% \paragraph{let-in expressions.} FOOL contains \LETIN\ expressions that can be used to introduce local function definitions. Consider the following \folb\ formula that expresses the fact that a binary relation $\mathit{Rel}$ on the set $\sigma$ is symmetric:
% \begin{equation*}
% \begin{aligned}
%   &\mathtt{let}\;\binding{\mathit{Inv}(x:\sigma, y:\sigma)}{\mathit{Rel}(y,x)}\;\mathtt{in}\\
%   &\quad(\forall x:\sigma)(\forall y:\sigma)(\mathit{Rel}(x,y) \liff \mathit{Inv}(x,y)))
% \end{aligned}
% \end{equation*}

% After the translation from~\cite{FOOL} is applied, it becomes $(\forall x:\sigma)(\forall y:\sigma)(\mathit{Rel}(x,y) \liff G(x,y))),$ where $C$ is a fresh predicate symbol with the definition $(\forall x:\sigma)(\forall y:\sigma)(G(x,y) \liff \mathit{Rel}(y,x)).$ This translation leads to the following set of clauses:
% \begin{equation*}
% \begin{aligned}
% &\{\neg\mathit{Rel}(x,y), G(x,y)\}, \{\neg G(x,y), \mathit{Rel}(x,y)\},\\
% &\{\neg G(x,y), \mathit{Rel}(y,x)\}, \{\neg\mathit{Rel}(y,x), G(x,y)\}.
% \end{aligned}
% \end{equation*}

% An equivalent smaller CNF can be obtained if $\mathit{Inv}$ is inlined in the \LETIN\ expression. The resulting set of clauses in such case will be $$\{\neg\mathit{Rel}(x,y), \mathit{Rel}(y,x)\}, \{\neg\mathit{Rel}(y,x), \mathit{Rel}(x,y)\}.$$

\section{Experimental Results}
\label{sec:newcnf/experiments}
Vampire is the first theorem prover to implement \newcnf{}. We extended Vampire's implementation of \newcnf{} to enable support for \folb{} formulas. This extension comprised about 500 lines of C++ code. %It will be included in the forthcoming official release of Vampire.

% Our implementation is available at \url{www.cse.chalmers.se/~evgenyk/fool-cnf-experiments/} and will be included in the forthcoming official release of Vampire.

In this section we present experimental results obtained by running Vampire on \folb{} problems. In particular, we compare performance of Vampire with the extended \newcnf{} algorithm and with the translation of \folb{} formulas to FOL presented in~\cite{FOOL}. In the sequel, by Vampire we will mean its version with the extended \newcnf{}. We will write \oldcnfVampire{} for its version with the translation of \folb{} formulas to FOL and enabled \folb{} paramodulation.

For our experiments we used two sets of problems. The first set is taken from our previous work~\cite{VampireAndFOOL} on the implementation of \folb{} in Vampire. The seconds set consists of problems from the SMT-LIB library~\cite{SMT-LIB}, a corpus of benchmarks for satisfiability modulo theory (SMT) solvers.

In our previous work~\cite{VampireAndFOOL} we experimented with our initial implementation of \folb{} in Vampire. For that experiment we generated two set of \folb{} problems.
\begin{enumerate}
  \item Problems from the higher-order part of the TPTP library~\cite{TPTP} that can be directly expressed in \folb{}. We translated these problems from the TPTP language of higher-order logic to the modification of TPTP that supports \folb{}.
  \item Problems about properties of (co)algebraic datatypes generated by the Isabelle theorem prover~\cite{Isabelle} to be checked by SMT solvers. We translated these problems from the SMT-LIB~2 language to TPTP using the SMTtoTPTP tool~\cite{SMTLIB2TPTP}.
\end{enumerate}

For this work we used the second set of problems and run Vampire in the matching experimental setup. We did not use the first set in this work~--- problems in this set are easy and all of them were already solved by Vampire before.

Our results are summarised in Tables~\ref{table:isabelle-results}--\ref{table:smt-lib-results2} and discussed below. 

\subsection[Experiments with Algebraic Datatypes Problems]{Experiments with Algebraic\\Datatypes Problems}
The set of problems about (co)algebraic datatypes generated by Isabelle and translated by us to the TPTP syntax contains 152 problems. All of them use \folb{} features: boolean variables occurring as formulas, formulas occurring as arguments to function and predicate symbols, and \ITE\ expressions. None of the 152 problems use \LETIN\ expressions.

We run Vampire twice on each problem: once using the option \verb'--mode' \verb'casc', and once using \verb'--mode casc_sat'. Both times the \ITE\ expansion threshold was set to 3, the default value. For each problem, we recorded the fastest successful run of Vampire. \iffalse The experiments were run on a MacBook Pro with a 2,9 GHz Intel Core i5 and 8 Gb RAM, and using the time limit of 60 seconds per problem.\fi We then compared Vampire with the results of \oldcnfVampire, CVC4~\cite{CVC4} and Z3~\cite{Z3}, taken from~\cite{VampireAndFOOL}. 

Table~\ref{table:isabelle-results} summarises the results of our experiments on these 152 problems. Vampire solved the largest number of problems, and all problems solved by \oldcnfVampire\ were also solved by Vampire. Figure~\ref{fig:isabelle-diagram} shows the Venn diagram of the sets of problems solved by Vampire, CVC4 and Z3, where the numbers denote the numbers of solved problems. Compared to \oldcnfVampire, Vampire solved one more problem that was previously only solved by Z3 and 18 more problems, not solved by either Z3 or CVC4. This is significant because the problems were tailored for SMT reasoning. Based on our experimental results we observe that our implementation of the extended \newcnf{} improved the performance of Vampire on this set of problems.

\begin{table}[tb]
  \caption{Runtimes in seconds of provers on the set of 152 algebraic datatypes problems.}
  \begin{center}
  \begin{tabular}{lrr}
    \hline Prover & Solved & Total time on solved problems \\ \hline
    Vampire & 78 & 19.416 \\
    \oldcnfVampire & 59 & 26.580 \\
    Z3 & 57 & 4.291 \\
    CVC4 & 53 & 25.480
  \end{tabular}
  \end{center}
  \label{table:isabelle-results}
\end{table}

\begin{figure}[tb]
  %\vspace{-0.3em}
  \centering
  \begin{tikzpicture}
    \draw (0,0) circle (1.5cm);
    \draw (50:1cm) circle (1.55cm);
    \draw (0cm:0.8cm) circle (1.4cm);
    \node at (0.8cm:0.5cm) {$51$}; %
    \node at (-2.2cm:1.2cm) {$1$};
    \node at (1cm:-1.1cm) {$1$};
    \node at (-2cm:-1.1cm) {$4$};
    \node at (-3.8cm:-1.9cm) {$22$}; %
    \node at (-0.95cm:1.775cm) {$0$}; %
    \node at (0.45cm:1.775cm) {$1$}; %
    \node at (2.7cm:2.65cm) {Vampire};
    \node at (-3.7cm:1.8cm) {Z3};
    \node at (-1.6cm:2.3cm) {CVC4};
  \end{tikzpicture}
  \vspace{-0.3em}
  \caption{Venn diagram of the subsets of the algebraic datatypes problems, solved by Vampire, CVC4 and Z3.}
  \label{fig:isabelle-diagram}
\end{figure}

\subsection{Experiments with SMT-LIB Problems}

\folb{} can be regarded as a superset of SMT-LIB core logic and problems of SMT-LIB core logic can be directly expressed in \folb{}. The language of \folb{} extends the SMT-LIB core language with local function definitions, using \LETIN\ expressions defining functions of arbitrary, and not just zero, arity. A theorem prover that supports \folb{} can be straightforwardly extended to read problems written in the SMT-LIB syntax.

For this experiment we used problems in quantified predicate logic with uninterpreted functions stored in the UF subspace of the SMT-LIB library. These problems are written in the SMT-LIB~2 syntax. In order to read them we implemented a parser for a sufficient subset of the SMT-LIB~2 language in Vampire. The implementation comprised about 2,500 lines of C++ code. 

In this experiment we evaluated performance of Vampire, \oldcnfVampire, and CVC4 on unsatisfiable problems of the UF subspace. Each problem in the SMT-LIB library is marked with one of the statuses \verb'sat', \verb'unsat' and \verb'unknown'. A problem is marked as \verb'sat' or \verb'unsat' when at least two SMT solved proved it to be satisfiable or unsatisfiable, respectively. Otherwise, a problem is marked as \verb'unknown'. In order to filter out satisfiable problems we run Vampire, \oldcnfVampire, and CVC4 on the problems marked as \verb'unsat' and \verb'unknown' and then recorded the results on the problems that were proven unsatisfiable by at least one prover. That gave us 2596 problems.

The problems in this set use \ITE\ expressions, \LETIN\ expressions that define constants, and formulas 
occurring as arguments to equality. None of the problems use quantifiers over the boolean sort.

We run Vampire twice on each problem: once with naming of \LETIN\ expressions, and once with inlining. Both times the \ITE\ expansion threshold was set to 3, the default value. In both runs we also used the option \verb'--mode casc'. For each problem, we recorded the fastest successful run of Vampire. We run \oldcnfVampire\ once on each problem with the option \verb'--mode casc'.

\begin{table}[tb]
  \caption{Runtimes in seconds of provers on the set of 2596 unsatisfiable SMT-LIB problems.}
  \begin{center}
  \begin{tabular}{lrr}
    \hline Prover & Solved & Total time on solved problems \\ \hline
    Vampire & 2329 & 11,057.374 \\
    CVC4 & 2084 & 26,309.466 \\
    \oldcnfVampire & 2060 & 14,189.568
  \end{tabular}
  \end{center}
  \label{table:smt-lib-results2}
\end{table}

\begin{figure}[tb]
  \centering
  \begin{tikzpicture}
    \draw (0,0) circle (1.5cm);
    \draw (50:1.5cm) circle (1.6cm);
    \draw (0cm:1.3cm) circle (1.45cm);
    \node at (0.8cm:0.75cm) {$1887$}; %
    \node at (-1.85cm:1.05cm) {$17$};
    \node at (0.8cm:-0.8cm) {$104$};
    \node at (-2.8cm:-1.0cm) {$76$};
    \node at (-4.1cm:-2.3cm) {$220$}; %
    \node at (-0.75cm:2cm) {$146$}; %
    \node at (0.65cm:1.95cm) {$146$}; %
    \node at (2.55cm:3.2cm) {Vampire};
    \node at (-3.7cm:1.8cm) {CVC4};
    \node at (-1.6cm:2.4cm) {\oldcnfVampire};
  \end{tikzpicture}
  \vspace{-0.3em}
  \caption{Venn diagram of the subsets of the unsatisfiable SMT-LIB problems, solved by Vampire, \oldcnfVampire\ and CVC4.}
  \label{fig:smt-lib-newcnf-diagram}
\end{figure}

Table~\ref{table:smt-lib-results2} summarises the results of our experiments on the SMT-LIB problems. These results are obtained on the StarExec compute cluster~\cite{starexec} using the time limit of 5 minutes per problem. Vampire solved the largest number of problems and was the fastest among the provers. None of the provers solved a superset of problems solved by another prover. Figure~\ref{fig:smt-lib-newcnf-diagram} shows the Venn diagram of the sets of problems solved by Vampire, \oldcnfVampire, and CVC4, where the numbers denote the numbers of solved problems. Vampire solved 296 problem not solved by \oldcnfVampire, and \oldcnfVampire\ solved 163 problems not solved by Vampire. Moreover, we recorded how different translations of \LETIN\ affected the performance of Vampire. Vampire with inlining of \LETIN\ expressions solved 314 problems not solved by Vampire without inlining of \LETIN\ expressions. Vampire without inlining of \LETIN\ expressions solved 95 problems not solved by Vampire without inlining of \LETIN\ expressions.

Based on the results of this experiment we make the following observations. Vampire solved new problems by inlining \LETIN\ expressions and expanding \ITE\ expressions. Vampire could not solve some of the problems that were solved by \oldcnfVampire, likely because of expanding of \ITE\ rather than naming. Both inlining and naming of \LETIN\ expressions can make a prover inefficient.

% On unsat:

% newcnf:
% - solved with on but not with off: 78
% - solved with off but not with on: 75
% - solved by both: 1776

% newcnf vs oldcnf:
% - solved by newcnf but not oldcnf: 70
% - solved by oldcnf but not newcnf: 14
% - solved by both: 1859
% - solved by neither: 96

% cvc vs oldcnf:
% - solved by cvc but not oldcnf: 102
% - solved by oldcnf but not cvc: 24
% - solved by both: 1905
% - solved by neither: 8

% On unknown:

% newcnf:
% - solved with on but not with off: 236
% - solved with off but not with on: 20
% - solved by both: 144

% newcnf vs oldcnf:
% - solved by newcnf but not oldcnf: 226
% - solved by oldcnf but not newcnf: 149
% - solved by both: 174
% - solved by either: 549

% cvc vs oldcnf:
% - solved by cvc but not oldcnf: 21
% - solved by oldcnf but not cvc: 267
% - solved by both: 56
% - solved by either: 344

% \begin{table}[tb]
%   \caption{Runtimes in seconds of provers on the set of 2039 unsatisfiable SMT-LIB problems.}
%   \begin{center}
%   \begin{tabular}{lrr}
%     \hline Prover & Solved & Total time on solved problems \\ \hline
%     CVC4 & 2007 & 18569.676 \\
%     Vampire & 1929 & 8540.553 \\
%     \oldcnfVampire & 1873 & 11734.979
%   \end{tabular}
%   \end{center}
%   \label{table:smt-lib-unsat}
% \end{table}

% \begin{table}[tb]
%   \caption{Runtimes in seconds of provers on the set of ? SMT-LIB problems with the status marked as unknown.}
%   \begin{center}
%   \begin{tabular}{lrr}
%     \hline Prover & Solved & Total time on solved problems \\ \hline
%     Vampire & 400 & 2516.821 \\
%     \oldcnfVampire & 187 & 2454.589 \\
%     CVC4 & 77 & 7739.790
%   \end{tabular}
%   \end{center}
%   \label{table:smt-lib-unknown}
% \end{table}

%\section{Related Work}
%\label{sec:newcnf/related}
%\EK{TODO}

\section{Conclusion and Future Work}
\label{sec:newcnf/conclusions}
We presented a clausification algorithm for \folb{}. It takes a \folb{} formula as input and produces an equisatisfiable set of first-order clauses. Our algorithm is based on the \newcnf{} clausification algorithm for first-order logic and extends it to support \folb{} formulas.

Our algorithm aims to minimise the number of clauses and the size of the resulting signature it produces. It combines translation of \folb{} to first-order logic and clausification. This combination allowed us to integrate into the translation clausification techniques such as skolemisation, formula naming and tautology elimination.

% The new translation is different from the old one:
% * boolean variables are skolemised with predicates and not functions
% * boolean variables that do not need to be skolemised are exhaustively instantiated with the two possible boolean constants $\true$ and $\false$ in a way that does not increase the size of the translation
% * it never introduces an equality that does not appear in the original formula (guards instead of definitions)
% * parametrised by a \ITE\ expansion threashold 
% * Don't need FOOL paramodulation

We implemented the extended \newcnf{} algorithm in the Vampire theorem prover. Our experimental results showed an increase of performance of Vampire compared to its version with the translation of \folb{} formulas to full first-order logic. We observed that new problems can be solved by expansion of \ITE\ and instantiation of boolean variables with boolean constants. We observed that both inlining and naming of \LETIN\ expressions can make a theorem prover succeed or fail. 

For future work we are interested in developing syntactical criteria that determine whether a given \LETIN\ or \ITE\ expression should be named or inlined, or expanded or inlined, respectively. 

%We do not have problems that use let with functions with arguments. They would've been useful though.



\bibliography{refs}

\end{document}