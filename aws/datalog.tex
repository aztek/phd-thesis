We answer a network question by translating the network model and the network question to a \Datalog program and then running it using the \Datalog engine \souffle\cite{souffle}. A \Datalog program consists of a \Datalog \emph{query} and a finite set of \Datalog \textit{rules}. We obtain a query from the network question and a set of rules from both the network model and the network question. We assume that each \Datalog rule has the standard form $A\rl L_1\comma\ldots\comma L_n$ $(n\ge0)$, where $A$ is an atom and each of $L_1,\ldots,L_n$ is a literal, and a \Datalog query has the form $L_1\comma\ldots\comma L_n$ $(n\ge0)$, where each of $L_1,\ldots,L_n$ is a literal. We also assume that all \Datalog rules use stratified negation.

% \EK{Network specification is intentional database, network snapshot is extensional database.}

\souffle accepts definitions of typed relations, contains the predefined symbol and numeric types, and accepts definitions of new types. The types in \souffle are interpreted under the open-world assumption. We model the types of the network models, interpreted as finite domains, using \Datalog relations with one argument. Let $\tau$ be a type and $c_1,\ldots,c_n$ ($n\ge0$) be constants of this type. We introduce a relation $\typedRel{\tau}$ and add the facts $\typed{\tau}{c_i}$, $1\le i\le n$ to the set of \Datalog rules. We use literals of the form $\typed{\tau}{t}$ in every \Datalog rule to guard the argument $t$ of the type $\tau$ in the head of the rule. \EK{Mention that we can simplify it and only use typing literals near negations.}

Let $p(t_1,\ldots,t_n)\leftarrow L_1\wedge\ldots\wedge L_m$ $(n\ge0,m\ge0)$ be a rule in the network model, where $p$ is a predicate of the type $\tau_1\times\ldots\times\tau_n$ and each of $L_1,\ldots,L_m$ is a literal. We translate $p$ to a \Datalog relation $r$ and translate this rule to the \Datalog rule $$r(t_1,\ldots,t_n)\rl\typed{\tau_1}{t_1}\comma\ldots\comma\typed{\tau_n}{t_n}\comma L_1\comma\ldots\comma L_m.$$

We translate a network question expressed as a first-order formula $\phi$ without function symbols to a \Datalog query and a set of \Datalog rules. We start by converting $\phi$ to a prenex disjunctive normal form that is $$(\forall x_1:\tau_1)\ldots(\forall x_n:\tau_n)(\exists y_1:\sigma_1)\ldots(\exists y_m:\sigma_m)(C_1\vee\ldots\vee C_k),$$ where $n\ge0$, $m\ge0$, $k\ge0$, and each of $C_1,\ldots,C_k$ is a conjunction of literals. Let $z_1:\upsilon_1,\ldots,z_l:\upsilon_l$ be all free variables of $\phi$. $l=0$ for formulas expressing boolean network questions and $l>0$ for formulas expressing list network question. We introduce two fresh relations $r$ and $q$ of the types $\tau_1\times\ldots\times\tau_n\times\upsilon_1\times\ldots\times\upsilon_l$ and $\upsilon_1\times\ldots\times\upsilon_l$, respectively. The translated set of \Datalog rules consists of $k+1$ rules: $k$ rules of the form
\begin{align*}
r(x_1,\ldots,x_n,z_1,\ldots,z_l)\rl\;&\typed{\tau_1}{x_1}\comma\ldots\comma\typed{\tau_n}{x_n}\comma~\\
                                     &\typed{\upsilon_1}{z_1}\comma\ldots\comma\typed{\upsilon_l}{z_l}\comma C_i
\end{align*}
for each $1\le i\le k$ and the rule
\begin{align*}
q(z_1,\ldots,z_l)\rl\typed{\upsilon_1}{z_1}\comma\ldots\comma\typed{\upsilon_l}{z_l}\comma\neg r(x_1,\ldots,x_n,z_1,\ldots,z_l).
\end{align*}
Note that we can use each conjunction $C_i$ in a \Datalog rule because each literal in $C_i$ only contains variables and constants~--- there are no function symbols in $\phi$ and they do not appear during a conversion to prenex disjunctive normal form. Finally, the \Datalog query is $\neg q(z_1,\ldots,z_l)$.

\EK{Explain why negations are needed.}

We translate types $\type{bits16}$ and $\type{bits32}$ to numeric types for 32 and 16-bit integers, respectively, and translate the predicates over bit vectors into their correspondent built-in \souffle operations.

We illustrate our translation using examples from Section~\ref{sect:aws/motivation}. We translate Formula~\ref{eq:bool-property} that expresses a boolean network question to the \Datalog rules
\begin{align*}
r(W,D)\rl\;& \typed{instance}{W}\wedge\neg\pred{instanceHasSubnet}(W,\const{subnet}_\text{Web}).\\
r(W,D)\rl\;& \typed{instance}{D}\wedge\neg\pred{instanceHasSubnet}(D,\const{subnet}_\text{Database}).\\
r(W,D)\rl\;& \typed{instance}{W}\wedge\typed{instance}{D}\:\wedge\:\\
           & \pred{instanceCanConnectToInstance}(W,D).\\
q\rl\;& \neg r(W,D).
\end{align*}
and the \Datalog query $\neg q$. Note that multiple rules in the definition of $r$ appear because of the translation to disjunctive normal form. We translate Formula~\ref{eq:list-property} that expresses a list network question to the \Datalog rules
\begin{align*}
r(I,E)\rl\;& \typed{instance}{I}\wedge\typed{eni}{E}\:\wedge\:\\
           & \pred{instanceHasEni}(I,E)\:\wedge\:\\
           &\begin{aligned}
               \pred{reachablePublicTcpUdp}(&\const{dir}_\text{ingress},\const{proto}_6,E,\const{port}_{22},\\
                                            &\const{publicIp}_\text{8:8:8:8},\const{port}_\text{40000}).
             \end{aligned}\\
q(I,E)\rl\;& \typed{instance}{I}\wedge\typed{eni}{E}\wedge\neg r(I,E).
\end{align*}
and the \Datalog query $\neg q(I,E)$.

%For example, consider a list query expressed by the formula $(\forall x:\tau)(\exists y:\sigma)(r(x, y) \wedge g(y))$, where $r$ and $g$ are predicates in the network model. We translate it to a \Datalog relation $p$ of the type $\tau\times\sigma$ and a nullary \Datalog relation $q$, two \Datalog rules $p(X,Y)\rl r(X,Y)\comma g(Y)$ and $q\rl \neg p(X,Y)$, and a \Datalog query $\neg q$.
