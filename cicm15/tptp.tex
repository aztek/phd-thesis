The typed monomorphic first-order formulas subset, called TFF0, of the TPTP language~\cite{TPTP}, is a representation language for many-sorted first-order logic. It contains \ITE\ and \LETIN\ constructs (see below), which is useful for applications, but is inconsistent in its treatment of the boolean sort. It has a predefined atomic sort symbol \lstinline'$o' denoting the boolean sort. However, unlike all other sort symbols, \lstinline'$o' can only be used to declare the return type of predicate symbols. This means that one cannot define a function having a boolean argument, use boolean variables or equality between booleans. 

Such an inconsistent use of the boolean sort results in having two kinds of \ITE\ expressions and four kinds of \LETIN\ expressions. For example, a \folb-term $\letin{f(x_1:\sigma_1, \ldots, x_n:\sigma_n)}{s}{t}$ can be represented using one of the four TPTP alternatives \lstinline'$let_tt', \lstinline'$let_tf', \lstinline'$let_ft' or \lstinline'$let_ff', depending on whether $s$ and $t$ are terms or formulas. 

Since the boolean type is second-class in TPTP, one cannot directly represent formulas coming from program analysis and interactive theorem provers, such as formulas \eqref{formula:contains} and \eqref{formula:subset-sorted} of Section~\ref{sec:cicm15/introduction}.

We propose to modify the TFF0 language of TPTP to coincide with \folb. It is not late to do so, since there is no general support for \ITE\ and \LETIN. To the best of our knowledge, Vampire is currently the only theorem prover supporting full TFF0. Note that such a modification of TPTP would make multiple forms of \ITE\ and \LETIN\ redundant. It will also make it possible to directly represent the SMT-LIB core theory.

We note that our changes and modifications on TFF0 can also be applied to the TFF1 language of TPTP~\cite{tff1}. TFF1 is  a polymorphic extension of TFF0 and its formalisation  does not treat the boolean sort. Extending our work to TFF1 should not be hard but has to be done in detail.