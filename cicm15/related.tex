% !TEX root = ../main.tex
Handling boolean terms as formulas is common in the SMT community. The SMT-LIB project~\cite{SMT-LIB} defines its core logic as first-order logic extended with the distinguished first-class boolean sort and the \verb'let'-\verb'in' expression used for local bindings of variables. The core theory of SMT-LIB defines logical connectives as boolean functions and the ad-hoc polymorphic \verb'if'-\verb'then'-\verb'else' ($ite$) function, used for conditional expressions. 
% SMT-solvers do not reason in the core logic, but use quantifier-free fragments of it with theories. 
The language \folb\ defined here extends the SMT-LIB core language with local function definitions,
using \verb'let'-\verb'in' expressions defining functions of arbitrary, and not just zero, arity. This, \folb\ contains both this language and the TFF0 subset of TPTP. Further, we present a translation of \folb\ to FOL and show how one can improve superposition theorem provers to reason with the boolean sort. 

% Unlike SMT-LIB, \folb{} defines logical connectives as interpreted functions and not as part of a theory, and the \verb'if'-\verb'then'-\verb'else' construct as part of the logic language.

Efficient superposition theorem proving in finite domains, such as the boolean domain, is also discussed in~\cite{HillenbrandWeidenbach13}. The approach of~\cite{HillenbrandWeidenbach13} sometimes falls back to enumerating instances of a clause by instantiating finite domain variables with all elements of the corresponding domains. We point out here that for the boolean (i.e., two-element) domain there is a simpler solution. However, the approach of~\cite{HillenbrandWeidenbach13} also allows one to handle domains with more than two elements. One can also generalise our approach to arbitrary finite domains by using binary encodings of finite domains, however, this will necessarily result in loss of efficiency, since a single variable over a domain with $2^k$ elements will become $k$ variables in our approach, and similarly for function arguments.
