% !TEX root = ../main.tex
\folb\ is a modification of FOL. Every FOL formula is syntactically a \folb\ formula and has the same models, but not the other way around. In this section we present a translation from \folb\ to FOL, which preserves models. This translation can be used for proving theorems of \folb\ using a first-order theorem prover. We do not claim that this translation is efficient -- more research is required on designing translations friendly for first-order theorem provers.

We do not formally define many-sorted FOL with equality here, since FOL is essentially a subset of \folb, which we will discuss now.  

We say that an occurrence of a subterm $s$ of the sort $\bool$ in a term $t$ is in a \emph{formula context} if it is an argument of a logical connective or the occurrence in either $(\forall x:\sigma)s$ or $(\exists x:\sigma)s$. We say that an occurrence of $s$ in $t$ is in a \emph{term context} if this occurrence is an argument of a function symbol, different from a logical connective, or an equality. We say that a formula of \folb\ is \emph{syntactically first order} if it contains no \ITE\ and \LETIN\ expressions, no variables occurring in a formula context and no formulas occurring in a term context. By restricting the definition of terms to the subset of syntactically first-order formulas, we obtain the standard definition of many-sorted first-order logic, with the only exception of having a distinguished Boolean sort and constants $\true$ and $\false$ occurring in a formula context.

Let $\phi$ be a closed $\Sigma$-formula of \folb{}. We will perform the following steps to translate $\phi$ into a first-order formula. During the translation we will maintain a set of formulas $D$, which initially is empty. The purpose of $D$ is to collect a set of formulas (definitions of new symbols), which guarantee that the transformation preserves models.

\begin{enumerate}
\item Make a sequence of translation steps obtaining a syntactically first order formula $\phi'$. During this translation we will introduce new function symbols and add their types to the type assignment $\context$. We will also add formulas describing properties of these symbols to $D$. The translation will guarantee that the formulas $\phi$ and $\bigwedge_{\psi \in D}\psi \wedge \phi'$ are equivalent, that is, have the same models restricted to $\Sigma$.

\item Replace the constants $\true$ and $\false$, standing in a formula context, by nullary predicates $\top$ and $\bot$ respectively, obtaining a first-order formula.

\item Add special Boolean sort axioms.
\end{enumerate}
During the translation, we will say that a function symbol or a variable is \emph{fresh} if it neither appears in $\phi$ nor in any of the definitions, nor in the domain of $\context$.

We also need the following definition. Let $\context \vdash t:\sigma$, and $x$ be a variable occurrence in $t$. The \emph{sort of this occurrence of $x$} is defined as follows:

\begin{enumerate}
\item any free occurrence of $x$ in a subterm $s$ in the scope of $(\forall x:\sigma')s$ or $(\exists x:\sigma')s$ has the sort $\sigma'$.
\item any free occurrence of $x_i$ in a subterm $s_1$ in the scope of \\$\letin{f(x_1:\sigma_1, \ldots, x_n:\sigma_n)}{s_1}{s_2}$ has the sort $\sigma_i$, where $i = 1,\ldots,n$. 
\item a free occurrence of $x$ in $t$ has the sort $\context(x)$.
\end{enumerate}
If $\context \vdash t:\sigma$, $s$ is a subterm of $t$ and $x$ a free variable in $s$, we say that $x$ has a sort $\sigma'$ in $s$ if its free occurrences in $s$ have this sort.

The translation steps are defined below. We start with an empty set $D$ and an initial \folb\ formula $\phi$, which we would like to change into a syntactically first-order formula. At every translation step we will select a formula $\chi$, which is either $\phi$ or a formula in $D$, which is not syntactically first-order, replace a subterm in $\chi$ it by another subterm, and maybe add a formula to $D$. The translation steps can be applied in any order.

\begin{enumerate}
  \item Replace a Boolean variable $x$ occurring in a formula context, by $x \eql \true$.

  \item Suppose that $\psi$ is a formula occurring in a term context such that (i) $\psi$ is different from $\true$ and $\false$, (ii) $\psi$ is not a variable, and (iii) $\psi$ contains no free occurrences of function symbols bound in $\chi$. Let $x_1,\ldots,x_n$ be all free variables of $\psi$ and $\sigma_1,\ldots,\sigma_n$ be their sorts. Take a fresh function symbol $g$, add the formula $(\forall x_1:\sigma_1)\ldots(\forall x_n:\sigma_n) (\psi \liff g(x_1,\ldots,x_n) \eql \true)$ to $D$ and replace $\psi$ by $g(x_1,\ldots,x_n)$. Finally, change $\context$ to $\context,g : \sigma_1 \times \ldots \times \sigma_n \to \bool$.

  \item Suppose that $\ite{\psi}{s}{t}$ is a term containing no free occurrences of function symbols bound in $\chi$. Let $x_1,\ldots,x_n$ be all free variables of this term and $\sigma_1,\ldots,\sigma_n$ be their sorts. Take a fresh function symbol $g$, add the formulas $(\forall x_1:\sigma_1)\ldots(\forall x_n:\sigma_n) (\psi \implies g(x_1,\ldots,x_n) \eql s)$ and $(\forall x_1:\sigma_1)\ldots(\forall x_n:\sigma_n) (\neg\psi \implies g(x_1,\ldots,x_n) \eql t)$ to $D$ and replace this term by $g(x_1,\ldots,x_n)$. Finally, change $\context$ to $\context,g : \sigma_1 \times \ldots \times \sigma_n \to \sigma_0$, where $\sigma_0$ is such that $\context,x_1:\sigma_1,\ldots,x_n:\sigma_n \vdash s : \sigma_0$.

  \item Suppose that $\letin{f(x_1:\sigma_1, \ldots, x_n:\sigma_n)}{s}{t}$ is a term containing no free occurrences of function symbols bound in $\chi$. Let $y_1,\ldots,y_m$ be all free variables of this term and $\tau_1,\ldots,\tau_m$ be their sorts. Note that the variables in $x_1,\ldots,x_n$ are not necessarily disjoint from the variables in $y_1,\ldots,y_m$. 

Take a fresh function symbol $g$ and fresh sequence of variables $z_1,\ldots,z_n$. Let the term $s'$ be obtained from $s$ by replacing all free occurrences of $x_1,\ldots,x_n$ by $z_1,\ldots,z_n$, respectively. Add the formula $(\forall z_1:\sigma_1)\ldots(\forall z_n:\sigma_n) (\forall y_1:\tau_1)\ldots(\forall y_m:\tau_m) (g(z_1,\ldots,z_n,y_1,\ldots,\allowbreak y_m) \eql s')$ to $D$. Let the term $t'$ be obtained from $t$ by replacing all bound occurrences of $y_1,\ldots,y_m$ by fresh variables and each application $f(t_1, \ldots, t_n)$ of a free occurrence of $f$ in $t$ by $g(t_1, \ldots, t_n,\allowbreak y_1, \ldots, y_m)$. Then replace $\letin{f(x_1:\sigma_1, \ldots, x_n:\sigma_n)}{s}{t}$ by $t'$. Finally, change $\context$ to $\context,g : \sigma_1 \times \ldots \times \sigma_n \times \tau_1 \times \ldots \times \tau_m \to \sigma_0$, where $\sigma_0$ is such that $\context,x_1:\sigma_1,\ldots,x_n:\sigma_n,y_1:\tau_1,\ldots,y_m:\tau_m \vdash s : \sigma_0$. 
\end{enumerate}
The translation terminates when none of the above rules apply.

We will now formulate several of properties of this translation, which will imply that, in a way, it preserves models. These properties are not hard to prove, we do not include proofs in this paper.

\begin{lemma}\label{lemma:step-preserves-equivalence}\rm
  Suppose that a single step of the translation changes a formula $\phi_1$ into $\phi_2$, $\delta$ is the formula added at this step (for step 1 we can assume $\true=\true$ is added), $\context$ is the type assignment before this step and $\context'$ is the type assignment after. Then for every $\context'$-interpretation $\intI$ we have $\intI \models \delta \implies (\phi_1 \liff \phi_2)$. \QED
\end{lemma}

By repeated applications of this lemma we obtain the following result.

\begin{lemma}\label{lemma:definitions-preserve-models}\rm
  Suppose that the translation above changes a formula $\phi$ into $\phi'$, $D$ is the set of definitions obtained during the translation, $\context$ is the initial type assignment and $\context'$ is the final type assignment of the translation. Let $I'$ be any interpretation of $\context'$. Then $I' \models \bigwedge_{\psi \in D} \psi \implies (\phi \Leftrightarrow \phi')$. \QED
\end{lemma}

We also need the following result.

\begin{lemma}\label{lem:termination}\rm
  Any sequence of applications of the translation rules terminates. \QED
\end{lemma}

The lemmas proved so far imply that the translation terminates and the final formula is equivalent to the initial formula in every interpretation satisfying all definitions in $D$. To prove model preservation, we also need to prove some properties of the introduced definitions. 

\begin{lemma}\label{lem:satisfy:definitions}\rm
  Suppose that one of the steps 2--4 of the translation translates a formula $\phi_1$ into $\phi_2$, $\delta$ is the formula added at this step, $\context$ is the type assignment before this step, $\context'$ is the type assignment after, and $g$ is the fresh function symbol introduced at this step. Let also $\intI$ be $\context$-interpretation. Then there exists a function $h$ such that $\replacement{\intI}{g}{h} \models \delta$. \QED
\end{lemma}

These properties imply the following result on model preservation.

\begin{theorem}\label{thm:model:preservation}\rm
  Suppose that the translation above translates a formula $\phi$ into $\phi'$, $D$ is the set of definitions obtained during the translation, $\context$ is the initial type assignment and $\context'$ is the final type assignment of the translation. 
  \begin{enumerate}
    \item Let $\intI$ be any $\context$-interpretation. Then there is a $\context'$-interpretation $I'$ such that $\intI'$ is an extension of $\intI$ and $\intI' \models \bigwedge_{\psi \in D} \psi \wedge \phi'$.
    \item Let $\intI'$ be a $\context'$-interpretation and $\intI' \models \bigwedge_{\psi \in D} \psi \wedge \phi'$. Then $\intI' \models \phi$. \QED
  \end{enumerate}
\end{theorem}
This theorem implies that $\phi$ and $\bigwedge_{\psi \in D} \psi \wedge \phi'$ have the same models, as far as the original type assignment (the type assignment of $\Sigma$) is concerned. The formula $\bigwedge_{\psi \in D} \psi \wedge \phi'$ in this theorem is syntactically first-order. Denote this formula by $\gamma$. Our next step is to define a model-preserving translation from syntactically first-order formulas to first-order formulas.

To make $\gamma$ into a first-order formula, we should get rid of $\true$ and $\false$ occurring in a formula context. To preserve the semantics, we should also add axioms for the Boolean sort, since in first-order logic all sorts are uninterpreted, while in \folb\ the interpretations of the Boolean sort and constants $\true$ and $\false$ are fixed. 

To fix the problem, we will add axioms expressing that the Boolean sort has two elements and that $\true$ and $\false$ represent the two distinct elements of this sort.
\begin{equation}\label{axiom:bool}
  \forall (x:\bool)(x \eql \true \vee x \eql \false) \wedge \true \not\eql \false.
\end{equation}
Note that this formula is a tautology in \folb, but not in FOL.

Given a syntactically first-order formula $\gamma$, we denote by $\toFOL{\gamma}$ the formula obtained from $\gamma$ by replacing all occurrences of $\true$ and $\false$ in a formula context by logical constants $\top$ and $\bot$ (interpreted as always true and always false), respectively and adding formula \eqref{axiom:bool}.

\begin{theorem}\label{thm:model:preservation:2}\rm
  Let $\context$ is a type assignment and $\gamma$ be a syntactically first-order formula such that $\context \vdash \gamma:\bool$.
  \begin{enumerate}
  \item Suppose that $\intI$ is a $\context$-interpretation and $\intI \models \gamma$ in \folb. Then $\intI \models \toFOL{\gamma}$ in first-order logic.
  \item Suppose that $\intI$ is a $\context$-interpretation and $\intI \models \toFOL{\gamma}$ in first-order logic. Consider the \folb-interpretation $\intI'$ that is obtained from $\intI$ by changing the interpretation of the Boolean sort $\bool$ by $\{0,1\}$ and the interpretations of $\true$ and $\false$ by the elements $1$ and $0$, respectively, of this sort. Then $\intI' \models \gamma$ in \folb. \QED
  \end{enumerate}
\end{theorem}

Theorems~\ref{thm:model:preservation} and~\ref{thm:model:preservation:2} show that our translation preserves models. Every model of the original formula can be extended to a model of the translated formulas by adding values of the function symbols introduced during the translation. Likewise, any first-order model of the translated formula becomes a model of the original formula after changing the interpretation of the Boolean sort to coincide with its interpretation in \folb.