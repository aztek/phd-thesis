% !TEX root = ../main.tex
We defined first-order logic with the first class Boolean sort (\folb{}). It extends ordinary many-sorted first-order logic (FOL) with (i) the Boolean sort such that terms of this sort are indistinguishable from formulas and (ii) \ITE\ and \LETIN\ expressions. The semantics of \LETIN\ expressions in \folb{} is essentially their semantics in functional programming languages, when they are not used for recursive definitions. In particular, non-recursive local functions can be defined and function symbols can be bound to a different sort in nested \verb'let'-\verb'in' expressions.

We argued that these extensions are useful in reasoning about problems coming from program analysis and interactive theorem proving. The extraction of properties from certain program definitions (especially in functional programming languages) into \folb{} formulas is more straightforward than into ordinary FOL formulas and potentially more efficient. In a similar way, a more straightforward translation of certain higher-order formulas into \folb{} can facilitate proof automation in interactive theorem provers.

\folb{} is a modification of FOL and reasoning in it reduces to reasoning in FOL. We gave a translation of \folb{} to FOL that can be used for proving theorems in \folb{} in a first-order theorem prover. We further discussed a modification of superposition calculus that can reason efficiently in presence of the Boolean sort. Finally, we pointed out that the TPTP language can be changed to support \folb{}, which will also simplify some parts of the TPTP syntax.

Implementation of theorem proving support for \folb{}, including its super\-po\-sition-friendly translation to CNF, is an important task for future work. Further, we are also interested in extending \folb{} with theories, such as the theory of integer linear arithmetic and arrays.