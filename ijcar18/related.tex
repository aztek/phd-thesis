Our previous work introduced FOOL~\cite{FOOL}, its implementation in Vampire~\cite{VampireAndFOOL}, and an efficient clausification algorithm for FOOL formulas~\cite{FOOLCNF}.

In~\cite{VampireAndFOOL} we sketched a tuple extension of FOOL and an algorithm for computing the next state relations of imperative programs that uses this extension. This paper extends and improves the algorithm. In particular, \begin{enumerate*}[label=(\roman*)]\item we described an encoding that uses FOOL in its current form, available in Vampire, \item we refined the encoding to only use in \LETIN\ the variables updated in program statements, \item we gave the definition of the encoding formally and in full detail, and \item we presented experimental results that confirm the described benefits of the encoding.\end{enumerate*}

Boogie is used as the name of both the intermediate verification language~\cite{leino2008boogie} and the automated verification framework~\cite{DBLP:conf/fmco/BarnettCDJL05}. The Boogie verifier encodes the next state relations of imperative programs in first-order logic by naming intermediate states of program variables~\cite{DBLP:journals/ipl/Leino05}.

BLT~\cite{CF-iFM17} is a tool that automatically generates verification conditions of Boogie programs. The aim of the BLT project is to use first-order theorem provers rather than SMT solvers for checking quantified program properties. BLT produces formulas written in the TPTP language and uses \ITE\ and \LETIN\ constructs of FOOL. BLT has an experimental option that introduces tuples for encoding of the next state relation. This option implements the encoding described in our earlier work~\cite{VampireAndFOOL}.