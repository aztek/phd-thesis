% !TEX root = ../main.tex
We presented an encoding of the next state relations of imperative programs in FOOL. Based on this encoding we defined a translation from imperative programs, annotated with their pre- and post-conditions, to FOOL formulas that encode partial correctness properties of these programs. We presented experimental results obtained by running the theorem prover Vampire on such properties. We generated these properties using our translation and verification tools Boogie and BLT. We described a polymorphic theory of first class tuples and its implementation in Vampire.

The formulas produced by our translation can be efficiently checked by automated theorem provers that support FOOL. The structure of our encoding closely resembles the structure of the program. The encoding contains neither new symbols nor duplicated parts of the program. This way, the efficient representation of the problem in plain first-order logic is left to the theorem prover that is better equipped to do it.

Our encoding is useful for automated program analysis and verification. Our experimental results show that Vampire was more efficient in terms of the number of solved problems and runtime on the problems obtained using our translation.

FOOL reduces the gap between programming languages and languages of automated theorem provers. Our encoding relies on tuple expressions and \LETIN\ with tuple definitions, available in FOOL. To our knowledge, these constructs are not available in any other logic efficiently implemented in automated theorem provers.

The polymorphic theory of first class tuples is a useful addition to a first-order theorem prover. On the one hand, it generalises and simplifies tuple expressions in FOOL. On the other hand, it is a convenient theory on its own, and can be used for expressing problems of program analysis and computer mathematics.

For future work we are interested in making automated first-order theorem provers friendlier to program analysis and verification. One direction of this work is design of an efficient translation of features of programming languages to languages of automated theorem provers. Another direction is extensions of first-order theorem provers with new theories, such as the theory of bit vectors. Finally, we are interested in further improving automated reasoning in combination of theories and quantifiers.